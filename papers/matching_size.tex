\documentclass{article}

\usepackage{hyperref}
\usepackage{comment}
\usepackage{polyglossia}
\setmainlanguage{russian}
\setotherlanguage{english}

\usepackage{geometry}
\geometry{
     top=18pt, bottom=14pt,
     left=.5cm,right=1cm,
     paperwidth=5.5in, paperheight=8.5in,
     }

\setmainfont[
 Ligatures=TeX,
 Extension=.otf,
 BoldFont=cmunbx,
 ItalicFont=cmunti,
 BoldItalicFont=cmunbi,
]{cmunrm}
% С засечками (для заголовков)
\setsansfont[
 Ligatures=TeX,
 Extension=.otf,
 BoldFont=cmunsx,
 ItalicFont=cmunsi,
]{cmunss}
\setmonofont[Scale=1.0,
    BoldFont=lmmonolt10-bold.otf,
    ItalicFont=lmmono10-italic.otf,
    BoldItalicFont=lmmonoproplt10-boldoblique.otf
]{lmmono9-regular.otf}

\usepackage[cache=true]{minted}
\renewcommand{\epsilon}{\ensuremath{\varepsilon}}
\usepackage{amsmath}

\begin{document}

\section{Попробую в общем виде}
Рассматриваем fully abstract типы с рекурсией. Конструкторы -- $C_1\dots C_n$. Количество рекурсивных аргументов в i-м конструкторе -- $a_i$, рекурсивных аргументов во всех конструкторах -- $\sum_{i=1}^{n} a_i$. Таким образом нерекурсивных аргументов как бы нет и новых жителей они не дают. Все типы записываются как полином
\[
t(x) = c_0 + c_1*t(x) + \dots + c_n*t^n(x) + ...
\]
Здесь степени -- это количество рекурсивных аргументов конструктора


$L(h) = \begin{cases}
O(1) & h=1\\
\sum{a_i} * L(h-1),&h>1
\end{cases} = O(c^h)$

Наибольшее влияние оказывает наибольшая степень (где больше всего рекурсивных аргументов)

$S(h) = \sum_{i=1}^{h} L(i)  = O(c^h)$


\section{Про количество примеров}

$L(h)$ -- количество жителей высоты $h\ge 1$

$S(h)$ -- количество жителей высоты от 1 до $h$

Очевидно, что на $L(h)$ больше влияет количество конструкторов, а на $S(h)$ -- на сколько сильно они ветвятся

Пару примеров
\begin{itemize}
\item Числа Пеано: $p(x) = 1 + p(x)$\\ $L(h) = 1$, $S(h) = h$

\item Числа Пеано с двумя видами листов: $p(x) = 2 + p(x)$\\
$L(h) = 2$, $S(h) = 2*h$
\item Деревья: $t(x) = 1 + t^2(x)$\\
$L(h) = \begin{cases}
1, & h=1\\
2*L(h-1),&h>1
\end{cases} = 2^{h-1}$

$S(h) = \sum_{i=1}^{h} L(i) =\sum_{i=1}^{h} 2^{i-1} =2^h - 1$

\item Деревья с двумя видами листов: $t(x) = 2 + t^2(x)$\\
$L(h) = 2^h$\\

$S(h) = \sum_{i=1}^{h} L(i) =\sum_{i=1}^{h} 2^{i} =2(2^h - 1)$

\end{itemize}


\end{document}