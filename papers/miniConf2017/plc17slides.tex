\documentclass[10pt, mathserif]{beamer}

\usepackage[utf8]{inputenc}
\usepackage[russian]{babel}

\usepackage{listings}
\usepackage{color}
\usepackage{amssymb, amsmath}
\usepackage[all]{xy}
\usepackage{alltt}
\usepackage{pslatex}
\usepackage{epigraph}
\usepackage{verbatim}
\usepackage{graphicx}
\usepackage{latexsym}
\usepackage{array}
\usepackage{changepage} % for adjustwidth
%\usepackage{courier}
\usepackage{pstricks}
\usepackage{tikz}
\usepackage{pdfpages}

\makeatletter
\newcolumntype{e}[1]{%--- Enumerated cells ---
   >{\minipage[t]{\linewidth}%
     \NoHyper%                Hyperref adds a vertical space
     \let\\\tabularnewline
     \enumerate
        \addtolength{\rightskip}{0pt plus 50pt}% for raggedright
        \setlength{\itemsep}{-\parsep}}%
   p{#1}%
   <{\@finalstrut\@arstrutbox\endenumerate
     \endNoHyper
     \endminipage}}

\newcolumntype{i}[1]{%--- Itemized cells ---
   >{\minipage[t]{\linewidth}%
        \let\\\tabularnewline
        \itemize
           \addtolength{\rightskip}{0pt plus 50pt}%
           \setlength{\itemsep}{-\parsep}}%
   p{#1}%
   <{\@finalstrut\@arstrutbox\enditemize\endminipage}}

\AtBeginDocument{%
    \@ifpackageloaded{hyperref}{}%
        {\let\NoHyper\relax\let\endNoHyper\relax}}
\makeatother

\definecolor{shadecolor}{gray}{1.00}
\definecolor{darkgray}{gray}{0.30}

\newcommand{\set}[1]{\{#1\}}
\newcommand{\angled}[1]{\langle {#1} \rangle}
\newcommand{\fib}{\rightarrow_{\mathit{fib}}}
\newcommand{\fibm}{\Rightarrow_{\mathit{fib}}}
\newcommand{\oo}[1]{{#1}^o}
\newcommand{\inml}[1]{\mbox{\lstinline{#1}}}

\setlength{\epigraphwidth}{.55\textwidth}

\definecolor{light-gray}{gray}{0.90}
\newcommand{\graybox}[1]{\colorbox{light-gray}{#1}}

\newcommand{\nredrule}[3]{
  \begin{array}{cl}
    \textsf{[{#1}]}&
    \begin{array}{c}
      #2 \\
      \hline
      \raisebox{-1pt}{\ensuremath{#3}}
    \end{array}
  \end{array}}

\newcommand{\naxiom}[2]{
  \begin{array}{cl}
    \textsf{[{#1}]} & \raisebox{-1pt}{\ensuremath{#2}}
  \end{array}}


\newcommand\Wider[2][3em]{%
\makebox[\linewidth][c]{%
  \begin{minipage}{\dimexpr\textwidth+#1\relax}
  \raggedright#2
  \end{minipage}
  }%
}

\lstdefinelanguage{ocaml}{
keywords={let, begin, end, in, match, type, and, fun,
function, try, with, class, object, method, of, rec, repeat, until,
while, not, do, done, as, val, inherit, module, sig, @type, struct,
if, then, else, open, virtual, new, fresh},
sensitive=true,
basicstyle=\small\ttfamily,
commentstyle=\scriptsize\rmfamily,
keywordstyle=\ttfamily\bfseries,
identifierstyle=\ttfamily,
basewidth={0.5em,0.5em},
columns=fixed,
fontadjust=true,
literate={->}{{$\to$}}1
         {===}{{$\equiv$}}1
}

\lstdefinelanguage{scheme}{
keywords={define, conde, fresh},
sensitive=true,
basicstyle=\small,
commentstyle=\scriptsize\rmfamily,
keywordstyle=\ttfamily\bfseries,
identifierstyle=\ttfamily,
basewidth={0.5em,0.5em},
columns=fixed,
fontadjust=true,
literate={==}{{$\equiv$}}1
}

\lstset{
basicstyle=\small,
identifierstyle=\ttfamily,
keywordstyle=\bfseries,
commentstyle=\scriptsize\rmfamily,
basewidth={0.5em,0.5em},
% fontadjust=true,
escapechar=!,
language=ocaml
}

\setbeamertemplate{footline}[frame number]
\setbeamertemplate{navigation symbols}{}
\setbeamertemplate{blocks}[rounded][shadow=true]
\beamertemplateballitem

\mode<presentation>{
  \usetheme{default}
}

\theoremstyle{definition}

\title{Уменьшение цены абстракции при типобезопасном встраивании реляционнного
языка программирования в OCaml}

\author{Дмитрий Косарев}

\institute[]{
\small{
\textbf{Санкт-Петербургский Государтсвенный Университет} \\
\textbf{JetBrains Research}
}
}

\date{
   \vskip 1cm
   \small{
   \textbf{Языки программирования и компиляторы}\\
   4 апреля, 2016 \\
   Ростов-на-Дону}
}

\begin{document}
\begin{frame}
  \titlepage
\end{frame}

\begin{frame}{Реляционное программирование на miniKanren}
 % \vskip1cm
 \begin{center}
 От программ-\emph{функций} к программам-\emph{отношениям}:
 \end{center}

 $$
 f \colon X \to Y\;\leadsto\;\oo{f} \subseteq X\times Y
 $$
 \vskip5mm
 \begin{tabular}{m{4cm}m{6cm}}
    \includegraphics[scale=0.3]{trs.jpg} &
    \begin{itemize}
       \item Изначально DSL для Scheme/Racket с довольно минималистичной реализацией
       \item Семейство языков ($\mu$Kanren, $\alpha$-Kanren, cKanren, и т.д.)
       \item Встраивается как DSL в широкий набор языков (включая OCaml, Haskell, Scala, и т.д.)
       \item Daniel P. Friedman, William Byrd and Oleg Kiselyov. \emph{The Reasoned Schemer},
       The MIT Press, Cambridge, MA, 2005
    \end{itemize}
 \end{tabular}
 \vskip 3cm
\end{frame}

\begin{frame}[fragile]{Пример: реляционное слияние списков (OCaml/OCanren/miniKanren)}
%\vskip5mm
\begin{adjustwidth}{-1.5em}{-1.5em}
\begin{tabular}{m{6cm}m{6cm}}
 \graybox{$\inml{append} \colon \alpha\:\inml{list} \to \alpha\:\inml{list} \to \alpha\:\inml{list}$} &
 \graybox{$\oo{\inml{append}}\subseteq\alpha\:\inml{list}\times\alpha\:\inml{list}\times\alpha\:\inml{list}$}\\
 \begin{lstlisting}[language=ocaml,keywordstyle=\bfseries]
let rec append xs ys =
  match xs with
  | []      -> ys
  | h :: tl ->
      h :: (append tl ys)!\pause!
 \end{lstlisting} &
 \begin{lstlisting}[mathescape=true,language=ocaml]
let rec append$^o$ xs ys xys =!\pause!
  ((xs === nil) &&& (xys === ys))!\pause!
  |||
  (fresh (h t tys)!\pause!
     (xs  === h % t)!\pause!
     (xys === h % tys)!\pause!
     (append$^o$ t ys tys) )
 \end{lstlisting}
\end{tabular}\pause
\begin{center}
\begin{minipage}{6cm}
\centering{\small\bf В оригинальной реализации:}
\begin{lstlisting}[mathescape=true,language=scheme]
(define (append$^o$ xs ys xys)
   (conde
      [(== '() xs) (== ys xys)]
      [(fresh (h t tys)
         (== `(,h . ,t) xs)
         (== `(,h . ,tys) xys)
         (append$^o$ t ys tys))]))
\end{lstlisting}
\end{minipage}
\end{center}

\end{adjustwidth}
\vskip5mm
\end{frame}

\begin{frame}[fragile]{Набросок минимальной реализации}
Jason Hemann, Daniel P. Friedman. \emph{$\mu$Kanren: A Minimal Functional
Core for Relational Programming} // Scheme'13:
\vskip5mm
\small
\pause
\begin{tabular}{p{5cm}c}
Логические переменные  & $X=\{x_1,x_2,\dots\}$ \vspace*{-\baselineskip} \\
Символы (конструкторы) & $S=\{s_1,s_2,\dots\}$ \vspace*{-\baselineskip} \\
Термы                  & $T=X\cup\{s\;(t_1,\dots,t_k)\mid s\in S,\; t_i \in T\}$ \vspace*{-\baselineskip} \\
Подстановки            & $\Sigma=T^X$ \vspace*{-\baselineskip} \\
\\\pause
Унификация & $(\equiv)\colon\Sigma\to T\to T\to\Sigma_\perp$ \vspace*{-\baselineskip}\\
\\\pause
State (подстановка + как создавать новые логические переменные) & $\sigma$ \vspace*{-\baselineskip}\\
Goal (функция из состояния в ленивый список состояний) & $g : \sigma\to\sigma\;\inml{stream}$ \vspace*{-\baselineskip} \\
Конъюнкция $g \wedge g$ & ``bind''  \vspace*{-\baselineskip} \\
Дизъюнкция $g \vee   g$ & ``mplus'' \vspace*{-\baselineskip} \\

Refinement: извлечение посчитанных ответов & $\inml{refine} \colon \sigma\to X\to T$ \vspace*{-\baselineskip}
\end{tabular}

% \begin{center}
% {\bf Unification and refinement are virtually the main things to implement}
% \end{center}
\end{frame}

% \begin{frame}[fragile]{Dealing with Typed Terms}
% \pause
% \begin{itemize}
%   \item Non-solution:
%      \begin{itemize}
%         \item implement unification for a fixed term representation;
%         \item convert user-type data to- and from that universal representation.
%       \end{itemize}\pause
%       {\small\bf\textcolor{red}{Does not help in detecting the mistakes statically}}\pause
%   \item Bad solution: generate boilerplate unification code for each user type
%       (there is no direct support for $ad-hoc$ polymorphism in OCaml yet).\pause
%
%       {\small\textcolor{red}{\bf Boilerplatish}}\pause
%
%       {\small\textcolor{red}{\bf Users would need to adjust their types to represent logical variables}}\pause
%   \item Polymorphic unification:
%
%     $$
%      \equiv\;\colon \Sigma\to\alpha\to\alpha\to\Sigma_{\perp}
%     $$\pause
%     \small
%     {\bf\textcolor{red}{Has to be implemented in an untyped manner}}
%     \pause
%
%     {\bf\textcolor{green}{Might be a good solution (lightweight, efficient), if
%     type safety is justified}}
% \end{itemize}
%
% \end{frame}

\begin{frame}[fragile]{Полиморфная унификация}

Работает для всех \emph{логических} типов $\alpha\; logic$ (он же $\oo{\alpha}$):

$$
\equiv\;\colon \Sigma\to\oo{\alpha}\to\oo{\alpha}\to\Sigma_{\perp}
$$
\pause

Реализована как сравнение представлений значений в памяти.
% \pause
% \begin{figure}
% \centering
% \includegraphics[width=0.5\textwidth]{img0.eps}
% \end{figure}

% \begin{lstlisting}[mathescape=true]
% type 'a logic = Var of int | Value of 'a
% ...
% val x : 'a logic
% val s : string logic
% ...!\pause!
% (x === s)
% ...
% !\pause!
% val n : int logic
% ... (x === n) ...
%
% Error: This expression has type int logic
%        but an expression was expected of type string logic
%        Type int is not compatible with type string
% \end{lstlisting}

%\vskip3mm
% Pitfalls:
%
% \begin{itemize}
% \item compiler loses the track of types after the results of unification
% are stored in a substitution $\leadsto$ refinement has to be implemented untyped as well;\pause
% \item the safety of unification/refinement implementation has to be justified separately;\pause
% \item states must not escape their scope (otherwise the coherence between variable
% types and terms, stored in states, can be lost).
% \end{itemize}
\end{frame}


\begin{frame}[fragile]{Систематическое конструирование логических типов}
  % mathescape=true,
  \begin{lstlisting}[mathescape=true]
  type $\alpha$ logic = Var of int | Value of $\alpha$
  ...!\pause!
  type ($\alpha$, $\beta$) glist = Nil | Cons of $\alpha$ * $\beta$ !\pause!

  type $\alpha$  list = ($\alpha$, $\alpha$ list) glist !\pause!

  type $\alpha$ llist = ($\alpha$, $\alpha$ llist) glist logic !\pause!

  ...
  # Value Nil
  -: $\alpha$ llist
  # Value (Cons (Value 1), Value Nil)
  -: int logic llist
  # Value (Cons (Var 101), Value Nil)
  -: int logic llist
  \end{lstlisting}
\end{frame}
% \begin{frame}[fragile]{Properties of Polymorphic Unification}
% It can be shown, that for our concrete implementation:
%
% \begin{itemize}
% \item variables in a substitution are always associated with the terms of the same type;
% \item all variables preserve their types, assigned by the compiler;
% \item all variables occur in terms only in a ``type-safe'' positions:
%
% $$
% t[x] \iff \mbox{the type of }x\mbox{ corresponds to the type of the hole of }t
% $$
% \end{itemize}\pause
%
% $$\leadsto$$
%
% the refinement is type-safe, if a variable is refined in a state, which is an inheritor
% of the state that variable was created in.
% \end{frame}

% \begin{frame}[fragile]{Capturing the States}
% States and refinement function are hidden and can not be accessed directly.\pause
% \vskip3mm
% The refinement is performed transparently as the top-level running primitive
% is invoked:\pause
% \vskip3mm
% \begin{lstlisting}[mathescape=true]
%     run $\bar{n}$ (fun $q_1\;q_2\;\dots\;q_n\;$ -> $\;\;$g) (fun $a_1\;a_2\;\dots\;a_n\;$ -> $\;\;$h)
% \end{lstlisting}\pause
%
% Here:
% {\small
% \begin{itemize}
% \item \lstinline{run}~--- the only way to run goals;
% \item $\bar{n}$~--- a \emph{numeral}, describing the number of fresh variables,
% available for running the goal \lstinline{g}; numerals can be manufactured
% \emph{quantum satis} using the successor function, which is provided as well;
% \item $q_1, q_2\dots q_n$~--- these fresh variables;
% \item $a_1, a_2\dots a_n$~--- the streams of \emph{refined} answers for the variables
% $q_1, q_2\dots q_n$ respectively;
% \item \lstinline{h}~--- a \emph{handler}, which can make use of refined answers.
% \end{itemize}
% }
%
% The framework guarantees, that variables are refined only in correct states.
%
% \end{frame}

% \begin{frame}[fragile]{Converting from- and to User Types}
%
% \small
% Injecting a user-type into logic domain and projecting the
% logical results back:
%
% $$
% \begin{array}{rcl}
% \uparrow_t&\colon &t \to \oo{t}\\
% \downarrow_t&\colon& \oo{t} \to t
% \end{array}
% $$
% \pause
% Can be done systematically using generic programming:
%
% \begin{itemize}
% \item ``$\uparrow_\forall$'', ``$\downarrow_\forall$'' are polymorphic shallow injection/projection;
% \item for the deep case, make the type a functor and use $fmap$.
% \end{itemize}
% \pause
% \begin{lstlisting}[mathescape=true]
% type tree = Leaf of int | Node of tree * tree!\pause!
% $\leadsto$
% type ('int, 'tree) tree$_f$ = Leaf of 'int | Node of 'tree * 'tree!\pause!
% type tree  = (int, tree) tree$_f$!\pause!
% type tree$^o$ = ((int$^o$, tree$^o$) tree$_f$)$^o$!\pause!
%
% let rec ($\uparrow_{\mbox{\small\ttfamily tree}}$) t = $\uparrow_\forall$ (!\graybox{$fmap_{\mbox{\small\ttfamily tree}_f}$}! ($\uparrow_\forall$) ($\uparrow_{\mbox{\small\ttfamily tree}}$) t)!\pause!
% let rec ($\downarrow_{\mbox{\small\ttfamily tree}}$) l = !\graybox{$fmap_{\mbox{\small\ttfamily tree}_f}$}! ($\downarrow_\forall$) ($\downarrow_{\mbox{\small\ttfamily tree}}$) ($\downarrow_\forall$ l)
%
% \end{lstlisting}
% \end{frame}

\begin{frame}[fragile]{Промежуточные результаты}
Были представлены на ML Workshop 2016 (совмещённым с ICFP 2016)
\begin{itemize}
\item Типобезопасное встраивание miniKanren в OCaml
\item Полиморфная унификация
\item Регулярный подход для описания типов
\end{itemize}

\end{frame}

%{
%\setbeamercolor{background canvas}{bg=}
% \includepdf[pages={1}]{histo1.pdf}
%}
\begin{frame}{Результаты сравнения производительности}
\centering
\includegraphics[scale=0.85]{histo1.pdf}
\end{frame}

\begin{frame}[fragile]{Дальнейшие задачи}
  \begin{itemize}
  \item Найти причину замедления
  \item Ускорить
  \item Подход должен остаться типобезопасным
  \end{itemize}

\end{frame}


% \begin{frame}[fragile]{Полиморфная унификация 2 из 7}
%   %\begin{figure}
%   \centering
%   %\includegraphics[width=0.6\textwidth]{img0.pdf}
%   \includegraphics[scale=0.5]{img0.png}
%   %\end{figure}
% \end{frame}

\begin{frame}[fragile]{Представление термов}
  \centering
  \includegraphics[width=0.9\textwidth]{img1.png}
  %\end{figure}
\end{frame}

% \begin{frame}[fragile]{Тегированное представление логических значений}
%   \begin{figure}
%   \centering
%   \includegraphics[width=0.9\textwidth]{img2.png}
%   \end{figure}
% \end{frame}

\begin{frame}[fragile]{Тегированное представление логических значений}
  \begin{figure}
  \centering
  \includegraphics[width=0.9\textwidth]{img3.png}
  \end{figure}
\end{frame}

\begin{frame}[fragile]{Рост размера термов из-за тегирования}
\begin{figure}
\centering
\includegraphics[width=0.9\textwidth]{img4.png}
\end{figure}
\end{frame}

\begin{frame}[fragile]{План улучшения реализации}
  \begin{itemize}
  \item Новое представление деревьев
    \begin{itemize}
      \item{Значению нельзя присвоить конкретный тип, нужен абстрактный тип значений.}
      \item{Предоставить интерфейс для конструирования логических значений}
      \item{Дополнительные действия по преобразованию абстрактного логического значения в типизируемое}
    \end{itemize}
  \item Модернизировать подход по описанию типов логических значений
  \item Не потерять типовую безопасность
  \end{itemize}
\end{frame}

\begin{frame}[fragile]{Основная идея}
\begin{itemize}
 \item Унифицировать нетипизированнные термы
 \item Преобразовывать к типизируемому представлению при refine
 \item Запоминать формальные типы значений при каждом преобразовании к логическому значению
\end{itemize}

\begin{figure}
\centering
\includegraphics[width=0.9\textwidth]{img5.png}
\end{figure}
\end{frame}

\begin{frame}[fragile]{Тип injected}
  \begin{lstlisting}[language=ml]
  type ('a, 'b) injected!
  \end{lstlisting}

\vskip 1cm
  Тип \verb='a= -- это исходный тип, а тип \verb='b= -- его логическое представление
  
  Представление ground-типов совпадает с представлением \verb='a=.
\end{frame}

\begin{frame}[fragile]{Конструирование логических значений для простых типов}
  \begin{lstlisting}[language=ml]
  type ('a, 'b) injected!\pause!

  val lift: 'a -> ('a,'a) injected
  val inj : ('a,'b) injected -> ('a,'b logic) injected
  \end{lstlisting}
  \pause
  Например для чисел
  \begin{lstlisting}[language=ml]
  # inj (lift 5)
  -: (int, int logic) injected
  \end{lstlisting}
  \pause
  Оба введенных примитива оставляют переданное значение как есть (identity)
\end{frame}

\begin{frame}[fragile]{Конструирование логических значений для сложных типов}
  \begin{lstlisting}[language=ml,mathescape=true]
  module Option = struct
    type $\alpha$ option = None | Some of $\alpha$
    let fmap = ....
  end!\pause!

  # Make1(Option).distrib
  ...!\pause!
  # let some x  = inj @@ distrib (Some x)
  -: ($\alpha$, $\beta$) injected -> $\qquad$($\alpha$ option, $\beta$ option logic) injected
  \end{lstlisting}
  \pause

  Здесь \emph{fmap} нужен для доказательства того, что тип является функтором, т.е.
  чтобы можно было описать примимитив \emph{distrib}, который позволяет ``снять'' 
  тип со значения, ничего не делая со значением (он тоже identity).

\end{frame}

\begin{frame}[fragile]{Восстановление посчитанных значений}
  Необходимо, так как значения в типе \emph{(\_,\_) injected} хранятся в
  нетипизированном виде.

  \begin{lstlisting}[language=ml,mathescape=true]
  module Option = struct
    type $\alpha$ option = None | Some of $\alpha$
    let fmap = ....
  end!\pause!

    # Make1(Option).reify
    -: ( ($\alpha$, $\beta$) injected -> $\qquad\beta$) -> !\pause!
       ($\alpha$ option, $\beta$ option logic) injected -> $\qquad\beta$ option logic

  \end{lstlisting}

  При построении \emph{reify} функция \emph{fmap} используется по существу.
\end{frame}

\begin{frame}[fragile]{Текущая реализация}
\begin{itemize}
\item Репозиторий: \url{https://github.com/dboulytchev/OCanren}
\item Реализация $\mu$Kanren с неравенствами (disequality constraints)
\item Работает на большинстве оригинальных примеров
\item Быстрее $\mu$Kanren (\url{https://github.com/Kakadu/ocanren-perf})
\end{itemize}

\end{frame}

% \begin{frame}[fragile]{}
% 
% \centering
% \begin{center}
% \Huge Вопросы?
% \end{center}
% 
% \end{frame}

\end{document}
