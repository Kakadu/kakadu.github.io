\section{Заключение}
\label{sec:futurework}

Существует несколько возможны направлений для дальнейшего развития проекта. Во-первых, в данной работе мы не касались вопросов производительности. Мы представляем преобразования в очень обобщенном виде, с несколькими слоями косвенности. Очевидно, что преобразования, реализованные с помощью нашей библиотеки будут работать медленнее, чем написанные вручную. Мы предполагаем, что производительность может быть
улучшено с помощью, так называемого, staging~\cite{Staged} или, возможно, с помощью оптимизаций, специфичных для объектов.

Другим важным направлением является поддержка большего разнообразия объявлений типов, а именно GADT и нерегулярных типов. Хотя уже сделаны некоторые наработки, получившиеся решение делает интерфейс всей библиотеки чересчур сложным даже для простых случаев.

Наконец, структура с информацией о типе, которую мы генерируем, может быть использована, чтобы симитировать \emph{ad-hoc} полиморфизм, так как они содержит реализацию функций, индексированных типами. Это в сумме с недавно предложенными расширениями~\cite{ModularImplicits} может открыть интересные перспективы.

