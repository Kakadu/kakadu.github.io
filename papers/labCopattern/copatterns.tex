\documentclass[10pt, mathserif]{beamer}

\usepackage[utf8]{inputenc}
\usepackage[russian]{babel}

\usepackage{listings}
\usepackage{color}
\usepackage{amssymb, amsmath}
\usepackage[all]{xy}
\usepackage{alltt}
\usepackage{pslatex}
\usepackage{epigraph}
\usepackage{verbatim}
\usepackage{graphicx}
\usepackage{latexsym}
\usepackage{array}
\usepackage{changepage} % for adjustwidth
\usepackage{pstricks}
\usepackage{tikz}
\usepackage{stmaryrd}
\usepackage{pdfpages}
\usepackage{fontawesome}
\usepackage[scaled]{DejaVuSansMono}
\usepackage[T1]{fontenc}


\makeatletter
\newcolumntype{e}[1]{%--- Enumerated cells ---
   >{\minipage[t]{\linewidth}%
     \NoHyper%                Hyperref adds a vertical space
     \let\\\tabularnewline
     \enumerate
        \addtolength{\rightskip}{0pt plus 50pt}% for raggedright
        \setlength{\itemsep}{-\parsep}}%
   p{#1}%
   <{\@finalstrut\@arstrutbox\endenumerate
     \endNoHyper
     \endminipage}}

\newcolumntype{i}[1]{%--- Itemized cells ---
   >{\minipage[t]{\linewidth}%
        \let\\\tabularnewline
        \itemize
           \addtolength{\rightskip}{0pt plus 50pt}%
           \setlength{\itemsep}{-\parsep}}%
   p{#1}%
   <{\@finalstrut\@arstrutbox\enditemize\endminipage}}

\AtBeginDocument{%
    \@ifpackageloaded{hyperref}{}%
        {\let\NoHyper\relax\let\endNoHyper\relax}}
\makeatother

\definecolor{shadecolor}{gray}{1.00}
\definecolor{darkgray}{gray}{0.30}

\newcommand{\set}[1]{\{#1\}}
\newcommand{\angled}[1]{\langle {#1} \rangle}
\newcommand{\fib}{\rightarrow_{\mathit{fib}}}
\newcommand{\fibm}{\Rightarrow_{\mathit{fib}}}
\newcommand{\oo}[1]{{#1}^o}
\newcommand{\inml}[1]{\mbox{\lstinline{#1}}}
\newcommand{\sem}[1]{\llbracket #1\rrbracket}

\setlength{\epigraphwidth}{.55\textwidth}

\definecolor{light-gray}{gray}{0.90}
\newcommand{\graybox}[1]{\colorbox{light-gray}{#1}}

\newcommand{\nredrule}[3]{
  \begin{array}{cl}
    \textsf{[{#1}]}&
    \begin{array}{c}
      #2 \\
      \hline
      \raisebox{-1pt}{\ensuremath{#3}}
    \end{array}
  \end{array}}

\newcommand{\naxiom}[2]{
  \begin{array}{cl}
    \textsf{[{#1}]} & \raisebox{-1pt}{\ensuremath{#2}}
  \end{array}}


\newcommand\Wider[2][3em]{%
\makebox[\linewidth][c]{%
  \begin{minipage}{\dimexpr\textwidth+#1\relax}
  \raggedright#2
  \end{minipage}
  }%
}
\newcommand*\bigcdot{\mathpalette\bigcdot@{.5}}

\lstdefinelanguage{ocaml}{
keywords={let, begin, end, in, match, type, and, fun, module,
function, try, with, class, object, method, of, rec, repeat, until,
corec, cofunction, lazy,
while, not, do, done, as, val, inherit, module, sig, @ type, struct,
if, then, else, open, virtual, new, fresh},
sensitive=true,
basicstyle=\small\ttfamily,
commentstyle=\scriptsize\rmfamily,
keywordstyle=\ttfamily\bfseries,
%identifierstyle=\ttfamily,
basewidth={0.5em,0.5em},
columns=fixed,
moredelim=**[is][\only<+>{\color{black}\lstset{style=highlight}}]{@}{@},
fontadjust=true,
literate={->}{{$\to$}}1
	 {<-}{{\leftarrow}}1
         {===}{{$\equiv$}}1
         {=/=}{{$\not\equiv$}}1
}

\lstdefinelanguage{scheme}{
keywords={define, conde, fresh},
sensitive=true,
basicstyle=\small,
commentstyle=\scriptsize\rmfamily,
keywordstyle=\ttfamily\bfseries,
identifierstyle=\ttfamily,
basewidth={0.5em,0.5em},
columns=fixed,
fontadjust=true,
literate={==}{{$\equiv$}}1
}

\lstset{
% basicstyle=\small,
basicstyle=\footnotesize, 
%identifierstyle=\ttfamily,
keywordstyle=\bfseries,
commentstyle=\scriptsize\rmfamily,
basewidth={0.5em,0.5em},
% fontadjust=true,
escapechar=!,
% escapeinside={(*@}{@*),
language=ocaml
}

\setbeamertemplate{footline}[frame number]
\setbeamertemplate{navigation symbols}{}
\setbeamertemplate{blocks}[rounded][shadow=true]
\beamertemplateballitem

\mode<presentation>{
  \usetheme{default}
}

\AtBeginSection[]
{
\begin{frame}{Table of Contents}
\tableofcontents[currentsection]
\end{frame}
}
\theoremstyle{definition}

\title{Copattern matching and first-class observations in OCaml, with a macro}
\author{Дмитрий Косарев}

\date{
  \vskip 2cm
  \small{
    \textbf{19 февраля, 2018}
  }
}


% some slides https://www.cl.cam.ac.uk/events/metaprog2016/codata-types-and-copattern-matching.pdf
% original paper https://hal.inria.fr/hal-01653261/document
\begin{document}
\begin{frame}
  \titlepage
\end{frame}


\section{Мотивация}

\begin{frame}{Мотивация}
 
\begin{itemize}
  \item Конечные
  \begin{itemize}
    \item Например: список, дерево, ...
    \item Индуктивные типы и pattern matching
  \end{itemize}
  \item Бесконечные
  \begin{itemize}
    \item Например: stream, бесконечное дерево, ...
    \item Коиндуктивные типы и copattern matching
  \end{itemize}
  \vspace{1in}
\end{itemize}
\pause
Copatterns: Programming Infinite Structures by Observations
Abel, Pientka, Thibodeau и Setzer (POPL, 2013)

и другие статьи от тех же авторов
% Есть две группы статей, про копаттерны&indexed codattypes от Thibadeau&Pientka
% И одна про unnesting
\end{frame}

\begin{frame}[fragile]{Объявление бесконечных значений}

\begin{lstlisting}[language=ocaml,mathescape=true]
let rec fib () = 0 :: 1 :: map2 (+) (fib ()) (tl (fib ()))
\end{lstlisting}
В call-be-value языках присутствуют проблемы.
\pause
\begin{lstlisting}[language=ocaml,mathescape=true]
type 'a lazy_list = C of 'a * 'a lazy_list Lazy.t

let rec fib = C (0, lazy (C (1, lazy (map2' (+) fib (tl' fib)))))

\end{lstlisting}
\end{frame}

\begin{frame}[fragile]{Идея}
\begin{itemize}
 \item Copattern matching вместо pattern matching
 \item Строим бесконечное вычисление из результатов ``наблюдений'' (observations)
 \item Copattern matching предлагает потенциальные варианты того, что можно
      построить
\end{itemize}

% Будем использовать copattern matching, который уже часто используется в 
% proof assistants.
% Паттерн матчинг обладает контролем какое значение построить. Copattern matching 
% предлагает потенциальные варианты, и позволяет контексту выбирать между ними.
% А observation -- что-то типа \lstinline{Lazy.force}
\end{frame}


\begin{frame}[fragile]{Copattern-matching. Пример. Фибоначчи}
Объявление корекурсивного типа
\begin{lstlisting}
type 'a stream = {
  Head: 'a        !$\leftarrow$! 'a stream
  Tail: 'a stream !$\leftarrow$! 'a stream
}
\end{lstlisting}

Копаттерн-матчинг. Пример использования в расширенном синтаксисе OCaml
\begin{lstlisting}
 let corec fib : int stream with
| .. #Head -> 0
| .. #Tail : int stream with
| .. #Tail#Head -> 1
| .. #Tail#Tail -> map2!$_{stream}$! (+) fib (fib#Tail)
\end{lstlisting}
% Значение fib в 1й строчке задаетпоток чисел, который строится по трём правилам
% Если наблюдается голова, то возвращаем 0
% второе правило про наблюдения, осуществляемые по цепочке
% Если наблюдается голова хвоста, то возвращаем 1
% Третий случай говорит. что надо сделать поточечное сложение двух потоков чисел.

%При обычном паттерн-матчинге все ветки возвращают одинаковый тип, но тут это не так.
%Тип веточки зависит от ипа наблюдения: если смотрим в голову, то надо возвращать значение
%типа int, а если хвост, то int stream. Опишем явно котип данных для стрима
\end{frame}

\begin{frame}[fragile]{Что хочется получить}
\begin{itemize}
 \item Копаттерны для OCaml
 \item Используя дуальность, реинтерпретировать это в терминах pattern matching
 \item Типобезопасно
 \item Эффективно
\end{itemize}
\end{frame}

\begin{frame}[fragile]{Что получилось}
\begin{itemize}
 \item \href{https://github.com/yurug/ocaml4.04.0-copatterns}{OCaml-4.04+copatterns \faGithub}.
 \item Типобезопасно, но с GADT и 2nd-order polymorphic types
 \item Эффективно (\lstinline=lazy cofunction=)
\end{itemize}
\end{frame}



\section{GADT}

\begin{frame}[fragile]{Синтаксис GADT}
\begin{lstlisting}[language=ocaml,mathescape=true]
type ($\alpha$,$\beta$, ... ) typ =
  | $C_1$ of $\tau_{11}$ * ... * $\tau_{1m_1}$
  | ...
  | $C_n$ of $\tau_{n1}$ * ... * $\tau_{nm_n}$

!\pause!

type ($\alpha$,$\beta$, ... ) typ =
  | $C_1$ !\colorbox{yellow}{:}! $\tau_{11}$ * ... * $\tau_{1m_1}$ !\colorbox{yellow}{$\rightarrow$ ($t_{11}$, $t_{12}$, ...) typ}!
  | ...
  | $C_n$ !\colorbox{yellow}{:}! $\tau_{n1}$ * ... * $\tau_{nm_n}$ !\colorbox{yellow}{$\rightarrow$ ($t_{n1}$, $t_{n2}$, ...) typ}!
\end{lstlisting}
% Они как бы guarded и generalized и equality-qualified
\end{frame}

\begin{frame}[fragile]{GADT}
Они же
\begin{itemize}
\item Generalized Algebraic Data Type
\item First-class phantom type (James\&Hinze, 2003)
\item Guarded recursive datatype (Xi\&Chen, 2003)
\item Equality-qualified type (Sheard\&Pasalic, \textit{tile here}, 2004)
\end{itemize}
\end{frame}

\begin{frame}[fragile]{Каноничный пример}
\begin{lstlisting}[language=ocaml,mathescape=true]
type 'a expr =
| Int      : int                 -> int expr
| Bool     : bool                -> bool expr
| EqualInt : int expr * int expr -> bool expr

EqualInt (Int 5, Int 6)

- : bool expr = EqualInt (Int 5, Int 6)

!\pause!

EqualInt (!\colorbox{red}{Bool true}, Int 6)
\end{lstlisting}

\begin{verbatim}
Error: This expression has type bool expr
       but an expression was expected of type int expr
       Type bool is not compatible with type int
\end{verbatim}
\end{frame}

\begin{frame}[fragile]{Равенство типов 1/2 (слайд, которого нет)}
\begin{lstlisting}
module Num = struct
  type t
end

(* type num  *)

type _ dataFormat =
  | Date  : Num.t  dataFormat
  | Number: Num.t  dataFormat
  | String: string dataFormat
  | Bool  : bool   dataFormat

let test: Num.t dataFormat -> unit = function
| Number -> ()
| Date   -> ()
| _      -> ()  (* required *) 
\end{lstlisting}
\end{frame}

\begin{frame}[fragile]{Равенство типов 2/2 (слайд, которого нет)}
\begin{lstlisting}
type (_,_) eq = Eq: ('a,'a) eq
module Num: sig
  type t
  val eq: (t,string) eq
end = struct .. end

type _ dateFormat =
  | Date  : Num.t  dateFormat
  | Number: Num.t  dateFormat
  | String: string dateFormat

let test: Num.t dateFormat -> unit = function
| Number -> ()
| Date   -> ()
| _      -> ()  (* required *) 

let () = match Num.eq with Eq -> test String

\end{lstlisting}
\end{frame}

\section{GADT и ленивые списки}

\begin{frame}[fragile]{GADT и наблюдение за ленивыми списками}
\begin{lstlisting}[language=ocaml,mathescape=true]
type ('o,'a) stream_query =
| Head : ('a,       'a) stream_query
| Tail : ('a stream,'a) stream_query
\end{lstlisting}
% Второй параметр -- что генерирует стрим
% первый -- что получает при наблюдении
\end{frame}

\begin{frame}[fragile]{Результат преобразования объявления типа коданных}
\begin{lstlisting}[language=ocaml]
and 'a stream = Stream of {
  dispatch : 'o . ('o, 'a) stream query -> 'o
}
\end{lstlisting}
% As our inversion of control turns a codata into a
% function defined by pattern matching over observation requests

\vspace{1cm}

А в System F тип конструктора \lstinline=Stream= записывается так:
\begin{lstlisting}[language=ocaml,mathescape=true]
$\forall$'a . ($\forall$'o . ('o,'a) stream_query ->   'o) -> 'a stream
\end{lstlisting}
\vspace{1cm}
\href{https://caml.inria.fr/pub/papers/garrigue_remy-poly-ic99.ps.gz}{Jacques Garrigue and Didier Rémy. 1999. }
% Внутренняя квантификация по типу 'o немного напоминает, answer types,
% которые получаются при преобразовании в CPS. Однако там ничего не известно
% про тип выходного значения, а тут всё норм благодаря GADT: всем возможные 
% инстанциации закодированы в конструкторах
\end{frame}

\begin{frame}[fragile]{Фибоначчи после трансформации}
\begin{lstlisting}[language=ocaml,numbers=left,stepnumber=1]
let rec fib : int stream =
  let dispatch : type o . ('o, int) stream query -> 'o = function
  | Head -> 0
  | Tail -> !\pause!
    let dispatch : type o .('o, int) stream query -> 'o = function
    | Head -> 1
    | Tail -> map2 (+) !\graybox{fib}! (!\graybox{tail fib}!)
    in Stream { dispatch }
  in Stream { dispatch }
\end{lstlisting}
\begin{lstlisting}[language=ocaml]
let tail (Stream { dispatch }) = dispatch Tail
\end{lstlisting}
% Типы тут хорошие не смотря на то, что 3я строчка возвращает число, а 4я
% стрим. Просто answer type `o` в разных ветках имеет разные типы из-за GADT stream_query,
% в 1й бранче 1 имеет тип о, а во втрой `Stream{dispatch}` тоже имеет тип о.

% Замечание, преобразование не очень эффективное, так как fib и tail fib не имеют общей части.

\end{frame}

\begin{frame}[fragile]{Исходные фибоначчи с копаттернами}
\begin{lstlisting}[language=ocaml]
let corec fib : int stream with
| ..#Head -> 0
| ..#Tail : int stream with
| ..#Tail#Head -> 1
| ..#Tail#Tail -> map2 (+) fib (fib#Tail)
\end{lstlisting}

При преобразовании произошел unnesting
% Anton Setzer, Brigitte Pientka, and David Thibodeau у них тоже был алгоритм, но другой
\end{frame}


\section{Формализация}

\begin{frame}[c]{Преобразование}
\centering
{\LARGE $\lambda^C \to \lambda^G$}
\end{frame}

\begin{frame}[c]{Термы}
\begin{tabular}{ l l r }
  $t,u $&$::=$ & \\
  &|  $x$ & Переменная \\
  &|  $D$ & \graybox{Observation request}\\
  &|  $K\; t$ & \graybox{Конструктор с 1 аргументом} \\
  &|  $t\; t$ & Применение  \\
  &|  $t \cdot t$  & \graybox{Observation}  \\
  &|  $\mu^+ f: \sigma := \lambda \bar{x}\{\bar{b}\}$ & Функция  \\
  &|  $\mu^- f: \sigma := \lambda \bar{x}\{\bar{b}\}$ & Коданные \\
\end{tabular}


% Здесь про выделенные надо что-то сказать
% Observation request раньше (у Thuibodeau&Pientka) писался как t\cdot D, 
% но у нас справа произвольные терм, который может быть простым Request или сложным.
% Таким образом у нас first-class observations

% Аргумент конструктора тут всегда один, но в ML бывает 0 или много. 
% Это всё не умаляет общность,
% так как пары и unit можно записать как котипы данных (пример 3.1 в статье). 
% Про это будет позже

% мюшки для рекурсии. Плюс и минус отвечает за то, как конструируются данные
% рекурсивно или корекурсивно. фигурные скобочки отделяют тело от аргументов
\end{frame}

\begin{frame}[c]{Ветки мэтчинга и значения}
\begin{tabular}{ l l r }
  $branch$ & $::=$ & \\
  &|  $\bullet \Rightarrow t$ & Suspension \\
  &|  $K\;x$ & Деконструирование \\
  &|  $\cdot D \Rightarrow t$  & Observation  \\
\end{tabular}
% Рекурсивные и корекурсивные функции используют один и другой вид бранчей соответственно
% лямбда-абстракция это ни то, ни другое, у неё первый случай, а жирная точка, это отсутсвие 
% копаттерна

\vspace{1cm}

\begin{tabular}{ l l r }
  $value$ & $::=$ & \\
  &|  $\lambda^{-} \bar{x}\{\bar{b}\} \Rightarrow t$ & Коданные\\
  &|  $\lambda^{+} \bar{x}\{\bar{b}\} \Rightarrow t$ & Функция \\
  &|  $K v$ & Данные\\
  &|  $D$   & Request\\
\end{tabular}
% вектор аргументов может быть пустой, это позволяет описывать рекурсивные данные и коданные.
% Пьетка (ICFP,2013) вводили понятие Generalized lambda-abstraction, то различать эти вещи проще.
% Синтаксис бранчей выглядит чересчур атомарным, чтобы выражить вложенные копаттерны,
% но тут поможет unnesting. Но об этом позже
\end{frame}

\begin{frame}[c]{Семантика малого шага для $\lambda^C$}
\begin{tabular}{ r c l }
  $E[t]$ & $\xrightarrow{\text{SCxt}}$ & $E[t'] \quad если \quad t\to t'$ \\
  
  $\mu^{\diamond} f:\sigma := \lambda \bar{x}\{\bar{b}\}$ & $\xrightarrow{\text{SUnr}}$ & 
     $\lambda^\diamond \bar{x}\{(\bar{b}[\;f \mapsto \mu^\diamond f:\sigma := \lambda \bar{x}\{\bar{b}\}])\}$ \\
     
  $(\lambda^\diamond x \bar{x}\{\bar{b}\})v$ & $\xrightarrow{\text{SPush}}$ & $\lambda^\diamond \bar{x}\{\bar{b}[x\mapsto v]\}$ если SEval не применим \\
  
  $(\lambda^\diamond x \{\bullet \Rightarrow t | \bar{b}\})v$ & $\xrightarrow{\text{SEval}}$ &  $t[x \mapsto v]$ \\
    
  $(\lambda^+ \bullet \{K\;x \Rightarrow t \mid \bar{b}\})(K\;\;v)$ & $\xrightarrow{\text{SDes}}$ &  $t[x \mapsto v]$ \\
  
  $(\lambda^+ \bullet \{K\;x \Rightarrow t \mid \bar{b}\})(K'\;v)$ & $\xrightarrow{\text{SDesF}}$ & 
    $(\lambda^+ \bullet \{\bar{b}\})(K' v)$, если $K \neq K'$ \\
  
  $(\lambda^- \bullet \{\cdot D \Rightarrow t \mid \bar{b}\})\;D\;$ & $\xrightarrow{\text{SObs}}$ &  
    $t$ \\
  
  $(\lambda^- \bullet \{\cdot D \Rightarrow t \mid \bar{b}\})\;D'$ & $\xrightarrow{\text{SObsF}}$ &
    $(\lambda^- \bullet \{\bar{b}\})D'$, если $D \neq D'$
\end{tabular}

\vspace{1cm}

где $\diamond \in \{+,-\}$
% SCxt вставляет терм в контексте
% SUnr (unroll) вставляет тело (ко)рекусривной функции в место вызова, делая 1 шаг
% Spush делает бета-редукцию пока больше 1го аргумента у лямбды

% SEval если только один формальный аргумент и первая бранча это suspended computation, то можно заменить вхождения икса

% Если лямбда-абстракция это рекурсивная функция с буллетом (нет формальных входных аргументов),
% то нам передали сконструированное значение и его надо мэтчить
% всё просто для SDes и SDesF

% C правилами для копаттернов всё примерно также, только  observation copattern не связывает никаких переменных,так что 
% нет никаких подстановок

%%%%%%%%%%%%%%%%%%%%%%%%%%%
% Тут в D нет вложенных копаттернов, они не нужны, выразительность не теряется, потому что
% разрешенно возвращать функцию, а потом с ней оперировать
% У Observations были аргументы в предыдущей версии, там был короче язык, но сложнее трансформация, 
% и её было сложнее перенести в OCaml. Потому оставили так.

\end{frame}

\begin{frame}[c]{Типы}
\begin{tabular}{l l r}
  $\tau$ & $::=$ & \\
  &| $\alpha$ & Типовая переменная \\
  &| $\varepsilon^+(\bar{\tau})$  & Данные \\
  &| $\varepsilon^-(\bar{\tau})$  & Коданные \\
  &| $\tau \to \tau$ & Стрелка \\
  &| $\tau \leftarrow \varepsilon^-(\bar{\tau})$ & Observation request \\
\end{tabular}

% С плюсиком GADT (они у нас везде), с минусиком котипы данных, стрелки и кострелки
% Потом должно быть понятно. почему кострелки справа налево
\vspace{1cm}
\centering
$\epsilon^+(\overline{\alpha})\;:=\; \Sigma_i K_i \;:\forall\overline{\alpha}.\tau_i \;\to       \;\epsilon^+(\bar{\tau}_i)$

$\epsilon^-(\overline{\alpha})\;\;:=\; \times_i D_i   :\forall\overline{\alpha}.\tau_i \;\leftarrow\;\epsilon^-(\bar{\tau}_i)$
\end{frame}

\begin{frame}[c]{Одного аргумента у $K$ -- достаточно}
% Всё это выглядит не очень выразительно по сравнению с предыдущими работами по котипам
% потому что там были констрейнты на равенство типов, а тут нет. Но всё хорошо
% 1) эпсилоны применяются не к типовым переменным, а к ground типам
% 2) И ещё можно с помощью GADT выразить равенства

% TODO: надо таки сделать как по Лейбницу.
\centering
$\epsilon^+(\overline{\alpha})\;:=\; \Sigma_i K_i \;:\forall\overline{\alpha}.\tau_i \;\to       \;\epsilon^+(\bar{\tau}_i)$

$\epsilon^-(\overline{\alpha})\;\;:=\; \times_i D_i   :\forall\overline{\alpha}.\tau_i \;\leftarrow\;\epsilon^-(\bar{\tau}_i)$
\vspace{1cm}

$unit = DUnit : unit \leftarrow unit$
% Тут может быть много определений юнита, как типа с одним жителем, но все определения будут 
% изоморфны
\vspace{.5cm}

$pair(\alpha,\beta) = \begin{cases}
    DFst : \forall\alpha \beta . \alpha \leftarrow pair(\alpha,\beta) \\
    DSnd : \forall\alpha \beta . \beta  \leftarrow pair(\alpha,\beta)
\end{cases}$
\end{frame}

\begin{frame}[c]{Окружения с именами типов и констрейнты}
% Тут ничего интересного
\begin{tabular}{l l r}
  $\Gamma$ & $::=$ & \\
  &| $\bullet$ & пусто \\
  &| $\Gamma \alpha$  & Связанная переменная типа \\
  &| $\Gamma (x: \alpha)$ & Связанная переменная \\
\end{tabular}

\vspace{1cm}

\begin{tabular}{l l r}
  $C$ & $::=$ & \\
  &| true  & Тривиальный \\
  &| false & Пустой \\
  &| $\tau=\tau$ & Атомарное равенство\\
  &| $C \bigwedge C $ & Конъюнкция\\
\end{tabular}
% Констрейнты нужны чтобы поддержать GADT (и индексированные котипы данных)

\end{frame}

\begin{frame}[fragile]{Правила типизации}
Тут показывать скриншот :)

% Тут и гамма, и констрейнты, рассказать как читать
% Г,С |- t:\tau означает что в окружении и констрейнтах t имеет мономорфный тип тау
% Г,С |- \x.b : \diamond\tau означает что в окружении такомто и при таких констрейнтах
% абстракция обозначает ромбик-функцию и имеет тип тау

% С||-\tau=\rho Означает, что из констрейнтов следует, что тип \tau эквив. \rho
% Sconv это дополнительное правило про равенство типов

% Тут нет никакой полиморфности, потому что полиморфные типовые переменные сразу же
% инстанциируются в нужные


% Svar и Sapply стандартные

% Srequest просто приписывает тип реквесту, ХЗ, непонятно что тут вообще, но вроде что-то очень простое
% SConstruct банален

% SObserve & SApply дуальны: просто проверяет что левая часть подходит под правую
% стрелочка справа налево подразумевает что reqest logically driving вычисления

% SLam это банальная лямбда абстракция
% SFun проверяет что (ко)рекурсивные функции хорошие. Стрёмно, но вторая строчка проверяет что тут идет 
% типовая аннотация функции хорошая. 
% Аннотация \forall \bar{\alpha} приписывается человеком к фнукции f . Проверка того, что 
% фнукции подходит тип, приписанный человеком, происходит во второй строчке

% Spat оно типа стандартное для GADTов. Задачей стоит определить тип икса, стоящего под 
% конструктором К. Входной тип стрелки должен использовать тот же самый конструктор типа
% как и та штука, которую мы мэтчим. Аргументы могут отличаться, и это отличие протаскивается
% с самого конструктора в правую часть бранча. Потому локально (в каждой отдельной бранче)
% подразумевается, что \bar{\tau}=\bar{\tau}' (3я строчка). Эти равенства локальные,
% и потому не могут сбежать наружу, чт обозначается гипотезой \bar{\beta}#Г

% C копаттернами аналогично.
% Блин, я это понял.
\end{frame}

% Я тут теоремы пропустил

\begin{frame}[fragile]{Теоремы $\lambda^C$}
Теорема (про редукцию):

$C$ выполнимо. Если $\Gamma,C \vdash t:\tau$ и $t\to t'$, тогда
$\Gamma, C \vdash t':\tau$ .

\vspace{1cm}

Теорема (Про прогресс).

Констрейнты $C$ выполнимы. Если $\Gamma,C \vdash t:\tau$, то либо $t$ это 
конечное значение или существует $t'$ такое что $t \to t'$.
\end{frame}

\begin{frame}[fragile]{Синтаксис и семантика $\lambda^G$}
\begin{itemize}
 \item Нет observations ни в синтаксисе, ни в правилах для семантики (просто выкинули)
 \item Конструкторы могут иметь 0 или много аргументов
\end{itemize}
\end{frame}


\begin{frame}[fragile]{Разница между $\lambda^C$ и $\lambda^G$}
% Типы в статье на 9й странице.
$\lambda^C:\qquad\epsilon^+(\overline{\alpha})\;:=\; \Sigma_i K_i \;:\forall\overline{\alpha}.\tau_i \;\to\;\epsilon^+(\bar{\tau}_i)$

$\lambda^G:\qquad\epsilon^+(\overline{\alpha})\;:=\; \Sigma_i K_i \;:\forall\overline{\alpha}.\overline{\sigma} \;\to\;\epsilon^+(\bar{\tau}_i)$

\vspace{1cm}
В $\lambda^G$ аргументы конструктора могут иметь полиморфный тип, а к самому конструктору
приписывается схема типов второго порядка.

% В $\lambda^C$, система типов основани на двух правилах: одно для термов,
% второе для абстракций. Главная разница в том, что между типами $\lambda^C$ и
% типами в $\lambda^G$, что теперь правила присваивают схему типов
% (не мономорфный тип) к терму.
% 
% Однако в ML, многие правил принимают только мономорфные типы
% чтобы поддерживать разрешимость вывода типов. Тут повторяются все трюки,
% что использует компилятор OCaml (см. статью Реми и Гаррика).
% Правила (TConv), (TApply), (TRecfun) и (TLam) сделаны по аналогии с $\lambda^C$.
% Правило (SVar) было разделено на (TVar) и (TInst).
% Это разделение позвляет полиморфно типизированным идентификаторам
% после извлечения из env встать на место, где ожидается полиморфный терм. 
% (This contrasts
% with usual syntax-directed typing rules of ML where polymorphic
% identifiers are immediately instantiated when extracted from the
% environment.) Смотря на правило (TConstructor), видно, что аргументы конструктора
% являются как раз таким местом: им присваиваются схемы типов, описанные при объявлении типа 
% данных. Это немного более строгий констрейнт на общность аргументов конструктора, 
% переменные связанные при паттерн матчинге могут быть полиморфными, что выражается
% новой гипотезой в правиле TPat, которая теперь вводит в окружение (\bar{x}:\bar{\sigma})
% чтобы протайпчекать тело ветки. Если закрыть на это глаза, то правила одинаковые с lambda^C
\end{frame}

\section{Трансляция}

\begin{frame}[fragile]{Схема трансляции}
% Тут три части
% 
Типы
\begin{tabular}{r c l}
  $\sem{\alpha}$ & = & $\alpha$ \\
  $\sem{\tau \to \rho}$ & = & $\sem{\tau}\to\sem{\rho}$ \\
  $\sem{\epsilon^+(\bar{\tau})}$ & = & $\epsilon^+(\sem{\bar{\tau}})$ \\
  $\sem{\tau \leftarrow \epsilon^-(\bar{\tau})}$ & = & $\epsilon_r^-(\sem{\tau}\sem{\bar{\tau}})$ \\
  $\sem{\epsilon^-(\bar{\tau})}$ & = & $\epsilon^-_d (\sem{\bar{\tau}})$ \\
\end{tabular}

\vspace{1cm}

Термы
\begin{tabular}{r c l}
$\sem{x}$        & = & $x$ \\
$\sem{tu}$       & = & $\sem{t}\sem{u}$ \\
$\sem{t\cdot u}$ & = & \graybox{$\sem{u}\sem{t}$} \\
$\sem{Kt}$       & = & $K\sem{t}$ \\
$\sem{D}$        & = & \graybox{$d$} \\
$\sem{\mu^- f:\forall \bar{\alpha}.\tau := \lambda \bar{x}\{\bar{b}\}}$ &=& 
  $\mu^+ f:\forall \bar{\alpha}.\sem{\tau} := \lambda \bar{x}.K^{\epsilon^-(\bar{\tau})}\sem{\bar{b}}^\bot_{\bar{\alpha},O(\tau)}$ \\
$\sem{\mu^+ f:\forall \bar{\alpha}.\tau := \lambda \bar{x}\{\bar{b}\}}$ &=& 
  $\mu^+ f:\forall \bar{\alpha}.\sem{\tau} := \lambda \bar{x}.\sem{\bar{b}}$ \\
$\sem{K x \Rightarrow t}$       & = & $K x \Rightarrow\sem{t}$ \\

$\sem{\vee \cdot D_i \Rightarrow t}^\bot_{\bar{\alpha},\epsilon^-{\bar{\tau}}}$       & = & 
  $\mu^+\_:\;\;\forall \beta.\epsilon^-_r(\beta,\bar{\tau})\to \beta \;\;:= \;\;\lambda \bullet .$ \graybox{$\vee_i D_i \Rightarrow \sem{t}$} \\

\end{tabular}
% d маленькое это та короткая реализация tail, которая на инверсии управления.
% Интересных мест тут мало
% 1) применение копаттерна, которое использует инверсию управления
% 2) Сам копаттерн заменяется, на функцию с маленькой буквы
% 3) Корекурсивная функция в последней строчке превращается в паттерн мэтчинг по 
%    reified observation
% По остальным термам идет в лоб, сохраняя структуру
\end{frame}

\begin{frame}[fragile]{Преобразование объявлений типов}
Индуктивные

$\sem{\epsilon^+(\bar{\alpha}):= \Sigma_i K_i :\forall\bar{\alpha}.\tau_i\to\epsilon^+(\bar{\tau}_i)}$ = 
  $\epsilon^+(\bar{\alpha}):= \Sigma_i K_i :\forall\bar{\alpha}.\sem{\tau_i}\to\epsilon^+(\sem{\bar{\tau}}_i) $\\
\vspace{1cm}
Коиндуктивные

$\sem{\epsilon^+(\bar{\alpha}):= \times_i D_i :\forall\bar{\alpha}.\tau_i \leftarrow\epsilon^-(\bar{\tau}_i)} $ =

\qquad
  $\begin{cases}
   \;\;\;\;\;\;\epsilon_r^-(\bar{\alpha}):=\Sigma_i D_i: \forall\bar{\alpha}.\epsilon_r^-(\sem{\tau_i},\sem{\bar{\tau}_i})\\
   \;\;\;\;\;\;\epsilon_d^-(\bar{\alpha}):=K^{\epsilon^-(\bar{\alpha})}:\forall\bar{\alpha}.(\forall\beta.\epsilon^-_r(\beta,\bar{\alpha})\to\beta)\to \epsilon_d^-(\bar{\alpha})\\
   \forall i,\;\;\;\;\;\;\;\;d_i:=\mu^+\_:\forall\bar{\alpha}.\epsilon^-_d(\sem{\bar{\tau}_i})\to \sem{\tau_i}:=\lambda\bullet.K^{\epsilon^-(\bar{\alpha})}c\Rightarrow c\, D_i\\
  \end{cases}$
\\
% Может тут надо привести пример про список чтобы показать, что есть что?
\end{frame}

\section{Реализация}

%\section{Unnesting}
\begin{frame}[fragile]{Unnesting паттернов и копаттернов. Пример.}
% делаем не как у Сецера и комании
\begin{lstlisting}[language=ocaml,mathescape=true]
type _ repr =                   type _ qrepr = {
| Int  : int  -> int repr          QInt  : int  !$\leftarrow$! int qrepr;
| Bool : bool -> bool repr         QBool : bool !$\leftarrow$! bool qrepr }
\end{lstlisting}

\begin{lstlisting}[language=ocaml,mathescape=true]
let corec f : type a . a repr -> a qrepr with
| (..(Int n))#QInt   -> n
| (..(Bool b))#QBool -> b
\end{lstlisting}
\centering{$\Downarrow$}
\begin{lstlisting}[language=ocaml,mathescape=true]
let f : type a . a repr -> a qrepr = fun x ->
  let dispatch : type o . (o, a) qrepr_query -> o = function
  | QInt  -> (match x with Int n  -> n)
  | QBool -> (match x with Bool b -> b)
  in 
  Qrepr { dispatch }
\end{lstlisting}
%\vspace{1cm}
% У Сецнера было преобразование которое не зависило от порядка, что является 
% стрёмным если пытаться запихнуть его в ML. Ткнцть пальцев во ``fresh splitting variables''
% x
\end{frame}


\begin{frame}[fragile]{``Лишние'' паттерны. Два примера}
\begin{lstlisting}[language=ocaml,mathescape=true]
let corec zeros : int stream with
| ..#Head -> 0
| ..#Head -> 1       (* <= to remove *)
| ..#Tail -> zeros
\end{lstlisting}
\pause
\begin{lstlisting}[language=ocaml,mathescape=true]
let corec f : int -> int stream with
| (.. n)#Head -> 0
| (.. n)#Tail -> f (n - 1)   (* <= to remove *)
| (.. n)#Tail : int stream with
| (.. n)#Tail#Head -> n
| (.. n)#Tail#Tail -> f (n + 1)
\end{lstlisting}
Более глубокие считаются более сильными
% Потому что если человек написал (f n)#Tail#Head то уж он наверняка желает 
% потом написать и случай для Tail, потому считать (f n)#Tail#Tail лишним
% не имеет смысла
\end{frame}

\begin{frame}[fragile]{Ленивые вычисления}
\begin{lstlisting}[language=ocaml]
let rec fib : int stream =
  let dispatch : type o. (o, int) stream query -> o = function
  | Head -> 0
  | Tail ->
    let dispatch : type o.(o, int) stream query -> o = function
    | Head -> 1
    | Tail -> map2 (+) !\graybox{fib}! (!\graybox{tail fib}!)
    in Stream { dispatch }
  in Stream { dispatch }
\end{lstlisting}

Можно заменить каждый \lstinline=dispatch= на dispatch с мемоизацией (на слайдах нет)
\end{frame}

\begin{frame}[fragile]{First-order и high-order observations}
\begin{itemize}
 \item По описанию копаттерна сгенерировались функции...
 \item ...их можно передавать в другие функции и получать high-order
 \item Это главный contribution.
\end{itemize}
% Их нельзя хешировать, но можно хешировать конструкторы копаттерна
% См. пример про Жизнь
\end{frame}

\begin{frame}[fragile]{Пример про жизнь и комонаду}
$ $
\end{frame}

\end{document}
