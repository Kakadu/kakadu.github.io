\newif\ifanswers
%\answerstrue % comment out to hide answers
% hack from https://tex.stackexchange.com/questions/33576/conditional-typesetting-build

\documentclass[aspectratio=169
  , xcolor={svgnames}
  , hyperref={ colorlinks,citecolor=DeepPink4
             , linkcolor=DarkRed,urlcolor=DarkBlue}
  , russian
  ]{beamer}
\usetheme{CambridgeUS}
\beamertemplatenavigationsymbolsempty % remove navigation bar

\usefonttheme{professionalfonts}
\makeatletter
\@ifclassloaded{beamer}{
  % get rid of header navigation bar
  \setbeamertemplate{headline}{}
  % get rid of bottom navigation symbols
  \setbeamertemplate{navigation symbols}{}
  % get rid of footer
  %\setbeamertemplate{footline}{}
}
{}
\makeatother
%%%%%%%%%%%%%%%%%%%%%%%%%%%%%%%%%%%%%%%%%%%%%
\usepackage{fontawesome}
% \newfontfamily{\FA}{Font Awesome 5 Free} % some glyphs missing
\expandafter\def\csname faicon@facebook\endcsname{{\FA\symbol{"F09A}}}
\def\faQuestionSign{{\FA\symbol{"F059}}}
\def\faQuestion{{\FA\symbol{"F128}}}
\def\faExclamation{{\FA\symbol{"F12A}}}
\def\faUploadAlt{{\FA\symbol{"F093}}}
\def\faLemon{{\FA\symbol{"F094}}}
\def\faPhone{{\FA\symbol{"F095}}}
\def\faCheckEmpty{{\FA\symbol{"F096}}}
\def\faBookmarkEmpty{{\FA\symbol{"F097}}}

\newcommand{\faGood}{\textcolor{ForestGreen}{\faThumbsUp}}
\newcommand{\faBad}{\textcolor{red}{\faThumbsODown}}
\newcommand{\faWrong}{\textcolor{red}{\faTimes}}
\newcommand{\faMaybe}{\textcolor{blue}{\faQuestion}}
\newcommand{\faCheckGreen}{\textcolor{ForestGreen}{\faCheck}}
%%%%%%%%%%%%%%%%%%%%%%%%%%%%%%%%%%%%%%%%%%%%%

\usepackage{fontspec}
\usepackage{xunicode}
\usepackage{xltxtra}
\usepackage{xecyr}
\usepackage{hyperref}

\setmainfont[
 Ligatures=TeX,
 Extension=.otf,
 BoldFont=cmunbx,
 ItalicFont=cmunti,
 BoldItalicFont=cmunbi,
% Scale = 1.1
]{cmunrm}
\setsansfont[
 Ligatures=TeX,
 Extension=.otf,
 BoldFont=cmunsx,
 ItalicFont=cmunsi,
%  Scale = 1.2
]{cmunss}
%\setmainfont[Mapping=tex-text]{DejaVu Serif}
%\setsansfont[Mapping=tex-text]{DejaVu Sans}
%\setmonofont{Fira Code}[Contextuals=AlternateOff]
\setmonofont{Fira Code}[Contextuals=Alternate,Scale=0.9]
\newfontfamily{\myfiracode}[Scale=1.5,Contextuals=Alternate]{Fira Code}
%\setmonofont[Scale=0.9,BoldFont={Inconsolata Bold}]{Inconsolata}

\usepackage{polyglossia}
\setmainlanguage{russian}
\setotherlanguage{english}


%\newfontfamily\dejaVuSansMono{DejaVu Sans Mono}
% https://github.com/vjpr/monaco-bold/raw/master/MonacoB/MonacoB.otf
%\newfontfamily\monacoB{MonacoB}
%%%%%%%%%%%%%%%%%%%%%%%%%%%%%%%%%%%%%%%%%%%%%%%5
\usepackage{soul} % for \st that strikes through
\usepackage[normalem]{ulem} % \sout

\usepackage{stmaryrd}
\newcommand{\sem}[1]{\ensuremath{\llbracket #1\rrbracket}}


\usepackage{listings}
%\lstdefinestyle{style1}{
%  language=haskell,
%  numbers=left,
%  stepnumber=1,
%  numbersep=10pt,
%  tabsize=4,
%  showspaces=false,
%  showstringspaces=false
%}
%\lstdefinestyle{hsstyle1}
%{ language=haskell
%%          , basicstyle=\monacoB
%         , deletekeywords={Int,Float,String,List,Void}
%         , breaklines=true
%         , columns=fullflexible
%         , commentstyle=\color{ForestGreen}
%         , escapeinside=§§
%         , escapebegin=\begin{russian}\commentfont
%         , escapeend=\end{russian}
%         , commentstyle=\color{ForestGreen}
%         , escapeinside=§§
%         , escapebegin=\begin{russian}\color{ForestGreen}
%         , escapeend=\end{russian}
%         , mathescape=true
%%          , backgroundcolor = \color{MyBackground}
%}
%
%\newcommand{\inline}[1]{\lstinline{haskell}{#1}}
%\def\hsinline{\mintinline{haskell}}
%\def\inline{\hsinline}
%
%\lstnewenvironment{hslisting} {
%    \lstset { style={hsstyle1} }
%  }
%  {}
%  
%%%%%%%%%%%%%%%%%%%%%%%%%%%%%%%%%%%%%%%%%%%%%%%%%%%%%%%%%%%  
%%\setmainfont[
%% Ligatures=TeX,
%% Extension=.otf,
%% BoldFont=cmunbx,
%% ItalicFont=cmunti,
%% BoldItalicFont=cmunbi,
%%]{cmunrm}
%%% С засечками (для заголовков)
%%\setsansfont[
%% Ligatures=TeX,
%% Extension=.otf,
%% BoldFont=cmunsx,
%% ItalicFont=cmunsi,
%%]{cmunss}
%% \setmonofont[Scale=0.6]{Monaco}
%
%\usefonttheme{professionalfonts}
%\usepackage{times}
\usepackage{tikz}
\usetikzlibrary{cd}
\usepackage{tikz-cd}
\usepackage{caption}
\usepackage{subcaption}

%\renewtheorem{definition}{برهان}[chapter]
%%\DeclareMathOperator{->}{\rightarrow}
%\newcommand\iso{\ensuremath{\cong}}
%\usepackage{verbatim}
%\usepackage{graphicx}
%\usetikzlibrary{arrows,shapes}

%\usepackage{amsmath}
%\usepackage{amsfonts}
\usepackage{scalerel}
\DeclareMathOperator*{\myvee}{\scalerel*{\vee}{\sum}}
\DeclareMathOperator*{\mywedge}{\scalerel*{\wedge}{\sum}}

%
%\usepackage{tabulary}
%
%% sudo aptget install ttf-mscorefonts-installer
%%\setmainfont{Times New Roman}
%%\setsansfont[Mapping=tex-text]{DejaVu Sans}
%
%%\setmonofont[Scale=1.0,
%%    BoldFont=lmmonolt10-bold.otf,
%%    ItalicFont=lmmono10-italic.otf,
%%    BoldItalicFont=lmmonoproplt10-boldoblique.otf
%%]{lmmono9-regular.otf}
%
\usepackage[cache=true]{minted}
\usemintedstyle{perldoc}

\def\hsinline{\mintinline{haskell}}
\def\mlinline{\mintinline{ocaml}}
% color options
\definecolor{YellowGreen} {HTML}{B5C28C}
\definecolor{ForestGreen} {HTML}{009B55}
\definecolor{MyBackground}{HTML}{F0EDAA}



\institute{матмех СПбГУ}

\addtobeamertemplate{title page}{}{
  \begin{center}{\tiny Дата сборки: \today}\end{center}
}

\usepackage{tabulary}
\usepackage{verbatim}
% \usepackage{tabularx}  % for 'tabularx' environment
% \usepackage{ragged2e} % for \Centering macro
% \newcolumntype{C}{>{\Centering\arraybackslash}X}m
% sudo aptget install ttf-mscorefonts-installer
\defaultfontfeatures{Ligatures={TeX}} 
\setmainfont{Times New Roman}
\setsansfont{CMU Sans Serif}

\setmonofont[Scale=1.0,
    BoldFont=lmmonolt10-bold.otf,
    ItalicFont=lmmono10-italic.otf,
    BoldItalicFont=lmmonoproplt10-boldoblique.otf
]{lmmono9-regular.otf}

\newcommand{\term}[2]{\textit{#1} (#2)}

\usepackage[cache=true]{minted}
\usepackage{amsthm}

\newtheorem{remark}{\textbf{Замечание}}[section]
\newtheorem{hint}{\textbf{Указание разработчикам}}[section]

\newtheoremstyle{exerciseStyle1}
{}                % Space above
{}                % Space below
{}                % Theorem body font % (default is "\upshape")
{}                % Indent amount
{\bfseries}       % Theorem head font % (default is \mdseries)
{.}               % Punctuation after theorem head % default: no punctuation
{ }               % Space after theorem head
{}                % Theorem head spec
\theoremstyle{exerciseStyle1}
\newtheorem{exercise}{\textbf{Упражнение}}[section]


\deftranslation[to=russian]{Theorem}{Теорема}
\deftranslation[to=russian]{theorem}{теорема}

\usepackage{tikz}
\usetikzlibrary{trees}
\usepackage{subcaption}

%%%%%%%%%%%%%%%%%%
\makeatletter
\newenvironment{tabminted}{%
  \let\FV@ListVSpace\relax  
  \minted
}{%
  \endminted
  \unskip   
  \aftergroup\@tabmintedend
}
\newcommand*{\tabminted@finalstrut}[1]{%
  \ifdim\prevdepth>0pt
    \ifdim\dp#1>\prevdepth
      \vskip\dimexpr(\dp#1)-\prevdepth\relax
    \fi
  \else
    \vskip\dimexpr(\dp#1)\relax
  \fi
}
\newcommand*{\@tabmintedend}{%
  \let\@finalstrut\tabminted@finalstrut
}
\renewcommand{\cite}[1]{}
\makeatother


%%%%%%%%%%%%%%%%%%%%%5
\title[]{Чисто функциональные структуры данных}
\subtitle{С примерами кода на Haskell}
\author{Косарев Дмитрий }

\institute{матмех СПбГУ}

\date{\today}
 
\AtBeginSection[]
{
  \begin{frame}<beamer>[allowframebreaks]
    \frametitle{Оглавление}
    \tableofcontents[currentsection]
  \end{frame}
}

\newcommand{\verbatimfont}[1]{\def\verbatim@font{#1}}
\setcounter{tocdepth}{1}  % part,chapters,sections 
\newcommand\chap[1]{
%  \chapter*{#1}
  \addcontentsline{toc}{chapter}{#1}
}

\usepackage{verbatimbox}

\begin{document}
\maketitle

% For every picture that defines or uses external nodes, you'll have to
% apply the 'remember picture' style. To avoid some typing, we'll apply
% the style to all pictures.
\tikzstyle{every picture}+=[remember picture] 

% By default all math in TikZ nodes are set in inline mode. Change this to
% displaystyle so that we don't get small fractions.
\everymath{\displaystyle}

% Uncomment these lines for an automatically generated outline.
\begin{frame}[allowframebreaks]{Оглавление}
  \tableofcontents
\end{frame}

\begin{frame}[fragile]{}
Мунк
\end{frame}

\section{Индуктивные типы данных}

%\begin{frame}{Устойчивоcть (persistence)}
Отличительной особенностью функциональных структур данных является то,
что они всегда \term{устойчивы}{persistent}~--- обновление
функциональной структуры не уничтожает старую версию, а создает
новую, которая с ней сосуществует. \\

Устойчивость достигается путем
\emph{копирования} затронутых узлов структуры данных, и все изменения
проводятся на копии, а не на оригинале. \\

Поскольку узлы никогда
напрямую не модифицируются, все незатронутые узлы могут
\term{совместно использоваться}{be shared} между старой и новой версией структуры
данных без опасения, что изменения одной версии непроизвольно окажутся
видны другой.

\end{frame}

%\chap{Некоторые известные структуры данных в функциональном окружении}

\section{Списки и их конкатенация}
\label{sc:3.1}

\begin{frame}[fragile]{Сигнатура Stack. Реализация через встроенные списки}
\begin{minipage}{.48\textwidth}
\inputminted[firstline=3,lastline=8]{haskell}{code/Stack.hs}
\end{minipage}
\begin{minipage}{.48\textwidth}
  \inputminted[firstline=21,lastline=30]{haskell}{code/Stacks.hs}
\end{minipage}
\end{frame}

\begin{frame}[fragile]{Сигнатура Stack. Реализация через новый тип данных}
\begin{minipage}{.48\textwidth}
  \inputminted[firstline=3,lastline=8]{haskell}{code/Stack.hs}
\end{minipage}
\begin{minipage}{.48\textwidth}
  \inputminted[firstline=37,lastline=48]{haskell}{code/Stacks.hs}
\end{minipage}
\end{frame}

\begin{frame}[fragile]{Конкатенация списков}
\begin{minted}{haskell}
(++) :: STACK l => l a -> l a -> l a
\end{minted}
В императивной среде легко сделать за O(1), если хранить указатель на конец.
\end{frame}

\begin{frame}[fragile]{Конкатенация в императивной среде}
\begin{figure}[h]
%	\centering
	\begin{tikzpicture}[thick,scale=0.5, every node/.style={scale=0.5}]
    \tikzstyle{marrs}=[very thick,-latex]

    \begin{scope}
    
        \foreach \x/\y in {0/0, 0/-2, 3/-2, 6/-2} {
            \draw (\x - 0.5, \y - 0.5) rectangle +(1, 1); \draw (\x + 1 - 0.5, \y - 0.5) rectangle +(1, 1);
        }
        \draw[marrs] (-1.5, 0) -> +(1, 0);
        \draw[marrs] (0, 0) -> +(0, -1.5);
        \draw[marrs] (1, -2) -> +(1.5, 0);
        \draw[marrs] (4, -2) -> +(1.5, 0);
        \draw[marrs] (1, 0) -- (5.5, 0) .. controls (5.75, 0) and (6, -0.25) .. (6, -0.5) -- (6, -1.5);
        
        { \huge
            \draw (-2, 0) node {$xs$};
            \draw (0, -2) node {$0$};
            \draw (3, -2) node {$1$};
            \draw (6, -2) node {$2$};
            \draw (7, -2) node {$\cdot$};
        }
    
    \end{scope}
    
    \begin{scope}[xshift=12cm]
    
        \foreach \x/\y in {0/0, 0/-2, 3/-2, 6/-2} {
            \draw (\x - 0.5, \y - 0.5) rectangle +(1, 1); \draw (\x + 1 - 0.5, \y - 0.5) rectangle +(1, 1);
        }
        \draw[marrs] (-1.5, 0) -> +(1, 0);
        \draw[marrs] (0, 0) -> +(0, -1.5);
        \draw[marrs] (1, -2) -> +(1.5, 0);
        \draw[marrs] (4, -2) -> +(1.5, 0);
        \draw[marrs] (1, 0) -- (5.5, 0) .. controls (5.75, 0) and (6, -0.25) .. (6, -0.5) -- (6, -1.5);
        
        { \huge
            \draw (-2, 0) node {$ys$};
            \draw (0, -2) node {$3$};
            \draw (3, -2) node {$4$};
            \draw (6, -2) node {$5$};
            \draw (7, -2) node {$\cdot$};
        }
    \end{scope}
    
    
    
\end{tikzpicture}\par
	(до)\par
%	\vspace{0.5cm}
	\documentclass[tikz]{standalone}
%\usepackage{fontawesome}

% \newfontfamily{\FA}{Font Awesome 5 Free} % some glyphs missing
\expandafter\def\csname faicon@facebook\endcsname{{\FA\symbol{"F09A}}}
\def\faQuestionSign{{\FA\symbol{"F059}}}
\def\faQuestion{{\FA\symbol{"F128}}}
\def\faExclamation{{\FA\symbol{"F12A}}}
\def\faUploadAlt{{\FA\symbol{"F093}}}
\def\faLemon{{\FA\symbol{"F094}}}
\def\faPhone{{\FA\symbol{"F095}}}
\def\faCheckEmpty{{\FA\symbol{"F096}}}
\def\faBookmarkEmpty{{\FA\symbol{"F097}}}

%\def\faCatt{{\FA\symbol{"F6BE}}}
%\def\faCat{\faicon{cat}}
%\def\faCat{\faicon{yoast}}
\expandafter\def\csname faicon@dog\endcsname{{\FA\symbol{"F4DA}}}
%\def\faDog{\faicon{dog}}
%\def\faDog{{\FA\symbol{"F4DA}}}
%\def\faDogg{{\FA\symbol{"F6D3}}}
%\def\faDogg{{\FA\symbol{"F596}}}

% /usr/share/texlive/texmf-dist/fonts/opentype/public/fontawesome5/FontAwesome5Free-Solid-900.otf
\newfontfamily{\FAS}{FontAwesome5Free-Solid-900.otf}
%\expandafter\def\csname faicon@download\endcsname{{\FAS\symbol{"F6D3}}}
\expandafter\def\csname faicon@cat\endcsname{{\FAS\symbol{"F6BE}}}
\def\faCat{\faicon{cat}}
\expandafter\def\csname faicon@dog\endcsname{{\FAS\symbol{"F6D3}}}
\def\faDog{\faicon{dog}}
\expandafter\def\csname faicon@dragon\endcsname{{\FAS\symbol{"F6D5}}}
\def\faDragon{\faicon{dragon}}
\expandafter\def\csname faicon@fish\endcsname{{\FAS\symbol{"F578}}}
\def\faFish{\faicon{fish}}
\expandafter\def\csname faicon@horse\endcsname{{\FAS\symbol{"F6F0}}}
\def\faHorse{\faicon{horse}}
\expandafter\def\csname faicon@spider\endcsname{{\FAS\symbol{"F717}}}
\def\faSpider{\faicon{spider}}

\expandafter\def\csname faicon@chessking\endcsname{{\FAS\symbol{"F43F}}}
\def\faChessKing{\faicon{chessking}}
\expandafter\def\csname faicon@chessqueen\endcsname{{\FAS\symbol{"F445}}}
\def\faChessQueen{\faicon{chessqueen}}
\expandafter\def\csname faicon@chessrook\endcsname{{\FAS\symbol{"F447}}}
\def\faChessRook{\faicon{chessrook}}
\expandafter\def\csname faicon@chesspawn\endcsname{{\FAS\symbol{"F443}}}
\def\faChessPawn{\faicon{chesspawn}}
\expandafter\def\csname faicon@chessknight\endcsname{{\FAS\symbol{"F441}}}
\def\faChessKnight{\faicon{chessknight}}
\expandafter\def\csname faicon@chessbishop\endcsname{{\FAS\symbol{"F43A}}}
\def\faChessBishop{\faicon{chessbishop}}
\expandafter\def\csname faicon@chess\endcsname{{\FAS\symbol{"F439}}}
\def\faChess{\faicon{chess}}







\usepackage{tikz}
\usetikzlibrary{positioning,trees,decorations.pathreplacing}

\begin{document}
\begin{tikzpicture}[thick,scale=0.5, every node/.style={scale=0.5}]
    \tikzstyle{marrs}=[very thick,-latex]

    
    \begin{scope}
    
        \foreach \x/\y in {0/0, 0/-2, 3/-2, 6/-2} {
            \draw (\x - 0.5, \y - 0.5) rectangle +(1, 1); \draw (\x + 1 - 0.5, \y - 0.5) rectangle +(1, 1);
        }
        \draw[marrs] (-1.5, 0) -> +(1, 0);
        \draw[marrs] (0, 0) -> +(0, -1.5);
        \draw[marrs] (1, -2) -> +(1.5, 0);
        \draw[marrs] (4, -2) -> +(1.5, 0);
        \draw[marrs] (1, 0) -- (17.5, 0) .. controls (17.75, 0) and (18, -0.25) .. (18, -0.5) -- (18, -1.5);
        \draw[marrs] (7, -2) -- +(4.5, 0);
        
        { \huge
            \draw (-2, 0) node {$zs$};
            \draw (0, -2) node {$0$};
            \draw (3, -2) node {$1$};
            \draw (6, -2) node {$2$};
        }
    
    \end{scope}
    
    \begin{scope}[xshift=12cm]
    
        \foreach \x/\y in {0/-2, 3/-2, 6/-2} {
            \draw (\x - 0.5, \y - 0.5) rectangle +(1, 1); \draw (\x + 1 - 0.5, \y - 0.5) rectangle +(1, 1);
        }
        
        \draw[marrs] (1, -2) -> +(1.5, 0);
        \draw[marrs] (4, -2) -> +(1.5, 0);
        
        { \huge
            \draw (0, -2) node {$3$};
            \draw (3, -2) node {$4$};
            \draw (6, -2) node {$5$};
            \draw (7, -2) node {$\cdot$};
        }
    \end{scope}
\end{tikzpicture}
\end{document}
\par
	(после)\par
%	\vspace{0.5cm}
	\caption{Выполнение \texttt{xs ++ ys} в императивной среде. Эта операция уничтожает списки-аргументы \texttt{xs} и \texttt{ys} (их использовать больше нельзя)}
	\label{fig:2.4}
\end{figure}
\end{frame}


\begin{frame}[fragile]{Конкатенация в функциональной среде}
В функциональной среде мы не можем деструктивно модифицировать. Поэтому
\begin{itemize}
\item добавляем последний элемент первого списка ко второму
\item добавляем \emph{пред}последний элемент первого списка к результату
\item и т.д.
\end{itemize}

\inputminted[firstline=50,lastline=54] {haskell}{code/Stacks.hs}
Если нам доступно внутреннее представление, то можно написать более короткий идиоматичный код
\inputminted[firstline=57,lastline=58,gobble=2] {haskell}{code/Stacks.hs}
\end{frame}

%\begin{frame}[fragile]{}
%\inputminted[firstline=50,lastline=54] {haskell}{code/Stacks.hs}
%Если нам доступно внутреннее представление, то можно написать более короткий идиоматичный код
%\inputminted[firstline=57,lastline=58,gobble=2] {haskell}{code/Stacks.hs}
%\end{frame}

\begin{frame}[fragile]{Конкатенация}
\begin{figure}[h]
	\centering
	\begin{tikzpicture}[thick,scale=0.5, every node/.style={scale=0.5}]
    \tikzstyle{marrs}=[very thick,-latex]

    \begin{scope}
    
        \foreach \x/\y in {0/0, 3/0, 6/0} {
            \draw (\x - 0.5, \y - 0.5) rectangle +(1, 1); \draw (\x + 1 - 0.5, \y - 0.5) rectangle +(1, 1);
        }
        \draw[marrs] (-1.5, 0) -> +(1, 0);
        \draw[marrs] (1, 0) -> +(1.5, 0);
        \draw[marrs] (4, 0) -> +(1.5, 0);
        
        { \huge
            \draw (-2, 0) node {$xs$};
            \draw (0, 0) node {$0$};
            \draw (3, 0) node {$1$};
            \draw (6, 0) node {$2$};
            \draw (7, 0) node {$\cdot$};
        }
    
    \end{scope}
    
    \begin{scope}[xshift=12cm]
    
        \foreach \x/\y in {0/0, 3/0, 6/0} {
            \draw (\x - 0.5, \y - 0.5) rectangle +(1, 1); \draw (\x + 1 - 0.5, \y - 0.5) rectangle +(1, 1);
        }
        \draw[marrs] (-1.5, 0) -> +(1, 0);
        \draw[marrs] (1, 0) -> +(1.5, 0);
        \draw[marrs] (4, 0) -> +(1.5, 0);
        
        { \huge
            \draw (-2, 0) node {$ys$};
            \draw (0, 0) node {$3$};
            \draw (3, 0) node {$4$};
            \draw (6, 0) node {$5$};
            \draw (7, 0) node {$\cdot$};
        }
    \end{scope}
    
    
    
\end{tikzpicture}\par
	(до)\par
	\vspace{0.5cm}
	\begin{tikzpicture}[thick,scale=0.5, every node/.style={scale=0.5}]
    \tikzstyle{marrs}=[very thick,-latex]
    
    
    
    \begin{scope}
    
        \foreach \x/\y in {0/0, 3/0, 6/0} {
            \draw (\x - 0.5, \y - 0.5) rectangle +(1, 1); \draw (\x + 1 - 0.5, \y - 0.5) rectangle +(1, 1);
        }
        \draw[marrs] (-1.5, 0) -> +(1, 0);
        \draw[marrs] (1, 0) -> +(1.5, 0);
        \draw[marrs] (4, 0) -> +(1.5, 0);
        
        { \huge
            \draw (-2, 0) node {$xs$};
            \draw (0, 0) node {$0$};
            \draw (3, 0) node {$1$};
            \draw (6, 0) node {$2$};
            \draw (7, 0) node {$\cdot$};
        }
    
    \end{scope}
    
    \begin{scope}[xshift=12cm]
    
        \foreach \x/\y in {0/0, 3/0, 6/0} {
            \draw (\x - 0.5, \y - 0.5) rectangle +(1, 1); \draw (\x + 1 - 0.5, \y - 0.5) rectangle +(1, 1);
        }
        \draw[marrs] (-1.5, 0) -> +(1, 0);
        \draw[marrs] (1, 0) -> +(1.5, 0);
        \draw[marrs] (4, 0) -> +(1.5, 0);
        
        { \huge
            \draw (-2, 0) node {$ys$};
            \draw (0, 0) node {$3$};
            \draw (3, 0) node {$4$};
            \draw (6, 0) node {$5$};
            \draw (7, 0) node {$\cdot$};
        }
    \end{scope}
    
    \begin{scope}[yshift=2cm]
    
        \foreach \x/\y in {0/0, 3/0, 6/0} {
            \draw (\x - 0.5, \y - 0.5) rectangle +(1, 1); \draw (\x + 1 - 0.5, \y - 0.5) rectangle +(1, 1);
        }
        \draw[marrs] (-1.5, 0) -> +(1, 0);
        \draw[marrs] (1, 0) -> +(1.5, 0);
        \draw[marrs] (4, 0) -> +(1.5, 0);
        
        { \huge
            \draw (-2, 0) node {$zs$};
            \draw (0, 0) node {$0$};
            \draw (3, 0) node {$1$};
            \draw (6, 0) node {$2$};
            
            \draw[marrs] (7, 0) -- (11.5, 0) .. controls (11.75, 0) and (12, -0.25) .. (12, -0.5) -- (12, -1.5);
        }
    
    \end{scope}
    
    
    
\end{tikzpicture}\par
	(после)\par
	\vspace{0.5cm}
	\caption{Выполнение \texttt{zs = xs ++ ys} в функциональной среде. Заметим, что списки-аргументы \texttt{xs} и \texttt{ys} не затронуты операцией.
	}
	\label{fig:2.5}
\end{figure}
Несмотря на большой объем копирования, заметим, что второй список копировать не пришлось
\end{frame}

\begin{frame}[fragile]{Update}
\inputminted[firstline=60,lastline=64] {haskell}{code/Stacks.hs}
%Если нам доступно внутреннее представление, то можно написать более короткий идиоматичный код
Здесь мы не копируем весь список-аргумент.\\

Копировать приходится
только сам узел, подлежащий модификации (узел $i$) и узлы,
содержащие прямые или косвенные указатели на $i$. \\

Другими словами,
чтобы изменить один узел, мы копируем все узлы на пути от корня
к изменяемому. Все узлы, не находящиеся на этом пути, используются как
исходной, так и обновленной версиями. 
%На Рис.
%~\ref{fig:2.6} 
%показан
%результат изменения третьего узла в пятиэлементном списке: первые
%три узла копируются, а последние два используются совместно.
\end{frame}

\begin{frame}[fragile]{}
\begin{figure}[h]
	\centering
	\begin{tikzpicture}[thick,scale=0.5, every node/.style={scale=0.5}]
    \tikzstyle{marrs}=[very thick,-latex]

    \begin{scope}
    
        \foreach \x/\y in {0/0, 3/0, 6/0, 9/0, 12/0} {
            \draw (\x - 0.5, \y - 0.5) rectangle +(1, 1); \draw (\x + 1 - 0.5, \y - 0.5) rectangle +(1, 1);
        }
        \draw[marrs] (-1.5, 0) -> +(1, 0);
        \foreach \x in {1, 4, 7, 10} {
            \draw[marrs] (\x, 0) -> +(1.5, 0);
        }
        
        { \huge
            \draw (-2, 0) node {$xs$};
            \foreach \x/\y in {0/0, 3/1, 6/2, 9/3, 12/4} {
                \draw (\x, 0) node {$\y$};
            }
            \draw (13, 0) node {$\cdot$};
        }
    
    \end{scope}
    
\end{tikzpicture}\par
	(до)\par
	\vspace{0.5cm}
	\begin{tikzpicture}[thick,scale=0.5, every node/.style={scale=0.5}]
    \tikzstyle{marrs}=[very thick,-latex]
    
    
    
    \begin{scope}
    
        \foreach \x/\y in {0/0, 3/0, 6/0, 9/0, 12/0} {
            \draw (\x - 0.5, \y - 0.5) rectangle +(1, 1); \draw (\x + 1 - 0.5, \y - 0.5) rectangle +(1, 1);
        }
        \draw[marrs] (-1.5, 0) -> +(1, 0);
        \foreach \x in {1, 4, 7, 10} {
            \draw[marrs] (\x, 0) -> +(1.5, 0);
        }
        
        { \huge
            \draw (-2, 0) node {$xs$};
            \foreach \x/\y in {0/0, 3/1, 6/2, 9/3, 12/4} {
                \draw (\x, 0) node {$\y$};
            }
            \draw (13, 0) node {$\cdot$};
        }   
    
    \end{scope}
    
    \begin{scope}[yshift=2cm]
    
        \foreach \x/\y in {0/0, 3/0, 6/0} {
            \draw (\x - 0.5, \y - 0.5) rectangle +(1, 1); \draw (\x + 1 - 0.5, \y - 0.5) rectangle +(1, 1);
        }
        \draw[marrs] (-1.5, 0) -> +(1, 0);
        \draw[marrs] (1, 0) -> +(1.5, 0);
        \draw[marrs] (4, 0) -> +(1.5, 0);
        
        { \huge
            \draw (-2, 0) node {$ys$};
            \draw (0, 0) node {$0$};
            \draw (3, 0) node {$1$};
            \draw (6, 0) node {$7$};
            
            \draw[marrs] (7, 0) -- (8.5, 0) .. controls (8.75, 0) and (9, -0.25) .. (9, -0.5) -- (9, -1.5);
        }
    
    \end{scope}
    
    
    
\end{tikzpicture}\par
	(после)\par
	\vspace{0.5cm}	
	\caption{Выполнение \texttt{ys = update xs 2 7}. Обратите
		внимание на совместное использование структуры списками \texttt{xs} и \texttt{ys}.}
	\label{fig:2.6}
\end{figure}
\end{frame}

\begin{frame}[fragile]
\begin{remark}
	Такой стиль программирования очень сильно упрощается при наличии
	автоматической сборки мусора. Очень важно освободить память от тех
	копий, которые больше не нужны, но многочисленные совместно используемые
	узлы делают ручную сборку мусора нетривиальной задачей.
\end{remark}
\begin{exercise}\label{ex:2.1}
  Напишите функцию \texttt{suffixes} типа \mintinline{haskell}{[a] -> [a]}, которая принимает как
  аргумент список \texttt{xs} и возвращает список всех его
  суффиксов в убывающем порядке длины. Например,
  \begin{minted}{haskell}
  suffixes [1,2,3,4] = [[1,2,3,4],[2,3,4],[3,4],[4],[]]
  \end{minted}
  Покажите, что список суффиксов можно породить за время $O(n)$ и
  занять при этом $O(n)$ памяти.
\end{exercise}

\end{frame}

\section{Двоичные деревья поиска}
\label{sc:2.2}

\begin{frame}{Двоичные деревья поиска}
Если узел структуры содержит более одного указателя, оказываются
возможны более сложные сценарии совместного использования памяти. Хорошим примером
совместного использования такого вида служат \emph{двоичные деревья поиска}.

\inputminted[firstline=10, lastline=10] {haskell}{code/SearchTree.hs}

Двоичные деревья поиска~--- это двоичные деревья, в которых элементы
хранятся во внутренних узлах в \term{симметричном}{symmetric}
порядке, то есть, элемент в каждом узле больше любого элемента в
левом поддереве этого узла и меньше любого элемента в правом
поддереве.
\end{frame}

\begin{frame}[fragile]{Сигнатура для множеств упорядоченных элементов}
\begin{figure}[h]
  \centering
  \inputminted[firstline=12, lastline=15]{haskell}{code/SearchTree.hs}
%  \caption{Сигнатура для множеств.}
\label{fig:2.7}
\end{figure}

Cигнатура для множеств значение <<пустое множество>>, а также функции добавления
нового элемента и проверки на членство. \\

 В более практической
реализации, вероятно, будут присутствовать и многие другие функции,
например, для удаления элемента или перечисления всех элементов.
\end{frame}

\begin{frame}[fragile]{Функция \hsinline{member}}
Ищет в дереве, сравнивая запрошенный элемент с находящимся в корне дерева. 

\inputminted[firstline=22, lastline=26] {haskell}{code/SearchTree.hs}


Если мы когда-либо натыкаемся на пустое дерево, значит,
запрашиваемый элемент не является членом множества, и мы возвращаем
значение \hsinline{False}. \\

Если запрошенный элемент \textbf{меньше}
корневого, мы рекурсивно ищем в левом поддереве.

Если он \textbf{больше}, рекурсивно ищем в правом поддереве. 

Наконец, в оставшемся случае
запрошенный элемент \textbf{равен} корневому, и мы возвращаем значение
\hsinline{True}.
\end{frame}

\begin{frame}[fragile]{Функция \hsinline{insert}}
%\inputminted[firstline=10, lastline=10] {haskell}{code/SearchTree.hs}
\inputminted[firstline=28, lastline=32,gobble=2] {haskell}{code/SearchTree.hs}

Функция \hsinline{insert} проводит поиск в дереве по той же стратегии,
что и \hsinline{member}, но только по пути она копирует каждый
элемент. \\

Когда, наконец, оказывается достигнут пустой узел, он
заменяется на узел, содержащий новый элемент.
 
\end{frame}

\begin{frame}[fragile]{}
\begin{minipage}{.48\textwidth}
		\documentclass[tikz]{standalone}
%\usepackage{fontawesome}

% \newfontfamily{\FA}{Font Awesome 5 Free} % some glyphs missing
\expandafter\def\csname faicon@facebook\endcsname{{\FA\symbol{"F09A}}}
\def\faQuestionSign{{\FA\symbol{"F059}}}
\def\faQuestion{{\FA\symbol{"F128}}}
\def\faExclamation{{\FA\symbol{"F12A}}}
\def\faUploadAlt{{\FA\symbol{"F093}}}
\def\faLemon{{\FA\symbol{"F094}}}
\def\faPhone{{\FA\symbol{"F095}}}
\def\faCheckEmpty{{\FA\symbol{"F096}}}
\def\faBookmarkEmpty{{\FA\symbol{"F097}}}

%\def\faCatt{{\FA\symbol{"F6BE}}}
%\def\faCat{\faicon{cat}}
%\def\faCat{\faicon{yoast}}
\expandafter\def\csname faicon@dog\endcsname{{\FA\symbol{"F4DA}}}
%\def\faDog{\faicon{dog}}
%\def\faDog{{\FA\symbol{"F4DA}}}
%\def\faDogg{{\FA\symbol{"F6D3}}}
%\def\faDogg{{\FA\symbol{"F596}}}

% /usr/share/texlive/texmf-dist/fonts/opentype/public/fontawesome5/FontAwesome5Free-Solid-900.otf
\newfontfamily{\FAS}{FontAwesome5Free-Solid-900.otf}
%\expandafter\def\csname faicon@download\endcsname{{\FAS\symbol{"F6D3}}}
\expandafter\def\csname faicon@cat\endcsname{{\FAS\symbol{"F6BE}}}
\def\faCat{\faicon{cat}}
\expandafter\def\csname faicon@dog\endcsname{{\FAS\symbol{"F6D3}}}
\def\faDog{\faicon{dog}}
\expandafter\def\csname faicon@dragon\endcsname{{\FAS\symbol{"F6D5}}}
\def\faDragon{\faicon{dragon}}
\expandafter\def\csname faicon@fish\endcsname{{\FAS\symbol{"F578}}}
\def\faFish{\faicon{fish}}
\expandafter\def\csname faicon@horse\endcsname{{\FAS\symbol{"F6F0}}}
\def\faHorse{\faicon{horse}}
\expandafter\def\csname faicon@spider\endcsname{{\FAS\symbol{"F717}}}
\def\faSpider{\faicon{spider}}

\expandafter\def\csname faicon@chessking\endcsname{{\FAS\symbol{"F43F}}}
\def\faChessKing{\faicon{chessking}}
\expandafter\def\csname faicon@chessqueen\endcsname{{\FAS\symbol{"F445}}}
\def\faChessQueen{\faicon{chessqueen}}
\expandafter\def\csname faicon@chessrook\endcsname{{\FAS\symbol{"F447}}}
\def\faChessRook{\faicon{chessrook}}
\expandafter\def\csname faicon@chesspawn\endcsname{{\FAS\symbol{"F443}}}
\def\faChessPawn{\faicon{chesspawn}}
\expandafter\def\csname faicon@chessknight\endcsname{{\FAS\symbol{"F441}}}
\def\faChessKnight{\faicon{chessknight}}
\expandafter\def\csname faicon@chessbishop\endcsname{{\FAS\symbol{"F43A}}}
\def\faChessBishop{\faicon{chessbishop}}
\expandafter\def\csname faicon@chess\endcsname{{\FAS\symbol{"F439}}}
\def\faChess{\faicon{chess}}







\usepackage{tikz}
\usetikzlibrary{positioning,trees,decorations.pathreplacing}

\begin{document}
\begin{tikzpicture}[thick,scale=0.5, every node/.style={scale=0.5},level distance=3cm]
    \tikzstyle{marrs}=[very thick,-latex]
    \tikzstyle{tnode}=[circle, draw=black,node distance=1.7cm]
    \tikzstyle{level 1}=[sibling distance=5.6cm]
    \tikzstyle{level 2}=[sibling distance=3.4cm]


    \huge

    \draw (0, 2.5) node {$xs$};
    \draw[marrs] (0, 2) -- (0, 0.7);
    \node[tnode] {d}
    child {node[tnode] {b}
        child {node[tnode] {a}}
        child {node[tnode] {c}}
    }
    child {node[tnode] {g}
        child {node[tnode] {f}}
        child {node[tnode] {h}}
    };
    
\end{tikzpicture}
\end{document}\par
\end{minipage}
\begin{minipage}{.48\textwidth}
	\begin{tikzpicture}[thick,scale=0.5, every node/.style={scale=0.5},level distance=3cm]
    \tikzstyle{marrs}=[very thick,-latex]
    \tikzstyle{tnode}=[circle, draw=black,node distance=1.7cm]
    \tikzstyle{level 1}=[sibling distance=5.6cm]
    \tikzstyle{level 2}=[sibling distance=3.4cm]
    
    \huge

    \draw (0, 2.5) node {$xs$};
    \draw[marrs] (0, 2) -- +(0, -1.3);
    
    \draw (1.7, 2.5) node {$ys$};
    \draw[marrs] (1.7, 2) -- +(0, -1.3);
    
    \node[tnode] (n_d) {d}
    child {node[tnode] (n_b) {b}
        child {node[tnode] (n_a) {a}}
        child {node[tnode] (n_c) {c}}
    }
    child {node[tnode] (n_g) {g}
        child {node[tnode] (n_f) {f}
        node[tnode] (n_f') [right of=n_f] {f} [clockwise from=250]
        child {node[tnode] (n_e) {e}}
        }
        child {node[tnode] (n_h) {h}}
        node[tnode] (n_g') [right of=n_g] {g}
    }
    node[tnode] (n_d') [right of=n_d]{d};
    
    \path (n_d') edge (n_b);
    \path (n_d') edge (n_g');
    \path (n_g') edge (n_f');
    \path (n_g') edge (n_h);
    
\end{tikzpicture}
\end{minipage}
Выполнение \texttt{ys = insert "e" xs}. 
%Как и прежде,
%обратите внимание на совместное использвание структуры деревьями \texttt{xs} и \texttt{ys}.

%Каждый скопированный узел использует одно из поддеревьев 
%совместно с исходным деревом; речь о том поддереве,
%которое не оказалось на пути поиска. 
Для большинства деревьев путь
поиска содержит лишь небольшую долю узлов в дереве. Громадное
большинство узлов находятся в совместно используемых поддеревьях.
\end{frame}

\begin{frame}
\begin{exercise}\textbf{Андерсон \cite{Andersson1991}}\label{ex:2.2}
  В худшем случае \lstinline{member} производит $2d$ сравнений, где
  $d$~--- глубина дерева. Перепишите ее так, чтобы она делала не более
  $d+1$ сравнений, сохраняя элемент, который \emph{может} оказаться
  равным запрашиваемому (например, последний элемент, для которого
  операция $<$ вернула значение <<истина>> или $\le$~--- <<ложь>>, и
  производя проверку на равенство только по достижении дна дерева.
\end{exercise}

\begin{exercise}\label{ex:2.3}
  Вставка уже существующего элемента в двоичное дерево поиска копирует
  весь путь поиска, хотя скопированные узлы неотличимы от
  исходных. Перепишите \lstinline{insert} так, чтобы она избегала
  копирования с помощью исключений. Установите только один обработчик
  исключений для всей операции поиска, а не по обработчику на итерацию.
\end{exercise}
\end{frame}

\ifanswers
\begin{frame}
\begin{exercise}\label{ex:2.4}
  Совместите улучшения из предыдущих двух упражнений, и получите
  версию \lstinline{insert}, которая не делает ненужного копирования и
  использует не более $d+1$ сравнений.
\end{exercise}
\end{frame}

\begin{frame}
\begin{exercise}\label{ex:2.5}
  Совместное использование может быть полезно и внутри одного объекта, не
  обязательно между двумя различными.  Например, если два поддерева
  одного дерева идентичны, их можно представить одним и тем же
  деревом.
  \begin{enumerate}
    \item Используя эту идею, напишите функцию \lstinline{complete} типа
    \mintinline{haskell}{Elem -> Int -> Tree}, такую, что
    \mintinline{haskell}{complete(x,d)} создает полное двоичное дерево глубины
    \lstinline{d}, где в каждом узле содержится \lstinline{x}.
    (Разумеется, такая функция бессмысленна для абстракции множества,
    но она может оказаться полезной для какой-либо другой абстракции,
    например, мультимножества.) Функция должна работать за время $O(d)$.
    \item Расширьте свою функцию, чтобы она строила сбалансированные
    деревья произвольного размера. Эти деревья не всегда будут полны,
    но они должны быть как можно более сбалансированными: для любого
    узла размеры поддеревьев должны различаться не более чем на
    единицу. Функция должна работать за время $O(\log n)$. (Подсказка:
    воспользуйтесь вспомогательной функцией \lstinline{create2},
    которая, получая размер $m$, создает пару деревьев~--- одно размера
    $m$, а другое размера $m+1$)
  \end{enumerate}
\end{exercise}
\end{frame}


\begin{frame}
\begin{exercise}\label{ex:2.6}
  Измените функтор \texttt{UnbalancedSet} так, чтобы он служил
  реализацией не множеств, а \term{конечных отображений}{finite maps}. На
  Рис.~\ref{fig:2.10} приведена минимальная сигнатура для конечных
  отображений. (Заметим, что исключение \texttt{NotFound} не
  является встроенным в Стандартный ML~--- Вам придется его определить
  самостоятельно. Это исключение можно было бы сделать частью
  сигнатуры \texttt{FiniteMap},  чтобы каждая реализация
  определяла собственное исключение \texttt{NotFound}, но удобнее,
  если все конечные отображения будут использовать одно и то же
  исключение.)
\end{exercise}
\end{frame}
\fi 

\section{Левоориентированные кучи}

\begin{frame}[fragile]{}
Как правило, множества и конечные отображения поддерживают эффективный
доступ к произвольным элементам. Однако иногда требуется эффективный
доступ только к \emph{минимальному} элементу.  Структура данных,
поддерживающая такой режим доступа, называется \term{очередь с
приоритетами}{priority queue} или \term{куча}{heap}.

\inputminted[firstline=4, lastline=12] {haskell}{code/Heap.lhs}
\end{frame}

\begin{frame}[fragile]{}
\begin{definition}[\term{Порядок кучи}{heap-ordered}]
Элемент при каждой вершине не больше элементов в поддеревьях.
\end{definition}
При таком упорядочении минимальный элемент дерева всегда находится в корне.

\begin{definition}[\term{Правая периферия}{right spine} узла]
Самого правого пути от данного узла до пустого 
\end{definition}
 Ранг узла определяется как длина его правой периферии. 
\begin{definition}[Свойство \term{левоориентированности}{leftist property}]
Ранг любого левого поддерева не меньше ранга его сестринской правой вершины. 
\end{definition}
 Простым
следствием свойства левоориентированности является то, что правая
периферия любого узла~--- кратчайший путь от него к пустому узлу.\\

Левоориентированные кучи \cite{Crane1972, Knuth1973a} представляют
собой двоичные деревья с порядком кучи, обладающие свойством
левоориентированности.

\end{frame}

\begin{frame}[fragile]{Левоориентированные кучи}
Если у нас есть некоторый тип упорядоченных элементов
\mintinline{haskell}{e}, 
мы можем представить левоориентированные кучи как
двоичные деревья, снабженные информацией о ранге.
\inputminted[firstline=6, lastline=6] {haskell}{code/LeftistHeap.lhs}

Заметим, что элементы правой периферии левоориентированной кучи (да и
любого дерева с порядком кучи) расположены в порядке возрастания.\\

Главная идея левоориентированной кучи заключается в том, что для
слияния двух куч достаточно слить их правые периферии как
упорядоченные списки, а затем вдоль полученного пути обменивать
местами поддеревья при вершинах, чтобы восстановить свойство
левоориентированности. 

\end{frame}

\begin{frame}[fragile]{}
\inputminted[firstline=20, lastline=25] {haskell}{code/LeftistHeap.lhs}

где \hsinline{makeT}~--- вспомогательная функция, вычисляющая ранг
вершины \hsinline{T} и, если необходимо, меняющая местами ее
поддеревья.

\inputminted[firstline=8, lastline=13] {haskell}{code/LeftistHeap.lhs}

Поскольку длина правой периферии любой вершины в худшем случае
логарифмическая, \hsinline{merge} выполняется за время $O(\log n)$.
\end{frame}

\begin{frame}[fragile]{}
\inputminted[firstline=27, lastline=32] {haskell}{code/LeftistHeap.lhs}

Поскольку \hsinline{merge} выполняется за время $O(\log n)$, столько
же занимают и \hsinline{insert} с \hsinline{deleteMin}.\\

Очевидно, что \hsinline{findMin} выполняется за $O(1)$. 
\end{frame}

\ifanswers
\begin{frame}[fragile]{}
\begin{exercise}\label{ex:3.2}
  Определите \hsinline{insert} напрямую, а не через обращение к \hsinline{merge}.
\end{exercise}

\begin{exercise}\label{ex:3.3}
  Реализуйте функцию \hsinline{fromList} типа \hsinline{Elem.T list $\to$ Heap},
  порождающую левоориентированную кучу из неупорядоченного списка
  элементов путем преобразования каждого элемента в одноэлементную
  кучу, а затем слияния получившихся куч, пока не останется
  одна. Вместо того, чтобы сливать кучи проходом слева направо или
  справа налево при помощи \hsinline{foldr} или \hsinline{foldl},
  слейте кучи за $\lceil \log n \rceil$ проходов, где на каждом
  проходе сливаются пары соседних куч. Покажите, что
  \hsinline{fromList} требует всего $O(n)$ времени.
\end{exercise}
\end{frame}

\begin{frame}[fragile]{}
\begin{exercise}\label{ex:3.4}
  Левоориентированные кучи
  со сдвинутым весом~--- альтернатива левоориентированным кучам, где
  вместо свойства левоориентированности используется свойство
  \term{левоориентированности, сдвинутой по весу}{weight-biased leftist
    property}: размер любого левого поддерева всегда не меньше размера
  соответствующего правого поддерева.
  \begin{enumerate}
    \item Докажите, что правая периферия левоориентированной кучи со
    сдвинутым весом содержит не более $\lfloor \log(n+1) \rfloor$ элементов.
    \item Измените реализацию, чтобы получились
    левоориентированные кучи со сдвинутым весом.
    \item Функция \lstinline!merge! сейчас выполняется в два прохода:
    сверху вниз, с вызовами \lstinline!merge!, и снизу вверх, с
    вызовами вспомогательной функции \lstinline!makeT!. Измените
    \lstinline!merge! для левоориентированных куч со сдвинутым весом
    так, чтобы она работала за один проход сверху вниз.
    \item Каковы преимущества однопроходной версии в
    условиях ленивого вычисления? Параллельного?
  \end{enumerate}
\end{exercise}
\end{frame}
\fi 


\section{Биномиальные кучи}
\label{sc:3.2}

\begin{frame}{Биномиальные кучи}
Биномиальные очереди \cite{Vuillemin1978, Brown1978}, которые мы,
чтобы избежать путаницы с очередями FIFO, будем называть \term{ биномиальными
  кучами}{binomial heaps}~--- ещё одна распространенная реализация
куч. \\

Биномиальные кучи устроены сложнее, чем левоориентированные, и, на
первый взгляд, не возмещают эту сложность никакими
преимуществами. \\

Однако, с помощью дополнительных хитростей (амортизация\cite{sc:5.3}), можно заставить \hsinline{insert} и
\hsinline{merge} выполняться за время $O(1)$.

\end{frame}

\begin{frame}[fragile]{Биномиальные деревья. Пример}
\begin{figure}[h]
  \centering
  \begin{tikzpicture}[thick,scale=0.5, every node/.style={scale=0.5},grow via three points={%
one child at (0,-1.5) and two children at (0,-1.5) and (-0.8,-1.5)}
]
    \tikzstyle{marrs}=[very thick,-latex]
    \tikzstyle{tnode}=[circle, fill=black, inner sep=1.5mm]
    \def\rstep{5cm}
    
    \huge
    
    \node[tnode] (0, 0) {};
            child { node[tnode] {} }
            child { node[tnode] {} };
    
    \begin{scope}
        \draw (0, 1) node {Ранг 0};
        \node[tnode] (0, 0) {};
    \end{scope}
    
    \begin{scope}[xshift=\rstep]
        \draw (0, 1) node {Ранг 1};
        \node[tnode] {}
            child {node[tnode] {} };
    \end{scope}
    
    \begin{scope}[xshift=2 * \rstep]
        \draw (0, 1) node {Ранг 2};
        \node[tnode] {}
            child {node[tnode] {} }
            child {node[tnode] {} 
                    child {node[tnode] {} }};
            
            
    \end{scope}
    
    \begin{scope}[xshift=3 * \rstep]
        \draw (0, 1) node {Ранг 3};
        \node[tnode] {}
            child {node[tnode] {} }
            child {node[tnode] {} 
                child {node[tnode] {} }
            }
            child {node[tnode] {} 
                child {node[tnode] {} }
                child {node[tnode] {} 
                    child {node[tnode] {} }
                }
          };
            
            
    \end{scope}
    
    
\end{tikzpicture}

  \caption{Биномиальные деревья рангов 0--3.}
  \label{fig:3.3}
\end{figure}

%\inputminted[firstline=5, lastline=6] {haskell}{code/BinomialHeap.lhs}

\end{frame}

\begin{frame}{Биномиальные деревья. Определение}
Биномиальные кучи строятся из более простых объектов, называемых
биномиальными деревьями. Биномиальные деревья индуктивно определяются
так:
\begin{itemize}
  \item Биномиальное дерево ранга 0 представляет собой одиночный узел.
  \item Биномиальное дерево ранга $r+1$ получается путем
  \term{связывания}{linking} двух биномиальных деревьев ранга $r$, так
  что одно из них становится самым левым потомком второго.
\end{itemize}
Из этого определения видно, что биномиальное дерево ранга $r$ содержит
ровно $2^r$ элементов.  \\

Существует второе, эквивалентное первому,
определение биномиальных деревьев, которым иногда удобнее
пользоваться: биномиальное дерево ранга $r$ представляет собой узел
с $r$ потомками $t_1\ldots t_r$, где каждое $t_i$ является
биномиальным деревом ранга $(r-i)$.
\end{frame}




\begin{frame}[fragile]{}
\inputminted[firstline=5, lastline=5] {haskell}{code/BinomialHeap.lhs}

Каждый список потомков хранится в \emph{убывающем} (sic!) порядке рангов, а элементы
хранятся вместе с рангом кучи.  Чтобы сохранять этот порядок рангов, мы всегда
привязываем дерево с большим корнем к дереву с меньшим.

\inputminted[firstline=11, lastline=15] {haskell}{code/BinomialHeap.lhs}

Будем привязывать деревья только с одинаковым рангом
\end{frame}


\begin{frame}[fragile]{}
Определяем биномиальную кучу как 
\begin{itemize}
  \item коллекцию биномиальных деревьев
  \item каждое из которых имеет порядок кучи
  \item никакие два дерева не совпадают по рангу
\end{itemize} 

Например, список деревьев в порядке возрастания ранга.
\inputminted[firstline=6, lastline=6] {haskell}{code/BinomialHeap.lhs}
\end{frame}


\begin{frame}[fragile]{Биномиальные кучи и числа}
Поскольку каждое биномиальное дерево содержит $2^r$ элементов, и
никакие два дерева по рангу не совпадают, деревья размера $n$ в
точности соответствуют единицам в двоичном представлении
$n$.\\

Например, число $21_{10} = 10101_2$, и поэтому
биномиальная куча размера 21 содержит одно дерево ранга 0, одно ранга
2, и одно ранга 4 (размерами, соответственно, 1, 4 и 16).\\

Заметим, что
так же, как двоичное представление $n$ содержит не более $\lfloor log
(n+1)\rfloor$ единиц, биномиальная куча размера $n$ содержит не более
$\lfloor log(n+1) \rfloor$ деревьев.
\end{frame}


\begin{frame}[fragile]{}
\begin{figure}[h]
  \centering
  \begin{tikzpicture}[thick,scale=0.5, every node/.style={scale=0.5},grow via three points={%
one child at (0,-1.5) and 
two children at (0,-1.5) and (-0.8,-1.5) 
}
]
    \tikzstyle{marrs}=[very thick,-latex]
    \tikzstyle{tnode}=[circle, fill=black, inner sep=1.5mm]
    \def\rstep{5cm}
    
    \huge
    
    \node[tnode] (0, 0) {};
            child { node[tnode] {} }
            child { node[tnode] {} };
    
    \begin{scope}
        \draw (0, 1) node {Ранг 0};
        \node[tnode] (0, 0) {};
    \end{scope}
        
    \begin{scope}[xshift=1 * \rstep]
        \draw (0, 1) node {Ранг 2};
        \node[tnode] {}
            child {node[tnode] {} }
            child {node[tnode] {} 
                    child {node[tnode] {} }};
    \end{scope}
    
    \begin{scope}[xshift=3 * \rstep]
      \draw (0, 1) node {Ранг 4};
      \node[tnode] {}
          child {node[tnode] {} }
          child {node[tnode] {} 
            child {node[tnode] {} }
          }
          child {node[tnode] {} 
            child {node[tnode] {} }
            child {node[tnode] {} 
              child {node[tnode] {} }
            }
          }
        child[missing] {}
        child {node[tnode] {}
          child {node[tnode] {} }
          child {node[tnode] {} 
            child {node[tnode] {} }
          }
          child {node[tnode] {} 
            child {node[tnode] {} }
            child {node[tnode] {} 
              child {node[tnode] {} }
            }
          }
     };
      \end{scope}
    
    
\end{tikzpicture}

  \caption{Число $21_{10} = 10101_2$, и поэтому
    биномиальная куча размера 21 содержит одно дерево ранга 0, одно ранга
    2, и одно ранга 4 (размерами, соответственно, 1, 4 и 16).}
\end{figure}

\end{frame}


\begin{frame}[fragile]{\hsinline{insert} -- аналогично сложению}
% (Мы укрепим эту аналогию в Главе~\ref{ch:9}.). \\

Чтобы внести элемент в кучу,
мы сначала создаем одноэлементное дерево (т.~е., биномиальное дерево
ранга 0), затем поднимаемся по списку существующих деревьев в порядке
возрастания рангов, связывая при этом одноранговые деревья. Каждое
связывание соответствует переносу в двоичной арифметике.

\inputminted[firstline=8,lastline=8] {haskell}{code/BinomialHeap.lhs}
\inputminted[firstline=17,lastline=19] {haskell}{code/BinomialHeap.lhs}
\inputminted[firstline=38,lastline=38,gobble=2] {haskell}{code/BinomialHeap.lhs}
В худшем случае, при вставке в кучу размера $n = 2^k -1$, требуется
$k$ связываний и $O(k) = O(\log n)$ времени.
\end{frame}


\begin{frame}[fragile]{\hsinline{merge} -- аналогично сложению}
При слиянии двух куч мы проходим через оба списка деревьев в порядке
возрастания ранга и связываем по пути деревья равного ранга. Как и
прежде, каждое связывание соответствует переносу в двоичной
арифметике.

\inputminted[firstline=21,lastline=26] {haskell}{code/BinomialHeap.lhs}

\inputminted[firstline=39,lastline=39,gobble=2] {haskell}{code/BinomialHeap.lhs}

\end{frame}

\begin{frame}[fragile]{Поиск минимального элемента}
Функции \hsinline{findMin} и \hsinline{deleteMin} вызывают
вспомогательную функцию \hsinline{removeMinTree}, которая находит
дерево с минимальным корнем, исключает его из списка и возвращает как
это дерево, так и список оставшихся деревьев.

\inputminted[firstline=28,lastline=32]{haskell}{code/BinomialHeap.lhs}

Функция \hsinline{findMin} просто возвращает корень найденного дерева

\inputminted[firstline=40,lastline=41,gobble=2] {haskell}{code/BinomialHeap.lhs}
\inputminted[firstline=9,lastline=9] {haskell}{code/BinomialHeap.lhs}
\end{frame}

\begin{frame}[fragile]{Удаление минимального элемента}
Функция \hsinline{deleteMin} устроена немного похитрее. \\

 Отбросив
корень найденного дерева, мы ещё должны вернуть его потомков в список
остальных деревьев. Заметим, что список потомков \emph{почти} уже
соответствует определению биномиальной кучи. Это коллекция
биномиальных деревьев с неповторяющимися рангами, но только
отсортирована она не по возрастанию, а по убыванию ранга. Таким
образом, обратив список потомков, мы преобразуем его в биномиальную
кучу, а затем сливаем с оставшимися деревьями.

\inputminted[firstline=43,lastline=44,gobble=2] {haskell}{code/BinomialHeap.lhs}
\end{frame}

\ifanswers
\begin{frame}[fragile]{}
\begin{exercise}\label{ex:3.5}
  Определите \lstinline!findMin! напрямую, без обращения к \lstinline!removeMinTree!.
\end{exercise}

\begin{exercise}\label{ex:3.6}
  Большая часть аннотаций ранга в нашем представлении биномиальных куч
  излишня, потому что мы и так знаем, что дети узла ранга $r$ имеют
  ранги $(r\!-\!1), \ldots, 0$. Таким образом, можно исключить
  поле-аннотацию ранга из узлов, а вместо этого помечать ранг корня
  каждого дерева, т.~е.,
  \begin{minted}{haskell}
  data Tree a = Node  a [Tree]
  type Heap = [(Int, Tree)]
  \end{minted}
  Реализуйте биномиальные кучи в таком представлении.
\end{exercise}
\end{frame}

% Ещё одно упражнение не скопипастьил

\fi

\section{Красно-чёрные деревья}
\label{sc:3.3}

\begin{frame}[fragile]{Красно-чёрные деревья}
Двоичные деревья поиска хорошо ведут себя на случайных или неупорядоченных данных,
однако на упорядоченных данных их производительность резко падает, и
каждая операция может занимать до $O(n)$  времени. \\

 Решение этой
проблемы состоит в том, чтобы каждое дерево поддерживать в
приблизительно сбалансированном состоянии. Тогда каждая операция
выполняется не хуже, чем за время $O(\log n)$. \\

 Одним из наиболее
популярных семейств сбалансированных двоичных деревьев поиска являются
красно-чёрные \cite{GuibasSedgewick1978}.
\end{frame}

\begin{frame}[fragile]{}
Красно-чёрное дерево представляет собой двоичное дерево поиска, в
котором каждый узел окрашен либо красным, либо чёрным. Мы добавляем
поле цвета в тип двоичных деревьев поиска %из раздела~\ref{sc:2.2}.

\inputminted[firstline=6,lastline=7] {haskell}{code/RedBlackSet.lhs}
Все пустые узлы считаются чёрными, поэтому пустой конструктор
\hsinline{E} в поле цвета не нуждается.

\end{frame}

\begin{frame}[fragile]{Красно-чёрные деревья. Инварианты}
Мы требуем, чтобы всякое красно-чёрное дерево соблюдало два
инварианта:
\begin{itemize}
  \item \textbf{Инвариант 1.} У красного узла не может быть красного ребёнка.
  \item \textbf{Инвариант 2.} Каждый путь от корня дерева до пустого
  узла содержит одинаковое количество чёрных узлов.
\end{itemize}
Вместе эти два инварианта гарантируют, что самый длинный возможный
путь по красно-чёрному дереву, где красные и чёрные узлы чередуются,
не более чем вдвое длиннее самого короткого, состоящего только из
чёрных узлов.

\ifanswers
\begin{exercise}\label{ex:3.8}
  Докажите, что максимальная глубина узла в красно-чёрном дереве
  размера $n$ не превышает $2 \lfloor \log (n+1) \rfloor$.
\end{exercise}
\fi
\end{frame}

\begin{frame}[fragile]{}
Функция \hsinline{member} для красно-чёрных деревьев не обращает
внимания на цвета. За исключением wildcard в варианте для конструктора
\hsinline{T}, она не отличается от функции \hsinline{member} для
несбалансированных деревьев.
\inputminted[firstline=18,lastline=21] {haskell}{code/RedBlackSet.lhs}
\end{frame}


\begin{frame}[fragile]{}
Функция \hsinline{insert} интереснее: она должна
поддерживать два инварианта балансировки.

\inputminted[firstline=23,lastline=29] {haskell}{code/RedBlackSet.lhs}

Эта функция содержит три существенных изменения по сравнению с \hsinline{insert} для
несбалансированных деревьев поиска. Во-первых, когда мы создаем новый
узел в ветке \hsinline{ins E}, мы сначала окрашиваем его в красный
цвет. Во-вторых, независимо от цвета, возвращаемого \hsinline{ins},
в окончательном результате мы корень окрашиваем чёрным. Наконец, в
ветках \hsinline{x < y} и \hsinline{x > y} мы вызовы конструктора
\hsinline{T} заменяем на обращения к функции
\hsinline{balance}. Функция \hsinline{balance} действует подобно
конструктору \hsinline{T}, но только она переупорядочивает свои
аргументы, чтобы обеспечить выполнение инвариантов баланса.
\end{frame}


\begin{frame}[fragile]{}
\inputminted[firstline=9,lastline=13] {haskell}{code/RedBlackSet.lhs}

Если новый узел окрашен красным, мы сохраняем Инвариант 2, но в
случае, если отец нового узла тоже красный, нарушается Инвариант 1. Мы
временно позволяем существовать одному такому нарушению, и переносим
его снизу вверх по мере перебалансирования. Функция
\hsinline{balance} обнаруживает и исправляет красно-красные нарушения,
когда обрабатывает чёрного родителя красного узла с красным
ребёнком. Такая чёрно-красно-красная цепочка может возникнуть в
четырёх различных конфигурациях, в зависимости от того, левым или
правым ребёнком является каждая из красных вершин. Однако в каждом из
этих случаев решение одно и то же: нужно преобразовать
чёрно-красно-красный путь в красную вершину с двумя чёрными детьми,
как показано на рисунке ниже.
\end{frame}


\begin{frame}[fragile]{}
После балансировки некоторого поддерева красный корень этого поддерева
может оказаться ребёнком ещё одного красного узла. Таким образом,
балансировка продолжается до самого корня дерева. На самом верху
дерева мы можем получить красную вершину с красным ребёнком, но без
чёрного родителя. С этим вариантом мы справляемся, всегда перекрашивая корень
в чёрное.

\end{frame}


\begin{frame}[fragile]{}
\begin{figure}[h]
  \centering
  \begin{tikzpicture}[thick,scale=0.5, every node/.style={scale=0.5},level distance=2cm, sibling distance=2cm]
    \tikzstyle{tblack}=[circle, line width=1mm, draw=black]
    \tikzstyle{tred}=[circle, draw=black]
    \def\xstep{7cm}
    \def\ystep{8cm}
    
    \huge
    
    % legend
    \begin{scope}[xshift=7cm, yshift=7cm]
        \def\inse{3.5mm}
     %   \draw (-1, -2.5) rectangle (5, 1);
        
        \node[tblack, inner sep=\inse] at (0,0) {};
        \node[tred, inner sep=\inse] at (0,-1.6) {};
        \node[right=1pt] at (1,0) { -- черный};
        \node[right=1pt] at (1,-1.6) { -- красный};
    \end{scope}
    
    \begin{scope}[yshift=\ystep]
        \node[tblack] {z}
            child { node[tred] {x}
                child { node {a} }
                child { node[tred] {y}
                    child {node {b}}
                    child {node {c}}
                }
            }
            child { node {d} };
    \end{scope}
    
    \begin{scope}[xshift=\xstep]
        \node[tblack] {x}
            child { node {a} }
            child { node[tred] {y}
                child { node {b} }
                child { node[tred] {z}
                    child {node {c}}
                    child {node {d}}
                }
            };
    \end{scope}
    
    \begin{scope}[xshift=-\xstep]
        \node[tblack] {z}
            child { node[tred] {y}
                child { node[tred] {x}
                    child {node {a}}
                    child {node {b}}
                }
                child { node {c} }
            }
            child { node {d} };
    \end{scope}
    
%    \begin{scope}[yshift=-\ystep]
%        \node[tblack] {x}
%            child { node {a} }
%            child { node[tred] {z}
%                child { node[tred] {y}
%                    child {node {b}}
%                    child {node {c}}
%                }
%                child { node {d} }
%            };
%    \end{scope}
    
    \begin{scope}[yshift=-1.5cm]
        \tikzstyle{level 1}=[sibling distance=3cm]
        \tikzstyle{level 2}=[sibling distance=2cm]
        \node[tred] {y}
            child { node[tblack] {x}
                child { node {a} }
                child { node {b} }
            }
            child { node[tblack] {z}
                child { node {c} }
                child { node {d} }
            };
    \end{scope}
    \Huge
    \draw (0, 0.5cm) node[rotate=-90] {$\Rightarrow$};
%    \draw (0, -8cm) node[rotate=90] {$\Rightarrow$};
    \draw (-4cm, -4cm) node[rotate=0] {$\Rightarrow$};
    \draw (4cm, -4cm) node[rotate=180] {$\Rightarrow$};
    
\end{tikzpicture}

%  \caption{Избавление от красных узлов с красными родителями.}
  \label{fig:3.5}
\end{figure}
\end{frame}

\begin{frame}[fragile]{}
\begin{figure}[h]
  \centering
  \begin{tikzpicture}[thick,scale=0.5, every node/.style={scale=0.5},level distance=2cm, sibling distance=2cm]
    \tikzstyle{tblack}=[circle, line width=1mm, draw=black]
    \tikzstyle{tred}=[circle, draw=black]
    \def\xstep{7cm}
    \def\ystep{8cm}
    
    \huge
    
%    % legend
%    \begin{scope}[xshift=7cm, yshift=7cm]
%        \def\inse{3.5mm}
%     %   \draw (-1, -2.5) rectangle (5, 1);
%        
%        \node[tblack, inner sep=\inse] at (0,0) {};
%        \node[tred, inner sep=\inse] at (0,-1.6) {};
%        \node[right=1pt] at (1,0) { -- черный};
%        \node[right=1pt] at (1,-1.6) { -- красный};
%    \end{scope}
    
%    \begin{scope}[yshift=\ystep]
%        \node[tblack] {z}
%            child { node[tred] {x}
%                child { node {a} }
%                child { node[tred] {y}
%                    child {node {b}}
%                    child {node {c}}
%                }
%            }
%            child { node {d} };
%    \end{scope}
    
    \begin{scope}[xshift=\xstep]
        \node[tblack] {x}
            child { node {a} }
            child { node[tred] {y}
                child { node {b} }
                child { node[tred] {z}
                    child {node {c}}
                    child {node {d}}
                }
            };
    \end{scope}
    
    \begin{scope}[xshift=-\xstep]
        \node[tblack] {z}
            child { node[tred] {y}
                child { node[tred] {x}
                    child {node {a}}
                    child {node {b}}
                }
                child { node {c} }
            }
            child { node {d} };
    \end{scope}
    
    \begin{scope}[yshift=-\ystep]
        \node[tblack] {x}
            child { node {a} }
            child { node[tred] {z}
                child { node[tred] {y}
                    child {node {b}}
                    child {node {c}}
                }
                child { node {d} }
            };
    \end{scope}
    
    \begin{scope}[yshift=-1.5cm]
        \tikzstyle{level 1}=[sibling distance=3cm]
        \tikzstyle{level 2}=[sibling distance=2cm]
        \node[tred] {y}
            child { node[tblack] {x}
                child { node {a} }
                child { node {b} }
            }
            child { node[tblack] {z}
                child { node {c} }
                child { node {d} }
            };
    \end{scope}
    \Huge
%    \draw (0, 0.5cm) node[rotate=-90] {$\Rightarrow$};
    \draw (0, -6.5cm) node[rotate=90] {$\Rightarrow$};
    \draw (-4cm, -4cm) node[rotate=0] {$\Rightarrow$};
    \draw (4cm, -4cm) node[rotate=180] {$\Rightarrow$};


    
\end{tikzpicture}

%  \caption{Избавление от красных узлов с красными родителями.}
  \label{fig:3.5.2}
\end{figure}
\end{frame}

\begin{frame}[fragile]{}
\begin{hint}
  Даже без дополнительных оптимизаций наша реализация сбалансированных
  двоичных деревьев поиска~--- одна из самых быстрых среди
  имеющихся. С оптимизациями вроде описанных в
  Упражнениях~\ref{ex:2.2} и \ref{ex:3.10} она просто летает!
\end{hint}
\end{frame}


\begin{frame}[fragile]{Почему это выглядит короче императивной реализации?}
\begin{remark}
  Одна из причин, почему наша реализация выглядит настолько проще, чем
  типичное описание красно-чёрных деревьев 
%  (напр., Глава~14 в книге~\cite{CormenLeisersonRivest1990})
  , состоит в том, что мы
  используем несколько другие преобразования перебалансировки. \\
  
  В
  императивных реализациях обычно наши четыре проблематичных случая
  разбиваются на восемь, в зависимости от цвета узла, соседствующего с
  красной вершиной с красным ребёнком.  Знание цвета этого узла в
  некоторых случаях позволяет совершить меньше присваиваний, а в
  некоторых других завершить балансировку раньше. Однако в
  функциональной среде мы в любом случае копируем все эти вершины, и
  таким образом, не можем ни сократить число присваиваний, ни
  прекратить копирование раньше времени, так что для использования
  более сложных преобразований нет причины.
\end{remark}

\end{frame}


\begin{frame}[fragile]{}
\begin{exercise}\label{ex:3.9}
  Напишите функцию \hsinline{fromOrdList} типа \hsinline{[a] -> Tree a},
  преобразующую отсортированный список без повторений в красно-чёрное
  дерево. Функция должна выполняться за время $O(n)$.
\end{exercise}

\begin{exercise}\label{ex:3.10}
  Приведенная нами функция \hsinline{balance} производит несколько
  ненужных проверок. Например, когда функция \hsinline{ins}
  рекурсивно вызывается для левого ребёнка, не требуется проверять
  красно-красные нарушения на правом ребёнке.
  \begin{enumerate}
    \item Разбейте \hsinline{balance} на две функции
    \hsinline{lbalance} и \hsinline{rbalance}, которые проверяют,
    соответственно, нарушения инварианта в левом и правом
    ребёнке. Замените обращения к \hsinline{balance} внутри
    \hsinline{ins} на вызовы \hsinline{lbalance} либо \hsinline{rbalance}.
    \item Ту же самую логику можно распространить ещё на шаг и убрать
    одну из проверок для внуков. Перепишите \hsinline{ins} так, чтобы
    она никогда не проверяла цвет узлов, не находящихся на пути поиска.
  \end{enumerate}
\end{exercise}
\end{frame}


%
%\section{Ленивое вычисление}


%\chap{Основы амортизации}


\section{Методы амортизированного анализа}
\label{sc:5.1}


\begin{frame}{Методы амортизированного анализа}
Реализации с амортизированными
характеристиками производительности часто оказываются проще и быстрее,
чем реализации со сравнимыми жёсткими характеристиками. \\

К сожалению, простой подход к амортизации, рассматриваемый в этой
главе, конфликтует с идеей устойчивости~--- эти структуры, будучи
используемы как устойчивые, могут быть весьма неэффективны. Однако на
практике многие приложения устойчивости не требуют, и часто для таких
приложений реализации, представленные в этой главе, могут быть
замечательным выбором. \\

Чтобы совместить амортизацию и устойчивость стоит применить 
\emph{ленивые вычисления}.

%В следующей главе мы увидим, как можно
%совместить понятия амортизации и устойчивости при помощи ленивого
%вычисления.
%
%TODO:
\end{frame}


\begin{frame}[fragile]{}
Понятие амортизации возникает из следующего наблюдения.  Имея
последовательность операций, мы можем интересоваться временем, которое
отнимает вся эта последовательность, однако при этом нам может быть
безразлично время каждой отдельной операции.\\

 Например, имея $n$
операций, мы можем желать, чтобы время всей последовательности было
ограничено показателем $O(n)$, не настаивая, чтобы каждая из этих
операций происходила за время $O(1)$. Нас может устраивать, чтобы
некоторые из операций занимали $O(\log n)$ или даже $O(n)$, при
условии, что общая стоимость всей последовательности будет
$O(n)$. \\

Такая дополнительная степень свободы открывает широкое
пространство возможностей при проектировании, и часто позволяет найти
более простые и быстрые решения, чем варианты с аналогичными жёсткими
ограничениями.

\end{frame}


\begin{frame}[fragile]{}
$$
\sum_{i=1}^m a_i \ge \sum_{i=1}^m t_i
$$
где $a_i$~--- амортизированная стоимость $i$-й операции, $t_i$~--- ее
реальная стоимость, а $m$~--- общее число операций.\\

 Обычно
доказывается несколько более сильный результат: что на любой
промежуточной стадии в последовательности операций общая текущая
амортизированная стоимость является верхней границей для общей текущей
реальной стоимости, т.~е. для любого $j$
$$
\sum_{i=1}^j a_i \ge \sum_{i=1}^j t_i
$$
\end{frame}


\begin{frame}[fragile]{}
\begin{definition}
Разница между общей текущей амортизированной стоимостью
и общей текущей реальной стоимостью называется
\term{текущие накопления}{accumulated savings}. 
\end{definition}
Таким образом, общая
текущая амортизированная стоимость является верхней границей для
общей текущей реальной стоимости тогда и только тогда, когда текущие
накопления неотрицательны.

\end{frame}


\begin{frame}[fragile]{}
Амортизация позволяет некоторым операциям быть дороже, чем их
амортизированная стоимость. Такие операции называются
\term{дорогими}{expensive}. Операции, для которых амортизированная
стоимость превышает реальную, называются
\term{дешевыми}{cheap}. Дорогие операции уменьшают текущие накопления,
а дешевые их увеличивают.\\

 Главное при доказательстве
амортизированных характеристик стоимости~--- показать, что дорогие
операции случаются только тогда, когда текущих накоплений хватает,
чтобы покрыть их дополнительную стоимость.
\end{frame}



\begin{frame}[fragile]{}
\begin{itemize}
  \item \term{Метод банкира}{banker's method} 
      \begin{itemize}
        \item \term{кредит}{credits}
      \end{itemize}
  \item \term{Метод физика}{physicist's method}
      \begin{itemize}
        \item \term{потенциал}{potential}
      \end{itemize}
\end{itemize}



Кредит и потенциал являются лишь средствами анализа; ни
то, ни другое не присутствует в тексте программы (разве что, возможно,
в комментариях).

\end{frame}

\begin{frame}[fragile]{}
 В методе банкира
текущие накопления представляются как \term{кредит}{credits},
привязанный к определенным ячейкам структуры данных. Этот кредит
используется, чтобы расплатиться за будущие операции доступа к этим
ячейкам.  Амортизированная стоимость операции определяется как ее
реальная стоимость плюс размер кредита, выделяемого этой операцией,
минус размер кредита, который она расходует, т.~е.,
$$
a_i = t_i + c_i - \bar{c}_i
$$
где $c_i$~--- размер кредита, выделяемого операцией $i$, а $\bar{c}_i$~---
размер кредита, расходуемого операцией $i$.


\end{frame}

\begin{frame}[fragile]{}
$$
a_i = t_i + c_i - \bar{c}_i
$$
где $c_i$~--- размер кредита, выделяемого операцией $i$, а $\bar{c}_i$~---
размер кредита, расходуемого операцией $i$.\\

 Каждая единица кредита
должна быть выделена, прежде чем израсходована, и нельзя расходовать
кредит дважды. Таким образом, $\sum c_i \ge \sum \bar{c}_i$, а
следовательно, как и требуется, $\sum a_i \ge \sum t_i$.\\

Как правило,
доказательства с использованием метода банкира определяют
\term{инвариант кредита}{credit invariant}, регулирующий распределение
кредита так, чтобы при всякой дорогой операции достаточное его
количество было выделено в нужных ячейках структуры для покрытия
стоимости операции.
\end{frame}

\begin{frame}[fragile]{Метод физика}
Определяется функция $\Phi$, отображающая всякий
объект $d$ на действительное число, называемое его
\term{потенциалом}{potential}.  Потенциал обычно выбирается так, чтобы
изначально равняться нулю и оставаться неотрицательным. В таком случае
потенциал представляет нижнюю границу  текущих накоплений.\\

Пусть объект $d_i$ будет результатом операции $i$ и аргументом
операции $i+1$. Тогда амортизированная стоимость операции $i$
определяется как сумма реальной стоимости и изменения потенциалов между
$d_{i-1}$ и $d_i$, т.~е.,
$$
a_i = t_i + \Phi(d_i) - \Phi(d_{i-1})
$$
текущих накоплений.


\end{frame}


\begin{frame}[fragile]{}
$$
a_i = t_i + \Phi(d_i) - \Phi(d_{i-1})
$$
Текущая реальная стоимость последовательности операций равна
$$
\begin{array}{rcl}
\sum_{i=1}^j t_i & = & \sum_{i=0}^j (a_i + \Phi(d_{i-1}) - \Phi(d_i))\\
\\
& = & \sum_{i=1}^j a_i + \sum_{i=1}^j (\Phi(d_{i-1}) - \Phi(d_i)) \\
\\
& = & \sum_{i=1}^j a_i + \Phi(d_0) - \Phi(d_j)
\end{array}
$$

Если $\Phi$ выбран таким образом, что
$\Phi(d_0)$ равен нулю, а $\Phi(d_j)$ неотрицателен, мы имеем
$\Phi(d_j) \ge \Phi(d_0)$, так что, как и требуется, текущая общая
амортизированная стоимость является верхней границей для текущей общей
реальной стоимости.

\end{frame}

\begin{comment}
\begin{remark}
Такое описание метода физика несколько упрощает
картину. Часто при анализе оказывается трудно втиснуть реальное
положение дел в указанные рамки. Например, что делать с функциями,
которые порождают или возвращают более одного объекта? Однако даже
упрощенное описание достаточно для демонстрации основных идей.
\end{remark}

\end{comment}

\section{Очереди}
\label{sc:5.2}


\begin{frame}[fragile]{}

\begin{minipage}{.4\textwidth}
  \inputminted[firstline=5,lastline=11] {haskell}{code/Queue.lhs}
\end{minipage}
\begin{minipage}{.55\textwidth}
  Самая распространенная чисто функциональная реализация очередей
  представляет собой пару списков, \hsinline{f} и \hsinline{r}, где
  \hsinline{f} содержит головные элементы очереди в правильном порядке,
  а \hsinline{r} состоит из хвостовых элементов в обратном порядке.\\
  
  Например, очередь, содержащая целые числа 1\ldots 6, может быть
  представлена списками \hsinline{f=[1,2,3]} и
  \hsinline{r=[6,5,4]}. Это представление можно описать следующим
  типом:
  \begin{minted}{haskell}
  data Queue a = Queue [a] [a]
  \end{minted}
  
\end{minipage}
\end{frame}


\begin{frame}[fragile]{}
В этом представлении голова очереди~--- первый элемент \hsinline{f},
так что функции \hsinline{head} и \hsinline{tail}
возвращают и отбрасывают этот элемент, соответственно.
\begin{minted}{haskell}
head (x : f, r) = x
tail (x : f, r) = f
\end{minted}
Подобным образом, хвостом очереди является первый элемент
\hsinline{r}, так что \hsinline{snoc} добавляет к \hsinline{r}
новый.
\begin{minted}{haskell}
snoc (f,r) x = (f, x : r)
\end{minted}

\end{frame}


\begin{frame}[fragile]{}
Элементы добавляются к \hsinline{r} и убираются из \hsinline{f}, так
что они должны как-то переезжать из одного списка в другой. Этот
переезд осуществляется путем обращения \hsinline{r} и установки его
на место \hsinline{f} всякий раз, когда в противном случае
\hsinline{f} оказался бы пустым.\\

 Одновременно \hsinline{r}
устанавливается в \hsinline{[]}. Наша цель~--- поддерживать
инвариант, что список \hsinline{f} может быть пустым только в том
случае, когда список \hsinline{r} также пуст (т.~е., пуста вся
очередь). \\

Заметим, что если бы \hsinline{f} был пустым при непустом
\hsinline{r}, то первый элемент очереди находился бы в конце
\hsinline{r}, и доступ к нему занимал бы $O(n)$ времени. Поддерживая
инвариант, мы гарантируем, что функция \hsinline{head} всегда может
найти голову очереди за $O(1)$ времени.

\end{frame}


\begin{frame}[fragile]{}
\begin{minted}{haskell}
snoc (([], _), x) = ([x], [])
snoc (( f, r), x) = (f,  x :: r)
tail ([x], r) = (rev r, [])
tail (x:f, r) = (f, r)
\end{minted}

Заметим, что в первой ветке \hsinline{snoc} используется
wildcard. В этом случае поле \hsinline{r} проверять не нужно,
поскольку из инварианта мы знаем, что если список \hsinline{f} равен
\hsinline{[]}, то \hsinline{r} также пуст.

\end{frame}


\begin{frame}[fragile]{}
Чуть более изящный способ записать эти функции~--- вынести те части
\hsinline{snoc} и \hsinline{tail}, которые поддерживают инвариант, в
отдельную функцию \hsinline{checkf}. Она заменяет \hsinline{f} на
\hsinline{rev r}, если \hsinline{f} пуст, а в противном случае
ничего не делает.

\inputminted[firstline=10,lastline=15] {haskell}{code/NaiveQueue.hs}

Функции
\hsinline{snoc} и \hsinline{head} всегда завершаются за время
$O(1)$, но \hsinline{tail} в худшем случае отнимает $O(n)$
времени. 

Однако, используя либо метод банкира, либо метод физика, мы
можем показать, что как \hsinline{snoc}, так и \hsinline{tail}
занимают амортизированное время $O(1)$.

\end{frame}


\begin{frame}[fragile]{Чисто функциональная очередь и метод банкира}

Инвариант: каждый элемент в
хвостовом списке связан с одной единицей кредита. \\

Каждый вызов
\hsinline{snoc} для непустой очереди занимает один реальный шаг и
выделяет одну единицу кредита для элемента хвостового списка; таким
образом, общая амортизированная стоимость равна двум. \\

Вызов
\hsinline{tail}, не обращающий хвостовой список, занимает один шаг,
не выделяет и не тратит никакого кредита, и, таким образом, имеет
амортизированную стоимость 1. \\

Наконец, вызов \hsinline{tail},
обращающий хвостовой список, занимает $m+1$ реальный шаг, где $m$~---
длина хвостового списка, и тратит $m$ единиц кредита, содержащиеся в
этом списке, так что амортизированная стоимость получается $m + 1 - m
= 1$.

\end{frame}


\begin{frame}[fragile]{Чисто функциональная очередь и метод физика}
В методе физика мы определяем функцию потенциала $\Phi$ как длину
хвостового списка. \\

Тогда всякий \hsinline{snoc} к непустой очереди
занимает один реальный шаг и увеличивает потенциал на единицу, так что
амортизированная стоимость равна двум. \\

Вызов \hsinline{tail} без
обращения хвостовой очереди занимает один реальный шаг и не изменяет
потенциал, так что амортизированная стоимость равна одному.\\

 Наконец,
вызов \hsinline{tail} с обращением очереди занимает $m+1$ реальный
шаг, но при этом устанавливает хвостовой список равным \hsinline{[]},
уменьшая таким образом потенциал на $m$, так что амортизированная
стоимость равна $m + 1 - m = 1$.

\end{frame}


\begin{frame}[fragile]{}
\begin{hint}
  Эта реализация очередей идеальна в приложениях, где не требуется
  устойчивости и где приемлемы амортизированные показатели
  производительности.
\end{hint}

\end{frame}

\ifanswers
\begin{frame}[fragile]{}
\begin{exercise}\label{ex:5.1}
  \textbf{Хогерворд \cite{Hoogerwoord1992}.}  Идея этих очередей легко
  может быть расширена на абстракцию \term{двусторонней очереди}{double-ended
    queue}, или \term{дека}{deque}, где чтение и запись разрешены с
  обоих концов очереди (см. Рис.~\ref{fig:5.3}). Инвариант делается
  симметричным относительно \lstinline!f! и \lstinline!r!: если
  очередь содержит более одного элемента, оба списка должны быть
  непустыми. Когда один из списков становится пустым, мы делим другой
  пополам и одну из половин обращаем.
  
  \begin{enumerate}
    \item Реализуйте эту версию деков.
    \item Докажите, что каждая операция занимает $O(1)$ амортизированного
    времени, используя функцию потенциала $\Phi(f,r) = abs(|f| -
    |r|)$, где $abs$~--- функция модуля.
  \end{enumerate}
\end{exercise}
\end{frame}
\fi

\section{Биномиальные кучи}
\label{sc:5.3}


\begin{frame}[fragile]{}
В Разделе~\ref{sc:3.2} мы показали, что вставка в биномиальную кучу
проходит в худшем случае за время $O(\log n)$. Здесь мы доказываем,
что на самом деле амортизированное ограничение на время вставки
составляет $O(1)$.\\

Метод физика. Потенциал биномиальной кучи -- число деревьев в ней. 

Заметим, что это число равно количеству
единиц в двоичном представлении $n$, числа элементов в куче.  Вызов
\hsinline{insert} занимает $k+1$ шаг, где $k$~--- число обращений к
\hsinline{link}. Если изначально в куче было $t$ деревьев, то после
вставки окажется $t - k + 1$ деревьев. Таким образом, изменение
потенциала составляет $(t - k + 1) - t = 1 - k$, а амортизированная
стоимость вставки $(k + 1) + (1 - k) = 2$.\\

\begin{exercise}\label{ex:5.2}
  Повторите доказательство с использованием метода банкира.
\end{exercise}

\end{frame}


\begin{frame}[fragile]{}
Для полноты картины нам нужно показать, что амортизированная стоимость
операций \hsinline{merge} и \hsinline{deleteMin} по-прежнему
составляет $O(\log n)$. \hsinline{deleteMin} не доставляет здесь
никаких трудностей, но в случае \hsinline{merge} требуется небольшое
расширение метода физика. \\

До сих пор мы определяли амортизированную
стоимость операции как
$$
a = t + \Phi(d_{\mbox{\textit{вых}}}) - \Phi(d_{\mbox{\textit{вх}}})
$$
где $d_{\mbox{\textit{вх}}}$~--- структура на входе операции, а $d_{\mbox{\textit{вых}}}$~---
структура на выходе. Однако если операция принимает либо возвращает
более одного объекта, это определение требуется обобщить до
$$
a = t + \sum_{d \in \mbox{\textit{Вых}}} \Phi(d) - \sum_{d \in \mbox{\textit{Вх}}} \Phi(d)
$$
где $\mbox{\textit{Вх}}$~--- множество входов, а $\mbox{\textit{Вых}}$~--- множество выходов. В этом
правиле мы рассматриваем только входы и выходы анализируемого типа.
\end{frame}

\ifanswers
\begin{frame}[fragile]{}
\begin{exercise}\label{ex:5.3}
  Докажите, что амортизированная стоимость операций \hsinline{merge}
  и \hsinline{deleteMin} по-прежнему составляет $O(\log n)$.
\end{exercise}
\end{frame}
\fi


\section{Расширяющиеся кучи}
\label{sc:5.4}

\begin{frame}{\term{Расширяющиеся деревья}{splay trees}}
\term{Расширяющиеся деревья}{splay trees} \cite{SleatorTarjan1985}~--- возможно, самая известная
и успешно применяемая амортизированная структура данных.\\

 Расширяющиеся
деревья являются ближайшими родственниками двоичных сбалансированных
деревьев поиска, но они не хранят никакую информацию о балансе
явно. \\

Вместо этого каждая операция перестраивает дерево при помощи
некоторых простых преобразований, которые имеют тенденцию увеличивать
сбалансированность. Несмотря на то, что каждая конкретная операция
может занимать до $O(n)$ времени, амортизированная стоимость ее, как
мы покажем, не превышает $O(\log n)$.
\end{frame}


\begin{frame}[fragile]{Расширяющиеся vs. деревья поиска}
Важное различие между расширяющимися и сбалансированными
двоичными деревьями поиска вроде красно-чёрных деревьев из
Раздела~\ref{sc:3.3} состоит в том, что расширяющиеся деревья
перестраиваются даже во время запросов (таких, как \hsinline{member}),
а не только во время обновлений (таких, как \hsinline{insert}). \\

Это
свойство мешает использованию расширяющихся деревьев для реализации
абстракций вроде множеств или конечных отображений в чисто
функциональном окружении, поскольку приходилось бы возвращать в
запросе новое дерево наряду с ответом на запрос\footnote{%
  В принципе можно было бы хранить корень расширяющегося дерева в
  ссылочной ячейке и обновлять значение по ссылке при каждом запросе, но
  такое решение не является чисто функциональным.
}.
\end{frame}


\begin{frame}[fragile]{}
Представление расширяющихся деревьев идентично представлению
несбалансированных двоичных деревьев поиска.
\inputminted[firstline=5,lastline=5] {haskell}{code/SplayHeap.lhs}


Однако в отличие от несбалансированных двоичных деревьев поиска из
Раздела~\ref{sc:2.2}, мы позволяем дереву содержать повторяющиеся
элементы. Эта разница не является фундаментальным различием расширяющихся
деревьев и несбалансированных двоичных деревьев поиска; она просто
отражает отличие абстракции множества от абстракции кучи.

\end{frame}


\begin{frame}[fragile]{Реализация \hsinline{insert} }
Разобьем существующее дерево на два поддерева, чтобы одно содержало все
элементы, меньше или равные новому, а второе все элементы, большие
нового. Затем породим новый узел из нового элемента и двух этих
поддеревьев. В отличие от вставки в обыкновенное двоичное дерево
поиска, эта процедура добавляет элемент как корень дерева, а не как
новый лист.

\begin{minted}{haskell}
insert x t = T (smaller x t) x (bigger x t)
\end{minted}

где \hsinline{smaller} выделяет дерево из элементов, меньше или равных
\hsinline{x}, а \hsinline{bigger} -- больших
\hsinline{x}. 

\end{frame}


\begin{frame}[fragile]{Наивная реализация \hsinline{bigger}}
По аналогии с фазой разделения быстрой сортировки,
назовем новый элемент \term{границей}{pivot}.

Можно наивно реализовать \hsinline{bigger} как

\begin{minted}{haskell}
bigger pivot E = E
bigger pivot (T a x b) =
  if x <= pivot 
  then bigger pivot b
  else T (bigger pivot a) x b
\end{minted}
однако при таком решении не делается никакой попытки перестроить
дерево, добиваясь лучшего баланса.
%\begin{minted}{haskell}
%insert x t = T (smaller (x, t))  x (bigger (x, t))
%\end{minted}
\end{frame}

\begin{frame}[fragile]{Правильная реализация \hsinline{bigger} }
Вместо этого мы применяем простую
эвристику для перестройки: каждый раз, пройдя по двум левым ветвям
подряд, мы проворачиваем два пройденных узла.

\begin{minted}{haskell}
bigger pivot E = E
bigger pivot (T a x b) =
  if x <= pivot 
  then bigger pivot b
  else case a of
    E         -> T E x b
    T a1 y a2 ->
        if y <= pivot 
        then T (bigger pivot a2) x b)
        else T (bigger pivot a1) y (T a2 x b)
\end{minted} 
\end{frame}

\begin{frame}[fragile]{}
\begin{figure}
  \centering
  \begin{tikzpicture}[thick,scale=0.5, every node/.style={scale=0.5},grow via three points={%
one child at (-0.5,-1.8) and two children at (-0.5,-1.8) and (0.5,-1.8)}]
    \tikzstyle{tblack}=[circle, line width=1mm, draw=black]
    \tikzstyle{tred}=[circle, draw=black]
    \def\xstep{7cm}
    \def\ystep{10cm}
    
    \huge
    
    \begin{scope}
        \node {7}
        child { node {6}
        child { node {5}
        child { node {4}
        child { node {3}
        child { node {2}
        child { node {1}
        }}}}}}
        ;
    \end{scope}
    
    \begin{scope}[xshift=5cm]
        \node {6}
        child { node {4}
            child { node {2}
                child { node {1} } 
                child { node {3} } 
            }
            child { node {5} }
        }
        child { node {7} };
    \end{scope}
    
    \Huge
    \draw (1.5, -3.6) node[rotate=0] {$\Rightarrow$};
    
\end{tikzpicture}
  \caption{Вызов функции \hsinline{bigger} с граничным элементом \hsinline{pivot} = 0 на сильно несбалансированном дереве.}
  \label{fig:5.4}
\end{figure}


\end{frame}

\begin{frame}[fragile]{}
На Рис.~\ref{fig:5.4} показано, как \lstinline!bigger! действует на
сильно несбалансированное дерево. \\

Несмотря на то, что результат
по-прежнему не является сбалансированным в обычном смысле, новое
дерево намного сбалансированнее исходного; глубина каждого узла
уменьшилась примерно наполовину, от $d$ до $\lfloor d/2 \rfloor$ или
$\lfloor d/2 \rfloor + 1$.\\

Разумеется, мы не всегда можем уполовинить
глубину каждого узла в дереве, но мы можем уполовинить глубину каждого
узла, лежащего на пути поиска. \\

В сущности, в этом и состоит принцип
расширяющихся деревьев: нужно перестраивать путь поиска так, чтобы
глубина каждого лежащего на пути узла уменьшалась примерно вполовину.
\end{frame}

\ifanswers
\begin{frame}[fragile]{}
\begin{exercise}\label{ex:5.4}
  Реализуйте операцию \lstinline!smaller!. Не забудьте, что
  \lstinline!smaller! должна сохранять элементы, равные границе (однако
  устраивать отдельную проверку на равенство не следует!).
\end{exercise}
\end{frame}
\fi 


\begin{frame}[fragile]{}
Рассмотрим теперь \hsinline{findMin} и
\hsinline{deleteMin}. Минимальный элемент расширяющегося дерева
хранится в самой левой его вершине типа \hsinline{T}. Найти эту
вершину несложно.
\inputminted[firstline=42,lastline=44,gobble=2] {haskell}{code/SplayHeap.lhs}

Функция \hsinline{deleteMin} должна уничтожить самый левый узел и
одновременно перестроить дерево таким же образом, как это делает
\hsinline{bigger}. Поскольку мы всегда рассматриваем только левую
ветвь, сравнения не нужны.
\inputminted[firstline=46,lastline=49,gobble=2] {haskell}{code/SplayHeap.lhs}


\end{frame}

\begin{frame}[fragile]{}
N.B. Функция слияния
\hsinline{merge} довольно неэффективна и для многих входов
занимает $O(n)$ времени.\\

Можно показать методом физика, что \hsinline{insert} выполняется за время
$O(\log n)$.
\end{frame}

\begin{comment}
\begin{frame}[fragile]{}
Теперь мы хотим показать, что \hsinline{insert} выполняется за время
$O(\log n)$. Пусть $\#t$ обозначает размер дерева $t$ плюс
один. Заметим, что если $\hsinline{t = T a x b}$, то $\#t =
\#a + \#b$. Пусть потенциал вершины $\phi(t)$ равен $\log(\# t)$, а
потенциал всего дерева равен сумме потенциалов его вершин. Нам
требуется следующее элементарное утверждение, касающееся логарифмов:
\begin{lemma}\label{lm:5.1}
  Для всех положительных $x, y, z$, таких, что $y + z \le x$,
  $$
  1 + \log y + \log z < 2 \log x
  $$
  
  \noindent
  \textit{Доказательство.} Без потери общности предположим, что $y \le  z$.
  Тогда $y \le x/2$ и $z \le x$, так что $1 + \log y \le \log x$ и
  $\log z < \log x$
\end{lemma}

\end{frame}

\begin{frame}[fragile]{}

Пусть $\mathcal{T}(t)$ обозначает реальную стоимость вызова
\lstinline!partition! для дерева $t$, что определяется как число
рекурсивных вызовов \lstinline!partition!. Пусть $\mathcal{A}(t)$~---
амортизированная стоимость такого вызова, определяемая как
$$
\mathcal{A}(t) = \mathcal{T}(t) + \Phi(a) + \Phi(b) - \Phi(t)
$$
где $a$ и $b$~--- возвращаемые функцией \lstinline!partition!
поддеревья.

\end{frame}


\begin{frame}[fragile]{}
\begin{theorem}\label{th:5.2}
  $\mathcal{A}(t) \le 1 + 2\phi(t) = 1 + 2\log(\#t)$
  
  \noindent\textbf{Доказательство.} Требуется рассмотреть два
  нетривиальных случая, называемые зиг-зиг и зиг-заг, в зависимости
  от того, проходит ли вызов \hsinline{partition} по двум левым
  ветвям (или, симметрично, по двум правым), либо по левой ветке, а
  затем правой (или, симметрично, по правой, а затем по левой).
  
  Для случая зиг-зиг предположим, что исходное и результирующее дерево
  имеют формы
  
  \begin{center}
    \begin{tikzpicture}[thick,scale=0.5, every node/.style={scale=0.5},grow via three points={%
one child at (-0.8,-2.3) and two children at (-0.8,-1.8) and (0.8,-1.8)}]
    \tikzstyle{tblack}=[circle, line width=1mm, draw=black]
    \tikzstyle{tred}=[circle, draw=black]
    \def\xstep{7cm}
    \def\ystep{10cm}
    
    \huge
    
    \begin{scope}[xshift=0.8cm]
        \node (x) {$x$}
            child {node (y) {$y$}
                child {node {$u$}}
                child {node {$c$}}
                node[left of=y, below=-0.6cm] {$t = $}
            }
            child {node {$d$}}
            node[left of=x] {$s = $};
        
    \end{scope}
    
    \begin{scope}[xshift=4cm]
        \node at (0, -1.8){$a$};
        \node at (1, -1.8) {$||$};
        \node (y) at (3, 0) {$y$}
            child {node {$b$}}
            child {node (x) {$x$}
                child {node {$c$}}
                child {node {$d$}}
                node [right of=x, below=-0.6cm] {$ = t'$}
            }
            node [right of=y, below=-0.7cm] {$ = s'$};
    \end{scope}
    
    \Huge
    \draw (2.8, -1.8) node[rotate=0] {$\Rightarrow$};
    
\end{tikzpicture}
  \end{center}
  где $a$ и $b$ являются результатами вызова \hsinline{partition (pivot, u)}.
\end{theorem}
TODO;
\end{frame}

\begin{frame}[fragile]{}
   Тогда
  $$
  \begin{array}{ll}
  & \mathcal{A}(s) \\
  = & \qquad\{\mbox{ по определению $\mathcal{A}$ }\} \\
  & \mathcal{T}(s) + \Phi(a) + \Phi(s') - \Phi(s) \\
  = & \qquad\{\mbox{ $\mathcal{T}(s) = 1 + \mathcal{T}(u)$ }\} \\
  & 1 + \mathcal{T}(u) + \Phi(a) + \Phi(s') - \Phi(s) \\
  = & \qquad\{\mbox{ $\mathcal{T}(u) = \mathcal{A}(u) - \Phi(a) - \Phi(b) + \Phi(u)$ }\} \\
  & 1 + \mathcal{A}(u) - \Phi(a) - \Phi(b) + \Phi(u) + \Phi(a) + \Phi(s') - \Phi(s) \\
  = & \qquad\{\mbox{ раскрываем $\Phi(s)$ и $\Phi(s')$, упрощаем }\} \\
  & 1 + \mathcal{A}(u) + \phi(s') + \phi(t') - \phi(s) - \phi(t) \\
  \le & \qquad\{\mbox{ по предположению индукции, $\mathcal{A}(u) \le 1 + 2\phi(u)$ } \} \\
  & 2 + 2\phi(u) + \phi(s') + \phi(t') - \phi(s) - \phi(t) \\
  < & \qquad \{\mbox{$\phi(u) < \phi(t)$, а $\phi(s') \le \phi(s)$}\} \\
  & 2 + \phi(u) + \phi(t') \\
  < & \qquad \{\mbox{ $\#u + \#t' < \#s$, а также Лемма~\ref{lm:5.1} }\} \\
  & 1 + 2\phi(s) \\
  \end{array}
  $$
  Доказательство случая зиг-заг мы оставляем как упражнение для читателя.
\end{frame}

\begin{frame}[fragile]{}
Дополнительная стоимость операции \hsinline{insert} по сравнению с
\hsinline{partition} составляет один реальный шаг плюс разница
потенциалов между двумя поддеревьями-результатами
\hsinline{partition} и деревом-окончательным результатом
\hsinline{insert}. Это изменение потенциала равно просто $\phi$ от
нового корня. Поскольку амортизированная стоимость
\hsinline{partition} ограничена $1 + 2\log(\#t)$, амортизированная
стоимость \hsinline{insert} ограничена
$2 + 2\log(\#t) + \log(\#t + 1) \approx 2 + 3\log(\#t)$.

TODO
\end{frame}

\end{comment}

\chap{Некоторые известные структуры данных в функциональном
  окружении}
\label{ch:9}

\begin{frame}[fragile]{Списки похожи на числа}

Рассмотрим обыкновенные представления списков и натуральных чисел, а
также несколько типичных функций над этими типами данных.\vspace{.5cm}

\begin{minipage}{.48\textwidth}
\begin{minted}{haskell}
data List a = Nil
            | Cons of a [a]
  
append Nil ys = ys          
append (Cons x xs) ys =      
  Cons x (append xs ys)
\end{minted}
\end{minipage}
\begin{minipage}{.48\textwidth}
\begin{minted}{haskell}
data Nat = Zero
         | Succ of Nat
 
plus Zero n = n
plus (Succ m) n) =
  Succ (plus m n)
\end{minted}
\end{minipage}\vspace{.5cm}

Помимо того, что списки содержат элементы, а натуральные числа нет,
эти две реализации практически совпадают. Подобным же образом
соотносятся биномиальные кучи и двоичные числа. Эти примеры наводят на
сильную аналогию между представлениями числа $n$ и представлениями
объектов-контейнеров размером $n$. 

\end{frame}

\begin{frame}[fragile]{}

Функции, работающие с контейнерами,
полностью аналогичны арифметическим функциям, работающим с
числами. 
\begin{itemize}
  \item  добавление нового элемента $\sim (+1)$ 
  \item удаление элемента $\sim (-1)$ 
  \item слияние двух контейнеров $\sim (+)$ 
\end{itemize}
 

Можно использовать эту аналогию для проектирования новых
представлений абстракций контейнеров~--- достаточно выбрать
представление натуральных чисел, обладающее заданными свойствами, и
соответствующим образом определить функции над
объектами-контейнерами. \\

Назовем реализацию, спроектированную при помощи
этого приёма, \term{числовым представлением}{numerical representation}.

\end{frame}

\begin{frame}[fragile]{}

Будем исследовать несколько числовых представлений для двух
различных абстракций:
\begin{itemize}
  \item   \term{куч}{heaps}
  \item  \term{списков со свободным
    доступом}{random-access lists} a.k.a. \term{гибкие массивы}{flexible arrays}
\end{itemize}\vspace{.5cm}


Эти две абстракции подчёркивают различные наборы
арифметических операций. 

Нужны  эффективные функции:
\begin{itemize}
  \item Для куч: увеличения на единицу и сложения
  \item Для списков со свободным доступом: увеличения и уменьшения на единицу
\end{itemize}


\end{frame}


\section{Позиционные системы счисления}
\label{sc:9.1}

\begin{frame}[fragile]{\term{Позиционная система счисления}{positional number system}}


Cпособ записи числа в виде последовательности
цифр $b_0\ldots b_{m-1}$. Цифра $b_0$ называется \term{младшим разрядом}{least
  significant digit}, а цифра $b_{m-1}$ \term{старшим разрядом}{most
  significant digit}. \\

Кроме обычных десятичных чисел, мы всегда будем
записывать последовательности цифр в порядке \emph{от младшего разряда к старшему}.\\

Каждый разряд $b_i$ имеет вес $w_i$, так что значение
последовательности $b_0\ldots b_{m-1}$ равно $\sum_{i=0}^{m-1}
b_iw_i$.
\end{frame}

\begin{frame}[fragile]{Пример: единичные и двоичные числа}

 Для каждой конкретной позиционной системы счисления
последовательность весов фиксирована, и фиксирован набор цифр $D_i$,
из которых выбирается каждая $b_i$. 

\begin{minted}{haskell}
data Nat = Zero | Succ of Nat
\end{minted}
Для \textbf{единичных} чисел $w_i = 1$ и $D_i = \{\mathtt{1}\}$ для всех $i$.\\

А для \textbf{двоичных} чисел $w_i = 2^i$,
а $D_i = \{\mathtt{0}, \mathtt{1}\}$. \\

Говорится, что число записано по основанию $B$, если $w_i =
B^i$, а $D_i = \{\mathtt{0}, \ldots, B-1\}$. \\

Чаще всего, но не всегда,
веса разрядов представляют собой увеличивающуюся степенную
последовательность, а множество $D_i$ во всех разрядах одинаково.
\end{frame}

\begin{frame}[fragile]{Избыточные системы счисления}

Система счисления называется \term{избыточной}{redundant}, если
некоторые числа могут быть представлены более, чем одним способом.


Например, можно получить избыточную систему двоичного счисления, взяв
$w_i = 2^i$ и $D_i = \{\mathtt{0}, \mathtt{1}, \mathtt{2}\}$. Тогда

$$
13_{10} = 1011 = 1201 = 122
$$

N.B. Младшие разряды слева (кроме чисел по основанию 10)\\

 Мы запрещаем нули в конце числа,
поскольку иначе почти все системы счисления будут тривиально
избыточны.
\end{frame}

\begin{frame}[fragile]{Плотные и разреженные представления}


\term{Плотное}{dense} представление~--- это просто список (или какая-то другая
последовательность) цифр, включая нули. \\

Напротив, при \term{разреженном}{sparse}
представлении нули пропускаются. В таком случае требуется хранить
информацию либо о ранге (т.~е., индексе), либо о весе каждой ненулевой
цифры.   \\

%На Рис.~\ref{fig:9.1} показаны два разных представления
%двоичных чисел в Стандартном ML, одно из которых плотное, второе
%разреженное, а также функции увеличения на единицу, уменьшения на
%единицу и сложения для каждого из них. Среди уже виденных нами
%числовых представлений биномиальные кучи с расписаниями
%(Раздел~\ref{sc:7.3}) используют плотное представление, а биномиальные
%кучи (Раздел~\ref{sc:3.2}) и ленивые биномиальные кучи
%(Раздел~\ref{sc:6.4.1})~--- разреженное представление.

\end{frame}

\begin{frame}[fragile]{}
\inputminted{haskell}{code/DenseNumbers.hs}
\end{frame}

\begin{frame}[fragile]{}
\inputminted{haskell}{code/SparseByWeight.hs}
\end{frame}


\section{Двоичные числа}
\label{sc:9.2}

\begin{frame}[fragile]{}


Имея позиционную систему счисления, мы можем реализовать числовое
представление на её основе в виде последовательности
деревьев. Количество и размеры деревьев, представляющих коллекцию
размера $n$, определяются положением $n$ в позиционной системе
счисления. Для каждого веса $w_i$ имеются $b_i$ деревьев
соответствующего размера. Например, двоичное представление числа 73
выглядит как \texttt{1001001}, так что коллекция размера 73 в двоичном
числовом представлении будет содержать три дерева размеров 1, 8 и 64.

Как правило, деревья в числовых представлениях обладают весьма
регулярной структурой. Например, в двоичных числовых представлениях
все деревья имеют размер-степень двойки. Три часто встречающихся типа
деревьев с такой структурой~--- \term{полные двоичные листовые
  деревья}{complete binary leaf trees} \cite{KaldewaijDielissen1996}, \term{биномиальные
  деревья}{binomial trees} \cite{Vuillemin1978} и
\term{подвешенные деревья}{pennants} \cite{SackStrothotte1990}.

\end{frame}

\begin{frame}[fragile]{}

\begin{definition}
  \textbf{(Полные двоичные листовые деревья)} Полное двоичное листовое
  дерево ранга 0~--- это лист; полное двоичное листовое дерево ранга
  $r > 0$ представляет собой узел с двумя поддеревьями, каждое из
  которых является полным двоичным листовым деревом ранга $r -
  1$. Листовое дерево~--- это дерево, хранящее элементы только в
  листовых узлах, в отличие от обычных деревьев, где элементы
  содержатся в каждом узле. Полное двоичное дерево ранга $r$ содержит
  $2^{r+1} - 1$ узлов, но только $2^r$ листьев. Следовательно, полное
  двоичное листовое дерево ранга $r$ содержит $2^r$ элементов.
\end{definition}
\end{frame}

\begin{frame}[fragile]{}
\begin{definition}
  \textbf{(Биномиальные деревья)} Биномиальное дерево ранга $r$
  представляет собой узел с $r$ дочерними деревьями $c_1 \ldots c_r$,
  где каждое $c_i$ является биномиальным деревом ранга $r -
  i$. Можно также определить биномиальное дерево ранга $r > 0$ как
  биномиальное дерево ранга $r - 1$, к которому в качестве самого
  левого поддерева добавлено другое биномиальное дерево ранга $r -
  1$. Из второго определения легко видеть, что биномиальное дерево
  ранга $r$ содержит $2^r$ узлов.
\end{definition}

\end{frame}

\begin{frame}[fragile]{}
\begin{figure}
  \begin{subfigure}[b]{0.3\textwidth}
    \centering
    \begin{tikzpicture}[thick,scale=0.5, every node/.style={scale=0.5},level distance=1.7cm]
    \tikzstyle{marrs}=[very thick,-latex]
    \tikzstyle{tnode}=[circle, fill=black, inner sep=1.5mm]
    \tikzstyle{temp}=[inner sep=0mm]
    \def\rstep{5cm}
    
    \huge
    
    \tikzstyle{level 1}=[sibling distance = 4cm];
    \tikzstyle{level 2}=[sibling distance = 2cm];
    \tikzstyle{level 3}=[sibling distance = 1cm];
    
    
    \node[tnode] {}
        child foreach \xi in {1, 2} { node[tnode] {}
            child foreach \yi in {1, 2} { node[tnode] {}
                child foreach \zi in {1, 2} { node[tnode] {}
                    node[tnode] {}
                }
            }
        };
            
\end{tikzpicture}\par\vspace{0.2cm}
    (a)
  \end{subfigure}
  \begin{subfigure}[b]{0.3\textwidth}
    \centering
    \begin{tikzpicture}[thick,scale=0.5, every node/.style={scale=0.5},grow via three points={%
one child at (0,-1.7) and two children at (0,-1.7) and (-0.8,-1.7)}
]
    \tikzstyle{marrs}=[very thick,-latex]
    \tikzstyle{tnode}=[circle, fill=black, inner sep=1.5mm]
    \def\rstep{5cm}
    
    \huge
    
    \node[tnode] {}
        child {node[tnode] {} }
        child {node[tnode] {} 
            child {node[tnode] {} }
        }
        child {node[tnode] {} 
            child {node[tnode] {} }
            child {node[tnode] {} 
                child {node[tnode] {} }
            }
        };
        
\end{tikzpicture}
\par\vspace{0.2cm}
    (b)
  \end{subfigure}
  \begin{subfigure}[b]{0.3\textwidth}
    \centering
    \begin{tikzpicture}[thick,scale=0.5, every node/.style={scale=0.5},level distance=1.7cm]
    \tikzstyle{marrs}=[very thick,-latex]
    \tikzstyle{tnode}=[circle, fill=black, inner sep=1.5mm]
    \tikzstyle{temp}=[inner sep=0mm]
    \def\rstep{5cm}
    
    \huge
    
    \tikzstyle{level 1}=[sibling distance = 4cm];
    \tikzstyle{level 2}=[sibling distance = 2cm];
    \tikzstyle{level 3}=[sibling distance = 1cm];
    
    
    \node[tnode] {}
        child foreach \xi in {1} { node[tnode] {}
            child foreach \yi in {1, 2} { node[tnode] {}
                child foreach \zi in {1, 2} { node[tnode] {}
                    node[tnode] {}
                }
            }
        };
            
\end{tikzpicture}\par\vspace{0.2cm}
    (c)
  \end{subfigure}
  \caption{Три дерева ранга 3: (a) полное двоичное листовое дерево,
    (b) биномиальное дерево и (c) подвешенное дерево.}
  \label{fig:9.2}
\end{figure}
\begin{definition}
  \textbf{(Подвешенные деревья)} Подвешенное дерево ранга 0 представляет собой один узел, а
  подвешенное дерево ранга $r > 0$ представляет собой узел с единственным
  поддеревом~--- полным двоичным деревом ранга $r - 1$. Полное
  двоичное дерево содержит $2^r - 1$ элементов, так что подвешенное дерево
  содержит $2^r$ элементов.
\end{definition}

\end{frame}

%\begin{frame}[fragile]{}
%\begin{figure}
%  \begin{subfigure}[b]{0.3\textwidth}
%    \centering
%    \begin{tikzpicture}[thick,scale=0.5, every node/.style={scale=0.5},level distance=1.7cm]
    \tikzstyle{marrs}=[very thick,-latex]
    \tikzstyle{tnode}=[circle, fill=black, inner sep=1.5mm]
    \tikzstyle{temp}=[inner sep=0mm]
    \def\rstep{5cm}
    
    \huge
    
    \tikzstyle{level 1}=[sibling distance = 4cm];
    \tikzstyle{level 2}=[sibling distance = 2cm];
    \tikzstyle{level 3}=[sibling distance = 1cm];
    
    
    \node[tnode] {}
        child foreach \xi in {1, 2} { node[tnode] {}
            child foreach \yi in {1, 2} { node[tnode] {}
                child foreach \zi in {1, 2} { node[tnode] {}
                    node[tnode] {}
                }
            }
        };
            
\end{tikzpicture}\par\vspace{0.2cm}
%    (a)
%  \end{subfigure}
%  \begin{subfigure}[b]{0.3\textwidth}
%    \centering
%    \begin{tikzpicture}[thick,scale=0.5, every node/.style={scale=0.5},grow via three points={%
one child at (0,-1.7) and two children at (0,-1.7) and (-0.8,-1.7)}
]
    \tikzstyle{marrs}=[very thick,-latex]
    \tikzstyle{tnode}=[circle, fill=black, inner sep=1.5mm]
    \def\rstep{5cm}
    
    \huge
    
    \node[tnode] {}
        child {node[tnode] {} }
        child {node[tnode] {} 
            child {node[tnode] {} }
        }
        child {node[tnode] {} 
            child {node[tnode] {} }
            child {node[tnode] {} 
                child {node[tnode] {} }
            }
        };
        
\end{tikzpicture}
\par\vspace{0.2cm}
%    (b)
%  \end{subfigure}
%  \begin{subfigure}[b]{0.3\textwidth}
%    \centering
%    \begin{tikzpicture}[thick,scale=0.5, every node/.style={scale=0.5},level distance=1.7cm]
    \tikzstyle{marrs}=[very thick,-latex]
    \tikzstyle{tnode}=[circle, fill=black, inner sep=1.5mm]
    \tikzstyle{temp}=[inner sep=0mm]
    \def\rstep{5cm}
    
    \huge
    
    \tikzstyle{level 1}=[sibling distance = 4cm];
    \tikzstyle{level 2}=[sibling distance = 2cm];
    \tikzstyle{level 3}=[sibling distance = 1cm];
    
    
    \node[tnode] {}
        child foreach \xi in {1} { node[tnode] {}
            child foreach \yi in {1, 2} { node[tnode] {}
                child foreach \zi in {1, 2} { node[tnode] {}
                    node[tnode] {}
                }
            }
        };
            
\end{tikzpicture}\par\vspace{0.2cm}
%    (c)
%  \end{subfigure}
%  \caption{Три дерева ранга 3: (a) полное двоичное листовое дерево,
%    (b) биномиальное дерево и (c) подвешенное дерево.}
%  \label{fig:9.2}
%\end{figure}
%\end{frame}

\begin{frame}[fragile]{}
%Три этих разновидности деревьев показаны на Рис.~\ref{fig:9.2}.
 Выбор разновидности для каждой структуры данных
зависит от свойств, которыми эта структура должна обладать, например,
от порядка, в котором требуется хранить элементы в деревьях. Важным
вопросом при оценке соответствия разновидности деревьев для конкретной
структуры данных будет то, насколько хорошо данная разновидность
поддерживает функции, аналогичные переносу и занятию в двоичной
арифметике.\\

 При имитации переноса мы \term{связываем}{link} два дерева
ранга $r$ и получаем дерево ранга $r+1$. Аналогично, при имитации
занятия мы \term{развязываем}{unlink} дерево ранга $r > 0$ и получаем
два дерева ранга $r-1$. На Рис.~\ref{fig:9.3} показана операция
связывания (обозначенная $\oplus$)
для каждой из трех разновидностей деревьев.\\

Если мы предполагаем, что
элементы не переупорядочиваются, любая из разновидностей может быть
связана или развязана за время $O(1)$.

\end{frame}

\begin{frame}[fragile]{}

\begin{figure}
  \centering
  \begin{subfigure}[b]{0.45\textwidth}
    \centering
    
\begin{tikzpicture}[thick,scale=0.5, every node/.style={scale=0.5},level distance=1.3cm]
    
    \newcommand{\subtrg}[2]{\draw[thick] (#1.center) -- ++(0.85cm, -2.2cm) -- ++(-1.7cm, 0) -- cycle; \node[yshift=-1.8cm] at (#1.center) {#2}}

    
    \tikzstyle{tnode}=[circle, fill=black, inner sep=1.5mm]
    \tikzstyle{level 1}=[sibling distance=2.3cm]
    
    \huge
    
    \def\xstep{1.3}
    \coordinate (l) at (0,1.5) {};
    \coordinate (r) at (2 * \xstep,1.5) {};
    \node at (1 * \xstep, 0.4) {$\oplus$};
    \node at (3 * \xstep, 0.4) {$\rightarrow$};
    \coordinate (u) at (5 * \xstep, 2.8)
        child { coordinate (ul) }
        child { coordinate (ur) };
    
    \subtrg{l}{$r$};
    \subtrg{r}{$r$};
    \subtrg{ul}{$r$};
    \subtrg{ur}{$r$};
            
\end{tikzpicture}
\par\vspace{0.2cm}
    (a)
  \end{subfigure}
  \begin{subfigure}[b]{0.45\textwidth}
    \centering
    \begin{tikzpicture}[thick,scale=0.5, every node/.style={scale=0.5},grow via three points={%
one child at (-1.9cm,-1.9cm) and two children at (0,-1.5) and (-0.8,-1.5)}]
    \newcommand{\subtrg}[2]{\draw[thick] (#1.center) -- ++(0.85cm, -2.2cm) -- ++(-1.7cm, 0) -- cycle; \node[yshift=-1.8cm] at (#1.center) {#2}}
    \tikzstyle{tnode}=[circle, fill=black, inner sep=1.5mm]
    \tikzstyle{level 1}=[sibling distance=2.7cm]
    
    \huge
    
    \def\xstep{1.3}
    \node[tnode] (l) at (0,1.5) {};
    \node[tnode] (r) at (2 * \xstep,1.5) {};
    \node at (1 * \xstep, 0.4) {$\oplus$};
    \node at (3 * \xstep, 0.4) {$\rightarrow$};
    \node[tnode] (u) at (5.5 * \xstep, 3.4) {} 
        child { node[tnode] (ul) {} };
    
    \subtrg{l}{$r$};
    \subtrg{r}{$r$};
    \subtrg{u}{$r$};
    \subtrg{ul}{$r$};
            
\end{tikzpicture}
\par\vspace{0.2cm}
    (b)
  \end{subfigure}\par\vspace{0.2cm}
  \begin{subfigure}[b]{1\textwidth}
    \centering
    \begin{tikzpicture}[thick,scale=0.5, every node/.style={scale=0.5},level distance=1.3cm]
    \newcommand{\subtrg}[2]{\draw[thick] (#1.center) -- ++(0.85cm, -2.2cm) -- ++(-1.7cm, 0) -- cycle; \node[yshift=-1.8cm] at (#1.center) {#2}}
    \tikzstyle{tnode}=[circle, fill=black, inner sep=1.5mm]
    \tikzstyle{level 2}=[sibling distance=2.3cm]
    
    \huge
    
    \def\xstep{1.3}
    \node[tnode] at (0, 1.5 + 1.3) {} child {
      node[tnode] (l) {}
    };
    \node[tnode] at (2 * \xstep, 1.5 + 1.3) {} child {
      node[tnode] (r) {}
    };

    \node at (\xstep, 0.4) {$\oplus$};
    \node at (3 * \xstep, 0.4) {$\rightarrow$};
    \node[tnode] (u1) at (5 * \xstep, 1.5 + 1.3 * 2) {} 
        child { node[tnode] (u2) {} 
            child { node[tnode] (ul) {} }
            child { node[tnode] (ur) {} }
        };
    
    \subtrg{l}{r$-$1};
    \subtrg{r}{r$-$1};
    \subtrg{ur}{r$-$1};
    \subtrg{ul}{r$-$1};
            
\end{tikzpicture}
\par\vspace{0.2cm}
    (c)
  \end{subfigure}

  \caption{Связывание двух деревьев ранга $r$ в дерево ранга $r+1$ для
    (a) полных двоичных листовых деревьев, (b) биномиальных деревьев и
    (c) подвешенных деревьев.}
  \label{fig:9.3}
\end{figure}

\end{frame}

\begin{comment}
В предыдущих главах мы уже видели несколько реализаций куч,
основанных на двоичной арифметике и биномиальных деревьях. Теперь мы
сначала рассмотрим простое числовое представление для списков с
произвольным доступом. Затем мы исследуем насколько вариаций двоичной
арифметики, позволяющих улучшить асимптотические показатели.
\end{comment}

\subsection{Двоичные списки с произвольным доступом}
\label{sc:9.2.1}

\begin{frame}[fragile]{\term{Список с произвольным доступом}{random access list}}
Он же односторонний гибкий массив~--- это структура данных,
поддерживающая, подобно массиву, функции доступа и модификации любого
элемента, а также обыкновенные функции для списков: \hsinline{cons},
\hsinline{head} и \hsinline{tail}. 
%Сигнатура списков с произвольным доступом приведена на Рис.~\ref{fig:9.4}.

%\begin{figure}
%  \centering
%
%  \caption{Сигнатура списков с произвольным доступом.}
%  \label{fig:9.4}
%\end{figure}
\inputminted[firstline=4]{haskell}{code/RandomAccessList.lhs}
\end{frame}

\begin{frame}[fragile]{}
Мы реализуем списки с произвольным доступом, используя двоичное
числовое представление. Двоичный список с произвольным доступом
размера $n$ содержит по дереву на каждую единицу в двоичном
представлении $n$. Ранг каждого дерева соответствует рангу
соответствующей цифры; если $i$-й бит $n$ равен единице, то список с
произвольным доступом содержит дерево размера $2^i$. Мы можем
использовать любую из трех разновидностей деревьев и либо плотное,
либо разреженное представление. Для этого примера мы используем
простейшее сочетание: полные двоичные листовые деревья и плотное
представление. Таким образом, тип \hsinline{BinaryList} выглядит так:

\inputminted[firstline=7,lastline=9]{haskell}{code/BinaryRandomAccessList.lhs}
\end{frame}

\begin{frame}[fragile]{}

Целое число в каждой вершине~--- размер дерева. 

Это число
избыточно, поскольку размер каждого дерева полностью определяется
размером его родителя или позицией в списке цифр, но мы всё равно его
храним ради удобства. Деревья хранятся в порядке возрастания размера,
а порядок элементов~--- слева направо, как внутри, так и между
деревьями. Таким образом, головой списка с произвольным доступом
является самый левый лист наименьшего дерева. 
\end{frame}

\begin{frame}[fragile]{}
На Рис.~\ref{fig:9.5}
показан двоичный список с произвольным доступом размера 7. Заметим,
что максимальное число деревьев в списке размера $n$ равно
$\lfloor \log (n+1) \rfloor$, а максимальная глубина дерева равна
$\lfloor \log n \rfloor$.



\begin{figure}
  \centering
  \begin{tikzpicture}[thick,scale=0.5, every node/.style={scale=0.5},level distance=1.3cm]
    \tikzstyle{marrs}=[very thick,-latex]
    \tikzstyle{tnode}=[circle, fill=black, inner sep=1.5mm]
    \tikzstyle{tempty}=[inner sep=0mm, line width=0mm, draw=black, circle,fill=black]
    \def\rstep{5cm}
    
    \huge   
    
    \tikzstyle{level 1}=[sibling distance = 4cm];
    \tikzstyle{level 2}=[sibling distance = 2cm];
    \tikzstyle{level 3}=[sibling distance = 1cm];
    
    \node[tnode] at (0, 0) {};
    \node at (1, 0) {\bf ,};
    {
        \tikzstyle{level 1}=[sibling distance = 1cm];
        \node[tempty] at (2.5, 1.3) {} 
            child foreach \x in {1,2} { node[tnode] {} }
        ;    
    }
    \node at (4, 0) {\bf ,};
    {
        \tikzstyle{level 1}=[sibling distance = 2cm];
        \tikzstyle{level 2}=[sibling distance = 1cm];
        \node[tempty] at (6.5, 2.6) {} 
            child foreach \x in {1,2} { 
                child foreach \x in {1,2} { 
                    node[tnode] {}
                }
            }
        ;    
    }
    \newcommand{\noop}{}
    \foreach \x/\v in {0/0, 1/\noop, 2/1, 3/2, 4/\noop, 5/3, 6/4, 7/5, 8/6} {
        \node at (\x, -0.7) {$\v$};
    }

    \draw[very thick] (-0.8, -0.2) -- ++(-0.2, 0) -- ++(0, 3.2) -- ++(0.2, 0);
    \draw[very thick] (8.8, -0.2) -- ++(0.2, 0) -- ++(0, 3.2) -- ++(-0.2, 0);
    
            
\end{tikzpicture}
  \caption{Двоичный список с произвольным доступом, содержащий элементы 0\ldots 6.}
  \label{fig:9.5}
\end{figure}
\end{frame}

\begin{frame}[fragile]{}
Вставка элемента в двоичный список с произвольным доступом (при помощи
\lstinline!cons!) аналогична увеличению двоичного числа на
единицу. Напомним функцию увеличения для двоичных чисел:

\inputminted[firstline=5,lastline=7]{haskell}{code/DenseNumbers.hs}


Чтобы добавить новый элемент к началу списка, мы сначала преобразуем
его в лист, а затем вставляем его в список деревьев с помощью
вспомогательной функции \lstinline!consTree!, которая следует образцу
\lstinline!inc!.

\inputminted[firstline=30,lastline=30,gobble=2]{haskell}{code/BinaryRandomAccessList.lhs}
\inputminted[firstline=16,lastline=18]{haskell}{code/BinaryRandomAccessList.lhs}

Вспомогательная функция \lstinline!link! порождает новое дерево из двух
поддеревьев одинакового размера и автоматически вычисляет его размер.

\inputminted[firstline=14]{haskell}{code/BinaryRandomAccessList.lhs}
\end{frame}

\begin{frame}[fragile]{}
Уничтожение элемента в двоичном списке с произвольным доступом (при
помощи \lstinline!tail!) аналогично уменьшению двоичного числа на
единицу. Напомним функцию уменьшения для плотных двоичных чисел:
\inputminted[firstline=8,lastline=10]{haskell}{code/DenseNumbers.hs}

Соответствующая функция для списков деревьев называется
\lstinline!unconsTree!. Будучи примененной к списку, чья первая цифра
имеет ранг $r$, \lstinline!unconsTree! возвращает пару, состоящую из
дерева ранга $r$ и нового списка без этого дерева.
\inputminted[firstline=20,lastline=24]{haskell}{code/BinaryRandomAccessList.lhs}
\end{frame}

\begin{frame}[fragile]{}
Функции \lstinline!head! и \lstinline!tail!  удаляют самый левый
элемент при помощи \lstinline!unconsTree!, а затем, соответственно,
либо возвращают этот элемент, либо отбрасывают.
\inputminted[firstline=31,lastline=32]{haskell}{code/BinaryRandomAccessList.lhs}

Функции \lstinline!lookup! и \lstinline!update! не соответствуют
никаким арифметическим операциям. Они просто пользуются организацией
двоичных списков произвольного доступа в виде списков логарифмической
длины, состоящих из деревьев логарифмической глубины. 

\end{frame}

\begin{frame}[fragile]{}

Поиск элемента
состоит из двух этапов. Сначала в списке мы ищем нужное дерево, а
затем в этом дереве ищем требуемый элемент. Вспомогательная функция
\lstinline!lookupTree! использует поле размера в каждом узле, чтобы
определить, находится ли $i$-й элемент в левом или правом
поддереве.
\inputminted[firstline=34,lastline=44,gobble=2]{haskell}{code/BinaryRandomAccessList.lhs}
\end{frame}

\begin{frame}[fragile]{}
\lstinline!update! действует аналогично, но вдобавок копирует путь от
корня до обновляемого листа.
\inputminted[firstline=48,lastline=60,gobble=2]{haskell}{code/BinaryRandomAccessList.lhs}

%Полный код этой реализации приведен на Рис.~\ref{fig:9.6}.

%\begin{figure}
%  \centering
%
%  \caption{Двоичные списки с произвольным доступом.}
%  \label{fig:9.6}
%\end{figure}
\end{frame}

\begin{frame}[fragile]{}

Функции \lstinline!cons!, \lstinline!head! и \lstinline!tail!
производят не более $O(1)$ работы на цифру, так что общее время их
работы $O(\log n)$ в худшем случае. \lstinline!lookup! и
\lstinline!update! требуют не более $O(\log n)$ времени на поиск
нужного дерева, а затем не более $O(\log n)$ времени на поиск нужного
элемента в этом дереве, так что общее время их работы также $O(\log
n)$ в худшем случае.

\end{frame}

\ifanswers
\begin{frame}[fragile]{}
\begin{exercise}\label{ex:9.1}
  Напишите функцию \lstinline!drop! типа
  \lstinline!int $\times$ $\alpha$ RList $\to$ $\alpha$ RList!, уничтожающую первые $k$
  элементов двоичного списка с произвольным доступом. Функция должна
  работать за время $O(\log n)$.
\end{exercise}

\begin{exercise}\label{ex:9.2}
  Напишите функцию \lstinline!create! типа
  \lstinline!int $\times$ $\alpha$ $\to$ $\alpha$ RList!, которая создает
  двоичный список с произвольным доступом, содержащий $n$ копий
  некоторого значения $x$. Функция также должна работать за время
  $O(\log n)$. (Может оказаться полезным вспомнить Упражнение~\ref{ex:2.5}.)
\end{exercise}

\begin{exercise}\label{ex:9.3}
  Реализуйте \lstinline!BinaryRandomAccessList! заново, используя
  разреженное представление
  \begin{lstlisting}
    datatype $\alpha$ Tree = Leaf of $\alpha$ | Node of int $\times$ $\alpha$ Tree $\times$ $\alpha$ Tree
    type $\alpha$ RList = $\alpha$ Tree list
  \end{lstlisting}
\end{exercise}
\end{frame}
\fi 

\subsection{Безнулевые представления}
\label{sc:9.2.2}

\begin{frame}[fragile]{}

В двоичных списках с произвольным доступом разочаровывает то, что
списковые функции \lstinline!cons!, \lstinline!head! и
\lstinline!tail! требуют $O(\log n)$ времени вместо $O(1)$. В
следующих трех подразделах мы исследуем варианты двоичных чисел,
улучшающие время работы всех трех функций до $O(1)$. В этом подразделе
мы начинаем с функции \lstinline!head!.

%\begin{remark}
%  Очевидное решение, позволяющее \lstinline!head! выполняться за время
%  $O(1)$~--- хранить первый элемент отдельно от остального списка,
%  подобно функтору \lstinline!ExplicitMin! из
%  Упражнения~\ref{ex:3.7}. Другое решение~--- использовать разреженное
%  представление и либо биномиальные деревья, либо подвешенные деревья, так что
%  головой списка будет корень первого дерева. Решение, которое мы
%  исследуем в этом подразделе, хорошо тем, что оно также немного
%  улучшает время работы \lstinline!lookup! и \lstinline!update!.
%\end{remark}
\end{frame}

\begin{frame}[fragile]{}
Сейчас \lstinline!head! у нас реализована через вызов
\lstinline!unconsTree!, которая выделяет первый элемент, а также
перестраивает список без этого элемента. При таком подходе мы получаем
компактный код, поскольку \lstinline!unconsTree! поддерживает как
\lstinline!head!, так и \lstinline!tail!, но теряется время
на построение списков, не используемых функцией
\lstinline!head!. Ради большей эффективности имеет смысл реализовать
\lstinline!head! напрямую. В качестве особого случая, легко заставить
\lstinline!head! работать за время $O(1)$, когда первая цифра не ноль.
\begin{lstlisting}
  fun head (One (Leaf x) :: _) = x
\end{lstlisting}
Вдохновленные этим правилом, мы хотели бы устроить так, чтобы первая
цифра \emph{никогда} не была нулем. Есть множество простых трюков,
достигающих именно этого, но более красивым решением будет
использовать \term{безнулевое}{zeroless} представление, где ни одна
цифра не равна нулю.
\end{frame}

\begin{frame}[fragile]{}
Безнулевые двоичные числа строятся из единиц и двоек, а не из единиц и
нулей. Вес $i$-й цифры по-прежнему равен $2^i$. Так, например,
десятичное число 16 можно записать как \texttt{2111} вместо
\texttt{00001}. Функция добавления единицы на безнулевых двоичных
числах реализуется так:
\inputminted[firstline=3,lastline=8]{haskell}{code/ZerolessNumbers.hs}

\begin{exercise}\label{ex:9.4}
  Напишите функции уменьшения на единицу и сложения для безнулевых
  двоичных чисел. Заметим, что переноситься при сложении может как
  единица, так и двойка.
\end{exercise}
\end{frame}

\begin{frame}[fragile]{}
Теперь если мы заменим тип цифр в двоичных списках с произвольным
доступом на
\begin{minted}{haskell}
  datatype Digit a = One of (Tree a) | Two of (Tree a) (Tree a)
\end{minted}
то можем реализовать \lstinline!head! как
\begin{minted}{haskell}
head (One (Leaf x) : _) = x
head (Two (Leaf x) (Leaf y) : _) = x
\end{minted}
Ясно, что эта функция работает за время $O(1)$.

\begin{exercise}\label{ex:9.5}
  Реализуйте оставшиеся функции для этого типа.
\end{exercise}

\end{frame}

\ifanswers
\begin{frame}[fragile]{}
\begin{exercise}\label{ex:9.6}
  Покажите, что теперь функции \lstinline!lookup! и
  \lstinline!update!, примененные к элементу $i$, работают за время
  $O(\log i)$.
\end{exercise}

\begin{exercise}\label{ex:9.7}
  При некоторых дополнительных условиях красно-черные деревья
  (Раздел~\ref{sc:3.3}) можно рассматривать как числовое
  представление. Сопоставьте безнулевые двоичные списки с произвольным
  доступом и красно-черные деревья, в которых вставка разрешена только
  в самую левую позицию. Обратите особое внимание на функции
  \lstinline!cons! и \lstinline!insert!, а также на инварианты формы
  структур, порождаемых этими функциями.
\end{exercise}
\end{frame}
\fi


\subsection{Ленивые представления}
\label{sc:9.2.3}

\begin{frame}[fragile]{Ленивые представления}

Допустим, мы представляем двоичные числа как потоки цифр, а не списки. \\

Тогда функция увеличения на единицу \hsinline{inc} будет работать за $O(1)$ амортизированного времени. Доказать можно, например, методом банкира.

\begin{comment}
Заметим, что функция эта пошаговая.

В Разделе~\ref{sc:6.4.1} мы видели, как с помощью ленивого вычисления
можно заставить вставку в биномиальные кучи работать за
амортизированное время $O(1)$, так что нас не должно удивлять, что
наша новая версия \lstinline!inc! также работает за амортизированное
время $O(1)$. Мы доказываем это по методу банкира.

\emph{Доказательство.} Пусть каждая цифра ноль несет одну единицу долга, а
цифра единица~--- ноль единиц долга. Допустим, \lstinline!ds!
начинается с $k$ единиц (\lstinline!One!), а затем имеет ноль
(\lstinline!Zero!). Тогда \lstinline!inc ds! заменяет все эти \lstinline!One!
на \lstinline!Zero!, а \lstinline!Zero! на \lstinline!One!.
Выделим по одной единице долга на каждый
из этих шагов. Теперь у каждого элемента \lstinline!Zero! есть одна
единица долга, а у \lstinline!One! две: одна, унаследованная от
исходной задержки в этом месте, и одна, созданная только
что. Высвобождение этих двух единиц долга восстанавливает
инвариант. Поскольку амортизированная стоимость функции равна ее
нераздельной стоимости (здесь это $O(1)$) плюс число высвобождаемых
единиц долга (здесь две), \lstinline!inc! работает за амортизированное
время $O(1)$.

Рассмотрим теперь функцию уменьшения.
\begin{lstlisting}
  fun lazy dec ($\$$Cons (One, $\$$Nil)) = $\$$Nil
         | dec ($\$$Cons (One, ds)) = $\$$Cons (Zero, ds)
         | dec ($\$$Cons (Zero, ds)) = $\$$Cons (One, dec ds)
\end{lstlisting}
Поскольку эта функция подобна \lstinline!inc!, но
со сменой ролей цифр, можно ожидать, что при помощи подобного
доказательства мы получим такое же ограничение. Так оно и есть, если
мы не используем \emph{обе} функции. Однако если используются как
\lstinline!inc!, так и \lstinline!dec!, по крайней мере одной из них
приходится приписывать амортизированное время $O(\log n)$. Чтобы понять,
почему, представим последовательность увеличений и уменьшений,
циклически переходящих от $2^k - 1$ к $2^k$ и обратно. Каждая операция
при этом затрагивает каждую цифру, и общее время получается $O(\log
n)$.

Но разве мы не доказали, что амортизированное время каждой из функций
$O(1)$? Что здесь неверно? Проблема в том, что эти два доказательства
требуют конфликтующих инвариантов долга. Чтобы доказать, что
\lstinline!inc! работает за амортизированное время $O(1)$, мы
требовали, чтобы каждому \lstinline!Zero! приписывалась одна единица
долга, а каждому \lstinline!One! ноль единиц. При доказательстве, что
\lstinline!dec! работает за амортизированное время $O(1)$, мы
приписывали одну единицу долга каждому \lstinline!One! и ноль единиц
каждому \lstinline!Zero!.

Главное свойство, которое как \lstinline!inc!, так и \lstinline!dec!
по отдельности имеют, состоит в том, что по крайней мере половина
операций, достигших какой-то позиции, на этой позиции
останавливаются. А именно, каждый вызов \lstinline!inc! или
\lstinline!dec! обрабатывает первую цифру, но только один вызов из
двух затрагивает вторую. Третью цифру обрабатывает один вызов из
четырех, и так далее. На интуитивном уровне, амортизированная
стоимость каждой операции получается
$$
O(1 + 1/2 + 1/4 + 1/8 + \ldots) = O(1)
$$
Разделим возможные цифры-заполнители каждой позиции на
\term{безопасные}{safe} и \term{опасные}{dangerous}: функция,
достигшая безопасной цифры, всегда на ней и завершается, а функция,
добравшаяся до опасной цифры, может проследовать к следующей
позиции. Чтобы доказать, что из двух последовательных операций никогда
обе не добираются до следующей позиции, нам нужно показать, что каждый
раз, когда операция обрабатывает опасную цифру, она заменяет её на
безопасную. Тогда следующая операция, которая доберется до данной
позиции, на ней и остановится. Формально мы доказываем, что каждая
операция работает за амортизированное время $O(1)$, устанавливая
инвариант долга, где каждой безопасной цифре приписывается одна
единица долга, а опасной ноль.

Функция увеличения требует считать опасной самую большую цифру, а
функция уменьшения считает опасной самую маленькую цифру. Чтобы
поддержать их обе, нам нужна третья безопасная цифра. Таким образом,
мы переключаемся на \term{избыточные}{redundant} двоичные числа, где
каждая цифра может быть нулем, единицей или двойкой. Тогда
\lstinline!inc! и \lstinline!dec! реализуются следующим образом:
\begin{lstlisting}
  datatype Digit = Zero | One | Two
  type Nat = Digit Stream

  fun lazy inc ($\$$Nil) = $\$$Cons (One, $\$$Nil)
         | inc ($\$$Cons (Zero, ds)) = $\$$Cons (One, ds)
         | inc ($\$$Cons (One, ds)) = $\$$Cons (Two, ds)
         | inc ($\$$Cons (Two, ds)) = $\$$Cons (One, inc ds)

  fun lazy dec ($\$$Cons (One, $\$$Nil) = $\$$Nil
         | dec ($\$$Cons (One, ds)) = $\$$Cons (Zero, ds)
         | dec ($\$$Cons (Two, ds)) = $\$$Cons (One, ds)
         | dec ($\$$Cons (Zero, ds)) = $\$$Cons (One, dec ds)
\end{lstlisting}
Обратите внимание, что рекурсивные предложения в \lstinline!inc! и
\lstinline!dec!~--- для \lstinline!Two! (двойки) и \lstinline!Zero! (ноля),
соответственно~--- оба порождают \lstinline!One! (единицу). При этом
\lstinline!One!~--- безопасная цифра, а \lstinline!Zero! и
\lstinline!Two!~--- опасные. Чтобы увидеть, как нам здесь помогает
избыточность, рассмотрим, как работает увеличение на единицу двоичного
числа \texttt{222222}, дающее \texttt{1111111}. Для этой операции
требуется семь шагов. Однако уменьшение этого значения не дает снова
\texttt{222222}, Вместо этого мы всего за один шаг получаем
\texttt{0111111}. Таким образом, чередование увеличений и уменьшений
больше не является проблемой.

Ленивые двоичные числа могут служить моделью для построения многих других
структур данных. В Главе~\ref{ch:11} мы обобщим эту модель и получим
метод проектирования под названием \term{неявное рекурсивное
  замедление}{implicit recursive slowdown}.
\end{comment}
\end{frame}

\ifanswers
\begin{frame}[fragile]{}
\begin{exercise}\label{ex:9.8}
  Докажите, что как \lstinline!inc!, так и \lstinline!dec! работают за
  амортизированное время $O(1)$ с помощью инварианта долга,
  присваивающего одну единицу долга цифре \lstinline!One! и ноль цифрам
  \lstinline!Zero! и \lstinline!Two!.
\end{exercise}

\begin{exercise}\label{ex:9.9}
  Реализуйте \lstinline!cons!, \lstinline!head! и \lstinline!tail! для
  списков с произвольным доступом на основе безнулевых избыточных
  двоичных чисел, используя тип
  \begin{lstlisting}
    datatype $\alpha$ Digit =
           One of $\alpha$ Tree
         | Two of $\alpha$ Tree $\times$ $\alpha$ Tree
         | Three of $\alpha$ Tree $\times$ $\alpha$ Tree $\times$ $\alpha$ Tree
    type $\alpha$ RList = Digit Stream
  \end{lstlisting}
  Покажите, что все три функции работают за время $O(1)$.
\end{exercise}

\begin{exercise}\label{ex:9.10}
  Как показано в Разделе~\ref{sc:7.3} на биномиальных кучах с
  расписаниями, можно снабдить ленивые двоичные числа расписаниями и
  получить ограничение $O(1)$ в худшем случае. Заново реализуйте
  \lstinline!cons!, \lstinline!head! и \lstinline!tail! из предыдущего
  упражнения, так, чтобы они работали за время $O(1)$ в худшем
  случае. Может оказаться полезным иметь два различных конструктора
  для цифры <<два>> (скажем, \lstinline!Two! и \lstinline!Two'!),
  чтобы различать рекурсивные и нерекурсивные варианты вызова \lstinline!cons!
  и \lstinline!tail!.
\end{exercise}
\end{frame}
\fi

\subsection{Сегментированные представления}
\label{sc:9.2.4}

\begin{frame}[fragile]{\term{Cегментированные}{segmented} двоичные
  числа}


Ещё одна разновидность двоичных чисел, дающая показатели $O(1)$ в
худшем случае.\\

 Проблема с обычными двоичными числами состоит в том, что
переносы и занятия могут происходить каскадом. Например, увеличение
$2^k - 1$ приводит в двоичной арифметике к $k$ переносам. Аналогично,
уменьшение $2^k$ ведет к $k$ занятиям. Сегментированные двоичные числа
решают эту проблему, позволяя нескольким переносам или занятиям
выполняться за один шаг.\\

\end{frame}

\begin{frame}[fragile]{}

Заметим, что увеличение двоичного числа требует $k$ шагов, когда число
начинается с последовательности в $k$ единиц. Подобным образом,
уменьшение двоичного числа требует $k$ шагов, когда число начинается
с $k$ нулей. Сегментированные двоичные числа объединяют непрерывные
последовательности одинаковых цифр в блоки, так что мы можем применить
перенос или занятие к целому блоку за один шаг. Мы представляем
сегментированные двоичные числа как список чередующихся блоков из
единиц и нулей согласно следующему объявлению типа:
\begin{minted}{haskell}
  data DigitBlock = Zeros Int | Ones Int
  type Nat = [DigitBlock]
\end{minted}
Целое число в каждом DigitBlock представляет длину блока.

\end{frame}

\begin{frame}[fragile]{}
Хотя сегментированные числа позволяют реализовать \hsinline{inc} и \hsinline{dec} за 
$O(1)$ в худшем случае
\begin{itemize}
  \item Реализация сложная
  \item Идеи плохо переносятся на деревья
\end{itemize}
\end{frame}

\begin{comment}

\begin{frame}[fragile]{}
Мы добавляем новые блоки к началу списка блоков с помощью
вспомогательных функций
\lstinline!zeros! (нули) и \lstinline!ones! (единицы). Эти функции
сливают идущие подряд блоки одинаковых цифр и отбрасывают
пустые блоки.  Кроме того, \lstinline!zeros! отбрасывает нули в конце
записи числа.
\begin{lstlisting}
  fun zeros (i, []) = []
    | zeros (0, blks) = blks
    | zeros (i, Zeros j :: blks) = Zeros (i+j) :: blks
    | zeros (i, blks) = Zeros i :: blks

  fun ones (0, blks) = blks
    | ones (i, Ones j :: blks) = Ones (i+j) :: blks
    | ones (i, blks) = Ones i :: blks
\end{lstlisting}
Теперь при увеличении сегментированного двоичного числа мы смотрим на
первый блок цифр (если он вообще есть). Если первый блок содержит
нули, то мы заменяем первый из этих нулей на единицу, создавая новый
единичный блок единиц, а длину блока нулей уменьшая на один. Если же
первый блок содержит $i$ единиц, то мы за один шаг проделываем $i$
переносов, заменяя единицы на нули и увеличивая следующую цифру.
\begin{lstlisting}
  fun inc [] = [Ones 1]
    | inc (Zeros i :: blks) = ones (1, zeros (i-1, blks))
    | inc (Ones i :: blks) = Zeros i :: inc blks
\end{lstlisting}
В третьей строке функции мы знаем, что рекурсивный вызов
\lstinline!inc! не зациклится, поскольку если следующий блок
присутствует, он будет содержать нули. Во второй строке
вспомогательная функция позаботится об особом случае, когда первый
блок содержит единственный ноль.

Уменьшение сегментированного двоичного числа выглядит почти точно так
же, только роли единиц и нулей меняются.
\begin{lstlisting}
  fun dec (Ones i :: blks) = zeros (1, ones (i-1, blks))
    | dec (Zeros i :: blks) = Ones i :: dec blks
\end{lstlisting}
Здесь мы тоже знаем, что рекурсивный вызов не зациклится, потому что в
следующем блоке должны быть единицы.

К сожалению, несмотря на то, что сегментированные двоичные числа
поддерживают операции \lstinline!inc! и \lstinline!dec! за время
$O(1)$ в худшем случае, числовые представления, основанные на них,
оказываются слишком сложными, чтобы иметь какое-либо практическое
значение. Проблема заключается в том, что понятие перевода целого
блока единиц в нули и наоборот плохо переводится на язык операций с
деревьями. Более практичные решения, однако, можно получить, если
сочетать сегментацию с избыточными двоичными числами. При этом мы
можем снова обрабатывать цифры (а следовательно, и деревья) по
одной. Сегментация позволяет нам обрабатывать цифры в середине
последовательности, а не только в начале.

Рассмотрим, например, избыточное представление, в котором блоки единиц
рассматриваются как единый сегмент.
\begin{lstlisting}
  datatype Digits = Zero | Ones of int | Two
  type Nat = Digits list
\end{lstlisting}
Определяем вспомогательную функцию \lstinline!ones!, обрабатывающую
блоки, идущие друг за другом, и уничтожающую пустые блоки.
\begin{lstlisting}
  fun ones (0, ds) = ds
    | ones (i, Ones j :: ds) = Ones (i+j) :: ds
    | ones (i, ds) = Ones i :: ds
\end{lstlisting}
Полезно рассматривать цифру \lstinline!Two! (два) как незаконченный
перенос. Чтобы не было каскадов переносов, нам надо гарантировать, что
две двойки никогда не идут подряд. Будем поддерживать инвариант, что
последняя не равная единице цифра перед каждой двойкой равна
нулю. Этот инвариант можно записать как регулярное выражение
$\mathtt{(0|1|01^*2)^*}$ либо, если ещё учесть отсутствие нулей в
конце, $\mathtt{(0^*1 | 0^+1^*2)^*}$. Заметим, что двойка никогда не
является первой цифрой. Таким образом, мы можем увеличить число на
единицу за время $O(1)$ в худшем случае, просто увеличивая первую
цифру.
\begin{lstlisting}
  fun simpleInc [] = [Ones 1]
    | simpleInc (Zero :: ds) = ones (1, ds)
    | simpleInc (Ones i :: ds) = Two :: ones (i-1, ds)
\end{lstlisting}
В третьей строке инвариант нарушается очевидным образом, поскольку
\lstinline!Two! оказывается в начале. Кроме этого, инвариант может
быть нарушен во второй строке, если первая не равная единице цифра
равна двойке. Мы восстанавливаем инвариант при помощи вспомогательной
функции \lstinline!fixup!, проверяющей, не является ли первая не
равная единице цифра двойкой. Если это так, \lstinline!fixup! заменяет
двойку на ноль и увеличивает следующую цифру, которая, в свою очередь,
двойкой быть не может.
\begin{lstlisting}
  fun fixup (Two :: ds) = Zero :: simpleInc ds
    | fixup (Ones i :: Two :: ds) = Ones i :: Zero :: fixup ds
    | fixup ds = ds
\end{lstlisting}
Во второй строке \lstinline!fixup! мы пользуемся тем, что представление
сегментировано, проскакивая блок единиц, за которыми следует
двойка. Наконец, \lstinline!inc! зовет сначала \lstinline!simpleInc!,
затем \lstinline!fixup!.
\begin{lstlisting}
  fun inc ds = fixup (simpleInc ds)
\end{lstlisting}

\end{frame}

\begin{frame}[fragile]{}

Эта реализация может служить образцом для многих других структур
данных. Такая структура представляет собой последовательность уровней,
каждый из которых имеет признак \emph{зелёный}, \emph{жёлтый} или
\emph{красный}. Каждый уровень
соответствует цифре в вышеописанной реализации. Зелёный соответствует
нулю-\lstinline!Zero!, жёлтый единице-\lstinline!One!, а красный
двойке-\lstinline!Two!. Операция над любым объектом может перекрасить
первый уровень из зеленого в жёлтый или из жёлтого в красный, но
никогда из зелёного в красный. Инвариант состоит в том, что первый
не-жёлтый уровень перед красным всегда зелёный. Процедура
восстановления инварианта проверяет, не является ли первый не-жёлтый
уровень красным и, если да, переводит этот уровень из красного в
зелёный и, возможно, ухудшает цвет следующего уровня из зелёного в
жёлтый или из жёлтого в красный. Последовательные жёлтые уровни
собираются в блок, чтобы облегчить доступ к первому не-жёлтому. Каплан
и Тарьян \cite{KaplanTarjan1995} называют эту общую методику
\term{рекурсивное замедление}{recursive slowdown}.

\begin{exercise}\label{ex:9.11}
  Добавьте сегментацию к биномиальным кучам, чтобы операция
  \lstinline!insert! работала за время $O(1)$ в худшем
  случае. Используйте тип
  \begin{lstlisting}
    datatype Tree = Node of Elem.T $\times$ Tree list
    datatype Digit = Zero | Ones of Tree list | Two of Tree $\times$ Tree
    type Heap = Digit list
  \end{lstlisting}
  Восстанавливайте инвариант после слияния куч, уничтожая все цифры \lstinline!Two!.
\end{exercise}

\begin{exercise}\label{ex:9.12}
  Пример реализации двоичных чисел на основе рекурсивного замедления
  поддерживает операцию \lstinline!inc! за время $O(1)$ в худшем
  случае, но для \lstinline!dec! может потребоваться $O(\log
  n)$. Реализуйте сегментированные избыточные двоичные числа,
  поддерживающие как \lstinline!inc!, так и \lstinline!dec! за время
  $O(1)$ в худшем случае, с цифрами \texttt{0}, \texttt{1},
  \texttt{2}, \texttt{3} и \texttt{4}, причем \texttt{0} и \texttt{4}
  красные, \texttt{1} и \texttt{3} жёлтые, а \texttt{2} зелёная.
\end{exercise}

\begin{exercise}\label{ex:9.13}
  Реализуйте \lstinline!cons!, \lstinline!head!, \lstinline!tail! и
  \lstinline!lookup! для числового представления списков с
  произвольным доступом на основе системы счисления из предыдущего
  упражнения. Ваше представление должно поддерживать \lstinline!cons!,
  \lstinline!head! и \lstinline!tail! за время $O(1)$ в худшем случае,
  а \lstinline!lookup! за время $O(\log n)$ в худшем случае.
\end{exercise}

\end{frame}
\end{comment}

\section{Скошенные двоичные числа}
\label{sc:9.3}

\begin{frame}[fragile]{}



При помощи ленивых двоичных чисел и сегментированных двоичных чисел мы
получили два метода улучшения асимптотического поведения функций
увеличения на единицу и уменьшения на единицу с $O(\log n)$ до
$O(1)$. \\


Рассмотрим третий метод; на практике он
обычно приводит к более простым и быстрым программам, однако этот
метод связан с более радикальным отходом от обыкновенных двоичных
чисел. \\

\end{frame}

\begin{frame}[fragile]{Скошенные двоичные числа (skew binary numbers)}

В \term{скошенных двоичных числах}{skew binary numbers}
\cite{Myers1983, Okasaki1995b} вес
$i$-й цифры $w_i$ равен не $2^i$, как в обыкновенных двоичных числах,
а $2^{(i+1)} - 1$. Используются цифры ноль, один и два (т.~е., $D_i =
\{\mathtt{0}, \mathtt{1}, \mathtt{2}\}$). Например, десятичное число
92 можно записать как \texttt{002101} (начиная с наименее значимой
цифры).\\

\begin{align*} 
002101_{skew} & =& (2^1-1)*0 + (2^2-1)*0 + (2^3-1)*2 + \\ 
              &  & + (2^4-1)*1 + (2^5-1)*0 + (2^6-1)*1 \\
& = &1*0 + 3*0 + 7*2 + 15*1 + 31*0 + 63*1              \\
& = &92
\end{align*}



\end{frame}

\begin{frame}[fragile]{}

Эта система счисления избыточна, однако мы можем вернуть уникальность
представления, если введём дополнительное требование, что лишь
самая младшая ненулевая цифра может быть двойкой.  Будем говорить, что
такое число записано в \term{каноническом виде}{canonical
  form}. Начиная с этого момента, будем предполагать, что все
скошенные двоичные числа записаны в каноническом виде.


\begin{theorem}\label{th:9.1}
  (Майерс \cite{Myers1983}) Каждое натуральное число можно
  единственным образом записать в скошенном двоичном каноническом виде.
\end{theorem}
\end{frame}

\begin{frame}[fragile]{}

Напомним, что вес $i$-й цифры равен $2^i - 1$, и заметим, что $1 +
2(2^{i+1} - 1) = 2^{i+2} - 1$. Отсюда следует, что мы можем добавить
единицу к скошенному двоичному числу, чья младшая ненулевая цифра равна двойке,
заменив эту двойку на ноль и увеличив следующую цифру с нуля до
единицы или с единицы до двух. (Следующая цифра не может уже равняться
двойке.) Увеличение на единицу скошенного двоичного числа, которое не
содержит двойки, ещё проще~--- надо только увеличить младшую цифру с
нуля до единицы или с единицы до двойки. В обоих случаях результат
снова оказывается в каноническом виде. Если предположить, что мы можем
найти младшую ненулевую цифру за время $O(1)$, в обоих случаях мы
тратим не более $O(1)$ времени!
\end{frame}

\begin{frame}[fragile]{}
Мы не можем использовать для скошенных двоичных чисел плотное
представление, потому что тогда поиск первой ненулевой цифры займет
больше времени, чем $O(1)$. Поэтому мы выбираем разреженное
представление и всегда имеем непосредственный доступ к младшей
ненулевой цифре.
\inputminted[firstline=3,lastline=3]{haskell}{code/SkewNumbers.hs}

\end{frame}

\begin{frame}[fragile]{}
\inputminted[firstline=3,lastline=3]{haskell}{code/SkewNumbers.hs}
Целые числа представляют либо ранг, либо вес ненулевых цифр. Мы пока
что используем веса. Веса хранятся в порядке возрастания, но два
наименьших веса могут быть одинаковы, показывая, что младшая ненулевая
цифра равна двум. При таком представлении мы реализуем \lstinline!inc!
следующим образом:
\inputminted[firstline=5,lastline=7]{haskell}{code/SkewNumbers.hs}

Первый вариант проверяет два первых веса на равенство, и либо сливает
их в следующий больший вес (увеличивая таким образом следующую цифру),
либо добавляет новый вес 1 (увеличивая самую младшую цифру). Второй
вариант обрабатывает случай, когда список \lstinline!ws! пуст или
содержит только один вес. Ясно, что эта процедура работает за время
$O(1)$ в худшем случае.
\end{frame}

\begin{frame}[fragile]{}
Уменьшение скошенного двоичного числа на единицу столь же просто, как
увеличение. Если младшая цифра не равна нулю, мы просто уменьшаем эту
цифру с двух до единицы или с единицы до нуля. В противном случае мы
уменьшаем самую младшую ненулевую цифру, а предыдущий ноль заменяем
двойкой. Это реализуется так:
\begin{lstlisting}
  fun dec (1 :: ws) = ws
    | dec (w :: ws) = (w div 2) :: (w div 2) :: ws
\end{lstlisting}
Во второй строке нужно учитывать, что если $w = 2^{k+1} - 1$, то
$\lfloor w/2 \rfloor = 2^k - 1$. Ясно, что \lstinline!dec! также
работает за время $O(1)$ в худшем случае.

\subsection{Скошенные двоичные списки с произвольным доступом}
\label{sc:9.3.1}

Теперь мы разработаем числовое представление для списков с
произвольным доступом на основе скошенных двоичных чисел.  Основа
представления данных~--- список деревьев, одно дерево для каждой
единицы и два дерева для каждой двойки. Деревья хранятся в порядке
возрастания размера, но если младшая ненулевая цифра двойка, то два
первых дерева будут одинакового размера.

Размеры деревьев соответствуют весам цифр в скошенных двоичных числах,
так что дерево, представляющее $i$-ю цифру, имеет размер $2^{i+1} -
1$. До сих пор мы в основном рассматривали деревья размером степень
двойки, но встречались и деревья нужного нам сейчас размера: полные
двоичные деревья. Таким образом, мы представляем скошенные двоичные
списки с произвольным доступом в виде списков полных двоичных
деревьев.

Чтобы эффективно поддержать операцию \lstinline!head!, мы должны
сделать первый элемент списка с произвольным доступом вершиной первого
дерева, так что элементы внутри каждого дерева мы будем хранить в
предпорядке слева направо; элементы каждого дерева предшествуют
элементам следующего дерева.

В предыдущих примерах мы хранили в каждой вершине её размер или ранг,
даже когда эта информация была избыточна. В этом примере мы используем
более реалистичный подход и храним размер только вместе с вершиной
каждого дерева, а не для всех поддеревьев. Следовательно, тип данных
для скошенных двоичных списков с произвольным доступом получается
\begin{lstlisting}
  datatype $\alpha$ Tree = Leaf of $\alpha$ | Node of $\alpha$ $\times$ $\alpha$ Tree $\times$ $\alpha$ Tree
  type $\alpha$ RList = (int $\times$ $\alpha$ Tree) list
\end{lstlisting}
Теперь можно определить \lstinline!cons! по аналогии с
\lstinline!inc!.
\begin{lstlisting}
  fun cons (x, ts as (w$_1$, t$_1$) :: (w$_2$, t$_2$) :: rest) =
        if w$_1$ = w$_2$ then (1+w$_1$+w$_2$, Node (x, t$_1$, t$_2$) :: rest)
        else (1, Leaf x) :: ts
    | cons (x, ts) = (1, Leaf x) :: ts
\end{lstlisting}
Функции \lstinline!head! и \lstinline!tail! работают с корнем первого
дерева. \lstinline!tail! возвращает дочерние узлы этого дерева (если
они есть) обратно в начало списка, где они будут представлять новую
цифру-двойку.
\begin{lstlisting}
  fun head ((1, Leaf x) :: ts) = x
    | head ((w, Node (x, t$_1$, t$_2$)) :: ts) = x
  fun tail ((1, Leaf x) :: ts) = ts
    | tail ((w, Node (x, t$_1$, t$_2$)) :: ts) = (w div 2, t$_1$) :: (w div 2, t$_2$) :: ts
\end{lstlisting}
Чтобы найти элемент, мы сначала ищем нужное дерево, а затем нужный
элемент в этом дереве. При поиске внутри дерева мы отслеживаем размер
текущего дерева.
\begin{lstlisting}
  fun lookup (i, (w, t) :: ts) =
        if i < w then lookupTree (w, i ,t)
        else lookup (i-w, ts)

  fun lookupTree (1, 0, Leaf x) = x
    | lookupTree (w, 0, Node (x, t$_1$, t$_2$)) = x
    | lookupTree (w, i, Node (x, t$_1$, t$_2$)) =
        if i < w div 2 then lookupTree (w div 2, i-1, t$_1$)
        else lookupTree (w div 2, i - 1 - w div 2, t$_2$)
\end{lstlisting}
Заметим, что в предпоследней строке мы отнимаем единицу от \lstinline!i!,
поскольку перескакиваем через \lstinline!x!. В последней строке мы
отнимаем $1 + \lfloor \lstinline!w!/2 \rfloor$ от \lstinline!i!,
поскольку перескакиваем через \lstinline!x! и через все элементы
\lstinline!t$_1$!. Функции \lstinline!update! и \lstinline!updateTree!
определяются подобным же образом. Они приведены на Рис.~\ref{fig:9.7}
наряду со всеми остальными деталями реализации.

\begin{figure}
  \centering

  \caption{Скошенные двоичные списки с произвольным доступом.}
  \label{fig:9.7}
\end{figure}

\end{frame}

\begin{frame}[fragile]{}

Нетрудно убедиться, что \lstinline!cons!, \lstinline!head! и
\lstinline!tail! работают за время $O(1)$ в худшем случае. Подобно
двоичным спискам с произвольным доступом, скошенные двоичные списки с
произвольным доступом представляют собой списки логарифмической длины,
состоящие из деревьев логарифмической глубины, так что
\lstinline!lookup! и \lstinline!update! работают за время $O(\log n)$
в худшем случае. На самом деле каждый неудачный шаг \lstinline!lookup!
или \lstinline!update! отбрасывает по крайней мере один элемент, так
что можно немного улучшить оценку до $O(\min(i, \log n))$.

\begin{hint}
  Скошенные двоичные списки с произвольным доступом являются хорошим
  выбором для приложений, активно использующих как спископодобные, так
  и массивоподобные функции в списках с произвольным
  доступом. Существуют более производительные реализации списков и
  более производительные реализации (устойчивых) массивов, но ни одна
  реализация не превосходит нашу в обеих классах функций \cite{Okasaki1995b}.
\end{hint}

\begin{exercise}\label{ex:9.14}
  Перепишите структуру \lstinline!HoodMelvilleQueue! из
  Раздела~\ref{sc:8.2.1}, чтобы она вместо обычных списков
  использовала скошенные двоичные списки с произвольным
  доступом. Реализуйте на получившейся структуре операции
  \lstinline!lookup! и \lstinline!update!.
\end{exercise}

\end{frame}

\subsection{Скошенные биномиальные кучи}
\label{sc:9.3.2}

\begin{frame}[fragile]{}



Наконец, рассмотрим гибридное числовое представление для куч,
основанное как на скошенных двоичных числах, так и на обыкновенных
двоичных числах. Реализация скошенного двоичного числа проста и
быстра, и отлично подходит как образец для функции
\lstinline!insert!. К сожалению, сложение двух скошенных двоичных
чисел весьма неудобно. Поэтому функцию \lstinline!merge! мы порождаем
на основе сложения обыкновенных двоичных чисел, а не сложения
скошенных чисел.

\term{Скошенное биномиальное дерево}{skew binomial tree} представляет
собой биномиальное дерево, в котором к каждому узлу приписан список
длиной до $r$ элементов, где $r$~--- ранг рассматриваемого узла.
\begin{lstlisting}
  datatype Tree = Node of int $\times$ Elem.T $\times$ Elem.T list $\times$ Tree list
\end{lstlisting}
В отличие от обыкновенных биномиальных деревьев, размер скошенного
биномиального дерева не полностью определяется его рангом; ранг
определяет диапазон возможных размеров.

\begin{lemma}
  \label{lm:9.2}
  Если $t$~--- скошенное биномиальное дерево ранга $r$, то $2^r \le
  |t| \le 2^{r+1} - 1$
  \begin{exercise}\label{ex:9.15}
    Докажите Лемму~\ref{lm:9.2}
  \end{exercise}
\end{lemma}

Над скошенными биномиальными деревьями можно производить операцию
\term{связывания}{linking} и \term{скошенного связывания}{skew linking}.
Функция связывания \lstinline!link! сочетает два дерева ранга $r$ и
получает одно дерево ранга $r+1$, делая дерево с большим корнем
ребенком дерева с меньшим корнем.
\begin{lstlisting}
  fun link (t$_1$ as Node (r, x$_1$, xs$_1$, c$_1$), t$_2$ as Node (_, x$_2$, xs$_2$, c$_2$) =
        if Elem.leq (x$_1$, x$_2$) then Node (r+1, x$_1$, xs$_1$, t$_2$ :: c$_1$)
        else Node (r+1, x$_2$, xs$_2$, t$_1$ :: c$_2$)
\end{lstlisting}
Функция скошенного связывания \lstinline!skewLink! сочетает два дерева
ранга $r$ и дополнительный элемент, получая дерево ранга
$r+1$. Сначала она связывает два дерева, а затем сравнивает корень
получившегося дерева с дополнительным элементом. Меньший из этих двух
элементов становится корнем, а больший добавляется к дополнительному
списку элементов.
\begin{lstlisting}
  fun skewLink (x, t$_1$, t$_2$) =
        let val Node (r, y, ys, c) = link (t$_1$, t$_2$)
        in
            if Elem.leq (x, y) then Node (r, x, y :: ys, c)
            else Node (r, y, x :: ys, c)
        end
\end{lstlisting}

Скошенная биномиальная куча представляет собой список скошенных
биномиальных деревьев, упорядоченных в порядке кучи, отсортированных
по возрастанию ранга, и только два первых дерева могут иметь
одинаковый ранг. Поскольку скошенные биномиальные деревья одного ранга
могут иметь различный размер, здесь уже нет прямого соответствия между
деревьями в куче и цифрами скошенного двоичного числа. представляющего
размер кучи.  Например, хотя скошенное двоичное представление числа 4
равно \texttt{11}, скошенная биномиальная куча размера 4 может
содержать либо одно дерево ранга 2 размера 4, либо два дерева ранга 1
размером 2, либо дерево ранга 1 размером 3 и дерево ранга 0, либо
дерево ранга 1 размером 2 и два дерева ранга 0. Однако максимальное
число деревьев в куче по-прежнему равно $O(\log n)$.

Большое преимущество скошенных биномиальных куч состоит в том, что
новый элемент вставляется за время $O(1)$. Сначала мы сравниваем ранги
двух наименьших деревьев. Если они совпадают, мы производим скошенное
связывание нового элемента с этими деревьями. В противном случае мы
создаем новое одноэлементное дерево и добавляем его к началу списка.
\begin{lstlisting}
  fun insert (x, ts as t$_1$ :: t$_2$ :: rest) =
        if rank t$_1$ = rank t$_2$ then skewLink (x, t$_1$, t$_2$) :: rest
        else Node (0, x, [], []) :: ts
    | insert (x, ts) = Node (0, x, [], []) :: ts
\end{lstlisting}

Оставшиеся функции почти такие же, как соответствующие функции
обыкновенных биномиальных куч. Мы изменяем имя старой функции
\lstinline!merge! на \lstinline!mergeTrees!. Она по-прежнему проходит
оба списка деревьев, проводя связывание (не скошенное связывание!)
каждый раз, когда видит два дерева одного ранга. Поскольку и
\lstinline!mergeTrees!, и её вспомогательная функция
\lstinline!insTree! ожидают списки деревьев строго возрастающего
ранга, функция \lstinline!merge! нормализует оба своих аргумента,
убирая дубликаты из начала списков, прежде чем позвать
\lstinline!mergeTrees!.
\begin{lstlisting}
  fun normalize [] = []
    | normalize (t :: ts) = insTree (t, ts)
  fun merge (ts$_1$, ts$_2$) = mergeTrees (normalize ts$_1$, normalize ts$_2$)
\end{lstlisting}
На функции \lstinline!findMin! и \lstinline!removeMinTree!
переключение на скошенные биномиальные кучи никак не влияет, поскольку
обе эти функции не заботятся о рангах, рассматривая только корень
каждого дерева. Функция \lstinline!deleteMin! изменяется лишь
ненамного. Как и раньше, изымается дерево с минимальным корнем, список
его детей обращается, и обращенный список детей сливается с
оставшимися деревьями.  Однако затем заново вставляются элементы
дополнительного списка, прикрепленного к уничтоженному корню.
\begin{lstlisting}
  fun deleteMin ts =
        let val (Node (_, x, xs, ts$_1$), ts$_2$) = removeMinTree ts
            fun insertAll ([], ts) = ts
              | insertAll (x :: xs, ts) = insertAll (xs, insert (x,
              ts))
        in insertAll (xs, merge (rev ts$_1$, ts$_2$)) end
\end{lstlisting}
На Рис.~\ref{fig:9.8} приведена полная реализация скошенных
биномиальных куч.

\begin{figure}
  \centering

  \caption{Скошенные биномиальные кучи.}
  \label{fig:9.8}
\end{figure}

Функция \lstinline!insert! работает за время $O(1)$ в худшем случае, а
\lstinline!merge!, \lstinline!findMin! и \lstinline!deleteMin!
работают за то же время, что и соответствующие функции для
обыкновенных биномиальных куч, то есть, за $O(\log n)$ в худшем
случае. Заметим, что каждая из различных фаз функции \lstinline!deleteMin!~---
поиск дерева с минимальным корнем, обращение его детей, слияние детей
с оставшимися деревьями и вставка дополнительных элементов,~---
занимает по $O(\log n)$.

Если нужно, мы можем улучшить время работы \lstinline!findMin! до
$O(1)$ при помощи функтора \lstinline!ExplicitMin! из
Упражнения~\ref{ex:3.7}. В Разделе~\ref{sc:10.2.2} мы увидим, как
улучшить также и время операции \lstinline!merge! до $O(1)$.

\begin{exercise}\label{ex:9.16}
  Допустим, нам нужна функция \lstinline!delete! типа
  \lstinline!Elem.T $\times$ Heap $\to$ Heap!.
  Напишите функтор, берущий реализацию куч \lstinline!H! и порождающий
  реализацию куч, поддерживающую наряду с обычными операциями над
  кучей функцию \lstinline!delete!. Используйте тип
  \begin{lstlisting}
    type Heap = H.Heap $\times$ H.Heap
  \end{lstlisting}
  где одна из элементарных куч представляет положительные вхождения
  элементов, а вторая~--- отрицательные вхождения. Отрицательное
  вхождение элемента в кучу означает, что этот элемент был уже
  уничтожен, но физически ещё не удален из кучи.  Положительные и
  отрицательные вхождения одного и того же элемента
  взаимоуничтожаются и физически удаляются из кучи, когда оба
  оказываются минимальными элементами своих куч.  Поддерживайте
  инвариант, что минимальный элемент положительной кучи строго меньше,
  чем минимальный элемент отрицательной кучи. (У этой реализации есть
  забавное свойство: элемент можно уничтожить прежде, чем он
  вставлен в кучу; для многих приложений это свойство безвредно.)
\end{exercise}

\end{frame}

\section{Троичные и четверичные числа}
\label{sc:9.4}

\begin{frame}[fragile]{}


В информатике мы настолько привыкли работать с двоичными числами, что
иногда забываем о существовании других оснований. В этом разделе мы
рассмотрим использование арифметики по основанию 3 и 4 в числовых
представлениях.

Вес каждой цифры при основании $k$ равен $k^r$, так что нам нужны
семейства деревьев, имеющих такие размеры. Можно построить обобщения
для каждого из семейств деревьев, используемых в двоичных числовых
представлениях:

\begin{definition}\label{def:9.4}
  \textbf{(Полные $k$-ичные листовые деревья)} \term{Полное $k$-ичное дерево}{complete
    $k$-ary tree} ранга 0 представляет собой лист, а полное $k$-ичное
  дерево ранга $r > 0$ представляет собой узел с $k$ поддеревьями,
  каждое из которых является полным $k$-ичным деревом ранга
  $r-1$. Полное $k$-ичное дерево ранга $r$ содержит $(k^{r+1} - 1) /
  (k - 1)$ узлов и $k^r$ листьев. Полное $k$-ичное листовое дерево~---
  это полное $k$-ичное дерево, где элементы содержатся только в листьях.
\end{definition}
\begin{definition}\label{def:9.5}
  \textbf{($k$-номиальные деревья)} \term{$k$-номиальное
    дерево}{$k$-nomial tree} ранга $r$ представляет собой узел, у
  которого есть для каждого ранга $q$ от $r-1$ до 0 по $k-1$
  поддерева, имеющих ранг $q$. Иначе выражаясь,
  $k$-номиальное дерево ранга $r > 0$ представляет собой
  $k$-номиальное дерево ранга $r-1$, к которому в качестве левых
  поддеревьев присоединены ещё $k-1$ $k$-номиальных дерева ранга
  $r-1$. Из второго определения легко увидеть, что $k$-номиальное
  дерево ранга $r$ содержит $k^r$ узлов.
\end{definition}

\begin{definition}\label{def:9.6}
  \textbf{($k$-ичные подвешенные деревья)} \term{$k$-ичное подвешенное
  дерево}{$k$-ary pennant} ранга 0 представляет собой единственную
вершину, а $k$-ичное подвешенное дерево ранга $r > 0$ представляет
собой вершину с $k-1$ поддеревьями, каждое из которых является полным
$k$-ичным деревом ранга $r-1$. Каждое из этих поддеревьев содержит
$(k^r - 1) / (k - 1)$ узлов, так что всё дерево целиком содержит $k^r$ узлов.
\end{definition}

Преимущество при использовании оснований больше двойки заключается в
том, что для представления каждого числа требуется меньше цифр. В то
время как число по основанию 2 содержит примерно $\log_2 n$ цифр,
число по основанию $k$ содержит приблизительно $\log_k n = \log_2 n /
\log_2 k$ цифр. Например, при основании 4 нужно примерно вдвое меньше
цифр, чем при основании 2. С другой стороны, теперь для каждой цифры
имеется больше возможных значений, так что обработка каждой цифры
может отнимать больше времени. В числовых представлениях обработка
одной цифры по основанию $k$ часто требует примерно $k+1$ шагов, так
что операция, затрагивающая каждую цифру, должна отнимать примерно
$(k+1) \log_k n = \frac{k+1}{\log_2 k} \log n$ шагов. В следующей
таблице приведены значения $(k + 1) / \log_2 k$ для $k = 2, \ldots,
8$.\\
\begin{tabular}{c|ccccccc}
  $k$ & 2 & 3 & 4 & 5 & 6 & 7 & 8 \\
$(k + 1) / \log_2 k$ &
   3.00 & 2.52 & 2.50 & 2.58 & 2.71 & 2.85 & 3.0 \\
\end{tabular}
\\
По этой таблице можно заключить, что числовые представления,
основанные на троичных или четверичных числах, могут выигрывать до
16\% у числовых представлений на основе двоичных чисел. Другие
факторы, например, размер кода, часто делают большие основания менее
эффективными при увеличении $k$, так что настолько большие ускорения
редко встречаются на практике. Более того, троичные и четверичные
представления на маленьких объемах данных часто работают хуже, чем
двоичные представления. Однако для больших объемов данных троичные и
четверичные представления часто приносят ускорение от 5 до 10\%.

\begin{exercise}\label{ex:9.17}
  Реализуйте триномиальные кучи, используя тип
  \begin{lstlisting}
    datatype Tree = Node of Elem.T $\times$ (Tree $\times$ Tree) list
    datatype Digit = Zero | One of Tree | Two of Tree $\times$ Tree
    type Heap = Digit list
  \end{lstlisting}
\end{exercise}

\begin{exercise}\label{ex:9.18}
  Реализуйте безнулевые четверичные списки с произвольным доступом на
  основе типа
  \begin{lstlisting}
    datatype $\alpha$ Tree = Leaf of $\alpha$ | Node of $\alpha$ Tree vector.
    datatype $\alpha$ RList = $\alpha$ Tree vector list
  \end{lstlisting}
  где каждый вектор в \lstinline!Node! содержит по четыре дерева, а
  каждый вектор в списке содержит от одного до четырёх деревьев.
\end{exercise}

\begin{exercise}\label{ex:9.19}
  Можно также приспособить к произвольному основанию понятие
  скошенного двоичного числа. В скошенных $k$-ичных числах $i$-я цифра
  имеет вес $(k^{i+1} - 1) / (k - 1)$. Цифры выбираются из набора
  $\{0, \ldots, k-1\}$, плюс младшая ненулевая цифра может равняться
  $k$. Реализуйте скошенные троичные списки с произвольным доступом на
  основе типа
  \begin{lstlisting}
    datatype $\alpha$ Tree = Leaf of $\alpha$ | Node of $\alpha$ $\times$ $\alpha$ Tree $\times$ $\alpha$ Tree $\times$ $\alpha$ Tree
    type $\alpha$ RList = (int $\times$ $\alpha$ Tree) list
  \end{lstlisting}
\end{exercise}

\end{frame}

\section{Примечания}
\label{sc:9.5}

\begin{frame}[fragile]{}


Структуры данных, которые можно описать как числовые представления,
встречаются на удивление часто, но явным образом связь с
каким-либо вариантом системы счисления упоминают лишь изредка
\cite{Guibas-etal1977, Myers1983, CarlssonMunroPoblete1988,
  KaplanTarjan1996b}. Скошенные списки с произвольным доступом впервые
появились в \cite{Okasaki1996b}. Скошенные биномиальные кучи
описаны в \cite{BrodalOkasaki1996}.

\end{frame}
%%% Local Variables:
%%% mode: latex
%%% TeX-master: "pfds"
%%% End:


\begin{frame}[fragile]{}
1
\end{frame}

\begin{frame}[fragile]{}
1
\end{frame}

\begin{frame}[fragile]{}
1
\end{frame}

\begin{frame}[fragile]{}
1
\end{frame}

\begin{frame}[fragile]{}
1
\end{frame}

\begin{frame}[fragile]{}
1
\end{frame}

\begin{frame}[fragile]{}
1
\end{frame}

\begin{frame}[fragile]{}
1
\end{frame}

\begin{frame}[fragile]{}
1
\end{frame}

\begin{frame}[fragile]{}
1
\end{frame}

\begin{frame}[fragile]{}
1
\end{frame}

\begin{frame}[fragile]{}
1
\end{frame}




% \begin{frame}[allowframebreaks]
%   \frametitle<presentation>{Ссылки}
%   \begin{thebibliography}{10}
%   \bibitem{paper}
%     \href{http://conal.net/papers/compiling-to-categories/compiling-to-categories.pdf}{paper}
%     \newblock {\em Conal Elliot }    
%   \bibitem{conal}
%     Slides
%     \newblock {\em Conal Elliot }
%     \newblock \href{http://conal.net/talks/compiling-to-categories.pdf}{ссылка}
%   \bibitem{video}    
%     \href{http://podcasts.ox.ac.uk/compiling-categories}{ICFP 2017 video}
%     \newblock {\em Conal Elliot }
%   \bibitem{}
%     \href{https://github.com/conal/concat}{Project repo}
%   \end{thebibliography}
% \end{frame}

\end{document}
