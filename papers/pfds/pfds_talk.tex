\documentclass[aspectratio=169
  , xcolor={svgnames}
  , hyperref={ colorlinks,citecolor=DeepPink4
             , linkcolor=DarkRed,urlcolor=DarkBlue}
  , russian
  ]{beamer}
\usetheme{CambridgeUS}
\beamertemplatenavigationsymbolsempty % remove navigation bar

\usefonttheme{professionalfonts}
\makeatletter
\@ifclassloaded{beamer}{
  % get rid of header navigation bar
  \setbeamertemplate{headline}{}
  % get rid of bottom navigation symbols
  \setbeamertemplate{navigation symbols}{}
  % get rid of footer
  %\setbeamertemplate{footline}{}
}
{}
\makeatother
%%%%%%%%%%%%%%%%%%%%%%%%%%%%%%%%%%%%%%%%%%%%%
\usepackage{fontawesome}
% \newfontfamily{\FA}{Font Awesome 5 Free} % some glyphs missing
\expandafter\def\csname faicon@facebook\endcsname{{\FA\symbol{"F09A}}}
\def\faQuestionSign{{\FA\symbol{"F059}}}
\def\faQuestion{{\FA\symbol{"F128}}}
\def\faExclamation{{\FA\symbol{"F12A}}}
\def\faUploadAlt{{\FA\symbol{"F093}}}
\def\faLemon{{\FA\symbol{"F094}}}
\def\faPhone{{\FA\symbol{"F095}}}
\def\faCheckEmpty{{\FA\symbol{"F096}}}
\def\faBookmarkEmpty{{\FA\symbol{"F097}}}

\newcommand{\faGood}{\textcolor{ForestGreen}{\faThumbsUp}}
\newcommand{\faBad}{\textcolor{red}{\faThumbsODown}}
\newcommand{\faWrong}{\textcolor{red}{\faTimes}}
\newcommand{\faMaybe}{\textcolor{blue}{\faQuestion}}
\newcommand{\faCheckGreen}{\textcolor{ForestGreen}{\faCheck}}
%%%%%%%%%%%%%%%%%%%%%%%%%%%%%%%%%%%%%%%%%%%%%

\usepackage{fontspec}
\usepackage{xunicode}
\usepackage{xltxtra}
\usepackage{xecyr}
\usepackage{hyperref}

\setmainfont[
 Ligatures=TeX,
 Extension=.otf,
 BoldFont=cmunbx,
 ItalicFont=cmunti,
 BoldItalicFont=cmunbi,
% Scale = 1.1
]{cmunrm}
\setsansfont[
 Ligatures=TeX,
 Extension=.otf,
 BoldFont=cmunsx,
 ItalicFont=cmunsi,
%  Scale = 1.2
]{cmunss}
%\setmainfont[Mapping=tex-text]{DejaVu Serif}
%\setsansfont[Mapping=tex-text]{DejaVu Sans}
%\setmonofont{Fira Code}[Contextuals=AlternateOff]
\setmonofont{Fira Code}[Contextuals=Alternate,Scale=0.9]
\newfontfamily{\myfiracode}[Scale=1.5,Contextuals=Alternate]{Fira Code}
%\setmonofont[Scale=0.9,BoldFont={Inconsolata Bold}]{Inconsolata}

\usepackage{polyglossia}
\setmainlanguage{russian}
\setotherlanguage{english}


%\newfontfamily\dejaVuSansMono{DejaVu Sans Mono}
% https://github.com/vjpr/monaco-bold/raw/master/MonacoB/MonacoB.otf
%\newfontfamily\monacoB{MonacoB}
%%%%%%%%%%%%%%%%%%%%%%%%%%%%%%%%%%%%%%%%%%%%%%%5
\usepackage{soul} % for \st that strikes through
\usepackage[normalem]{ulem} % \sout

\usepackage{stmaryrd}
\newcommand{\sem}[1]{\ensuremath{\llbracket #1\rrbracket}}


\usepackage{listings}
%\lstdefinestyle{style1}{
%  language=haskell,
%  numbers=left,
%  stepnumber=1,
%  numbersep=10pt,
%  tabsize=4,
%  showspaces=false,
%  showstringspaces=false
%}
%\lstdefinestyle{hsstyle1}
%{ language=haskell
%%          , basicstyle=\monacoB
%         , deletekeywords={Int,Float,String,List,Void}
%         , breaklines=true
%         , columns=fullflexible
%         , commentstyle=\color{ForestGreen}
%         , escapeinside=§§
%         , escapebegin=\begin{russian}\commentfont
%         , escapeend=\end{russian}
%         , commentstyle=\color{ForestGreen}
%         , escapeinside=§§
%         , escapebegin=\begin{russian}\color{ForestGreen}
%         , escapeend=\end{russian}
%         , mathescape=true
%%          , backgroundcolor = \color{MyBackground}
%}
%
%\newcommand{\inline}[1]{\lstinline{haskell}{#1}}
%\def\hsinline{\mintinline{haskell}}
%\def\inline{\hsinline}
%
%\lstnewenvironment{hslisting} {
%    \lstset { style={hsstyle1} }
%  }
%  {}
%  
%%%%%%%%%%%%%%%%%%%%%%%%%%%%%%%%%%%%%%%%%%%%%%%%%%%%%%%%%%%  
%%\setmainfont[
%% Ligatures=TeX,
%% Extension=.otf,
%% BoldFont=cmunbx,
%% ItalicFont=cmunti,
%% BoldItalicFont=cmunbi,
%%]{cmunrm}
%%% С засечками (для заголовков)
%%\setsansfont[
%% Ligatures=TeX,
%% Extension=.otf,
%% BoldFont=cmunsx,
%% ItalicFont=cmunsi,
%%]{cmunss}
%% \setmonofont[Scale=0.6]{Monaco}
%
%\usefonttheme{professionalfonts}
%\usepackage{times}
\usepackage{tikz}
\usetikzlibrary{cd}
\usepackage{tikz-cd}
\usepackage{caption}
\usepackage{subcaption}

%\renewtheorem{definition}{برهان}[chapter]
%%\DeclareMathOperator{->}{\rightarrow}
%\newcommand\iso{\ensuremath{\cong}}
%\usepackage{verbatim}
%\usepackage{graphicx}
%\usetikzlibrary{arrows,shapes}

%\usepackage{amsmath}
%\usepackage{amsfonts}
\usepackage{scalerel}
\DeclareMathOperator*{\myvee}{\scalerel*{\vee}{\sum}}
\DeclareMathOperator*{\mywedge}{\scalerel*{\wedge}{\sum}}

%
%\usepackage{tabulary}
%
%% sudo aptget install ttf-mscorefonts-installer
%%\setmainfont{Times New Roman}
%%\setsansfont[Mapping=tex-text]{DejaVu Sans}
%
%%\setmonofont[Scale=1.0,
%%    BoldFont=lmmonolt10-bold.otf,
%%    ItalicFont=lmmono10-italic.otf,
%%    BoldItalicFont=lmmonoproplt10-boldoblique.otf
%%]{lmmono9-regular.otf}
%
\usepackage[cache=true]{minted}
\usemintedstyle{perldoc}

\def\hsinline{\mintinline{haskell}}
\def\mlinline{\mintinline{ocaml}}
% color options
\definecolor{YellowGreen} {HTML}{B5C28C}
\definecolor{ForestGreen} {HTML}{009B55}
\definecolor{MyBackground}{HTML}{F0EDAA}



\institute{матмех СПбГУ}

\addtobeamertemplate{title page}{}{
  \begin{center}{\tiny Дата сборки: \today}\end{center}
}

\usepackage{tabulary}
\usepackage{verbatim}
% \usepackage{tabularx}  % for 'tabularx' environment
% \usepackage{ragged2e} % for \Centering macro
% \newcolumntype{C}{>{\Centering\arraybackslash}X}m
% sudo aptget install ttf-mscorefonts-installer
\defaultfontfeatures{Ligatures={TeX}} 
\setmainfont{Times New Roman}
\setsansfont{CMU Sans Serif}

\setmonofont[Scale=1.0,
    BoldFont=lmmonolt10-bold.otf,
    ItalicFont=lmmono10-italic.otf,
    BoldItalicFont=lmmonoproplt10-boldoblique.otf
]{lmmono9-regular.otf}

\usepackage[cache=true]{minted}
\usepackage{amsthm}
\newtheorem*{remark}{Remark}

\usepackage{tikz}
\usetikzlibrary{trees}

%%%%%%%%%%%%%%%%%%
\makeatletter
\newenvironment{tabminted}{%
  \let\FV@ListVSpace\relax  
  \minted
}{%
  \endminted
  \unskip   
  \aftergroup\@tabmintedend
}
\newcommand*{\tabminted@finalstrut}[1]{%
  \ifdim\prevdepth>0pt
    \ifdim\dp#1>\prevdepth
      \vskip\dimexpr(\dp#1)-\prevdepth\relax
    \fi
  \else
    \vskip\dimexpr(\dp#1)\relax
  \fi
}
\newcommand*{\@tabmintedend}{%
  \let\@finalstrut\tabminted@finalstrut
}
\makeatother
%%%%%%%%%%%%%%%%%%%%%5
\title[]{Чисто функциональные структуры данных}
\author{Косарев Дмитрий }

\institute{матмех СПбГУ}

\date{\today}
 
\AtBeginSection[]
{
  \begin{frame}<beamer>
    \frametitle{Оглавление}
    \tableofcontents[currentsection,currentsubsection]
  \end{frame}
}

\newcommand{\verbatimfont}[1]{\def\verbatim@font{#1}}

\usepackage{verbatimbox}

\begin{document}
\maketitle

% For every picture that defines or uses external nodes, you'll have to
% apply the 'remember picture' style. To avoid some typing, we'll apply
% the style to all pictures.
\tikzstyle{every picture}+=[remember picture] 

% By default all math in TikZ nodes are set in inline mode. Change this to
% displaystyle so that we don't get small fractions.
\everymath{\displaystyle}

% Uncomment these lines for an automatically generated outline.
\begin{frame}{Оглавление}
  \tableofcontents
\end{frame}

\begin{frame}[fragile]{}
\begin{figure}[h]
%	\centering
	\begin{tikzpicture}[thick,scale=0.5, every node/.style={scale=0.5}]
    \tikzstyle{marrs}=[very thick,-latex]

    \begin{scope}
    
        \foreach \x/\y in {0/0, 0/-2, 3/-2, 6/-2} {
            \draw (\x - 0.5, \y - 0.5) rectangle +(1, 1); \draw (\x + 1 - 0.5, \y - 0.5) rectangle +(1, 1);
        }
        \draw[marrs] (-1.5, 0) -> +(1, 0);
        \draw[marrs] (0, 0) -> +(0, -1.5);
        \draw[marrs] (1, -2) -> +(1.5, 0);
        \draw[marrs] (4, -2) -> +(1.5, 0);
        \draw[marrs] (1, 0) -- (5.5, 0) .. controls (5.75, 0) and (6, -0.25) .. (6, -0.5) -- (6, -1.5);
        
        { \huge
            \draw (-2, 0) node {$xs$};
            \draw (0, -2) node {$0$};
            \draw (3, -2) node {$1$};
            \draw (6, -2) node {$2$};
            \draw (7, -2) node {$\cdot$};
        }
    
    \end{scope}
    
    \begin{scope}[xshift=12cm]
    
        \foreach \x/\y in {0/0, 0/-2, 3/-2, 6/-2} {
            \draw (\x - 0.5, \y - 0.5) rectangle +(1, 1); \draw (\x + 1 - 0.5, \y - 0.5) rectangle +(1, 1);
        }
        \draw[marrs] (-1.5, 0) -> +(1, 0);
        \draw[marrs] (0, 0) -> +(0, -1.5);
        \draw[marrs] (1, -2) -> +(1.5, 0);
        \draw[marrs] (4, -2) -> +(1.5, 0);
        \draw[marrs] (1, 0) -- (5.5, 0) .. controls (5.75, 0) and (6, -0.25) .. (6, -0.5) -- (6, -1.5);
        
        { \huge
            \draw (-2, 0) node {$ys$};
            \draw (0, -2) node {$3$};
            \draw (3, -2) node {$4$};
            \draw (6, -2) node {$5$};
            \draw (7, -2) node {$\cdot$};
        }
    \end{scope}
    
    
    
\end{tikzpicture}\par
	(до)\par
%	\vspace{0.5cm}
	\documentclass[tikz]{standalone}
%\usepackage{fontawesome}

% \newfontfamily{\FA}{Font Awesome 5 Free} % some glyphs missing
\expandafter\def\csname faicon@facebook\endcsname{{\FA\symbol{"F09A}}}
\def\faQuestionSign{{\FA\symbol{"F059}}}
\def\faQuestion{{\FA\symbol{"F128}}}
\def\faExclamation{{\FA\symbol{"F12A}}}
\def\faUploadAlt{{\FA\symbol{"F093}}}
\def\faLemon{{\FA\symbol{"F094}}}
\def\faPhone{{\FA\symbol{"F095}}}
\def\faCheckEmpty{{\FA\symbol{"F096}}}
\def\faBookmarkEmpty{{\FA\symbol{"F097}}}

%\def\faCatt{{\FA\symbol{"F6BE}}}
%\def\faCat{\faicon{cat}}
%\def\faCat{\faicon{yoast}}
\expandafter\def\csname faicon@dog\endcsname{{\FA\symbol{"F4DA}}}
%\def\faDog{\faicon{dog}}
%\def\faDog{{\FA\symbol{"F4DA}}}
%\def\faDogg{{\FA\symbol{"F6D3}}}
%\def\faDogg{{\FA\symbol{"F596}}}

% /usr/share/texlive/texmf-dist/fonts/opentype/public/fontawesome5/FontAwesome5Free-Solid-900.otf
\newfontfamily{\FAS}{FontAwesome5Free-Solid-900.otf}
%\expandafter\def\csname faicon@download\endcsname{{\FAS\symbol{"F6D3}}}
\expandafter\def\csname faicon@cat\endcsname{{\FAS\symbol{"F6BE}}}
\def\faCat{\faicon{cat}}
\expandafter\def\csname faicon@dog\endcsname{{\FAS\symbol{"F6D3}}}
\def\faDog{\faicon{dog}}
\expandafter\def\csname faicon@dragon\endcsname{{\FAS\symbol{"F6D5}}}
\def\faDragon{\faicon{dragon}}
\expandafter\def\csname faicon@fish\endcsname{{\FAS\symbol{"F578}}}
\def\faFish{\faicon{fish}}
\expandafter\def\csname faicon@horse\endcsname{{\FAS\symbol{"F6F0}}}
\def\faHorse{\faicon{horse}}
\expandafter\def\csname faicon@spider\endcsname{{\FAS\symbol{"F717}}}
\def\faSpider{\faicon{spider}}

\expandafter\def\csname faicon@chessking\endcsname{{\FAS\symbol{"F43F}}}
\def\faChessKing{\faicon{chessking}}
\expandafter\def\csname faicon@chessqueen\endcsname{{\FAS\symbol{"F445}}}
\def\faChessQueen{\faicon{chessqueen}}
\expandafter\def\csname faicon@chessrook\endcsname{{\FAS\symbol{"F447}}}
\def\faChessRook{\faicon{chessrook}}
\expandafter\def\csname faicon@chesspawn\endcsname{{\FAS\symbol{"F443}}}
\def\faChessPawn{\faicon{chesspawn}}
\expandafter\def\csname faicon@chessknight\endcsname{{\FAS\symbol{"F441}}}
\def\faChessKnight{\faicon{chessknight}}
\expandafter\def\csname faicon@chessbishop\endcsname{{\FAS\symbol{"F43A}}}
\def\faChessBishop{\faicon{chessbishop}}
\expandafter\def\csname faicon@chess\endcsname{{\FAS\symbol{"F439}}}
\def\faChess{\faicon{chess}}







\usepackage{tikz}
\usetikzlibrary{positioning,trees,decorations.pathreplacing}

\begin{document}
\begin{tikzpicture}[thick,scale=0.5, every node/.style={scale=0.5}]
    \tikzstyle{marrs}=[very thick,-latex]

    
    \begin{scope}
    
        \foreach \x/\y in {0/0, 0/-2, 3/-2, 6/-2} {
            \draw (\x - 0.5, \y - 0.5) rectangle +(1, 1); \draw (\x + 1 - 0.5, \y - 0.5) rectangle +(1, 1);
        }
        \draw[marrs] (-1.5, 0) -> +(1, 0);
        \draw[marrs] (0, 0) -> +(0, -1.5);
        \draw[marrs] (1, -2) -> +(1.5, 0);
        \draw[marrs] (4, -2) -> +(1.5, 0);
        \draw[marrs] (1, 0) -- (17.5, 0) .. controls (17.75, 0) and (18, -0.25) .. (18, -0.5) -- (18, -1.5);
        \draw[marrs] (7, -2) -- +(4.5, 0);
        
        { \huge
            \draw (-2, 0) node {$zs$};
            \draw (0, -2) node {$0$};
            \draw (3, -2) node {$1$};
            \draw (6, -2) node {$2$};
        }
    
    \end{scope}
    
    \begin{scope}[xshift=12cm]
    
        \foreach \x/\y in {0/-2, 3/-2, 6/-2} {
            \draw (\x - 0.5, \y - 0.5) rectangle +(1, 1); \draw (\x + 1 - 0.5, \y - 0.5) rectangle +(1, 1);
        }
        
        \draw[marrs] (1, -2) -> +(1.5, 0);
        \draw[marrs] (4, -2) -> +(1.5, 0);
        
        { \huge
            \draw (0, -2) node {$3$};
            \draw (3, -2) node {$4$};
            \draw (6, -2) node {$5$};
            \draw (7, -2) node {$\cdot$};
        }
    \end{scope}
\end{tikzpicture}
\end{document}
\par
	(после)\par
%	\vspace{0.5cm}
	\caption{Выполнение xs concat ys в императивной среде. Эта операция уничтожает списки-аргументы xs и ys.}
	\label{fig:2.4}
\end{figure}
\end{frame}


\begin{frame}[fragile]{}
\begin{figure}[h]
	\centering
	\begin{tikzpicture}[thick,scale=0.5, every node/.style={scale=0.5}]
    \tikzstyle{marrs}=[very thick,-latex]

    \begin{scope}
    
        \foreach \x/\y in {0/0, 3/0, 6/0} {
            \draw (\x - 0.5, \y - 0.5) rectangle +(1, 1); \draw (\x + 1 - 0.5, \y - 0.5) rectangle +(1, 1);
        }
        \draw[marrs] (-1.5, 0) -> +(1, 0);
        \draw[marrs] (1, 0) -> +(1.5, 0);
        \draw[marrs] (4, 0) -> +(1.5, 0);
        
        { \huge
            \draw (-2, 0) node {$xs$};
            \draw (0, 0) node {$0$};
            \draw (3, 0) node {$1$};
            \draw (6, 0) node {$2$};
            \draw (7, 0) node {$\cdot$};
        }
    
    \end{scope}
    
    \begin{scope}[xshift=12cm]
    
        \foreach \x/\y in {0/0, 3/0, 6/0} {
            \draw (\x - 0.5, \y - 0.5) rectangle +(1, 1); \draw (\x + 1 - 0.5, \y - 0.5) rectangle +(1, 1);
        }
        \draw[marrs] (-1.5, 0) -> +(1, 0);
        \draw[marrs] (1, 0) -> +(1.5, 0);
        \draw[marrs] (4, 0) -> +(1.5, 0);
        
        { \huge
            \draw (-2, 0) node {$ys$};
            \draw (0, 0) node {$3$};
            \draw (3, 0) node {$4$};
            \draw (6, 0) node {$5$};
            \draw (7, 0) node {$\cdot$};
        }
    \end{scope}
    
    
    
\end{tikzpicture}\par
	(до)\par
	\vspace{0.5cm}
	\begin{tikzpicture}[thick,scale=0.5, every node/.style={scale=0.5}]
    \tikzstyle{marrs}=[very thick,-latex]
    
    
    
    \begin{scope}
    
        \foreach \x/\y in {0/0, 3/0, 6/0} {
            \draw (\x - 0.5, \y - 0.5) rectangle +(1, 1); \draw (\x + 1 - 0.5, \y - 0.5) rectangle +(1, 1);
        }
        \draw[marrs] (-1.5, 0) -> +(1, 0);
        \draw[marrs] (1, 0) -> +(1.5, 0);
        \draw[marrs] (4, 0) -> +(1.5, 0);
        
        { \huge
            \draw (-2, 0) node {$xs$};
            \draw (0, 0) node {$0$};
            \draw (3, 0) node {$1$};
            \draw (6, 0) node {$2$};
            \draw (7, 0) node {$\cdot$};
        }
    
    \end{scope}
    
    \begin{scope}[xshift=12cm]
    
        \foreach \x/\y in {0/0, 3/0, 6/0} {
            \draw (\x - 0.5, \y - 0.5) rectangle +(1, 1); \draw (\x + 1 - 0.5, \y - 0.5) rectangle +(1, 1);
        }
        \draw[marrs] (-1.5, 0) -> +(1, 0);
        \draw[marrs] (1, 0) -> +(1.5, 0);
        \draw[marrs] (4, 0) -> +(1.5, 0);
        
        { \huge
            \draw (-2, 0) node {$ys$};
            \draw (0, 0) node {$3$};
            \draw (3, 0) node {$4$};
            \draw (6, 0) node {$5$};
            \draw (7, 0) node {$\cdot$};
        }
    \end{scope}
    
    \begin{scope}[yshift=2cm]
    
        \foreach \x/\y in {0/0, 3/0, 6/0} {
            \draw (\x - 0.5, \y - 0.5) rectangle +(1, 1); \draw (\x + 1 - 0.5, \y - 0.5) rectangle +(1, 1);
        }
        \draw[marrs] (-1.5, 0) -> +(1, 0);
        \draw[marrs] (1, 0) -> +(1.5, 0);
        \draw[marrs] (4, 0) -> +(1.5, 0);
        
        { \huge
            \draw (-2, 0) node {$zs$};
            \draw (0, 0) node {$0$};
            \draw (3, 0) node {$1$};
            \draw (6, 0) node {$2$};
            
            \draw[marrs] (7, 0) -- (11.5, 0) .. controls (11.75, 0) and (12, -0.25) .. (12, -0.5) -- (12, -1.5);
        }
    
    \end{scope}
    
    
    
\end{tikzpicture}\par
	(после)\par
	\vspace{0.5cm}
	\caption{Выполнение \texttt{zs = xs ++ ys} в функциональной среде. Заметим, что списки-аргументы \texttt{xs} и \texttt{ys} не затронуты операцией.
	}
	\label{fig:2.5}
\end{figure}
\end{frame}

\begin{frame}[fragile]{}
\begin{figure}[h]
	\centering
	\begin{tikzpicture}[thick,scale=0.5, every node/.style={scale=0.5}]
    \tikzstyle{marrs}=[very thick,-latex]

    \begin{scope}
    
        \foreach \x/\y in {0/0, 3/0, 6/0, 9/0, 12/0} {
            \draw (\x - 0.5, \y - 0.5) rectangle +(1, 1); \draw (\x + 1 - 0.5, \y - 0.5) rectangle +(1, 1);
        }
        \draw[marrs] (-1.5, 0) -> +(1, 0);
        \foreach \x in {1, 4, 7, 10} {
            \draw[marrs] (\x, 0) -> +(1.5, 0);
        }
        
        { \huge
            \draw (-2, 0) node {$xs$};
            \foreach \x/\y in {0/0, 3/1, 6/2, 9/3, 12/4} {
                \draw (\x, 0) node {$\y$};
            }
            \draw (13, 0) node {$\cdot$};
        }
    
    \end{scope}
    
\end{tikzpicture}\par
	(до)\par
	\vspace{0.5cm}
	\begin{tikzpicture}[thick,scale=0.5, every node/.style={scale=0.5}]
    \tikzstyle{marrs}=[very thick,-latex]
    
    
    
    \begin{scope}
    
        \foreach \x/\y in {0/0, 3/0, 6/0, 9/0, 12/0} {
            \draw (\x - 0.5, \y - 0.5) rectangle +(1, 1); \draw (\x + 1 - 0.5, \y - 0.5) rectangle +(1, 1);
        }
        \draw[marrs] (-1.5, 0) -> +(1, 0);
        \foreach \x in {1, 4, 7, 10} {
            \draw[marrs] (\x, 0) -> +(1.5, 0);
        }
        
        { \huge
            \draw (-2, 0) node {$xs$};
            \foreach \x/\y in {0/0, 3/1, 6/2, 9/3, 12/4} {
                \draw (\x, 0) node {$\y$};
            }
            \draw (13, 0) node {$\cdot$};
        }   
    
    \end{scope}
    
    \begin{scope}[yshift=2cm]
    
        \foreach \x/\y in {0/0, 3/0, 6/0} {
            \draw (\x - 0.5, \y - 0.5) rectangle +(1, 1); \draw (\x + 1 - 0.5, \y - 0.5) rectangle +(1, 1);
        }
        \draw[marrs] (-1.5, 0) -> +(1, 0);
        \draw[marrs] (1, 0) -> +(1.5, 0);
        \draw[marrs] (4, 0) -> +(1.5, 0);
        
        { \huge
            \draw (-2, 0) node {$ys$};
            \draw (0, 0) node {$0$};
            \draw (3, 0) node {$1$};
            \draw (6, 0) node {$7$};
            
            \draw[marrs] (7, 0) -- (8.5, 0) .. controls (8.75, 0) and (9, -0.25) .. (9, -0.5) -- (9, -1.5);
        }
    
    \end{scope}
    
    
    
\end{tikzpicture}\par
	(после)\par
	\vspace{0.5cm}	
	\caption{Выполнение \texttt{ys = update(xs, 2, 7)}. Обратите
		внимание на совместное использование структуры списками \texttt{xs} и \texttt{ys}.}
	\label{fig:2.6}
\end{figure}
\end{frame}

\begin{frame}
\begin{remark}
	Такой стиль программирования очень сильно упрощается при наличии
	автоматической сборки мусора. Очень важно освободить память от тех
	копий, которые больше не нужны, но многочисленные совместно используемые
	узлы делают ручную сборку мусора нетривиальной задачей.
\end{remark}
\end{frame}

\section{Двоичные деревья поиска}

\begin{frame}[fragile]{}
\begin{minipage}{.48\textwidth}
		\documentclass[tikz]{standalone}
%\usepackage{fontawesome}

% \newfontfamily{\FA}{Font Awesome 5 Free} % some glyphs missing
\expandafter\def\csname faicon@facebook\endcsname{{\FA\symbol{"F09A}}}
\def\faQuestionSign{{\FA\symbol{"F059}}}
\def\faQuestion{{\FA\symbol{"F128}}}
\def\faExclamation{{\FA\symbol{"F12A}}}
\def\faUploadAlt{{\FA\symbol{"F093}}}
\def\faLemon{{\FA\symbol{"F094}}}
\def\faPhone{{\FA\symbol{"F095}}}
\def\faCheckEmpty{{\FA\symbol{"F096}}}
\def\faBookmarkEmpty{{\FA\symbol{"F097}}}

%\def\faCatt{{\FA\symbol{"F6BE}}}
%\def\faCat{\faicon{cat}}
%\def\faCat{\faicon{yoast}}
\expandafter\def\csname faicon@dog\endcsname{{\FA\symbol{"F4DA}}}
%\def\faDog{\faicon{dog}}
%\def\faDog{{\FA\symbol{"F4DA}}}
%\def\faDogg{{\FA\symbol{"F6D3}}}
%\def\faDogg{{\FA\symbol{"F596}}}

% /usr/share/texlive/texmf-dist/fonts/opentype/public/fontawesome5/FontAwesome5Free-Solid-900.otf
\newfontfamily{\FAS}{FontAwesome5Free-Solid-900.otf}
%\expandafter\def\csname faicon@download\endcsname{{\FAS\symbol{"F6D3}}}
\expandafter\def\csname faicon@cat\endcsname{{\FAS\symbol{"F6BE}}}
\def\faCat{\faicon{cat}}
\expandafter\def\csname faicon@dog\endcsname{{\FAS\symbol{"F6D3}}}
\def\faDog{\faicon{dog}}
\expandafter\def\csname faicon@dragon\endcsname{{\FAS\symbol{"F6D5}}}
\def\faDragon{\faicon{dragon}}
\expandafter\def\csname faicon@fish\endcsname{{\FAS\symbol{"F578}}}
\def\faFish{\faicon{fish}}
\expandafter\def\csname faicon@horse\endcsname{{\FAS\symbol{"F6F0}}}
\def\faHorse{\faicon{horse}}
\expandafter\def\csname faicon@spider\endcsname{{\FAS\symbol{"F717}}}
\def\faSpider{\faicon{spider}}

\expandafter\def\csname faicon@chessking\endcsname{{\FAS\symbol{"F43F}}}
\def\faChessKing{\faicon{chessking}}
\expandafter\def\csname faicon@chessqueen\endcsname{{\FAS\symbol{"F445}}}
\def\faChessQueen{\faicon{chessqueen}}
\expandafter\def\csname faicon@chessrook\endcsname{{\FAS\symbol{"F447}}}
\def\faChessRook{\faicon{chessrook}}
\expandafter\def\csname faicon@chesspawn\endcsname{{\FAS\symbol{"F443}}}
\def\faChessPawn{\faicon{chesspawn}}
\expandafter\def\csname faicon@chessknight\endcsname{{\FAS\symbol{"F441}}}
\def\faChessKnight{\faicon{chessknight}}
\expandafter\def\csname faicon@chessbishop\endcsname{{\FAS\symbol{"F43A}}}
\def\faChessBishop{\faicon{chessbishop}}
\expandafter\def\csname faicon@chess\endcsname{{\FAS\symbol{"F439}}}
\def\faChess{\faicon{chess}}







\usepackage{tikz}
\usetikzlibrary{positioning,trees,decorations.pathreplacing}

\begin{document}
\begin{tikzpicture}[thick,scale=0.5, every node/.style={scale=0.5},level distance=3cm]
    \tikzstyle{marrs}=[very thick,-latex]
    \tikzstyle{tnode}=[circle, draw=black,node distance=1.7cm]
    \tikzstyle{level 1}=[sibling distance=5.6cm]
    \tikzstyle{level 2}=[sibling distance=3.4cm]


    \huge

    \draw (0, 2.5) node {$xs$};
    \draw[marrs] (0, 2) -- (0, 0.7);
    \node[tnode] {d}
    child {node[tnode] {b}
        child {node[tnode] {a}}
        child {node[tnode] {c}}
    }
    child {node[tnode] {g}
        child {node[tnode] {f}}
        child {node[tnode] {h}}
    };
    
\end{tikzpicture}
\end{document}\par
\end{minipage}
\begin{minipage}{.48\textwidth}
	\begin{tikzpicture}[thick,scale=0.5, every node/.style={scale=0.5},level distance=3cm]
    \tikzstyle{marrs}=[very thick,-latex]
    \tikzstyle{tnode}=[circle, draw=black,node distance=1.7cm]
    \tikzstyle{level 1}=[sibling distance=5.6cm]
    \tikzstyle{level 2}=[sibling distance=3.4cm]
    
    \huge

    \draw (0, 2.5) node {$xs$};
    \draw[marrs] (0, 2) -- +(0, -1.3);
    
    \draw (1.7, 2.5) node {$ys$};
    \draw[marrs] (1.7, 2) -- +(0, -1.3);
    
    \node[tnode] (n_d) {d}
    child {node[tnode] (n_b) {b}
        child {node[tnode] (n_a) {a}}
        child {node[tnode] (n_c) {c}}
    }
    child {node[tnode] (n_g) {g}
        child {node[tnode] (n_f) {f}
        node[tnode] (n_f') [right of=n_f] {f} [clockwise from=250]
        child {node[tnode] (n_e) {e}}
        }
        child {node[tnode] (n_h) {h}}
        node[tnode] (n_g') [right of=n_g] {g}
    }
    node[tnode] (n_d') [right of=n_d]{d};
    
    \path (n_d') edge (n_b);
    \path (n_d') edge (n_g');
    \path (n_g') edge (n_f');
    \path (n_g') edge (n_h);
    
\end{tikzpicture}
\end{minipage}
Выполнение \texttt{ys = insert("e", xs)}. Как и прежде,
обратите внимание на совместное использвание структуры деревьями \texttt{xs} и \texttt{ys}.
\end{frame}


% 
% \begin{frame}[allowframebreaks]
%   \frametitle<presentation>{Ссылки}
%   \begin{thebibliography}{10}
%   \bibitem{paper}
%     \href{http://conal.net/papers/compiling-to-categories/compiling-to-categories.pdf}{paper}
%     \newblock {\em Conal Elliot }    
%   \bibitem{conal}
%     Slides
%     \newblock {\em Conal Elliot }
%     \newblock \href{http://conal.net/talks/compiling-to-categories.pdf}{ссылка}
%   \bibitem{video}    
%     \href{http://podcasts.ox.ac.uk/compiling-categories}{ICFP 2017 video}
%     \newblock {\em Conal Elliot }
%   \bibitem{}
%     \href{https://github.com/conal/concat}{Project repo}
%   \end{thebibliography}
% \end{frame}

\end{document}
