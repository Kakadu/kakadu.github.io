\begin{frame}{Ленивые вычисления}
%Две основые идеи

\begin{idea}[Ленивые вычисления]
Если надо что-то сделать, то выполняем, когда понадобится результат этого действия, т.е. откладываем вычисление на потом
\end{idea}

\begin{idea}[Мемоизация ленивых вычислений]
Если вычисление понадобилось, то мы вычисляем и \emph{запоминаем} результат. Когда оно понадобится кому-то ещё, \emph{вернем уже посчитанный} результат
\end{idea}

\end{frame}

\begin{frame}{Ленивые списки (потоки)}
\begin{definition}[Поток (stream)]
Это список, где вычисления подсписков в нём отложены на потом, а вычисление элементов не отложено на потом (или не обязательно отложено)
\end{definition}
С потоками легко описать, например, "все возможные натуральные числа"\vspace{1em}

\begin{notation}
Добавление элемента $x$ к хвосту $xs$: $\cons{x}{xs}$

Пустой поток: $\nil$

Откладывание на потом $f$: $\texttt{\$}f$
\end{notation}
\begin{remark}
Потоки могут быть конечными, а могут быть бесконечными. Пока до конца не посчитаешь -- не поймешь
\end{remark}
\end{frame}

\begin{frame}{Пример: фибоначчи}
Пусть будет функция $\texttt{zip}: stream\times stream \rightarrow stream$, которая складывает потоки поэлементно.\\

Поток чисел фибоначчи описывается так:
\[
fibs \equiv \cons{1}{zip(fibs, tail(fibs))}
\]
\begin{center}

\only<2>{
\includegraphics[page=1]{tikzpics/fibs.pdf}
}
\only<3>{
\includegraphics[page=2]{tikzpics/fibs.pdf}
}
\only<4>{
\includegraphics[page=3]{tikzpics/fibs.pdf}
}
\only<5>{
\includegraphics[page=4]{tikzpics/fibs.pdf}
}
\only<6>{
\includegraphics[page=5]{tikzpics/fibs.pdf}

и т.д.
}
\end{center}
\end{frame}
