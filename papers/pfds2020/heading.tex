\makeatletter
\@ifclassloaded{beamer}{
  \usetheme{CambridgeUS}
  % get rid of header navigation bar
  \setbeamertemplate{headline}{}
  % get rid of bottom navigation symbols
  \setbeamertemplate{navigation symbols}{}
  % get rid of footer
  %\setbeamertemplate{footline}{}
}
{}
\makeatother
%%%%%%%%%%%%%%%%%%%%%%%%%%%%%%%%%%%%%%%%%%%%%
%\usepackage{fontawesome}
%% \newfontfamily{\FA}{Font Awesome 5 Free} % some glyphs missing
%\expandafter\def\csname faicon@facebook\endcsname{{\FA\symbol{"F09A}}}
%\def\faQuestionSign{{\FA\symbol{"F059}}}
%\def\faQuestion{{\FA\symbol{"F128}}}
%\def\faExclamation{{\FA\symbol{"F12A}}}
%\def\faUploadAlt{{\FA\symbol{"F093}}}
%\def\faLemon{{\FA\symbol{"F094}}}
%\def\faPhone{{\FA\symbol{"F095}}}
%\def\faCheckEmpty{{\FA\symbol{"F096}}}
%\def\faBookmarkEmpty{{\FA\symbol{"F097}}}
\usepackage{fontawesome}

% \newfontfamily{\FA}{Font Awesome 5 Free} % some glyphs missing
\expandafter\def\csname faicon@facebook\endcsname{{\FA\symbol{"F09A}}}
\def\faQuestionSign{{\FA\symbol{"F059}}}
\def\faQuestion{{\FA\symbol{"F128}}}
\def\faExclamation{{\FA\symbol{"F12A}}}
\def\faUploadAlt{{\FA\symbol{"F093}}}
\def\faLemon{{\FA\symbol{"F094}}}
\def\faPhone{{\FA\symbol{"F095}}}
\def\faCheckEmpty{{\FA\symbol{"F096}}}
\def\faBookmarkEmpty{{\FA\symbol{"F097}}}

%\def\faCatt{{\FA\symbol{"F6BE}}}
%\def\faCat{\faicon{cat}}
%\def\faCat{\faicon{yoast}}
\expandafter\def\csname faicon@dog\endcsname{{\FA\symbol{"F4DA}}}
%\def\faDog{\faicon{dog}}
%\def\faDog{{\FA\symbol{"F4DA}}}
%\def\faDogg{{\FA\symbol{"F6D3}}}
%\def\faDogg{{\FA\symbol{"F596}}}

% /usr/share/texlive/texmf-dist/fonts/opentype/public/fontawesome5/FontAwesome5Free-Solid-900.otf
\newfontfamily{\FAS}{FontAwesome5Free-Solid-900.otf}
%\expandafter\def\csname faicon@download\endcsname{{\FAS\symbol{"F6D3}}}
\expandafter\def\csname faicon@cat\endcsname{{\FAS\symbol{"F6BE}}}
\def\faCat{\faicon{cat}}
\expandafter\def\csname faicon@dog\endcsname{{\FAS\symbol{"F6D3}}}
\def\faDog{\faicon{dog}}
\expandafter\def\csname faicon@dragon\endcsname{{\FAS\symbol{"F6D5}}}
\def\faDragon{\faicon{dragon}}
\expandafter\def\csname faicon@fish\endcsname{{\FAS\symbol{"F578}}}
\def\faFish{\faicon{fish}}
\expandafter\def\csname faicon@horse\endcsname{{\FAS\symbol{"F6F0}}}
\def\faHorse{\faicon{horse}}
\expandafter\def\csname faicon@spider\endcsname{{\FAS\symbol{"F717}}}
\def\faSpider{\faicon{spider}}

\expandafter\def\csname faicon@chessking\endcsname{{\FAS\symbol{"F43F}}}
\def\faChessKing{\faicon{chessking}}
\expandafter\def\csname faicon@chessqueen\endcsname{{\FAS\symbol{"F445}}}
\def\faChessQueen{\faicon{chessqueen}}
\expandafter\def\csname faicon@chessrook\endcsname{{\FAS\symbol{"F447}}}
\def\faChessRook{\faicon{chessrook}}
\expandafter\def\csname faicon@chesspawn\endcsname{{\FAS\symbol{"F443}}}
\def\faChessPawn{\faicon{chesspawn}}
\expandafter\def\csname faicon@chessknight\endcsname{{\FAS\symbol{"F441}}}
\def\faChessKnight{\faicon{chessknight}}
\expandafter\def\csname faicon@chessbishop\endcsname{{\FAS\symbol{"F43A}}}
\def\faChessBishop{\faicon{chessbishop}}
\expandafter\def\csname faicon@chess\endcsname{{\FAS\symbol{"F439}}}
\def\faChess{\faicon{chess}}







\newcommand{\faGood}{\textcolor{ForestGreen}{\faThumbsUp}}
\newcommand{\faBad}{\textcolor{red}{\faThumbsODown}}
\newcommand{\faWrong}{\textcolor{red}{\faTimes}}
\newcommand{\faMaybe}{\textcolor{blue}{\faQuestion}}
\newcommand{\faCheckGreen}{\textcolor{ForestGreen}{\faCheck}}
%%%%%%%%%%%%%%%%%%%%%%%%%%%%%%%%%%%%%%%%%%%%%

\usepackage{fontspec}
\usepackage{hyperref}

\setmainfont[
 Ligatures=TeX,
 Extension=.otf,
 BoldFont=cmunbx,
 ItalicFont=cmunti,
 BoldItalicFont=cmunbi,
% Scale = 1.1
]{cmunrm}
\setsansfont[
 Ligatures=TeX,
 Extension=.otf,
 BoldFont=cmunsx,
 ItalicFont=cmunsi,
%  Scale = 1.2
]{cmunss}
%\setmainfont[Mapping=tex-text]{DejaVu Serif}
%\setsansfont[Mapping=tex-text]{DejaVu Sans}
%\setmonofont{Fira Code}[Contextuals=AlternateOff]
%\setmonofont{Fira Code}[Contextuals=Alternate,Scale=0.9]
%\setmonofont{Dejavu Sans Mono}[Contextuals=Alternate,Scale=0.9]
%\newfontfamily{\myfiracode}[Scale=1.5,Contextuals=Alternate]{Fira Code}
%\setmonofont[Scale=0.9,BoldFont={Inconsolata Bold}]{Inconsolata}

% https://tex.stackexchange.com/a/91510
\newfontfamily{\cyrillicfonttt}{Monaco for Powerline}[Contextuals=Alternate,Scale=0.8]
%\newfontfamily{\cyrillicfonttt}{Fira Code}

\usepackage{polyglossia}
\setmainlanguage{russian}
\setotherlanguage{english}


%\newfontfamily\dejaVuSansMono{DejaVu Sans Mono}
% https://github.com/vjpr/monaco-bold/raw/master/MonacoB/MonacoB.otf
%\newfontfamily\monacoB{MonacoB}
%%%%%%%%%%%%%%%%%%%%%%%%%%%%%%%%%%%%%%%%%%%%%%%5
\usepackage{soul} % for \st that strikes through
\usepackage[normalem]{ulem} % \sout

\usepackage{stmaryrd}
\newcommand{\sem}[1]{\ensuremath{\llbracket #1\rrbracket}}
\newcommand\Haskell{\textsc{Haskell}}
\newcommand\haskell{\Haskell}
\newcommand\OCaml{\textsc{OCaml}}

\usepackage{listings}
%\lstdefinestyle{style1}{
%  language=haskell,
%  numbers=left,
%  stepnumber=1,
%  numbersep=10pt,
%  tabsize=4,
%  showspaces=false,
%  showstringspaces=false
%}
%\lstdefinestyle{hsstyle1}
%{ language=haskell
%%          , basicstyle=\monacoB
%         , deletekeywords={Int,Float,String,List,Void}
%         , breaklines=true
%         , columns=fullflexible
%         , commentstyle=\color{ForestGreen}
%         , escapeinside=§§
%         , escapebegin=\begin{russian}\commentfont
%         , escapeend=\end{russian}
%         , commentstyle=\color{ForestGreen}
%         , escapeinside=§§
%         , escapebegin=\begin{russian}\color{ForestGreen}
%         , escapeend=\end{russian}
%         , mathescape=true
%%          , backgroundcolor = \color{MyBackground}
%}
%
%\newcommand{\inline}[1]{\lstinline{haskell}{#1}}
%\def\hsinline{\mintinline{haskell}}
%\def\inline{\hsinline}
%
%\lstnewenvironment{hslisting} {
%    \lstset { style={hsstyle1} }
%  }
%  {}
%
%%%%%%%%%%%%%%%%%%%%%%%%%%%%%%%%%%%%%%%%%%%%%%%%%%%%%%%%%%%
%%\setmainfont[
%% Ligatures=TeX,
%% Extension=.otf,
%% BoldFont=cmunbx,
%% ItalicFont=cmunti,
%% BoldItalicFont=cmunbi,
%%]{cmunrm}
%%% С засечками (для заголовков)
%%\setsansfont[
%% Ligatures=TeX,
%% Extension=.otf,
%% BoldFont=cmunsx,
%% ItalicFont=cmunsi,
%%]{cmunss}
%% \setmonofont[Scale=0.6]{Monaco}
%
%\usefonttheme{professionalfonts}
%\usepackage{times}
\usepackage{tikz}
\usetikzlibrary{cd}
\usepackage{tikz-cd}
\usepackage{caption}
\usepackage{subcaption}

%\renewtheorem{definition}{برهان}[chapter]
%%\DeclareMathOperator{->}{\rightarrow}
%\newcommand\iso{\ensuremath{\cong}}
%\usepackage{verbatim}
%\usepackage{graphicx}
%\usetikzlibrary{arrows,shapes}

%\usepackage{amsmath}
%\usepackage{amsfonts}
\usepackage{scalerel}
\DeclareMathOperator*{\myvee}{\scalerel*{\vee}{\sum}}
\DeclareMathOperator*{\mywedge}{\scalerel*{\wedge}{\sum}}

%
%\usepackage{tabulary}
%
%% sudo aptget install ttf-mscorefonts-installer
%%\setmainfont{Times New Roman}
%%\setsansfont[Mapping=tex-text]{DejaVu Sans}
%
%%\setmonofont[Scale=1.0,
%%    BoldFont=lmmonolt10-bold.otf,
%%    ItalicFont=lmmono10-italic.otf,
%%    BoldItalicFont=lmmonoproplt10-boldoblique.otf
%%]{lmmono9-regular.otf}
%

% color options
\definecolor{YellowGreen} {HTML}{B5C28C}
\definecolor{ForestGreen} {HTML}{009B55}
\definecolor{MyBackground}{HTML}{F0EDAA}
\usepackage{verbatimbox}


%\author{Косарев Дмитрий}
%\institute{JetBrains Reasearch, Лаборатория языковых инструментов\\матмех СПбГУ}
%
%\addtobeamertemplate{title page}{}{
%  \begin{center}{\tiny Дата сборки: \today}\end{center}
%}


%\makeatletter
%\@ifclassloaded{beamer}{
%  % get rid of header navigation bar
%  \setbeamertemplate{headline}{}
%  % get rid of bottom navigation symbols
%  \setbeamertemplate{navigation symbols}{}
%  % get rid of footer
%  %\setbeamertemplate{footline}{}
%}
%{}
%\makeatother
%
%\usepackage{xcolor}
%\definecolor{YellowGreen} {HTML}{B5C28C}
%\definecolor{ForestGreen} {HTML}{009B55}
%\definecolor{MyBackground}{HTML}{F0EDAA}
%
%
%\usepackage{xltxtra} % load xunicode
%\usefonttheme{professionalfonts}
%
%
%
%\newfontfamily\dejaVuSansMono{DejaVu Sans Mono}
%% https://github.com/vjpr/monaco-bold/raw/master/MonacoB/MonacoB.otf
%\newfontfamily\monacoB{MonacoB}
%
%\usepackage{fontspec}
%
%\defaultfontfeatures{Ligatures={TeX}}
%\setmainfont{Times New Roman}
%\setsansfont{CMU Sans Serif}
%
%%%%%%%%%%%%%%%%%%%%%%%%%%%%%%%%%%%%%%%%%%%%%%%%%%%%%%%%%%%
%\setmainfont[
% Ligatures=TeX,
% Extension=.otf,
% BoldFont=cmunbx,
% ItalicFont=cmunti,
% BoldItalicFont=cmunbi,
%]{cmunrm}
%\setsansfont[
% Ligatures=TeX,
% Extension=.otf,
% BoldFont=cmunsx,
% ItalicFont=cmunsi,
%]{cmunss}
%\setmonofont{Fira Code}[Contextuals=Alternate,Scale=0.9]
%\newfontfamily{\myfiracode}[Scale=1.5,Contextuals=Alternate]{Fira Code}
%
%%\newfontfamily\cyrillicfont[Script=Cyrillic]{Charis SIL}
%%%%%%%%%%%%%%%%%%%%%%%%%%%%%%%%%%%%%%%%%%%%%%%%%%%%%%%%%%%

%
%\usepackage{listings}
%\lstdefinestyle{style1}{
%  language=haskell,
%  numbers=left,
%  stepnumber=1,
%  numbersep=10pt,
%  tabsize=4,
%  showspaces=false,
%  showstringspaces=false
%}
%\lstdefinestyle{hsstyle1}
%{ language=haskell
%%          , basicstyle=\monacoB
%         , deletekeywords={Int,Float,String,List,Void}
%         , breaklines=true
%         , columns=fullflexible
%         , commentstyle=\color{ForestGreen}
%         , escapeinside=§§
%         , escapebegin=\begin{russian}\commentfont
%         , escapeend=\end{russian}
%         , commentstyle=\color{ForestGreen}
%         , escapeinside=§§
%         , escapebegin=\begin{russian}\color{ForestGreen}
%         , escapeend=\end{russian}
%         , mathescape=true
%%          , backgroundcolor = \color{MyBackground}
%}
%

\usepackage[cache=true]{minted}
\usemintedstyle{perldoc}
%\def\hsinline{\mintinline{haskell}}
\def\mlinline{\mintinline[mathescape=true,escapeinside=||]{ocaml}}

%\lstdefinelanguage{ocaml}{
%keywords={fresh, conde, let, begin, end, in, match, type, and, fun, function, try, with, when, class,
%object, method, of, rec, repeat, until, while, not, do, done, as, val, inherit,
%new, module, sig, deriving, datatype, struct, if, then, else, open, private, virtual, include, @type},
%sensitive=true,
%commentstyle=\small\itshape\ttfamily,
%keywordstyle=\ttfamily\underbar,
%identifierstyle=\ttfamily,
%basewidth={0.5em,0.5em},
%columns=fixed,
%fontadjust=true,
%literate={->}{{$\to\;\;$}}3 {*}{{$\times$}}3 {=/=}{{$\not\equiv$}}3 {|>}{{$\triangleright$}}3,
%morecomment=[s]{(*}{*)}
%}
%
%\usepackage{listings}
%\def\mlinline{\lstinline[language=ocaml]}

% TODO: minted и lstlisting дают ту же производительность компиляции.
% Разобраться должно ли быть так

%
%\lstnewenvironment{hslisting} {
%    \lstset { style={hsstyle1} }
%  }
%  {}
%
%
%\usefonttheme{professionalfonts}
%\usepackage{times}
%\usepackage{tikz}
%\usetikzlibrary{cd}
%% \usepackage{tikz-cd}
%\usepackage{amsmath}
%%\DeclareMathOperator{->}{\rightarrow}
%\newcommand\iso{\ensuremath{\cong}}
%\usepackage{verbatim}
%\usepackage{graphicx}
%\usetikzlibrary{arrows,shapes}
%
%%\usepackage{fontawesome}
%%% \newfontfamily{\FA}{Font Awesome 5 Free} % some glyphs missing
%%\expandafter\def\csname faicon@facebook\endcsname{{\FA\symbol{"F09A}}}
%%\def\faQuestionSign{{\FA\symbol{"F059}}}
%%\def\faQuestion{{\FA\symbol{"F128}}}
%%\def\faExclamation{{\FA\symbol{"F12A}}}
%%\def\faUploadAlt{{\FA\symbol{"F093}}}
%%\def\faLemon{{\FA\symbol{"F094}}}
%%\def\faPhone{{\FA\symbol{"F095}}}
%%\def\faCheckEmpty{{\FA\symbol{"F096}}}
%%\def\faBookmarkEmpty{{\FA\symbol{"F097}}}
%%
%%\newcommand{\faGood}{\textcolor{ForestGreen}{\faThumbsUp}}
%%\newcommand{\faBad}{\textcolor{red}{\faThumbsODown}}
%%\newcommand{\faWrong}{\textcolor{red}{\faTimes}}
%%\newcommand{\faMaybe}{\textcolor{blue}{\faQuestion}}
%%\newcommand{\faCheckGreen}{\textcolor{ForestGreen}{\faCheck}}
%
%\usepackage{soul} % for \st that strikes through
%\usepackage[normalem]{ulem} % \sout

\newcommand\mdoubleplus{\ensuremath{\mathbin{+\mkern-10mu+}}}
\newcommand\concat{\mdoubleplus}
\newcommand\card[1]{\ensuremath{\mid\!\!#1\!\mid}}
\newcommand\nil{\ensuremath{\texttt{\$Nil}}}
\newcommand\cons[2]{\ensuremath{\texttt{\$Cons}(#1,#2)}}

\def\Oconst{\ensuremath{O(1)}}
\def\Ologn{\ensuremath{O(log~n)}}

\usepackage{amsthm}
\newtheorem{remark}{\textbf{Замечание}}[section]
\newtheorem{notation}{\textbf{Нотация}}[section]
\newtheorem{idea}{\textbf{Идея}}
\newtheorem{hint}{\textbf{Указание разработчикам}}[section]
\deftranslation[to=russian]{Theorem}{Теорема}
\deftranslation[to=russian]{theorem}{теорема}