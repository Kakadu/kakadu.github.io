 
\subsection{Левоориентированные кучи}

\begin{frame}{Левоориентированные (leftist) кучи}
Как правило, множества и конечные отображения поддерживают эффективный
доступ к произвольным элементам. \\

Однако иногда требуется эффективный
доступ только к \emph{минимальному} элементу.  Структура данных,
поддерживающая такой режим доступа, называется \term{очередь с
приоритетами}{priority queue} или \term{куча}{heap}.\\

Операции:
\begin{itemize}
\item Вставка: \mlinline{int|$\times$|heap -> heap}
\item Слияние: \mlinline{heap|$\times$|heap -> heap}
\item Минимум: \mlinline{heap -> int} (если пустая -- исключение)
\item Удаление минимума: \mlinline{heap -> heap} (если пустая -- исключение)
\end{itemize}
\end{frame}

\begin{frame}{}
\begin{figure}[ht]
\begin{subfigure}{.7\textwidth}
\begin{definition}[\term{Порядок кучи}{heap-ordered}]
Элемент при каждой вершине не больше элементов в поддеревьях.
\end{definition}
При таком упорядочении минимальный элемент дерева всегда находится в корне, \emph{но это не дерево поиска}\\

\begin{definition}[\term{Правая периферия}{right spine} узла]
Самого правый путь от данного узла до пустого
\end{definition}
\begin{definition}[Ранг]
Ранг узла --- длина его правой периферии. 
\end{definition}

\end{subfigure}\hspace{1em}
\begin{subfigure}{.25\textwidth}
\includegraphics[page=4,scale=1.3]{tikzpics/LeftistExample.pdf}
\end{subfigure}
\end{figure}


%Левоориентированные кучи \cite{Crane1972, Knuth1973a} представляют
%собой двоичные деревья с порядком кучи, обладающие свойством
%левоориентированности.

\end{frame}

\begin{frame}[fragile]{Левоориентированные кучи}
\begin{definition}[Свойство \term{левоориентированности}{leftist property}]
Ранг любого левого поддерева не меньше ранга его сестринской правой вершины. 
\end{definition}
Простым
следствием свойства левоориентированности является то, что правая
периферия любого узла~--- кратчайший путь от него к пустому узлу.\\


%\inputminted[firstline=6, lastline=6] {haskell}{code/LeftistHeap.lhs}
\begin{block}{Представление левоорентированных куч}
Двоичные деревья, снабженные информацией о ранге, т.е.
\begin{itemize}
\item В листьях ничего нет (и ранг всегда 0)
\item В узлах: хранимый элемент, два поддерева и ранг (int)
\end{itemize}
\end{block}

Заметим, что элементы правой периферии левоориентированной кучи (да и
любого дерева с порядком кучи) расположены в порядке возрастания.\\

%Главная идея левоориентированной кучи заключается в том, что для
%слияния двух куч достаточно слить их правые периферии как
%упорядоченные списки, а затем вдоль полученного пути обменивать
%местами поддеревья при вершинах, чтобы восстановить свойство
%левоориентированности. 

\end{frame}

\begin{frame}[fragile]{Слияние}
\begin{idea}
Достаточно слить их правые периферии как
упорядоченные списки, а затем вдоль полученного пути обменивать
местами поддеревья при вершинах, чтобы восстановить свойство
левоориентированности.
\end{idea}

Функция \mlinline{merge: heap|$\times$|heap -> heap} 
\begin{itemize}
\item Если одна из куч пустая -- возвращаем другую
\item Если имеем два узла: \mlinline{h1}, состоящий из \mlinline{(x,l1,r1)} и  \mlinline{h2} --- \mlinline{(x,l2,r2)} 
\begin{itemize}
\item При x$\leqslant$y возвращаем \mlinline{makeT(x,l1, merge(r1, h2))}
\item Иначе \mlinline{makeT(y,l2, merge(h1, r2))}
\end{itemize}
\end{itemize}

Функция \mlinline{makeT: int|$\times$|heap|$\times$|heap -> heap} принимает \mlinline{(x,l,r)}:
\begin{itemize}
\item Если \mlinline{rank(l) >= rank(r)} то строим дерево из \mlinline{(1+rank(b), x, a, b)}
\item Иначе как \mlinline{(1+rank(a), x, b, a)}
\end{itemize}
%где \hsinline{makeT}~--- вспомогательная функция, вычисляющая ранг
%вершины \hsinline{T} и, если необходимо, меняющая местами ее
%поддеревья.

%\inputminted[firstline=8, lastline=13] {haskell}{code/LeftistHeap.lhs}

Поскольку длина правой периферии любой вершины в худшем случае
логарифмическая, \hsinline{merge} выполняется за время $O(\log n)$.
\end{frame}

\def\Olog{\ensuremath{O(log\ n)}}
\begin{frame}[fragile]{Итого: сложность левоориентированных куч }
%\inputminted[firstline=27, lastline=32] {haskell}{code/LeftistHeap.lhs}

\begin{itemize}
\item Слияние  (\Olog)
\item Минимум -- заглядывание в корень ($O(1)$)
\item Вставка -- это слияние с одноэлементным деревом (\Olog)
\item Удаление -- слияние левого поддерева с правым (\Olog)
\end{itemize}
%Поскольку \hsinline{merge} выполняется за время $O(\log n)$, столько
%же занимают и \hsinline{insert} с \hsinline{deleteMin}.\\
%
%Очевидно, что \hsinline{findMin} выполняется за $O(1)$. 
\end{frame}

\begin{frame}{Пример: слияние двух левоориентированных куч}
\begin{figure}[ht]
\begin{subfigure}{.25\textwidth}
\includegraphics[page=1,scale=1.3]{tikzpics/LeftistExample.pdf}
\end{subfigure}
\begin{subfigure}{.25\textwidth}
\includegraphics[page=2,scale=1.3]{tikzpics/LeftistExample.pdf}
\end{subfigure}
\begin{subfigure}{.1\textwidth}
\Large $\Rightarrow$
\end{subfigure}
\begin{subfigure}{.3\textwidth}
\includegraphics[page=3,scale=1.3]{tikzpics/LeftistExample.pdf}
\end{subfigure}
\end{figure}
\end{frame}


\ifanswers
\begin{frame}[fragile]{}
\begin{exercise}\label{ex:3.2}
 Определите \hsinline{insert} напрямую, а не через обращение к \hsinline{merge}.
\end{exercise}

\begin{exercise}\label{ex:3.3}
 Реализуйте функцию \hsinline{fromList} типа \hsinline{Elem.T list $\to$ Heap},
 порождающую левоориентированную кучу из неупорядоченного списка
 элементов путем преобразования каждого элемента в одноэлементную
 кучу, а затем слияния получившихся куч, пока не останется
 одна. Вместо того, чтобы сливать кучи проходом слева направо или
 справа налево при помощи \hsinline{foldr} или \hsinline{foldl},
 слейте кучи за $\lceil \log n \rceil$ проходов, где на каждом
 проходе сливаются пары соседних куч. Покажите, что
 \hsinline{fromList} требует всего $O(n)$ времени.
\end{exercise}
\end{frame}

\begin{frame}[fragile]{}
\begin{exercise}\label{ex:3.4}
 Левоориентированные кучи
 со сдвинутым весом~--- альтернатива левоориентированным кучам, где
 вместо свойства левоориентированности используется свойство
 \term{левоориентированности, сдвинутой по весу}{weight-biased leftist
   property}: размер любого левого поддерева всегда не меньше размера
 соответствующего правого поддерева.
 \begin{enumerate}
   \item Докажите, что правая периферия левоориентированной кучи со
   сдвинутым весом содержит не более $\lfloor \log(n+1) \rfloor$ элементов.
   \item Измените реализацию, чтобы получились
   левоориентированные кучи со сдвинутым весом.
   \item Функция \lstinline!merge! сейчас выполняется в два прохода:
   сверху вниз, с вызовами \lstinline!merge!, и снизу вверх, с
   вызовами вспомогательной функции \lstinline!makeT!. Измените
   \lstinline!merge! для левоориентированных куч со сдвинутым весом
   так, чтобы она работала за один проход сверху вниз.
   \item Каковы преимущества однопроходной версии в
   условиях ленивого вычисления? Параллельного?
 \end{enumerate}
\end{exercise}
\end{frame}
\fi 


\subsection{Биномиальные кучи}

\begin{frame}{Биномиальные кучи}
Биномиальные очереди \cite{Vuillemin1978, Brown1978}, которые мы,
чтобы избежать путаницы с очередями FIFO, будем называть \term{ биномиальными
 кучами}{binomial heaps}~--- ещё одна распространенная реализация
куч. \\

Биномиальные кучи устроены сложнее, чем левоориентированные, и, на
первый взгляд, не возмещают эту сложность никакими
преимуществами. \\

Однако, с помощью дополнительных ухищрений (избавление от амортизации), можно заставить \hsinline{insert} и
\hsinline{merge} выполняться за время $O(1)$.\\

Биномиальные кучи строятся из более простых объектов, называемых
 биномиальными деревьями. 
\end{frame}

\begin{frame}[fragile]{Биномиальные деревья. Пример}
\begin{figure}[ht]
\begin{subfigure}{.58\textwidth}
\only<1>{
\begin{definition}[Биномиальные деревья (опр. 1, индуктивное)]
\begin{itemize}
\item Биномиальное дерево ранга 0 представляет собой одиночный узел.
\item Биномиальное дерево ранга $r+1$ получается путем
  \term{связывания}{linking} двух биномиальных деревьев ранга $r$, так
что одно из них становится самым левым потомком второго.
\end{itemize}
\end{definition}
Из этого определения видно, что биномиальное дерево ранга $r$ содержит
 ровно $2^r$ элементов. 
}
\only<2>{
Существует второе, эквивалентное первому, определение биномиальных деревьев, которым иногда удобнее пользоваться.
 
\begin{definition}[Биномиальные деревья (опр. 2)]
Биномиальное дерево ранга $r$ представляет собой узел с $r$ потомками $t_1\ldots t_r$, где каждое $t_i$ является биномиальным деревом ранга $(r-i)$.
\end{definition}
}
\end{subfigure}\hspace{5mm}
\begin{subfigure}{.37\textwidth}
\includegraphics[scale=1]{tikzpics/BinomialTrees.pdf}
\end{subfigure}
\end{figure}
\end{frame}

%\begin{frame}{Биномиальные деревья. Определение}
%\begin{definition}[Биномиальные деревья (опр. 1, индуктивное)]
%\begin{itemize}
%\item Биномиальное дерево ранга 0 представляет собой одиночный узел.
%\item Биномиальное дерево ранга $r+1$ получается путем
%  \term{связывания}{linking} двух биномиальных деревьев ранга $r$, так
%что одно из них становится самым левым потомком второго.
%\end{itemize}
%\end{definition}
%Из этого определения видно, что биномиальное дерево ранга $r$ содержит
% ровно $2^r$ элементов.  \\
%
% \end{frame}
 
 
 
 
\begin{frame}[fragile]{}
Реализация биномиальных деревьев:
\begin{itemize}
\item Храним узлы с рангом, значением, список деревьев-потомков
\item Потомки хранятся в порядке \emph{убывания}\footnote{Это будет важно при удалении} ранга
\item Элементы подвешиваются согласно "порядку кучи" (деревья с большими корнями подвешиваются к узлам с меньшими)
\end{itemize}

Элементы хранятся в порядке кучи.  Чтобы сохранять этот порядок кучи, мы всегда
подцепляем дерево с большим корнем к узлу с меньшим.\\
 
Функция \mlinline{link: heap|$\times$|heap -> heap} принимает дерево \mlinline{t1} из \mlinline{(r,x1,c1)} и дерево \mlinline{t2} из \mlinline{(r2,x2,c2)}
\begin{itemize}
\item Если \mlinline{x1<x2} строим дерево из \mlinline{(r+1, x1, t2::c1)}
\item Иначе \mlinline{(r+1, x2, t1::c1)}
\item Инвариант: связываем деревья только с одинаковым рангом: \mlinline{assert(r1===r2)}
\end{itemize}
\end{frame}
 
 
\begin{frame}[fragile]{}
Реализация биномиальной кучи:
\begin{itemize}
\item Список биномиальных деревьев с "порядком кучи"
\item Отсортированный в порядке \emph{возрастания}\footnote{Это будет важно при удалении} рангов
\end{itemize}
\vspace{1em}

Поскольку каждое биномиальное дерево содержит $2^r$ элементов, и
 никакие два дерева по рангу не совпадают, деревья размера $n$ в
 точности соответствуют единицам в двоичном представлении
 $n$.\\
 
Например, число $21_{10} = 10101_2$, и поэтому биномиальная куча размера 21 содержит одно дерево ранга 0, одно ранга 2, и одно ранга 4 (размерами, соответственно, 1, 4 и 16).\\
 
Заметим, что так же, как двоичное представление $n$ содержит не более $\lfloor log (n+1)\rfloor$ единиц, биномиальная куча размера $n$ содержит не более $\lfloor log(n+1) \rfloor$ деревьев.
\end{frame}
 
 
 \begin{frame}[fragile]{}
 \begin{figure}[h]
   \centering
   \begin{tikzpicture}[thick,scale=0.5, every node/.style={scale=0.5},grow via three points={%
one child at (0,-1.5) and 
two children at (0,-1.5) and (-0.8,-1.5) 
}
]
    \tikzstyle{marrs}=[very thick,-latex]
    \tikzstyle{tnode}=[circle, fill=black, inner sep=1.5mm]
    \def\rstep{5cm}
    
    \huge
    
    \node[tnode] (0, 0) {};
            child { node[tnode] {} }
            child { node[tnode] {} };
    
    \begin{scope}
        \draw (0, 1) node {Ранг 0};
        \node[tnode] (0, 0) {};
    \end{scope}
        
    \begin{scope}[xshift=1 * \rstep]
        \draw (0, 1) node {Ранг 2};
        \node[tnode] {}
            child {node[tnode] {} }
            child {node[tnode] {} 
                    child {node[tnode] {} }};
    \end{scope}
    
    \begin{scope}[xshift=3 * \rstep]
      \draw (0, 1) node {Ранг 4};
      \node[tnode] {}
          child {node[tnode] {} }
          child {node[tnode] {} 
            child {node[tnode] {} }
          }
          child {node[tnode] {} 
            child {node[tnode] {} }
            child {node[tnode] {} 
              child {node[tnode] {} }
            }
          }
        child[missing] {}
        child {node[tnode] {}
          child {node[tnode] {} }
          child {node[tnode] {} 
            child {node[tnode] {} }
          }
          child {node[tnode] {} 
            child {node[tnode] {} }
            child {node[tnode] {} 
              child {node[tnode] {} }
            }
          }
     };
      \end{scope}
    
    
\end{tikzpicture}

   \caption{Число $21_{10} = 10101_2$, и поэтому
     биномиальная куча размера 21 содержит одно дерево ранга 0, одно ранга
     2, и одно ранга 4 (размерами, соответственно, 1, 4 и 16).}
 \end{figure}
 
 \end{frame}
 
 
\begin{frame}{Вставка аналогично инкременту}
Чтобы внести элемент в кучу, мы сначала создаем одноэлементное дерево (т.~е., биномиальное дерево ранга 0), затем поднимаемся по списку существующих деревьев в порядке возрастания рангов, связывая при этом одноранговые деревья. Каждое связывание соответствует переносу в двоичной арифметике.\\
 
Функция \mlinline{insTree: tree |$\times$| tree list -> tree}
\begin{itemize}
\item Вставка в пустой список возвращает одноэлементный
\item Нужно посмотреть на вставляемое дерево $t$ и на головное дерево \mlinline{t2} из списка \mlinline{ts} 
\begin{itemize}
\item Если \mlinline{rank(t)<rank(t2)}, то вернуть \mlinline{t::ts} 
\item Иначе вернуть  \mlinline{insTree(link(t,t2), tail(ts))}
\end{itemize}
\end{itemize}
\vspace{1em}
%\inputminted[firstline=8,lastline=8] {haskell}{code/BinomialHeap.lhs}
%\inputminted[firstline=17,lastline=19] {haskell}{code/BinomialHeap.lhs}
%\inputminted[firstline=38,lastline=38,gobble=2] {haskell}{code/BinomialHeap.lhs}

В худшем случае, при вставке в кучу размера $n = 2^k -1$, требуется
 $k$ связываний и $O(k) = O(\log n)$ времени.
\end{frame}
 
 
\begin{frame}{Слияние -- аналогично сложению}
 При слиянии двух куч мы проходим через оба списка деревьев в порядке
 возрастания ранга и связываем по пути деревья равного ранга. Как и
 прежде, каждое связывание соответствует переносу в двоичной
 арифметике.\\

Функция \mlinline{merge: heap * heap -> heap}
\begin{itemize}
\item Если одна из куч пустая, то возвращаем другую
\item Иначе у нас есть \mlinline{ts1===t1::ts1'} и \mlinline{ts2===t2::ts2'} 
\item если \mlinline{rank(t1)<rank(t2)}, выдаем \mlinline{t1 :: merge (ts1', ts2)}
\item если \mlinline{rank(t2)<rank(t1)}, выдаем \mlinline{t2 :: merge (ts1, ts2')}
\item если равны, то \mlinline{insTree(link(t1,t2), merge (ts1', ts2')}
\end{itemize}
 \end{frame}
 
\begin{frame}{Операции работы с минимумом}
Дополнительная функция \mlinline{removeMinTree: tree list -> tree|$\times$|(tree list)}\\
удаляет из списка дерево с минимальным значением в корне, и выдает это дерево и остаток списка\\

Функция \mlinline{findMin} -- вызвать \mlinline{removeMinTree} и заглянуть в корень полученного дерева\\

Функция \mlinline{delete} -- вызвать \mlinline{removeMinTree}, взять список потомков полученного дерева, перевернуть и слить с учетом рангов со вторым списком
\end{frame}
 
% \begin{frame}[fragile]{Поиск минимального элемента}
% Функции \hsinline{findMin} и \hsinline{deleteMin} вызывают
% вспомогательную функцию \hsinline{removeMinTree}, которая находит
% дерево с минимальным корнем, исключает его из списка и возвращает как
% это дерево, так и список оставшихся деревьев.
% 
% \inputminted[firstline=28,lastline=32]{haskell}{code/BinomialHeap.lhs}
% 
% Функция \hsinline{findMin} просто возвращает корень найденного дерева
% 
% \inputminted[firstline=40,lastline=41,gobble=2] {haskell}{code/BinomialHeap.lhs}
% \inputminted[firstline=9,lastline=9] {haskell}{code/BinomialHeap.lhs}
% \end{frame}
 
% \begin{frame}[fragile]{Удаление минимального элемента}
% Функция \hsinline{deleteMin} устроена немного похитрее. \\
% 
%  Отбросив
% корень найденного дерева, мы ещё должны вернуть его потомков в список
% остальных деревьев. Заметим, что список потомков \emph{почти} уже
% соответствует определению биномиальной кучи. Это коллекция
% биномиальных деревьев с неповторяющимися рангами, но только
% отсортирована она не по возрастанию, а по убыванию ранга. Таким
% образом, обратив список потомков, мы преобразуем его в биномиальную
% кучу, а затем сливаем с оставшимися деревьями.
% 
% \inputminted[firstline=43,lastline=44,gobble=2] {haskell}{code/BinomialHeap.lhs}
% \end{frame}



\begin{frame}
\begin{center}
%\begin{tabular}{ |>{\centering\arraybackslash}p{3cm}|>{\centering\arraybackslash}p{2cm}|>{\centering\arraybackslash}p{3cm}|>{\centering\arraybackslash}p{2.5cm}|>{\centering\arraybackslash}p{2cm}|>{\centering\arraybackslash}p{2cm}| } 
% \hline
%Оп-я \textbackslash Куча  & Leftist & Биномиальная & Бин. + амортизация & Бин. + расписания & 11\\ \hline
% findMin  & \Oconst & \Oconst\footnote{Изложено \Ologn{}, но можно сделать \Oconst{}} & & & \\  \hline
% deleteMin &  \Ologn{} & \Ologn{} &  & \\  \hline
% insert &\Ologn{} & \Ologn{} & \Oconst$^*$ & \Oconst & \\  \hline
% 
%  merge & \Ologn{} & \Ologn{} & & & \\ \hline
%\end{tabular}
\begin{tabular}{ |>{\centering\arraybackslash}p{3cm}|>{\centering\arraybackslash}p{2cm}|>{\centering\arraybackslash}p{3cm}|>{\centering\arraybackslash}p{2.5cm}|>{\centering\arraybackslash}p{2cm}| } 
 \hline
 Куча\textbackslash Операция  & findMin & deleteMin & insert & merge \\ \hline
 Leftist  & \Oconst & \Ologn{} &\Ologn{}& \Ologn{} \\  \hline
 Биномиальная &  \Oconst\footnote{Изложено \Ologn{}, но можно сделать \Oconst{}} & \Ologn{} & \Ologn{} &\Ologn{} \\  \hline
 Бин. + амортизация &  &   & \Oconst$^*$ &  \\  \hline
  Бин. + расписания &  &   & \Oconst &  \\ \hline
bootstrapped & \Oconst & \Ologn{} &\Oconst  & \Oconst \\ \hline
\end{tabular}
\end{center}

Пропуск означает, что точную оценку забыли подсмотреть в литературе

Амортизированные оценки обозначаются с$^*$.
\end{frame}


 \ifanswers
 \begin{frame}[fragile]{}
 \begin{exercise}\label{ex:3.5}
   Определите \lstinline!findMin! напрямую, без обращения к \lstinline!removeMinTree!.
 \end{exercise}
 
 \begin{exercise}\label{ex:3.6}
   Большая часть аннотаций ранга в нашем представлении биномиальных куч
   излишня, потому что мы и так знаем, что дети узла ранга $r$ имеют
   ранги $(r\!-\!1), \ldots, 0$. Таким образом, можно исключить
   поле-аннотацию ранга из узлов, а вместо этого помечать ранг корня
   каждого дерева, т.~е.,
   \begin{minted}{haskell}
   data Tree a = Node  a [Tree]
   type Heap = [(Int, Tree)]
   \end{minted}
   Реализуйте биномиальные кучи в таком представлении.
 \end{exercise}
 \end{frame}
 
 % Ещё одно упражнение не скопипастьил
 
 \fi
 