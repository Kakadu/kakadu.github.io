\newcommand\exscore[3]{#1/#2/#3}


\begin{frame}{Общие замечания по упражнениям}
Если явно не оговорено иное, то...
\vspace{2em}

Предлагается реализовать устойчивую структуру данных на вашем любимом языке обывательском (C\#, C, etc), и на каком-нибудь функциональном языке (\OCaml{}, \Haskell{}, F\#, Scala 3), а затем сравнить размер/сложность двух реализаций.

Ожидаются реализации в виде чистых функций (ну может понадобится присваивание для эмуляции мемоизации, в остальном его использовать нельзя)
\vspace{2em}

%Обозначение \exscore{A}{B}{C} говорит, что за решение на \sout{обывательском}вашем любимом языке будет начислено A баллов, на классическом функциональном B; С баллов начисляется дополнительно, если обучающийся может сравнить реализации, указать на преимущества и недостатки одной и второй, и т.д. Итого за задачу можно будет получить максимум A+B+C баллов
\end{frame}



\begin{frame}[allowframebreaks]{Упражнения на ленивость (максимум 8 баллов)}
\begin{exercise}[2 балла]
По аналогии с вычислением последовательности фибоначчи, сделайте вычисление простых чисел решетом Эратосфена
\end{exercise}

\begin{exercise}[2 балла]
Дан поток потоков чисел $xss$. Функция \mlinline{merge} должна объединить $xss$ в один поток чисел. \\
Ограничение: $\forall (i<\infty) \forall (j<\infty)\quad \big((xss[i][j]= n)\quad \Longrightarrow \quad\exists (k<\infty)\quad (merge(xss)[k]= n)\big)$
\end{exercise}

\begin{exercise}[4 балла]
Дано дерево с числами только в листьях. Построить новое дерево, где все числа предыдущего дерева заменены на минимум от этих чисел. Ограничение: за один проход.
\begin{remark}
Крайне рекомендуется использовать язык, где все вычисления ленивы по умолчанию (например, \Haskell)
\end{remark}
\end{exercise}
\end{frame}



\begin{frame}{Упражнения на очереди (14 баллов максимум)}
Реализуйте:
\begin{itemize}
\item чисто функциональную очередь 
\item очередь банкира 
\item очередь реального времени 
\end{itemize}
Награды:
\begin{itemize}
\item (4=2+1+1) балла за условный \textsc{Python}
\item (7=3+2+2) баллов за условный \OCaml{}
\item Сравнить всех со всеми: очереди между собой и реализации на разных языках (+3)
\end{itemize}
\end{frame}


%
%\begin{frame}[allowframebreaks]{Упражнения на очереди}
%\begin{exercise}[\exscore{?}{?}{?}]
%Реализуйте чисто функциональную очередь
%\end{exercise}
%
%\begin{exercise}[\exscore{?}{?}{?}]
%Реализуйте очередь банкира
%\end{exercise}
%
%\begin{exercise}[\exscore{?}{?}{?}]
%Реализуйте очередь реального времени
%\end{exercise}
%\end{frame}


\begin{frame}{Упражнения на деревья (максимум 15)}
\begin{exercise}
Реализуйте красно-черное дерево, где балансировка делает меньше проверок (упражнение в книге 3.10) \\
\begin{itemize}
\item 2 балла за условный \textsc{Python}
\item 3 балла за условный \OCaml{}
\item 1 за сравнение всех со всеми
\end{itemize}
\end{exercise}
\begin{exercise}
\begin{itemize}
\item Реализуйте префиксное дерево и HAMT
\item 3 балла за условный \textsc{Python}
\item 4 балла за условный \OCaml{}
\item 2 за сравнение всех со всеми
\end{itemize}
\end{exercise}
\end{frame}



\begin{frame}{Упражнения на кучи (20 баллов максимум)}

Реализуйте:
\begin{itemize}
\item левоориентированную кучу
\item weight-biased левоориентированную кучу (упражнение в книге 3.4)
\item  биномиальную кучу, храня аннотации ранга реже (упражнение в книге 3.6)
\item биномиальную кучу с явным минимумом (упражнение в книге 3.7)
\end{itemize}

Награды
\begin{itemize}
\item (6=2+1+2+1) балла за условный \textsc{Python}
\item (10=3+2+3+2) баллов за условный \OCaml{}
\item Сравнить всех со всеми: очереди между собой и реализации на разных языках (+4)
\end{itemize}


\end{frame}


