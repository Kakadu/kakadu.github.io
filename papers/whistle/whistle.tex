\documentclass[a5paper,12pt]{article}
\usepackage[cache=true]{minted}
\usepackage{polyglossia}
\setmainlanguage{russian}
\let\cyrillicfonttt\monofamily
\usepackage{comment}
\usepackage{stmaryrd}

\usepackage[ left=1cm
           , right=2cm
           , top=1cm
           , bottom=1.5cm
           ]{geometry}
\usepackage{amssymb,amsmath,amsthm,amsfonts} 

\usepackage{fontspec}
% \DeclareMathSizes{22}{30}{24}{20}

%\usefonttheme{professionalfonts}
\defaultfontfeatures{Ligatures={TeX}}
%\setmainfont[Scale=1.5]{Times New Roman}
%\setmainfont{Latin Modern Roman}
\setmainfont [ Scale=1]{CMU Serif Roman}
\setsansfont[Scale=1]{CMU Sans Serif}

%\setmonofont[ BoldFont=lmmonolt10-bold.otf
%			, ItalicFont=lmmono10-italic.otf
%			, BoldItalicFont=lmmonoproplt10-boldoblique.otf
%			, Scale=1.5
%]{lmmono9-regular.otf}
%\setmonofont[Scale=1.5]{CMU Typewriter Text}
\setmonofont{CMU Typewriter Text}

\usepackage{unicode-math}
\setmathfont{Latin Modern Math}[Scale=1]
\newcommand*{\arr}{\ensuremath{\rightarrow}}

% Doesn't work?
\renewcommand{\epsilon}{\ensuremath{\varepsilon}}
%\renewcommand{\sigma}{\ensuremath{\varsigma}}
\newcommand{\inbr}[1]{\left<{#1}\right>}
\newcommand{\ruleno}[1]{\mbox{[\textsc{#1}]}}
\newcommand{\bigslant}[2]{{\raisebox{.2em}{$#1$}\left/\raisebox{-.2em}{$#2$}\right.}}
\newcommand{\sem}[1]{\llbracket #1 \rrbracket}
\begin{document}


\section{Свистки}

Свистки на откладывание конъюнкта
\begin{itemize}
\item \textit{Имени Розплохаса} Аргументы \textit{нового} вызова более общие, чем аргументы какого-то \textit{предыдущего} вызова.\\
Критерий расходимости -- этот конъюнкт никогда не досчитается.
Консервативный (редко свистит) и достаточный (конъюнкт точно расходится)

\item \textit{Гомеоморфное вложение} в текущее дерево стрима, какого-то дерева стрима, полученного до этого.\\
\textbf{Недостатки}: тормозит.\\
\textbf{Замечание}: это свисток не на обрыв, а свисток на откладывание вычислений в текущем дереве стрима до тех пор, пока не уточнится подстановка в парном дереве стрима.

\end{itemize}

Свистки на гарантию пустого стрима ответов
\begin{itemize}
\item \textit{Имени Kakadu} Аргументы какого-то \textit{старого} вызова более общие, чем аргументы \textit{нового } вызова\\
Сам по себе ничего не говорит, надо знать об ответах, которые случились между двумя вызовами. Если их не было, то и не будет (можно прерваться). Если они были, то ничего нельзя сказать.

\item \textit{$\alpha$-эквивалентность} Сам по себе без знания ответов детектирует расходимость. При учете ответов похож на мемоизацию

\end{itemize}

\section{Свисток имени Kakadu}

\begin{minted}{ocaml}
let smart_or f g st = 
  match f st with 
  | Nil -> g st 
  | cons (x,xs) -> cons (x, (g ||| xs st))
  | Whistle t -> 
    match g st with 
    | Nil -> Whistel t
    | cons (ys, ys) -> cons (y, (f ||| ys st))
    | Whistle tau -> Whistle (t ||| tau st)

let smart_and f g st = 
  match f st with 
  | nil -> nil 
  | cons (x, xs) -> mplus (g x) ( xs &&& g st)
  | Whistle t -> nil
\end{minted}

\textit{Замечание от Петра. } Сейчас не понятно, чтобы будет если две части дизъюнкции свистнули по разным реляциям. Что возвращать наверх? И вообще, что возвращать наверх? Текущее дерево поиска решений: дизъюнкцию конъюнкций гоалов?


\section{Параллельная конъюнкция}
Идея: вычисляем два конъюнкта параллельно (конкурентно), когда один из них досчитается, то приостанавливаем вычисление второго и действуем как с обычной конъюнкцией.\\

Пример ($\bigwedge$ -- параллельная конъюнкция)

$$
(x\equiv 1) \vee ((1\equiv 1) \wedge (1\equiv 1) \wedge (0\equiv 1)) \bigwedge ( (1\equiv 1) \wedge (x\equiv 1))
$$
% Левый конъюнкт выплевывает ответ $(x\equiv 1)$
Основной вопрос: если левая часть $f$ выдала ответ и continuation для вычислений $c$, то имеем ли мы право продолжать делать параллельную конъюнкцию $c$ с $g$?

В четверг на доске был пример где такое вроде как вызывает дуплицирование ответов, нужно этот пример аккуратненько записать и проверить.

\end{document}
