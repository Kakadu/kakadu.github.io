\documentclass{beamer}

\usepackage{amssymb, amsmath}
%\usepackage[all]{xy}

\usepackage{alltt}
\usepackage{pslatex} 
\usepackage{epigraph}
\usepackage{verbatim}

\usepackage{graphicx}
\usepackage{latexsym}
\usepackage{array}
\usepackage{comment}
\usepackage{makeidx}
\usepackage{listings}
\usepackage{indentfirst}
\usepackage{verbatim}
\usepackage{color}
\usepackage{url} 
\usepackage{xspace}
\usepackage{hyperref}

\usepackage{stmaryrd}
\usepackage[normalem]{ulem}
\usepackage{xcolor}
\usepackage{wasysym}
\usepackage{cancel} 
\usepackage{amsmath, amsthm, amssymb}
\usepackage{graphicx}
\usepackage{euscript}
\usepackage{mathtools}
\usepackage{tikz}
\usepackage{tikz-qtree}
\usepackage{bchart}

\definecolor{shadecolor}{gray}{1.00}
\definecolor{darkgray}{gray}{0.30}

\newcommand{\set}[1]{\{#1\}}
\newcommand{\angled}[1]{\langle {#1} \rangle}
\newcommand{\fib}{\rightarrow_{\mathit{fib}}}
\newcommand{\fibm}{\Rightarrow_{\mathit{fib}}}
\newcommand{\oo}[1]{{#1}^o}
\newcommand{\inml}[1]{\mbox{\lstinline{#1}}}
\newcommand{\goal}{\mathfrak G}
\newcommand{\inmath}[1]{\mbox{$#1$}}
\newcommand{\sembr}[1]{\llbracket{#1}\rrbracket}

\newcommand{\withenv}[2]{{#1}\vdash{#2}}
\newcommand{\ruleno}[1]{\eqno[\textsc{#1}]}
\newcommand{\trule}[2]{\dfrac{#1}{#2}}

\definecolor{light-gray}{gray}{0.90}
\definecolor{dark-green}{rgb}{0.30, 0.60, 0.1}
\definecolor{dark-red}{rgb}{0.80, 0.1, 0.1}

\newcommand{\graybox}[1]{\colorbox{light-gray}{#1}}

\newcommand{\nredrule}[3]{
  \begin{array}{cl}
    \textsf{[{#1}]}& 
    \begin{array}{c}
      #2 \\
      \hline
      \raisebox{-1pt}{\ensuremath{#3}}
    \end{array}
  \end{array}}

\newcommand{\naxiom}[2]{
  \begin{array}{cl}
    \textsf{[{#1}]} & \raisebox{-1pt}{\ensuremath{#2}}
  \end{array}}

\lstdefinelanguage{ocaml}{
keywords={let, begin, end, in, match, type, and, fun, 
function, try, with, class, object, method, of, rec, repeat, until,
while, not, do, done, as, val, inherit, module, sig, @type, struct, 
if, then, else, open, virtual, new, fresh},
sensitive=true,
basicstyle=\small,
commentstyle=\ttfamily,
keywordstyle=\ttfamily\bfseries,
identifierstyle=\ttfamily,
basewidth={0.5em,0.5em},
columns=fixed,
fontadjust=true,
literate={->}{{$\to$}}3,
morecomment=[s]{(*}{*)}
}

\lstset{
mathescape=true,
basicstyle=\small,
identifierstyle=\ttfamily,
keywordstyle=\bfseries, 
commentstyle=\scriptsize\rmfamily,
basewidth={0.5em,0.5em},
fontadjust=true,
%escapechar=!,
language=ocaml,
moredelim=[is][\ttfamily\color{blue}]{(&}{&)}
}

\sloppy

\newcommand{\miniKanren}{\texttt{miniKanren}\xspace}
\newcommand{\ocanren}{\texttt{OCanren}\xspace}
\newcommand{\ocaml}{\texttt{OCaml}\xspace}

\newcommand\redsout{\bgroup\markoverwith{\textcolor{red}{\rule[0.5ex]{2pt}{0.9pt}}}\ULon}

\setbeamertemplate{footline}[frame number]
\setbeamertemplate{navigation symbols}{}
\setbeamertemplate{blocks}[rounded][shadow=true] 
\beamertemplateballitem


\mode<presentation>{
  \usetheme{default}
}

\theoremstyle{definition}

\newtheorem{thm}{Theorem}[section] % the main one
% other statement types

% for specifying a name
\theoremstyle{plain} % just in case the style had changed
\newcommand{\thistheoremname}{}
\newtheorem{genericthm}[thm]{\thistheoremname}
\newenvironment{namedthm}[1]
  {\renewcommand{\thistheoremname}{#1}%
   \begin{genericthm}}
  {\end{genericthm}}

\title{Generic Programming with \\ Combinators and Objects}

\author{ Dmitrii Kosarev }

\institute[]{
  \textbf{
  St. Petersburg State University\\
  JetBrains Research}
}

\date{
   \vskip 1cm
   \small{
   \textbf{ML-2018}\\
   September 28, 2018 \\
   St. Louis, Missouri, the USA}
}

\begin{document}
\begin{frame}[plain]
  \titlepage
\end{frame}

% ============================== Slide 2 ==================================== %

\begin{frame}[fragile]{Motivation}
A part 
\end{frame}

\begin{comment}
  The main idea in the area of datatype-generic programming is that a type can be used to derive a set of useful
  transformations of data of that type.
  
  A very common approach is to generate the implementation for the transformation of interest directly from the type definition. It's a powerful
  approach since it allows implementing a wide range of unrelated specific transformations in an ad hoc manner. On the other hand, every
  transformation has to be implemented in full from the scratch, and abstracting over such transformations is somewhat problematic, since they
  may not share common properties (for example, type).  
  
  Another approach is to develop a common representation for all types and implement a transformation of interest in
  terms of this representation. Thus, for a concrete type it becomes sufficient to translate the representation of a data in terms of its type
  into the representation in terms of representation of that type. Making new transformation becomes easier this way, but there is an essential
  performance penalty for re-encoding data representation (addressed in the relevant literature).

  Both approaches lack the same property: it is not possible to easy derive one transformation from another (or to reuse a part of already
  existing transformation). We argue, that this is a very frequent and important case, and, in addition, solving this problem would facilitate
  the application of generative approaches since it would make it possible to generate a wide range of ``default'' very generic transformations to
  derive the concrete ones from.

  In a parallel world of object-oriented programming this particular problem has long been solved: one just needs to inherit from
  an existing transformation and redefine it where needed. There are other problems with object-oriented languages (lack of convenient
  higher-order functions, overuse of objects (objects for abstraction, objects for polymorphism, etc.)), but this concrete problem
  is dealt with perfectly in object-oriented settings.

  In our work we present a ``happy marriage'' of object-oriented and functional approaches to generic programming in OCaml. Our framework
  allows to define datatype-generic functions, which can be used in convenient combinator way. However, under the hood the
  implementation uses object-orientd encoding. Thus, new transformations can be derived from existing ones with a massive
  reuse. There are also other useful properties --- all transformations are instances of the same scheme, only one
  generic function per type is required, our approach is compatible with other means of code reuse, such as polymorphic variants.
\end{comment}

\begin{frame}[fragile]{What and Why}
  Datatype-generic programming [Gibbons, 2006] --- use types to define transformations  
  \vskip5mm
  
  Two main approaches:

  \begin{itemize}
    \item generate a transformation from type definition (\lstinline[basicstyle=\large]|deriving|, etc.)
    \item define a representation for all types, implement a transformation of interest in terms of representation, translate a concrete
      type into its representation [W.Swierstra, 2008], [Chakravarty \emph{et al}, 2009]
  \end{itemize}
  \vskip5mm
  
  Observation: in many cases, a transformation of interest is a (small) modification of some other transformation
  \vskip5mm
  
  Approach: utilize object-oriented features to inherit one transformations from others
\end{frame}


% ================================================== Slide 3 ========================================================= %

\begin{comment}
  In our work we borrow some terminology and a meta-approach from the area of attribute grammars. Attribute grammars were
  introduced by Knuth as a mean to express a class of syntax-directed transformations, for which regular context-free
  grammars were not powerful or convenient enough. The similarity between attribute evaluation procedures and datatype-generic
  transformations was addressed multiple times. However, we do not make use of concrete algorithms for various classes of
  attribute grammars; we just consider AG terminology and approach very generic and convenient.

  We deal only with the transformations of types $\iota \to t \to \sigma$, where $\iota$ is a type for inherited attribute,
  $t$~--- a type for transforming data structure, and $\sigma$~--- a type for synthesized attribute. Informally speaking,
  synthesized attribute is the result of a transformation, while inherited~--- some auxiliary value, which may turn out
  to be useful is some specific cases. For parametric types this signature is enriched with similar transformations for
  type parameters. Note, both inherited and synthesized attributes for the transformations of type parameters can
  be different from each other and from those for the main type.

  This is a very generic scheme of transformations. All transformations, expressible in our framework, are
  instances of this scheme, and if a transformation does not fit into it, it cannot be implemented.  
\end{comment}

\begin{frame}[fragile]{Attribute Grammars and Catamorphisms}

  Attribute grammars (AG) [Knuth, 1968] as a way of describing catamorphisms [Fokkinga \emph{et al}, 1990], [Swierstra \emph{et al}, 2000, 2009]
  
  \vskip5mm
  
  A common shape of all transformations for a type $t$:

\[
\overbrace{\iota_t}^{\mbox{\scriptsize inherited attribute}} \to \underbrace{t}_{\mbox{\scriptsize transforming value}} \to \overbrace{\sigma_t}^{\mbox{\scriptsize{synthesized attribute}}}
\]
\vskip5mm
For a parametric type $(\alpha\,...)\;t$:

\[
\overbrace{(\iota_{\alpha} \to \alpha \to \sigma_{\alpha}) \to ...}^{\mbox{\scriptsize transformations for type parameters}}
\to \overbrace{\iota_t \to (\alpha\,...) t \to \sigma_t}^{\mbox{\scriptsize transformation for the type itself}}
\]

\end{frame}

% ================================================== Slide 4 ========================================================= %

\begin{comment}
  For given algebraic datatype $t$ we can define a single transformation function \lstinline|transform$_t$|. This
  function takes three arguments; two last ones are an inherited attribute and a subject (the data structure of type $t$). The first
  one is a \emph{transformation object}. This object represents the essense of the transformation on a method-per-constructor
  basis. The transformation function performs a pattern-matching on the subject and dispatched the control to the corresponding
  methods, passing as their arguments the matched sub-values and the inherited attribute. Note, the transformation function
  is not recursive, it performs a shallow traversal.
\end{comment}

\begin{frame}[fragile]{Transformation Function}

  A type:
  
\begin{lstlisting}
   type ($\alpha\,...$) $t$ = C of $q$ * ... | ...
\end{lstlisting}
\vskip5mm

A \emph{transformation function}:

\begin{lstlisting}
   let transform$_t$ $\omega$ $\iota$ $t$ =
     match $t$ with
     | C ($x$, ...) -> $\omega$#C $\iota$ $x$ ...
     ...
\end{lstlisting}
\vskip5mm

A \emph{transformation object} $\omega$: implements a transformation on a method-per-constructor basis
\vskip5mm
\end{frame}

% ================================================== Slide 5 ========================================================= %

\begin{comment}
  Transformation objects potentially can be constructed in any way one would prefer; however, we provide a common
  base abstract class for all transformation objects (or classes). As it implements the transformation of the shape we
  dealt with on the slide 3, the transformation class is highly polymorphic~--- it has $3*n+2$ type parameters (where
  $n$ is the arity of the type (plus one extra parameter for polymorphic variants). The signatures of methods are
  expressed in terms of the class' type parameters and types of arguments for corresponding constructor. 
\end{comment}

\begin{frame}[fragile]{Transformation Class}
An abstract class to mass-produce all transformation objects:
\vskip1cm

\begin{lstlisting}
  class virtual [$\overbrace{\iota_\alpha, \alpha, \sigma_\alpha,...,\iota,\sigma}^{\mbox{\scriptsize all type parameters for a transformation}}$] $t_t$ =
     object
        method virtual C : ...
        ...
     end     
\end{lstlisting}

\end{frame}

% ================================================== Slide 6 ========================================================= %

\begin{comment}
  We consider a small observable example right away. For the shown simple type with two constructors the framework
  automatically generates the following transformation function and the following transformation class. As the type
  is non-polymorphic, the signature of the class is rather simple: it parameterized only by the types of
  inherited and synthezised attributes. The signatures of two methods are rather obvious. 
\end{comment}

\begin{frame}[fragile]{An Observable Example: Generic Definitions}
  \begin{lstlisting}
    type expr = Const of int | Add of expr * expr

    
    let transform$_{expr}$ $\omega$ $\iota$ = function
    | Const i     -> $\omega$#c_Const $\iota$ i
    | Add  (x, y) -> $\omega$#c_Add   $\iota$ x y

    
    class virtual [$\iota$, $\sigma$] expr$_t$ =
    object
      method virtual c_Const : $\iota$ -> int -> $\sigma$
      method virtual c_Add   : $\iota$ -> expr -> expr -> $\sigma$
    end    
  \end{lstlisting}
\end{frame}

% ================================================== Slide 7 ========================================================= %

\begin{comment}
  Departing from those abstract generic definitions, a concrete transformation~--- show~--- can be implemented.
  First, we need a transformation object. For this we derive a concrete transformation class (we could
  construct an immediate object equally). Note, this class is parameterized with an additional argument~--- a
  function to transform the very same type we deal with. This function is used to implement recursion, and
  we need a fixpoint operator to tie the knot. Note, the type of class \lstinline|expr$_{show}$|  is not
  polymorphic any more; it instantiates all the parameters of the base class into ground types. Note also that
  we reused the generic transformation function~--- there is no pattern matching in the implemetation.
\end{comment}

\begin{frame}[fragile]{An Observable Example: Concrete Transformation}
  \begin{lstlisting}
    class expr$_{show}$ (fself : unit -> expr -> string) =
    object
      inherit [unit, string] expr$_t$ 
      method c_Const () i  =
        sprintf "Const (%d)" i      
      method c_Add () x y =
        sprintf "Add (%s, %s)" (fself x) (fself y)
    end

    
    let show$_{expr}$ e =
      fix (fun f -> transform$_{expr}$ (new expr$_{show}$ f)) () e
  \end{lstlisting}
\end{frame}

% ================================================== Slide 8 ========================================================= %

\begin{comment}
  We now recap/state the major design choices and properties of our framework:

  1. We use open recursion~--- the generic transformation function is shallow and non-recursive, and we tie the
  knot at the concrete transformation object creation time.

  2. The single generic transformation function is used to implement all concrete transformations.

  3. Neither the generic function nor the generic transformation class are parameterized with transformations for type-parameters.
  This, again, is done on a concrete transformation class/object implementation level.  
\end{comment}

\begin{frame}[fragile]{Design Choices and Properties}

  The transformation function is shallow (non-recursive)
  \vskip5mm
  The transformation function is \emph{not} parameterized by transformations for type parameters
  \vskip5mm
  A single per-type transformation function is used to implement all transformations for the type
  \vskip5mm
  All transformation objects can be inherited from a single class
  \vskip5mm
  For recursive types the knot has to be tied at the transformation object creation time
  
\end{frame}

% ================================================== Slide 9 ========================================================= %

\begin{comment}
  We swept a number of subtleties under the carpet:

  1. The types of transformation classes/objects can be verbose and scary; it is rather easy to get type parameters
  wrong, especially for the first few times; the right skill comes later. However, as long as only predefined
  transformations are used, an end-user never touches the object level.

  2. The described scheme has to be elaborated further to work with polymorphic variants.

  3. The support for extensibility in mutually-recursive case is even more complicated (but doable, and done). 

  4. A naive implementation of \lstinline|fix| creates a transformation object per recursive call; this can be avoided.  
\end{comment}

\begin{frame}[fragile]{Subtleties}
  
  The types of transformation objects can be (and, as a rule, are) scary 
  \vskip5mm

  The support for polymorphic variants requires extra work
  \vskip5mm

  The support for mutually-recursive types requires extra extra work
  \vskip5mm

  The \lstinline[basicstyle=\large]|fix| can be improved to not to create an identical object for each
  node of the data structure to transform
\end{frame}

% ================================================== Slide 10 ========================================================= %

\begin{comment}
  Before starting to show-off the examples, we summarize here the implementation details:

  1. Our framework generates the generic transformation function and abstract transformation class for a wide range
  of practically important kinds of type definitions.

  2. Moreover, our framework is equipped with a plugin system, which additionally allows to generate a number of
  concrete transformation classes, implementing transformations, which generally considered useful (like \lstinline|show|, etc.)
  Again, all these implementations reuse the single generic transformation function. It is interesting that, due to
  a very generic shape of transformation, chosen from the very beginning, it turned out to be possible to express some
  transformations, which have to be implemented in an ad hoc manner in other frameworks (for example, \lstinline|compare|
  and \lstinline|eq|).
 
  3. The plugin subsystem is available for an end user~--- it is possible to add a custom plugin, which will be
  dynamically loaded along with the standard ones. Due to the general design choices, the implementation of a plugin is
  easy~--- there is no need to traverse the type definition or to generate a transformation function. A plugin is only
  responsible for generation of methods-per-constructor bodies.

  4. The framework is available both in terms of \lstinline|ppxlib| and \lstinline|camlp5| syntax extension.  
\end{comment}

\begin{frame}[fragile]{Implementation Overview}
  The transformation function and abstract transformation class are generated by the framework from a type declaration
  \vskip5mm
  
  Supported: regular ADTs, type applications, structures, polymorphic variants
  \vskip5mm
  
  Not supported: extensible types (``\lstinline|...|'', ``\lstinline{+=}''), GADTs, module and object types
  \vskip5mm

  Plugins: predefined (\lstinline[basicstyle=\large]|show|, \lstinline[basicstyle=\large]|map|, \lstinline[basicstyle=\large]|fold|, etc.) and user-defined
  \vskip5mm

  Available by means of both \lstinline[basicstyle=\large]|camlp5| and \lstinline[basicstyle=\large]|ppxlib|  
\end{frame}

% ================================================== Slide 11 ========================================================= %

\begin{comment}
  Now we consider a number of examples; we concentrate on the cases which makes use of the object-oriented layer of the
  framework.

  The first example is connected with the utilization of the tranformation \lstinline|map|; in our case it is implemented
  by a plugin \lstinline|gmap|. Here we used a \lstinline|camlp5| syntax extension: \lstinline|@type| makes the type definition
  visible for the framework, and \lstinline|with gmap| additionally asks to generate the implementation for the corresponding
  plugin. The construct \lstinline|@expr[gmap]| is expanded into a class name for the generated transformation. For the shown type,
  this plugin generates no more than a copy transformation (indeed, the type is not polymorphic). 
  However, we can make use of this transformation, since we can inherit from it and modify its behaviour only for an ``interesting''
  case. We implemented here a transformation, which substitutes all variables in an expresion tree by their values in given
  state \lstinline|st| (we used here an immediate object). This example shows, that with object-oriented approach we can make a good
  use of seemingly useless transformations by reusing a part of their semantics.
\end{comment}

\begin{frame}[fragile]{Example: \lstinline[basicstyle=\Large]|gmap|}

\begin{lstlisting}
@type expr =
  Var   of string
| Add   of expr * expr
| Const of int with gmap
\end{lstlisting}
\vskip5mm

The default transformation \lstinline[basicstyle=\large]|@expr[gmap]| just makes a copy
\vskip5mm
This default behavior can be overriden, obtaining a new transformation:
\vskip5mm

\begin{lstlisting}
let substitute st e =
  fix (fun f ->
         transform(expr) (object inherit [_] @expr[gmap] f
                              method c_Var _ x = Const (st x)
                            end)
                           ()
      )
  e                            
\end{lstlisting}

\end{frame}

% ================================================== Slide 12 ========================================================= %

\begin{comment}
  The next example is somewhat similar: this time we reuse another typical transformation, \lstinline|foldr|, which ``folds'' a
  data structure in a bottom-up manner. Note, the type $t$ is not parametric, so the type of the transformation is
  $\iota \to t \to \iota$ (the inherited and synthesized attributes have the same type this time). In other words, there is
  no place for a function ``to fold with'', and the implementation does nothing except for a threading the inherited attribute
  through all the nodes and returning it back unchanged.
  Yet we can make use of this seemingly useless transformation: if we properly redefine its behaviour for one constructor (\lstinline|Var|), it
  will collect all the variables for us. Again, we modified the behaviour only for the ``interesting case''.
\end{comment}

\begin{frame}[fragile]{Example: \lstinline[basicstyle=\Large]|foldr|}
  
\begin{lstlisting}
@type expr =
| Var   of string
| Binop of expr * expr with foldr
\end{lstlisting}

The default transformation makes nothing (just threads the accumulator value in a bottom-up manner)
\vskip5mm

Again, by overriding only the ``interesting case'' a new transformation can be constructed:
\vskip5mm

\begin{lstlisting}
module S = Set.Make (String)
                    
let vars e = fix (fun f ->
  transform(expr)
    (object inherit [S.t, expr] @expr[foldr] f
         method c_Var vs s = S.add s vs
      end)
  ) S.empty e
\end{lstlisting}
\end{frame}

% ================================================== Slide 13 ========================================================= %

\begin{comment}
  The previous example can be equally implemented for polymorphic variant types. This time we use an open recursion: our
  type in not recursive, but polymorphic. We also implemened variable collector as a class, not as an immediate
  object. Of course, this is a preparation for the ``expression problem'' solution.
\end{comment}

\begin{frame}[fragile]{Example: Polymorphic Variant Support}

  We can equally use a polymorphic variant version (note, here we define a separate class rather than immediate object):
  \vskip5mm
  
\begin{lstlisting}
@type 'e base = [`Var of string | `Binop of 'e * 'e] with foldr

class ['e, 's] vars_base fself fe =
  object inherit ['e, S.t, 's] @base[foldr] fself fe 
    method c_Var vs s = S.add s vs
  end
\end{lstlisting}

\end{frame}

% ================================================== Slide 14 ========================================================= %

\begin{comment}
  The next step, another polymorphic variant type, this time for a binding construct. Nothing specifically interesting
  here, just reiteration of the same pattern. 
\end{comment}

\begin{frame}[fragile]{Example: Polymorphic Variant Support, More}

  We can declare another polymorphic variant type, now representing a binding construct,
  and implement a custom class:
  \vskip5mm
  
\begin{lstlisting}
@type 'e lam = [`Lam of string * 'e] with foldr

class ['e, 's] vars_lam fself fe =
  object inherit ['e, S.t, 's] @lam[foldr] fself fe
    method c_Lam vs s e = S.union vs (S.remove s @@ fe S.empty e)
  end
\end{lstlisting}
\end{frame}

% ================================================== Slide 15 ========================================================= %

\begin{comment}
  We can combine the two types into one (no surprise, just the best practices from OCaml world), but we can also
  combine the transformations in a purely declarative manner using the inheritance. Note, that the instantiation
  of the type parameter to the same type is done only within the definition of \lstinline|vars|, i.e. in the
  latest possible moment. The type itself remains polymorphic (= openly recursive), as well as the transformation,
  and can be extended further.
\end{comment}

\begin{frame}[fragile]{Example: Polymorphic Variant Support, Even More}

... and combine these two types and those two classes:
\vskip4mm
  
\begin{lstlisting}
@type 'e expr = ['e base | 'e lam] with foldr

class ['e, 's] vars_expr fself fe =
  object
    inherit ['e, 's] vars_base fself fe
    inherit ['e, 's] vars_lam  fself fe
  end
                                              
let vars e =
  fix (fun f -> transform(expr) (new vars_expr f f)) S.empty e
\end{lstlisting}

\end{frame}

% ================================================== Slide 16 ========================================================= %

\begin{comment}
  Our final example is not for children. We are going to implement an html-based navigation through OCaml compiler IR tree,
  providing a way to navigate from identifiers to their binding poistions. We need a typed representation for this purpose,
  and this representation is comprised of 60 mutually-recursive types.
  Here we import the representation of types from corresponding interface files and generate our generic decorations, including
  a transformation for generating html (yes, we already have one). This time we use \lstinline|ppxlib| version of our
  framework. The problem so far is that we can generate html, ok, but there will be no cross-references. The solution, as always,
  is to take this useless transformation and forge a useful one on its basis. In short, we have to redefine its
  behavior in two places: in a binding position and in a referencing position.
\end{comment}


% ================ Slide 20 ================================ %

\begin{comment}
Our library is rather similar to `visitors` library that is being used in so-called 
`polymorphic mode`. Two main differencies are 1) universal quantification of methods in `visitors` and 2) more explicit manner to specify type parameters of a class
\end{comment}

\begin{frame}[fragile]{P.S. Alternatives}
Our work can be compared to \href{https://gitlab.inria.fr/fpottier/visitors}{Visitors} from Fran\c cois Pottier. Two main differencies are:
\begin{itemize}
 \item Universal quantification of methods in visitors
 \item A more explicit way to specify type parameters of a class in our approach
\end{itemize}
\end{frame}

% ================ Slide 21 ================================ %
\begin{comment}
Although universal quantification of methods in `visitors` enables (it seems to) support 
of non-regular
data types, it creates a few issues with type abbrevations. In examples below
the first one is processed as expected. 
(NEXT SLIDE) The 2nd and 3rd ones lead to compilation error 
that we can... let's say repair...  by removing manually quantifiers from the 
method types in generated code. 
(NEXT SLIDE)
Then issue appears only in `polymorphic mode`, in monomorphic mode visitors works as 
expected. We want these examples to work as it is (because of some legacy code)
so we decided to upgrade our old work about the topic instead of changing the code
to make `visitors` applicable.

Also, in `visitors` some transformations (more precisely: `fold`) are available only in monomorphic mode. Our work doesn't have this restriction.
\end{comment}

\begin{frame}[fragile]{Universal quantification in methods leads to errors}
\begin{lstlisting}[language=ocaml] 
(* no error yet *)
type ('a, 'b) t = Foo of 'a * 'b  
[@@deriving visitors { variety = "map"; polymorphic = true }]
\end{lstlisting}
\pause
\begin{lstlisting}[language=ocaml] 
(* Code generation leads to compilation error below *)
type 'a t2 = ('a, int) t 
[@@deriving visitors { variety = "map"; polymorphic = true
                     ; name="map1"}]

(* And here too *)
type 'a t2 = T2 of ('a, int) t [@@unboxed]
[@@deriving visitors { variety = "map"; polymorphic = true
                     ; name="map2" }]
\end{lstlisting}\pause
But in \textit{monomorphic mode} visitors library works fine
\begin{lstlisting}[language=ocaml] 
type 'a t2 = T2 of ('a, int) t [@@unboxed]
[@@deriving visitors { variety = "map"; polymorphic = false
                     ; name="map3" }]
\end{lstlisting}
\end{frame}

% ================ Slide 22 ================================ %
\begin{comment}
Objects and classes generated by visitors have a very peculiar way to specify type 
parameters for a class, like this. In our library we generate a lot of type parameters
but there is an important one: the opened version of type that is being processed. It 
matters when we are generating `map` transformation (a.k.a. functor) for polymorphic
variant types. This explicit type parameter allows to take two classes that perform
transformations of polymorphic variant types and join them using inheritance. The 
visitors officially doesn't support polymorphic variants but we beleive that with 
their approach it will be troublesome to implement this.

And the last thing that we want to mention here is that `visitors` library 
relies on a type checker a lot and it can't
generate type signatures in the interface file by design. Our work doesn't suffer 
from this problem which should be better from engineering point of view.
\end{comment}

\begin{frame}[fragile]{Type parameters of transformation classes}
\begin{lstlisting}[language=ocaml] 
type ($\alpha\,...$) $t$ = ...  
\end{lstlisting}
In visitors: 
\begin{lstlisting}[language=ocaml] 
class ['self] c = object (self: 'self) ... end
\end{lstlisting}
\begin{itemize}
 \item In our library we have many type parameters for a class but there is an important one: the \textit{opened} version of type $t$

$\Rightarrow$ joining of polymorphic variants becomes doable.
\item Visitors are failing to generate a signature.
\end{itemize}
\end{frame}

% ================ Slide 23 ================================ %
\begin{comment}
  This is almost all. A number of concluding remarks:

  1. We are near the end of development, there are still some things to improve

  2. We know that there is some performance penalty due to the intensive use of objects. Perhaps, we should
  investigate this issue more closely and, perhaps, spend some time optimizing (specialization?)

  3. We support only homogenious plugins so far (``\lstinline|show|'' can not use ``\lstinline|compare|''); we are not
  sure we should go farther in this direction

  4. It is interesting if the framework can be used to implement ad-hoc polymorphism (actually, yes) and existentials (don't know)

  5. Our work gives a good case of using objects when they are really needed  
\end{comment}

\begin{frame}[fragile]{Concluding Remarks}

  The framework is still in development, but the fixpoint is close
  \vskip5mm
  
  Performance issues: method invocation is slower, than a direct function call
  \vskip5mm
  
  Only homogeneous plugins are supported so far
  \vskip5mm
  
  Ad hoc polymorphism? Existentials?
  \vskip5mm
  
  Object/classes can be good, when cooked properly
  
\end{frame}

\begin{frame}[fragile]

  \begin{center}
    {\textbf{\Large Thank you!}}
  \end{center}

\end{frame}

\begin{frame}[fragile]{Example: Hyperlinked OCaml Typedtree (18+)}
\begin{lstlisting}[language=ocaml] 
type pattern = [%import: Typedtree.pattern]
and pat_extra = [%import: Typedtree.pat_extra]
and pattern_desc = [%import: Typedtree.pattern_desc]
and expression = [%import: Typedtree.expression]
and expression_desc = [%import: Typedtree.expression_desc]
and exp_extra = [%import: Typedtree.exp_extra]
...    
\end{lstlisting}

(\textbf{53 more} mutually recursive types from the module \lstinline[basicstyle=\large]|Typedtree|)

\begin{lstlisting}
...   
and 'a class_infos = [%import: 'a Typedtree.class_infos]
[@@deriving gt ~options: {html; fmt}]
\end{lstlisting}
\end{frame}

% =========================== Slide 17 ====================================== %

\begin{comment}
  This slide shows the redefinition for a binding position (namely, in patterns). As we said earlier, a full-fledged
  support for mutually-recursive types is tricky. Here the transformation class is additionally parameterized
  by a collection of transformation functions for all members of mutually-recursive cluster of types (called here
  \lstinline|mut_trfs|). Thus, along with \lstinline|fself|, it represents the second level of open recursion. Besides that,
  nothing is particularly interesting or non-trivial here, just some boring HTML stuff.
\end{comment}

\begin{frame}[fragile]{Example: Patterns (Hyperlink Destinations)}
\begin{lstlisting}[language=ocaml] 
class ['self] pattern_desc_link mut_trfs fself = 
object
  inherit ['self] html_pattern_desc_t_stub mut_trfs fself
  method! c_Tpat_var () { Ident.name } nameloc =
    let loc_str = 
      Location.show_location nameloc.Asttypes.loc 
    in
    HTML.ul @@ HTML.seq
        [ HTML.anchor loc_str @@ HTML.string @@ 
          sprintf "%S from %s" name loc_str
        ]
end
\end{lstlisting}
\end{frame}

% ================================================== Slide 18 ========================================================= %

\begin{comment}
  Equally, here we see the redefinition for a referencing position (in expressions). An interesting part is that we do not
  see a method-per-constructor here; this is because the expression with its environment in a typed tree is represented by a structure, and
  when a type is a structure we generate a method called \lstinline|do_$\mbox{this type}$|. 
\end{comment}

\begin{frame}[fragile]{Example: Identifiers (Hyperrefs)}
\begin{lstlisting}[language=ocaml] 
class ['self] expression_with_link mut_trfs fself = 
object
  inherit ['self] html_expression_t_stub mut_trfs fself as super

  method! do_expression () e =
    match e.exp_desc with
    | Texp_ident (p,lloc,vd) ->
      let where = Location.show_t
        (Ocaml_common.Env.find_value p e.exp_env).val_loc 
      in
      HTML.ref ("#" ^ where) @@ HTML.string @@
      sprintf "%s  from %s" (Longident.show_t lloc.txt) where
    | _ -> super#do_expression () e
end
\end{lstlisting}
\end{frame}

% ================================================== Slide 19 ========================================================= %

\begin{comment}
  Finally, we have to create the desired transformation using two new classes we have just redefined and 58 old classes,
  generated by our framework. That what we do here. \lstinline|htm_fix_case| is a fixpoint combinator, specific for the given
  plugin (\lstinline|html|) and given mutually-recursive type declaration.
\end{comment}

\begin{frame}[fragile]{Example: Tying the Knot}
\begin{lstlisting}[language=ocaml] 
let html_structure =
  let { html_structure } = html_fix_case
    ~expression0:
      { html_expression_func = new expression_with_link }
    ~pattern_desc0:
      { html_pattern_desc_func = new pattern_desc_with_link }
    ()
  in
  html_structure.html_structure_trf
\end{lstlisting}
\end{frame}

\end{document}
