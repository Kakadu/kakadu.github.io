% !TeX spellcheck = ru_RU
\documentclass[aspectratio=169
  , xcolor={svgnames}
  , hyperref=
      { colorlinks
      , urlcolor=DarkBlue
      }  
  , 12pt
  , russian  % This line affects translation of theorem titles
  ]{beamer}
\usepackage[svgnames]{xcolor}
\usetheme{CambridgeUS}
\usefonttheme{professionalfonts}

\makeatletter
\@ifclassloaded{beamer}{
  % get rid of header navigation bar
  \setbeamertemplate{headline}{}
  % get rid of bottom navigation symbols
  \setbeamertemplate{navigation symbols}{}
  % get rid of footer
  %\setbeamertemplate{footline}{}
}
{}
\makeatother
%%%%%%%%%%%%%%%%%%%%%%%%%%%%%%%%%%%%%%%%%%%%%
\usepackage{fontawesome}
% \newfontfamily{\FA}{Font Awesome 5 Free} % some glyphs missing
\expandafter\def\csname faicon@facebook\endcsname{{\FA\symbol{"F09A}}}
\def\faQuestionSign{{\FA\symbol{"F059}}}
\def\faQuestion{{\FA\symbol{"F128}}}
\def\faExclamation{{\FA\symbol{"F12A}}}
\def\faUploadAlt{{\FA\symbol{"F093}}}
\def\faLemon{{\FA\symbol{"F094}}}
\def\faPhone{{\FA\symbol{"F095}}}
\def\faCheckEmpty{{\FA\symbol{"F096}}}
\def\faBookmarkEmpty{{\FA\symbol{"F097}}}

\newcommand{\faGood}{\textcolor{ForestGreen}{\faThumbsUp}}
\newcommand{\faBad}{\textcolor{red}{\faThumbsODown}}
\newcommand{\faWrong}{\textcolor{red}{\faTimes}}
\newcommand{\faMaybe}{\textcolor{blue}{\faQuestion}}
\newcommand{\faCheckGreen}{\textcolor{ForestGreen}{\faCheck}}
%%%%%%%%%%%%%%%%%%%%%%%%%%%%%%%%%%%%%%%%%%%%%

\usepackage{fontspec}
\usepackage{xunicode}
\usepackage{xltxtra}
\usepackage{xecyr}
\usepackage{hyperref}

\setmainfont[
 Ligatures=TeX,
 Extension=.otf,
 BoldFont=cmunbx,
 ItalicFont=cmunti,
 BoldItalicFont=cmunbi,
% Scale = 1.1
]{cmunrm}
\setsansfont[
 Ligatures=TeX,
 Extension=.otf,
 BoldFont=cmunsx,
 ItalicFont=cmunsi,
%  Scale = 1.2
]{cmunss}
%\setmainfont[Mapping=tex-text]{DejaVu Serif}
%\setsansfont[Mapping=tex-text]{DejaVu Sans}
%\setmonofont{Fira Code}[Contextuals=AlternateOff]
\setmonofont{Fira Code}[Contextuals=Alternate,Scale=0.9]
\newfontfamily{\myfiracode}[Scale=1.5,Contextuals=Alternate]{Fira Code}
%\setmonofont[Scale=0.9,BoldFont={Inconsolata Bold}]{Inconsolata}

\usepackage{polyglossia}
\setmainlanguage{russian}
\setotherlanguage{english}


%\newfontfamily\dejaVuSansMono{DejaVu Sans Mono}
% https://github.com/vjpr/monaco-bold/raw/master/MonacoB/MonacoB.otf
%\newfontfamily\monacoB{MonacoB}
%%%%%%%%%%%%%%%%%%%%%%%%%%%%%%%%%%%%%%%%%%%%%%%5
\usepackage{soul} % for \st that strikes through
\usepackage[normalem]{ulem} % \sout

\usepackage{stmaryrd}
\newcommand{\sem}[1]{\ensuremath{\llbracket #1\rrbracket}}


\usepackage{listings}
%\lstdefinestyle{style1}{
%  language=haskell,
%  numbers=left,
%  stepnumber=1,
%  numbersep=10pt,
%  tabsize=4,
%  showspaces=false,
%  showstringspaces=false
%}
%\lstdefinestyle{hsstyle1}
%{ language=haskell
%%          , basicstyle=\monacoB
%         , deletekeywords={Int,Float,String,List,Void}
%         , breaklines=true
%         , columns=fullflexible
%         , commentstyle=\color{ForestGreen}
%         , escapeinside=§§
%         , escapebegin=\begin{russian}\commentfont
%         , escapeend=\end{russian}
%         , commentstyle=\color{ForestGreen}
%         , escapeinside=§§
%         , escapebegin=\begin{russian}\color{ForestGreen}
%         , escapeend=\end{russian}
%         , mathescape=true
%%          , backgroundcolor = \color{MyBackground}
%}
%
%\newcommand{\inline}[1]{\lstinline{haskell}{#1}}
%\def\hsinline{\mintinline{haskell}}
%\def\inline{\hsinline}
%
%\lstnewenvironment{hslisting} {
%    \lstset { style={hsstyle1} }
%  }
%  {}
%  
%%%%%%%%%%%%%%%%%%%%%%%%%%%%%%%%%%%%%%%%%%%%%%%%%%%%%%%%%%%  
%%\setmainfont[
%% Ligatures=TeX,
%% Extension=.otf,
%% BoldFont=cmunbx,
%% ItalicFont=cmunti,
%% BoldItalicFont=cmunbi,
%%]{cmunrm}
%%% С засечками (для заголовков)
%%\setsansfont[
%% Ligatures=TeX,
%% Extension=.otf,
%% BoldFont=cmunsx,
%% ItalicFont=cmunsi,
%%]{cmunss}
%% \setmonofont[Scale=0.6]{Monaco}
%
%\usefonttheme{professionalfonts}
%\usepackage{times}
\usepackage{tikz}
\usetikzlibrary{cd}
\usepackage{tikz-cd}
\usepackage{caption}
\usepackage{subcaption}

%\renewtheorem{definition}{برهان}[chapter]
%%\DeclareMathOperator{->}{\rightarrow}
%\newcommand\iso{\ensuremath{\cong}}
%\usepackage{verbatim}
%\usepackage{graphicx}
%\usetikzlibrary{arrows,shapes}

%\usepackage{amsmath}
%\usepackage{amsfonts}
\usepackage{scalerel}
\DeclareMathOperator*{\myvee}{\scalerel*{\vee}{\sum}}
\DeclareMathOperator*{\mywedge}{\scalerel*{\wedge}{\sum}}

%
%\usepackage{tabulary}
%
%% sudo aptget install ttf-mscorefonts-installer
%%\setmainfont{Times New Roman}
%%\setsansfont[Mapping=tex-text]{DejaVu Sans}
%
%%\setmonofont[Scale=1.0,
%%    BoldFont=lmmonolt10-bold.otf,
%%    ItalicFont=lmmono10-italic.otf,
%%    BoldItalicFont=lmmonoproplt10-boldoblique.otf
%%]{lmmono9-regular.otf}
%
\usepackage[cache=true]{minted}
\usemintedstyle{perldoc}

\def\hsinline{\mintinline{haskell}}
\def\mlinline{\mintinline{ocaml}}
% color options
\definecolor{YellowGreen} {HTML}{B5C28C}
\definecolor{ForestGreen} {HTML}{009B55}
\definecolor{MyBackground}{HTML}{F0EDAA}



\institute{матмех СПбГУ}

\addtobeamertemplate{title page}{}{
  \begin{center}{\tiny Дата сборки: \today}\end{center}
}

%%%%%%%%%%%%%%%%%%%%%%%%%%%%%%%%%%%%%%%%

\title{Направления работ на 2020/2021 учебные года}

\institute[]{Лаборатория Языковых Инструментов JetBrains}
\author[Косарев Дмитрий]{Косарев Дмитрий a.k.a. Kakadu}
\date{17 сентября 2020 г.}


\newcommand{\verbatimfont}[1]{\def\verbatim@font{#1}}
\usepackage{verbatimbox}

%\setbeamertemplate{section in toc}{\inserttocsectionnumber.~\inserttocsection}
\begin{document}
\maketitle

% For every picture that defines or uses external nodes, you'll have to
% apply the 'remember picture' style. To avoid some typing, we'll apply
% the style to all pictures.
%\tikzstyle{every picture}+=[remember picture] 

% By default all math in TikZ nodes are set in inline mode. Change this to
% displaystyle so that we don't get small fractions.
\everymath{\displaystyle}

% Uncomment these lines for an automatically generated outline.
\begin{frame}{Чем занимается Лаборатория Языковых Инструментов}
\begin{itemize}
\item Разнообразная верификация программ
\item Теоретические модели памяти для параллельных программ
\item [\faGood] Синтаксический анализ и всё связанное с ним
\item [\faGood] Функциональное программирование
\item [\faGood] Реляционное (a.k.a. логическое) программирование
\end{itemize}
\vspace{3em}

Я занимаюсь тем, что отмечено \faGood

Подробнее тут: \href{http://kakadu.github.io/fp2020/projects.html}{http://kakadu.github.io/fp2020/projects.html}
\end{frame}

\begin{frame}[fragile]{Про функциональный язык программирования OCaml }
\framesubtitle{и IDE для него на основе Language Server Protocol}

\begin{itemize}
\item Семантическая подсветка (и/или) идентация \href{http://kakadu.github.io/fp2020/projects.html\#semantich-highlighting}{\faGithub}
\begin{itemize}
\item Для встраиваемых языков программирования, оформленных в виде библиотеки
\item Подход более популярный в функциональном программировании, чем в мейнстриме
\item Примеры из языка C\#: LINQ, в некоторой степени \mintinline{csharp}{select * from expr}
\end{itemize}\pause

\item Поддержка compile-time синтаксических расширений языка
\href{http://kakadu.github.io/fp2020/projects.html#ide-camlp5}{\faGithub}
\begin{itemize}
\item Уже работают расширения из синтаксиса OCaml в синтаксис OCaml
\item Хочется поддержать более могучие расширения
\begin{itemize}
\item Например, которые превращают синтаксис Scheme в синтаксис OCaml
\end{itemize}
\end{itemize}

\end{itemize}
\end{frame}





\begin{frame}[fragile]{Про язык ReScript}
\framesubtitle{диалект OCaml и \sout{испорченным} исправленным синтаксисом}

\begin{itemize}
\item Восстановление от ошибок для ReScript
\href{http://kakadu.github.io/fp2020/projects.html#rescript-recovery}{\faGithub}
\begin{itemize}
\item В ReScript добавили синтаксис, который сильно напоминает Javascript
\item Поэтому OCaml программистом пересаживаться на ReScript больно
\item Необходимо научить парсер ReScript распознавать наиболее частые "камлизмы" и автоматически исправлять их
\end{itemize}\pause

\item Парсер ReScript на основе парсер-комбинаторов
\href{http://kakadu.github.io/fp2020/projects.html#rescript-combinators}{\faGithub}
\begin{itemize}
\item Теоретически исследованные подходы на основе грамматик плохо работают на практике
\item Рекурсивный спуск, используемый на практике, больно писать и поддерживать
\item Парсер-комбинаторы пытаются совместить лучшее из двух подходов
\item Надо проверить это заявление в рамках курсовой
\end{itemize}
\end{itemize}

\end{frame}



\begin{frame}[fragile]{Мелкие темы (на уровне курсовых) }
\begin{itemize}
\item Доработка конвертера документов Pandoc (написан на \textsc{Haskell})
\begin{itemize}
\item хочется не писать РПД в Word
\item а использовать latex/markdown и т.п.
\item Нужно доработать Pandoc, поддержкой титульной страницы, таблиц и т.п.
\end{itemize}\pause

\item Ещё одна реализация tabling (кэширования) для \textsc{miniKanren}
\begin{itemize}
\item \textsc{miniKanren} -- реинкарнация логического программирования
\item Нужно будет прочитать статью и реализовать на алгоритм OCaml
\end{itemize}
\end{itemize}

\end{frame}

\end{document}
