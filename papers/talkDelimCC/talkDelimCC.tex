\documentclass{beamer}

\beamertemplatenavigationsymbolsempty % remove navigation bar

\usepackage[utf8]{inputenc} 
\usetheme{default}
% \usetheme{Frankfurt}
\usetheme{Boadilla}

% Define my settings
\definecolor{mygreen}{cmyk}{0.82,0.11,1,0.25}
\setbeamertemplate{blocks}[rounded][shadow=false]
% \addtobeamertemplate{block begin}{\pgfsetfillopacity{0.8}}{\pgfsetfillopacity{1}}
\setbeamercolor{structure}{fg=mygreen}
% \setbeamercolor*{block title example}{fg=blue!50,bg= blue!10}
% \setbeamercolor*{block body example}{fg= blue,bg= blue!5}

\makeatletter
\@ifclassloaded{beamer}{
  % get rid of header navigation bar
  \setbeamertemplate{headline}{}
  % get rid of bottom navigation symbols
  \setbeamertemplate{navigation symbols}{}
  % get rid of footer
  %\setbeamertemplate{footline}{}
}
{}
\makeatother
%%%%%%%%%%%%%%%%%%%%%%%%%%%%%%%%%%%%%%%%%%%%%
\usepackage{fontawesome}
% \newfontfamily{\FA}{Font Awesome 5 Free} % some glyphs missing
\expandafter\def\csname faicon@facebook\endcsname{{\FA\symbol{"F09A}}}
\def\faQuestionSign{{\FA\symbol{"F059}}}
\def\faQuestion{{\FA\symbol{"F128}}}
\def\faExclamation{{\FA\symbol{"F12A}}}
\def\faUploadAlt{{\FA\symbol{"F093}}}
\def\faLemon{{\FA\symbol{"F094}}}
\def\faPhone{{\FA\symbol{"F095}}}
\def\faCheckEmpty{{\FA\symbol{"F096}}}
\def\faBookmarkEmpty{{\FA\symbol{"F097}}}

\newcommand{\faGood}{\textcolor{ForestGreen}{\faThumbsUp}}
\newcommand{\faBad}{\textcolor{red}{\faThumbsODown}}
\newcommand{\faWrong}{\textcolor{red}{\faTimes}}
\newcommand{\faMaybe}{\textcolor{blue}{\faQuestion}}
\newcommand{\faCheckGreen}{\textcolor{ForestGreen}{\faCheck}}
%%%%%%%%%%%%%%%%%%%%%%%%%%%%%%%%%%%%%%%%%%%%%

\usepackage{fontspec}
\usepackage{xunicode}
\usepackage{xltxtra}
\usepackage{xecyr}
\usepackage{hyperref}

\setmainfont[
 Ligatures=TeX,
 Extension=.otf,
 BoldFont=cmunbx,
 ItalicFont=cmunti,
 BoldItalicFont=cmunbi,
% Scale = 1.1
]{cmunrm}
\setsansfont[
 Ligatures=TeX,
 Extension=.otf,
 BoldFont=cmunsx,
 ItalicFont=cmunsi,
%  Scale = 1.2
]{cmunss}
%\setmainfont[Mapping=tex-text]{DejaVu Serif}
%\setsansfont[Mapping=tex-text]{DejaVu Sans}
%\setmonofont{Fira Code}[Contextuals=AlternateOff]
\setmonofont{Fira Code}[Contextuals=Alternate,Scale=0.9]
\newfontfamily{\myfiracode}[Scale=1.5,Contextuals=Alternate]{Fira Code}
%\setmonofont[Scale=0.9,BoldFont={Inconsolata Bold}]{Inconsolata}

\usepackage{polyglossia}
\setmainlanguage{russian}
\setotherlanguage{english}


%\newfontfamily\dejaVuSansMono{DejaVu Sans Mono}
% https://github.com/vjpr/monaco-bold/raw/master/MonacoB/MonacoB.otf
%\newfontfamily\monacoB{MonacoB}
%%%%%%%%%%%%%%%%%%%%%%%%%%%%%%%%%%%%%%%%%%%%%%%5
\usepackage{soul} % for \st that strikes through
\usepackage[normalem]{ulem} % \sout

\usepackage{stmaryrd}
\newcommand{\sem}[1]{\ensuremath{\llbracket #1\rrbracket}}


\usepackage{listings}
%\lstdefinestyle{style1}{
%  language=haskell,
%  numbers=left,
%  stepnumber=1,
%  numbersep=10pt,
%  tabsize=4,
%  showspaces=false,
%  showstringspaces=false
%}
%\lstdefinestyle{hsstyle1}
%{ language=haskell
%%          , basicstyle=\monacoB
%         , deletekeywords={Int,Float,String,List,Void}
%         , breaklines=true
%         , columns=fullflexible
%         , commentstyle=\color{ForestGreen}
%         , escapeinside=§§
%         , escapebegin=\begin{russian}\commentfont
%         , escapeend=\end{russian}
%         , commentstyle=\color{ForestGreen}
%         , escapeinside=§§
%         , escapebegin=\begin{russian}\color{ForestGreen}
%         , escapeend=\end{russian}
%         , mathescape=true
%%          , backgroundcolor = \color{MyBackground}
%}
%
%\newcommand{\inline}[1]{\lstinline{haskell}{#1}}
%\def\hsinline{\mintinline{haskell}}
%\def\inline{\hsinline}
%
%\lstnewenvironment{hslisting} {
%    \lstset { style={hsstyle1} }
%  }
%  {}
%  
%%%%%%%%%%%%%%%%%%%%%%%%%%%%%%%%%%%%%%%%%%%%%%%%%%%%%%%%%%%  
%%\setmainfont[
%% Ligatures=TeX,
%% Extension=.otf,
%% BoldFont=cmunbx,
%% ItalicFont=cmunti,
%% BoldItalicFont=cmunbi,
%%]{cmunrm}
%%% С засечками (для заголовков)
%%\setsansfont[
%% Ligatures=TeX,
%% Extension=.otf,
%% BoldFont=cmunsx,
%% ItalicFont=cmunsi,
%%]{cmunss}
%% \setmonofont[Scale=0.6]{Monaco}
%
%\usefonttheme{professionalfonts}
%\usepackage{times}
\usepackage{tikz}
\usetikzlibrary{cd}
\usepackage{tikz-cd}
\usepackage{caption}
\usepackage{subcaption}

%\renewtheorem{definition}{برهان}[chapter]
%%\DeclareMathOperator{->}{\rightarrow}
%\newcommand\iso{\ensuremath{\cong}}
%\usepackage{verbatim}
%\usepackage{graphicx}
%\usetikzlibrary{arrows,shapes}

%\usepackage{amsmath}
%\usepackage{amsfonts}
\usepackage{scalerel}
\DeclareMathOperator*{\myvee}{\scalerel*{\vee}{\sum}}
\DeclareMathOperator*{\mywedge}{\scalerel*{\wedge}{\sum}}

%
%\usepackage{tabulary}
%
%% sudo aptget install ttf-mscorefonts-installer
%%\setmainfont{Times New Roman}
%%\setsansfont[Mapping=tex-text]{DejaVu Sans}
%
%%\setmonofont[Scale=1.0,
%%    BoldFont=lmmonolt10-bold.otf,
%%    ItalicFont=lmmono10-italic.otf,
%%    BoldItalicFont=lmmonoproplt10-boldoblique.otf
%%]{lmmono9-regular.otf}
%
\usepackage[cache=true]{minted}
\usemintedstyle{perldoc}

\def\hsinline{\mintinline{haskell}}
\def\mlinline{\mintinline{ocaml}}
% color options
\definecolor{YellowGreen} {HTML}{B5C28C}
\definecolor{ForestGreen} {HTML}{009B55}
\definecolor{MyBackground}{HTML}{F0EDAA}



\institute{матмех СПбГУ}

\addtobeamertemplate{title page}{}{
  \begin{center}{\tiny Дата сборки: \today}\end{center}
}

\usepackage{ebproof}
\usepackage{syntax}

% http://pllab.is.ocha.ac.jp/~asai/cw2011tutorial/slides.pdf
\title{Gentle introduction to delimited continuations}
\author{Дмитрий Косарев}

\begin{document}
\begin{frame} \titlepage \end{frame}

\begin{frame}{Обзор}
 
 \begin{itemize}
  \item call-with-current-continuation (callCC)
  \item[\faCheck] \camline=shift=/\camline=reset=
  \item[\faCheck] Примеры
  \item[\faCheck] Предикативный полиморфизм
  \item Импредикативный полиморфизм
  \item CPS-преобразование
  \item \camline=prompt=
  \item \camline=cupto=
  \item Через монады \cite{Dyvbig2007}
 \end{itemize}

\end{frame}


\begin{frame}[fragile]{Что такое продолжение (continuation)?}
 \begin{block}{Continuation}
Это остаток вычисления
\end{block}
\begin{itemize}
 \item Текущее вычисление: внутри  \hsinline=[]=
 \item Остаток вычисления: снаружи  \hsinline=[]=
\end{itemize}
Пример: \hsinline=3+[5*2]-1=
\begin{itemize}
 \item Текущее вычисление: 5*2
 \item Остаток вычисления: \hsinline=3+[$\cdot$]-1= 
\end{itemize}
\vspace{1cm}
``Если дали значение для "дырки" \hsinline=[=$\cdot$\hsinline=]=, то прибавить 3 и вычесть 1``, т.е. \camline=fun x -> 3 + x - 1=.
\end{frame}

\begin{frame}[fragile]{Что такое продолжение (continuation)?}
 \begin{block}{Continuation}
Это остаток вычисления
\end{block}
Продолжения можно потерять по мере вычисления. 

Например: \hsinline=3+[5*2]-1=
\begin{itemize}
 \item Заменим \hsinline=[$\cdot$]=  на \camline=raise Abort=:\\
    \camline=3+[raise Abort]-1=
 \item Теряющееся продолжение \hsinline=3+[$\cdot$]-1= является текущим
\end{itemize}
\vspace{1cm}
\end{frame}

\begin{frame}[fragile]{Разграниченные продолжения (continuation)?}
\begin{block}{Continuation}
Это остаток вычисления
\end{block}
\begin{alertblock}{Синтаксис} 
\camline=reset (fun () -> M)=
\end{alertblock}

Например: \camline=@reset (fun () ->@ 3 + [5*2]@)@ - 1 =
\begin{itemize}
 \item Текущее вычисление: \camline=5*2=
 \item Текущее разграниченное продолжение: \hsinline=3+[$\cdot$]=
\end{itemize}
\end{frame}

\begin{frame}[fragile]{Shift}
\begin{alertblock}{Синтаксис} 
\camline=shift (fun k -> M)=
\end{alertblock}
\begin{itemize}
 \item забывает текущее продолжение
 \item сохраняет забытое как k
 \item и исполняет \verb=M=
\end{itemize}

Например: 
\begin{center}
\camline=reset (fun () -> 3 + [shift (fun k -> M)]) - 1 =

{\Large $\Downarrow$}

\camline=reset (fun () -> [shift (fun k -> M)]) - 1 =

где k = reset (fun () -> 3 + [$\cdot$])
\end{center}

%{\Large ??? }
\end{frame}

\begin{frame}[fragile]{Полученными продолжениями можно не пользоваться}
\begin{alertblock}{ }
\camline=shift (fun _ -> M)=
\end{alertblock}
\begin{itemize}
 \item Сохраненное продолжение просто отбрасывается
 \item Очень похоже на исключения
\end{itemize}

Пример: 

\begin{center}
\camline=reset (fun () -> 3 + [shift (fun _ -> 2)]) - 1 =

{\Large $\Downarrow$}

\camline=reset (fun () ->                       2) - 1 =

где k = reset (fun () -> 3 + [$\cdot$])

{\Large $\Downarrow$}

\camline=2 - 1=

{\Large $\Downarrow$}

\camline=1=
\end{center}
\end{frame}

\begin{frame}[fragile]{Упражнение}
Дан список чисел, нужно их перемножить, а если встретился ноль, то \textit{сразу} вернуть ноль.
\begin{lstlisting}[style={camlstyle1}]
let rec times lst = match lst with
  | [] -> 1
  | 0 :: tl -> @???@
  | h :: tl -> h * times tl
\end{lstlisting}
Вызывать функцию будем так:
\begin{lstlisting}[style={camlstyle1}]
reset (fun () -> times [1; 2; 0; 4])
\end{lstlisting}
\end{frame}

\begin{frame}[fragile]{Ответ на упражнение}
Дан список чисел, нужно их перемножить, а если встретился ноль, то \textit{сразу} вернуть ноль.
\begin{lstlisting}[style={camlstyle1}]
# let rec times lst = match lst with
  | [] -> 1
  | 0 :: tl -> @shift (fun _ -> 0)@
  | h :: tl -> h * times tl
;;
times : int list => int = <fun>
# reset (fun () -> times [1;2;0;4]) ;;
- : int = 0
# reset (fun () -> times [1;2;3;4]) ;;
- : int = 24
\end{lstlisting}
\end{frame}

\begin{frame}[fragile]{Как сохранять продолжения}
\begin{block}{}
\camline=shift (fun k -> k)=
\end{block}
\begin{itemize}
\item Возвращаем продолжение сразу и как есть
\item А потом его можно вызывать!
\end{itemize}

Пример: \camline=reset (fun () -> 3 + [...] - 1)=
\begin{lstlisting}[style={camlstyle1}]
# let f   = reset (fun () -> 3 + @shift (fun k -> k)@ - 1) ;;
f : int => int = <fun>
# f 10;;
- : int = 12
\end{lstlisting}

\begin{lstlisting}[style={camlstyle1}]
# let f @x@  = reset (fun () -> 3 + shift (fun k -> k) - 1) @x@;;
f : int -> int = <fun>
# f 10;;
- : int = 12
\end{lstlisting}
\end{frame}

\begin{frame}[fragile]{Упражнение}
\begin{block}{}
\camline=shift (fun k -> k)=
\end{block}

Вот identity для списка:
\begin{lstlisting}[style={camlstyle1}]
(* id : 'a list -> 'a list *)
let rec id lst = match lst with 
  | [] -> []                     (* modify here *)
  | h :: tl -> h :: id tl;; 
\end{lstlisting}
Внесите изменение в строчке, чтобы извлечь продолжение, если функция вызывается вот так:
\begin{lstlisting}[style={camlstyle1}]
reset (fun () -> id[1;2;3]) ;;
\end{lstlisting}
Что это продолжение делает?
\end{frame}

\begin{frame}[fragile]{Решение упражнения}
\begin{lstlisting}[style={camlstyle1}]
(* id : 'a list -> 'a list *)
let rec id lst = match lst with 
  | [] -> @shift (fun k -> k)@
  | h :: tl -> h :: id tl;; 
id : int list => int list = <fun>
\end{lstlisting}
\pause
\begin{lstlisting}[style={camlstyle1}]
# let append123 = reset (fun () -> id[1;2;3]) ;;
append123 : int list => int list = <fun>
# append123 [4;5;6];;
- : int list = [1;2;3;4;5;6]
\end{lstlisting}
\end{frame}


\begin{frame}[fragile]{Композиция}
% Можно увидеть некоторый шаблон:
\begin{center}
\camline!fun x -> reset (f x)!

{\Large $\Downarrow$ }

\camline!compose f g  = fun x -> reset (f (g x))!
\end{center}
\end{frame}

\begin{frame}[fragile]{Fix с callCC и delimCC}

\begin{center}
\begin{lstlisting}[style={camlstyle1}]
let k = callCC (fun x -> c) in 
k k
\end{lstlisting}

\begin{lstlisting}[style={camlstyle1}]
let fix0 f =
  reset (let x = shift (fun c -> c c) in
         f (fun a -> x x a))
\end{lstlisting}
\end{center}
\end{frame}


\begin{frame}[fragile]{Answer types}
\begin{lstlisting}[style={camlstyle1}]
let rec append lst = match lst with 
  | [] -> shift (fun k -> k)
  | h :: tl -> h :: append tl;;
\end{lstlisting}
\begin{lstlisting}[style={camlstyle1}]
let append123 = reset (fun () -> append [1;2;3]) ;;
\end{lstlisting}
\noindent\makebox[\linewidth]{\rule{\paperwidth}{0.4pt}}
\begin{center}
$ 1 :: 2 :: 3 :: \bullet$  \camline=  : 'a list=

$\Downarrow$ shift

$\lambda x\!s \rightarrow 1 :: 2 :: 3 :: x\!s$ \camline=  : 'a list -> 'a list =
\end{center}
Новый вид записи типов:

\camline='a list / 'a list -> 'a list / ('a list -> 'a list)= 
\vspace{0.5cm}

т.е. $\forall$ 'a функция по значению с типом \camline='a list= возвращает \camline='a list= в непосредственном контексте; однако, в процессе тип результата (answer type)  текущего контекста модифицируется до \camline='a list -> 'a list=.

\end{frame}

\begin{frame}[fragile]{Полиморфизм по answer type (1/2)}
Произвольные функции с типом \camline=S -> T= должны рассматриваться как полиморфные функции с типом \camline=S/'a -> T/'a=.\\ \vspace{0.5cm}
Рассмотрим (как бы мономорфную) функцию \camline!let add1 x = 1+x!

\begin{lstlisting}[style={camlstyle1}]
 reset (fun () -> add1 x; ())
 reset (fun () -> add1 x; true)
\end{lstlisting}

Два типа \camline=int/unit -> int/unit= и \camline=int/unit -> int/unit= (анти)унифицируются до \camline=int/'a -> int/'a=.
\end{frame}

\begin{frame}[fragile]{Полиморфизм по answer type (2/2)}
\begin{lstlisting}[style={camlstyle1}]
let rec visit lst = match lst with 
| [] -> @shift@ (fun h -> [])
| h :: tl -> h :: @shift@ (fun k -> 
                            (k []) :: reset (k (visit rest)))
let rec prefix lst = reset (visit lst)
\end{lstlisting}

Первый \camline=shift= начинает конструирование префиксов, возвращая \camline=[] : 'a list list=.

Второй \camline=shift= выражает consing и применяется два раза: 1) к пустому списку чтобы получить текущий ответ и 2) чтобы сконструировать список длинных префиксов.

Продолжение \camline=k= используется два раза в разных контекстах
\begin{itemize}
 \item \camline='a list / 'a list list -> 'a list / 'a list list=
 \item \camline='a list / 'a list -> 'a list / 'a list=
\end{itemize}

\end{frame}

\begin{frame}[fragile]{Пару слов про Prompt и STLC}
STLC обладает свойством strong normalization: последовательность редукций любого терма завершается. С добавлением delimCC -- уже нет.

\begin{lstlisting}[style={camlstyle1}]
let loop () =
  let p = new_prompt () in
  let delta () = shift p (fun f v -> f v v) () in
  push_prompt p (fun () -> let r = delta () in
                           fun v -> r
                ) delta ;;
\end{lstlisting}
Выводится тип \camline=loop : unit -> 'a=, но по сути это функция \camline=fun () -> omega= и она виснет. 

% Факт того, что функция impure затерялся в типе \camline='a=.

% ???
\end{frame}

\begin{frame}[fragile]{Типобезопасный printf}
\begin{lstlisting}[style={camlstyle1}]
let int x = string_of_int x
let str (x:string) = x

let % to_str = shift (fun k -> fun x -> k (to_str x))
let sprintf p = reset p
\end{lstlisting}
\begin{lstlisting}[style={camlstyle1}]
sprintf (fun () -> "Hello world!")
sprintf (fun () -> "Hello " ^ % str ^ "!") "world"
sprintf (fun () -> "The value " ^ % str ^ " is " ^ % int) "x" 4
\end{lstlisting} \vspace{1cm}

У \hsinline=sprintf= ''зависимое`` поведение с типом \camline=(unit / string -> string / 'a) -> 'a=. Без полиморфизма так нельзя было.
\end{frame}

\begin{frame}[fragile]{State monad (1/2)}
Создание:
\begin{lstlisting}[style={camlstyle1}]
reset (fun () -> M) 3
\end{lstlisting}
Доступ к состоянию:
\begin{lstlisting}[style={camlstyle1}]
# let get () =
    shift (fun k -> fun state -> k state state) ;;
get : unit => 'a = <fun>
\end{lstlisting}
Запускаем вычисление:
\begin{lstlisting}[style={camlstyle1}]
# let run_state thunk =
    reset (fun k -> let result = think () in
                    fun state -> result) 0 ;;
run_state : (unit => 'a) => 'b = <fun>
\end{lstlisting}

\end{frame} 



\begin{frame}[fragile]{State monad (2/2)}
Работаем с состоянием (пример):
\begin{lstlisting}[style={camlstyle1}]
# let tick () =
    shift (fun k -> fun state -> k () (state+1) ) ;;
tick : unit => unit = <fun>
\end{lstlisting}

\begin{lstlisting}[style={camlstyle1}]
# run_state (fun () ->
     tick ();                   (* state = 1 *)
     tick ();                   (* state = 2 *)
     let a = get () in
     tick ();                   (* state = 3 *)
     get () - a);;
- : int = 1
\end{lstlisting}
\end{frame}

\begin{frame}[fragile]{Вызываем несколько раз}
\begin{lstlisting}[style={camlstyle1}]
(* either : 'a -> 'a -> 'a *)
let either a b () = shift (fun k -> k a; k b)
\end{lstlisting}
\vspace{1cm}
\begin{lstlisting}[style={camlstyle1}]
# reset (fun () -> 
    let x = either 0 1 in
    print_int x
    print_newline ());;
0
1
- : unit = ()
\end{lstlisting}
\end{frame}

\begin{frame}[fragile]{Generate\&test}
\begin{center}
$(P \vee Q) \wedge (P \vee \neg Q) \wedge (\neg P \vee \neg Q)$
\end{center}
\vspace{1cm}
\begin{lstlisting}[style={camlstyle1}]
# reset (fun () -> 
    let p = either true false in
    let q = either true false in
    if (p||q) && (p || not q) && (not p || not q)
    then (print_string (string_of_bool p);
          print_string ", ";
          print_string (string_of_bool q);
          print_newline () );;
    
true, false
- : unit = ()
\end{lstlisting}
\end{frame}

\begin{frame}[fragile]{Дифференцирование парсеров (1/2)} 
Подробнее у \href{http://okmij.org/ftp/continuations/differentiating-parsers.html}{Олега }

\begin{lstlisting}[style={camlstyle1}]
type stream_req = ReqDone 
                | ReqChar of int * (char option -> stream_req)

let stream_inv p = fun pos -> 
  shift p (fun sk -> ReqChar (pos,sk))
val stream_inv : stream_req Delimcc.prompt -> 
                 int -> char option = <fun>

let cont str (ReqChar (pos,k) as req) = filler str pos req;;
val cont : string -> stream_req -> stream_req = <fun>
     
let finish (ReqChar (pos,k)) = filler "" pos (k None);;
val finish : stream_req -> stream_req = <fun>
\end{lstlisting}
\end{frame}

\begin{frame}[fragile]{Дифференцирование парсеров (2/2)} 

\begin{lstlisting}[style={camlstyle1}]
let rec filler buffer buffer_pos = function 
  | ReqDone -> ReqDone
  | ReqChar (pos,k) as req ->
       let pos_rel = pos - buffer_pos in
       let _ = 
         (* we don't accumulate already emptied buffers. We could. *)
         assert (pos_rel >= 0) 
       in
       if pos_rel < String.length buffer then
         (* desired char is already in buffer *)
         filler buffer buffer_pos (k (Some buffer.[pos_rel])) 
       else
         (* buffer underflow. Ask the user to fill the buffer *)
         req
;;
val filler : string -> int -> stream_req -> stream_req = <fun>
\end{lstlisting}
\end{frame}


\begin{frame}[fragile]{Синтаксис для $\lambda_{let}^{s/r}$}
\begin{itemize}
 \item Значения \\
 $v$ ::= $c \mid x \mid \lambda x.e \mid $ \verb=fix= $f.x.e$
 \item Выражения \\
$e ::= v \mid e_1e_2\mid \mathcal{S}k.e \mid \langle e\rangle \mid$ let $x$ = $e_1$ \verb=in= $e_2 \mid $ \verb=if= $e_1$ \verb=then= $e_2$ \verb=else= $e_3$
\item Мономорфные типы\\
$\alpha,\beta,\gamma,\delta ::= t \mid b \mid (\alpha/\gamma \rightarrow \beta/\delta)$
\item Полиморфные типы\\
$A::= \alpha \mid \forall t . A$

\item Evaluation context (e-context):\\
E ::= $[] \mid vE \mid Ee \mid $ 
\verb=let= $x$ = $E$ \verb=in= $e \mid $
\verb=if= $E$ \verb=then= $e$ \verb=else= $e \mid
\langle E\rangle  $

\item Pure e-context:\\
F ::= $[] \mid vF \mid Fe \mid $ 
\verb=let= $x$ = $F$ \verb=in= $e \mid $
\verb=if= $F$ \verb=then= $e$ \verb=else= $e$

\item RedEx:\\ 
R ::= $(\lambda x.e)v \mid$ \verb=let= $x$ = $F$ \verb=in= $e \mid $
\verb=if= $F$ \verb=then= $e$ \verb=else= $e$ $\mid \langle E\rangle$ $\mid $ $\langle F[\mathcal{S}k.e]\rangle$
\end{itemize}
\end{frame}

\begin{frame}[fragile]{Правила редукции для $\lambda_{let}^{s/r}$}
\begin{tabular}{ r  c  l }
$(\lambda x.e) v$ & $\rightsquigarrow$ & $e[v/x]$ \\
$($\textbf{fix} $ f.x.e) v$ & $\rightsquigarrow$ & $e[$ \textbf{fix} $ f.x.e/f][v/x]$ \\
$\langle v\rangle$ & $\rightsquigarrow$ & $v$ \\
$\langle F[\mathcal{S}k.e]\rangle$ & $\rightsquigarrow$ & $\langle$\textbf{let} $k = \lambda x.\langle F[x]\rangle$ \textbf{in} $e\rangle$ \\
\textbf{let} $x = v$ \textbf{in} $e$ & $\rightsquigarrow$ & $e[v/x]$ \\
\textbf{if true  then} $e_1$ \textbf{else} $e_2$ & $\rightsquigarrow$ & $e_1$ \\
\textbf{if false then} $e_1$ \textbf{else} $e_2$ & $\rightsquigarrow$ & $e_2$ \\
\end{tabular}
\end{frame}

\begin{frame}[fragile]{Пример редукции в $\lambda_{let}^{s/r}$}
\begin{tabular}{ l l }
                   & \camline=prefix [1;2]= \\
$\rightsquigarrow$ & $\langle$1 :: $\mathcal{S}k.(k[] :: \langle k($\texttt{visit [2]}$)\rangle)\rangle$ \\
$\rightsquigarrow$ & $\langle$\texttt{let} k = $\lambda x.\langle1::x\rangle$ \texttt{in} $k$\texttt{[]} :: $\langle k($\texttt{visit [2]}$)\rangle\rangle$ \\
$\rightsquigarrow$ & $\langle(\lambda x.\langle1::x\rangle)$ \texttt{[]} $:: \langle(\lambda x.\langle1::x\rangle) ($\texttt{visit [2]}$)\rangle\rangle$ \\
% $\rightsquigarrow^+$ & \camline!<<[1]::<<(\lambda x.<<1::x>>)(2::\mathS k.(k[]::(k visit [])>>)) >> >>! \\ 
$\rightsquigarrow^+$ & $\langle$\texttt{[1]}$::\langle(\lambda x.\langle 1::x\rangle)(2::\mathcal{S} k.(k$\texttt{[]}$::(k($\texttt{visit []}$)\rangle)) \rangle\rangle$\\ 
$\rightsquigarrow$ & $\langle$\texttt{[1]}$::\langle$\texttt{let k = }$\lambda x.\langle (\lambda x.\langle 1::x\rangle)(2::x)\rangle$\texttt{ in } \\
& \texttt{    }$\qquad \quad k$\texttt{[]} $:: \langle k($\texttt{visit []}$)\rangle\rangle\rangle$
\\
$\rightsquigarrow$ & $\langle$\texttt{[1]}$::\langle(\lambda x.\langle (\lambda x.\langle 1::x\rangle)(2::x)\rangle)$\texttt{ [] ::} \\
& $\qquad\quad\langle(\lambda x.\langle (\lambda x.\langle 1::x\rangle)(2::x)\rangle)($\texttt{visit []}$)\rangle\rangle\rangle$
\\
$\rightsquigarrow^+$ & $\langle$\texttt{[1]}$::\langle$\texttt{[1;2]} $::\langle(\lambda x.\langle (\lambda x.\langle 1::x\rangle)(2::x)\rangle)(\mathcal{S}h.$\texttt{[]}$)\rangle\rangle\rangle$ \\
$\rightsquigarrow$ & $\langle$\texttt{[1]}$::\langle$\texttt{[1;2]} $::$ \texttt{let} $h$ = $\lambda x.\langle (\lambda x.\langle 1::x\rangle)(2::x)\rangle)x\rangle$\texttt{ in []}$\rangle\rangle$\\
$\rightsquigarrow$ & $\langle$\texttt{[1]}$::\langle$\texttt{[1;2]} $::$ \texttt{[]}$\rangle\rangle$\\
$\rightsquigarrow^+$ & \texttt{[[1]; [1;2]]} \\
\end{tabular}
\end{frame}

\begin{frame}[fragile]{Вывод типов}

\begin{center}$\Gamma, \alpha \vdash e:\tau, \beta$\end{center}
В контексте $\Gamma$ выражение $e$ имеет тип $\tau$ и процесс вычисления $e$ изменяет answer type с $\alpha$ на $\beta$.\vspace{1cm}

При CPS-преобразовании тип у образа $e$ был бы $(\tau^* \rightarrow \alpha^*)\rightarrow \beta^*$. \vspace{1cm}

При добавлении типов хотелось бы сохранить \textit{type preservation property}: при вычислении выражения тип не должен меняться.
\end{frame}

\begin{frame}[fragile]{Система типов}
 Мономорфная система типов для \texttt{shift} и \texttt{reset} есть у Danvy \& Filinski \cite{Danvy89afunctional}. \vspace{0.5cm}
 
 Полиморфизм туда можно добавлять разными способами, ограничивая полиморфизм выражения \hsinline!let x = $e_1$ in $e_2$!.
 
 \begin{itemize}
  \item Value restriction: $e_1$ должно быть значением.
  \item ``Слабые'' типовые переменные: тип $e_1$ может быть обобщен (generalized) только когда он не связан с побочными эффектами.
  \item Полиморфизм по имени: вычисление $e_1$ откладывается до тех пор, когда $x$ таки начнет использоваться в $e_2$, это приводит к call-by-name семантикe для $e_1$.
  \item Pure выражения из \cite{asai2007} .
 \end{itemize}

\end{frame}

\begin{frame}[fragile]{Pure выражения}
Ограничим  \hsinline!let x = $e_1$ in $e_2$!, чтобы $e_1$ было чисто от эффектов управления, т.е. являлось \textit{pure}.\vspace{0.5cm}

Pure $\sim$ полиморфно по answer types.
\begin{center}$\Gamma, \alpha \vdash e:\tau, \alpha$\end{center}

Примеры:
\begin{itemize}
 \item значения
 \item $\langle e\rangle$, т.к. все эффекты отделены reset'ом.
\end{itemize}\vspace{0.5cm}

Обозначать будем так: $\Gamma \vdash_p e:\tau$.\vspace{0.5cm}

\end{frame}

\begin{frame}[fragile]{Правила вывода типов (1/2)}
$A \succ \tau$: инстанциация полиморфных переменных из $A$ какими-то мономорфными типами. \vspace{0.5cm}

Gen$(\sigma,\Gamma) \sim \forall t_1 ... t_n . \sigma$, где ${t_1 ... t_n}=\mathtt{FTV}(\sigma) - \mathtt{FTV}(\Gamma)$. \vspace{0.5cm}
\hline

\begin{prooftree}
\hypo{ (x : \sigma) \in \Gamma }
\hypo{ \sigma \succ \tau }
\infer2[(var)]{ \Gamma &\vdash_p x:\tau }
\end{prooftree}
\vspace{0.5cm}

% exp
\begin{prooftree}
\hypo{ \Gamma &\vdash_p M : \tau }
\infer1[(exp)]{ \Gamma, \alpha &\vdash_p M : \tau,\alpha }
\end{prooftree}
\vspace{0.5cm}

% let
\begin{prooftree}
\hypo{ \Gamma &\vdash M_1:\tau_1 }
\hypo{ \Gamma,x: \mathtt{Gen}(\tau_1,\Gamma),\alpha &\vdash M_2 : \tau_2, \beta}
\infer2[(let)]{ \Gamma,\alpha &\vdash \mathbf{let~x~=~} M_1 \mathtt{~in~} M_2 : \tau_2, \beta}
\end{prooftree}
\end{frame}

\begin{frame}[fragile]{Правила вывода типов (2/2)}
% shift
\begin{prooftree}
\hypo{ \Gamma, k: \forall t . (\tau/t \rightarrow \alpha/t),\gamma &\vdash M:\gamma,\beta}
\infer1[(shift)]{ \Gamma,\alpha &\vdash \mathcal{S}k.M : \tau, \beta}
\end{prooftree} \vspace{0.5cm}

% reset
\begin{prooftree}
\hypo{ \Gamma, \gamma &\vdash M:\gamma,\tau}
\infer1[(reset)]{ \Gamma,\alpha &\vdash_p \langle M\rangle:\tau}
\end{prooftree} \vspace{0.5cm}

% fun
\begin{prooftree}
\hypo{ \Gamma, x : \tau_1, \alpha &\vdash M : \tau_2,\beta }
\infer1[(fun)]{ \Gamma &\vdash_p \lambda x . M : (\tau_1/\alpha \rightarrow \tau_2/\beta) }
\end{prooftree} \vspace{0.5cm}

% app
\begin{prooftree}
\hypo{ \Gamma,\gamma &\vdash M_1:(\tau_1/\alpha \rightarrow \tau_2/\beta),\delta }
\hypo{ \Gamma,\beta &\vdash M_2 : \tau_1, \gamma}
\infer2[(app)]{ \Gamma,\alpha &\vdash M_1 M_2 : \tau_2, \delta}
\end{prooftree}
\end{frame}


\begin{frame}[fragile]{Свойства системы типов (1/2)}
\begin{block}{Subject reduction}
Если и $\Gamma; \alpha \vdash e_1:\tau; \beta$ выводимо, и $e_1  \rightsquigarrow^+ e_2$, тогда $\Gamma; \alpha \vdash e_2:\tau; \beta$. Аналогично, если $\Gamma \vdash_p e_1:\tau$, то $\Gamma \vdash_p e_2:\tau$ 
\end{block}\vspace{0.5cm}

Слабая непротиворечивость системы типов (\emph{weak type soundness}): правильно протипизированные программы работают хорошо. \vspace{0.5cm}

Сильная непротиворечивость системы типов (\emph{strong type soundness}): результат такого же типа, что исходный терм.

\begin{block}{Прогресс и единственность разложения}
Если выводится $\vdash_p \langle e\rangle:\tau$, то либо  $e$ просто значение, либо $\langle e\rangle$ можно единственным образом разложить в форму $E[R]$, где $E$ -- контекст, а $R$ -- RedEx.
\end{block}
Из двух свойств следует непротиворечивость(soundness).
\end{frame}

\begin{frame}[fragile]{Свойства системы типов (2/2)}
$W' : (\Gamma,e) \mapsto (\theta; \alpha, \tau, \beta)$ как расширение HM.

\begin{block}{Principal type и вывод типов}
Можно построить алгоритм $W'$ для $\lambda_{let}^{s/r}$ такой что 
\begin{enumerate}
 \item $W'$ завершается
 \item Если $W'$ вернул $(\theta; \alpha, \tau, \beta)$, то $\Gamma\theta; \alpha \vdash e:\tau, \beta$ выводимо. Кроме этого, для любых таких $(\theta'; \alpha',\tau',\beta')$, что $\Gamma\theta'; \alpha' \vdash e:\tau', \beta'$ выводимо, верно $(\Gamma\theta'; \alpha',\tau',\beta') \equiv (\theta; \alpha, \tau, \beta)\phi$ для некоторой подстановки $\phi$.
 \item Если $W'$ завершился с ошибкой, то $\Gamma\theta; \alpha \vdash e:\tau, \beta$ не выводимо ни для каких $(\theta; \alpha, \tau, \beta)$.
\end{enumerate}
\end{block}
\end{frame}

\begin{frame}[fragile]{}
\begin{block}{Confluence \& strong normalization}
\begin{enumerate}
 \item Редукции $\rightsquigarrow$ в $\lambda_{let}^{s/r}$ не зависят от порядка.
 \item Редукции $\rightsquigarrow$ в $\lambda_{let}^{s/r}$ без \emph{fix} всегда завершаются.
\end{enumerate}
\end{block}
\end{frame}


\begin{frame}[fragile]{}
\begin{center}
{\Huge Конец}
\vspace{0.5cm}

Дальше только список литературы
\end{center}

\end{frame}



\begin{frame}[allowframebreaks]
        \frametitle{Ссылки}
        \bibliographystyle{amsalpha}
        \bibliography{talkDelimCC.bib}
\end{frame}

\end{document}
