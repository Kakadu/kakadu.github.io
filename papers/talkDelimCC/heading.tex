%%%%%%%%%%%%%%%%%%%%%%%%%%%%%%%%%%%%%%%%%%%%%%%%%%%%%%%%%%  
\usepackage{fontspec}
\setmainfont[
 Ligatures=TeX,
 Extension=.otf,
 BoldFont=cmunbx,
 ItalicFont=cmunti,
 BoldItalicFont=cmunbi,
]{cmunrm}
% С засечками (для заголовков)
\setsansfont[
 Ligatures=TeX,
 Extension=.otf,
 BoldFont=cmunsx,
 ItalicFont=cmunsi,
]{cmunss}

\setmonofont[
 Ligatures=TeX,
 Extension=.otf,
 BoldFont=cmuntb,
 ItalicFont=cmunvi,
]{cmuntt}


\usepackage{xcolor}
\definecolor{YellowGreen} {HTML}{B5C28C}
\definecolor{ForestGreen} {HTML}{009B55}
\definecolor{DarkGreen}   {HTML}{009B55}
\definecolor{MyBackground}{HTML}{F0EDAA}

\usepackage{xltxtra} % load xunicode
\usepackage{polyglossia}
\setmainlanguage{russian}
\setotherlanguage{english}

% \let\cyrillicfonttt\monofamily
% \usepackage[TT={Scale=0.88,FakeStretch=0.9},
% SS={Scale=0.9},
% RM={Scale=0.9},
% DefaultFeatures={Ligatures=TeX}]{dejavu-otf} % support opentype DejaVu fonts
% https://github.com/vjpr/monaco-bold/raw/master/MonacoB/MonacoB.otf
% \newfontfamily\monacoB{MonacoB} 

% \usepackage{inconsolata}
\usepackage{listings}
% \lstdefinestyle{style1}{
%   language=haskell,
%   numbers=left,
%   stepnumber=1,
%   numbersep=10pt,
%   tabsize=4,
%   showspaces=false,
%   showstringspaces=false
% }
\lstdefinestyle{hsstyle1}
         { showstringspaces=false
         , keywords={data}         
         , basicstyle=\ttfamily
%          , showspaces=false
%          , showstringspaces=false
%          , columns=fixed 
         , language=haskell
%          , deletekeywords={Int,Float,String,List,Void}
%          , breaklines=true
         %, columns=fullflexible
          , escapeinside={§§}
%          , escapebegin=\begin{russian}\monacoB\color{DarkGreen}
%          , escapeend=\end{russian}
          , commentstyle=\color{DarkGreen}
%          , escapebegin=\begin{russian}\monacoB\color{ForestGreen}
%          , escapeend=\end{russian}
         , mathescape=true
%          , backgroundcolor = \color{MyBackground}
}

\lstdefinelanguage{ocaml}{
keywords={@type, function, fun, let, in, match, with, when, class, type,
object, method, of, rec, repeat, until, while, not, do, done, as, val, inherit,
new, module, sig, deriving, datatype, struct, if, then, else, open, private, virtual, include, success, failure,
assert, true, false, end},
sensitive=true,
commentstyle=\small\itshape\ttfamily,
keywordstyle=\ttfamily\underbar,
identifierstyle=\ttfamily,
basewidth={0.5em,0.5em},
columns=fixed,
fontadjust=true,
literate={->}{{$\to$}}3 ,
morecomment=[s]{(*}{*)}
}
\lstdefinestyle{camlstyle1}
  { language=ocaml
  , moredelim=**[is][\color{red}]{@}{@}
}

\newcommand{\inline}[1]{\lstinline{haskell}{#1}}

\def\hsinline{\lstinline[style={hsstyle1}]}
\def\camline{\lstinline[style={camlstyle1}]}
\def\inline{\hsinline}
\def\myinline{\lstinline[basicstyle=\monacoB]}
\def\hline{\noindent\makebox[\linewidth]{\rule{\paperwidth}{0.4pt}}}

% \renewcommand{\vskip}{\vspace{1cm}}

\lstnewenvironment{hslisting} {
    \lstset { style={hsstyle1} }
  }
  {}
  
\lstset
  { language=haskell
%   , basicstyle=\monacoB         % print whole listing small
%   , stringstyle=\monacoB      % typewriter type for strings
  , showstringspaces=false       % no special string spaces
  }

\usefonttheme{professionalfonts}
\usepackage{times}
\usepackage{tikz}
\usetikzlibrary{cd}
% \usepackage{tikz-cd}
\usepackage{amsmath}
%\DeclareMathOperator{->}{\rightarrow}
\newcommand\iso{\ensuremath{\cong}}
\usepackage{verbatim}
\usepackage{graphicx}
\usetikzlibrary{arrows,shapes}
\usepackage{verbatimbox} % for verbnobox

\usepackage{fontawesome}
% \newfontfamily{\FA}{Font Awesome 5 Free} % some glyphs missing
\expandafter\def\csname faicon@facebook\endcsname{{\FA\symbol{"F09A}}}
\def\faQuestionSign{{\FA\symbol{"F059}}}
\def\faQuestion{{\FA\symbol{"F128}}}
\def\faExclamation{{\FA\symbol{"F12A}}}
\def\faUploadAlt{{\FA\symbol{"F093}}}
\def\faLemon{{\FA\symbol{"F094}}}
\def\faPhone{{\FA\symbol{"F095}}}
\def\faCheckEmpty{{\FA\symbol{"F096}}}
\def\faBookmarkEmpty{{\FA\symbol{"F097}}}
\def\faCheck{{\FA\symbol{"F00C}}}

\newcommand{\faGood}{\textcolor{ForestGreen}{\faThumbsUp}}
\newcommand{\faBad}{\textcolor{red}{\faThumbsODown}}
\newcommand{\usefulExercise}{\textcolor{red}{\faPlus}}

\usepackage{soul} % for \st that strikes through
\usepackage[normalem]{ulem} % \sout
% \newcommand{\hline}{\noindent\makebox[\linewidth]{\rule{\paperwidth}{0.4pt}}}
% \usepackage{minted}
% \newcommand{\inline}[1]{\mintinline{haskell}{#1}}

\hypersetup{%
  colorlinks=true, linkcolor=blue
}
% \usepackage[colorlinks=true, linkcolor=blue]{hyperref} 
