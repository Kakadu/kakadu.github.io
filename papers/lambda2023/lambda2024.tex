% !TeX spellcheck = ru_RU
% !TEX TS-program = xelatex

\documentclass[aspectratio=169
  , xcolor={svgnames}
  , hyperref=
      { colorlinks
      , urlcolor=DarkBlue
      }
  , russian  % This line affects translation of theorem titles
  ]{beamer}

\usetheme{CambridgeUS}

\makeatletter
\@ifclassloaded{beamer}{
  % get rid of header navigation bar
  \setbeamertemplate{headline}{}
  % get rid of bottom navigation symbols
  \setbeamertemplate{navigation symbols}{}
  % get rid of footer
  %\setbeamertemplate{footline}{}
  \setbeamertemplate{section in toc} {\inserttocsectionnumber.~\inserttocsection}
  \usefonttheme{professionalfonts}
}
{}
\makeatother
%%%%%%%%%%%%%%%%%%%%%%%%%%%%%%%%%%%%%%%%%%%%%
\usepackage{ifxetex,ifluatex}
\newif\ifxetexorluatex
    \ifxetex
        \xetexorluatextrue
    \else
        \ifluatex
            \xetexorluatextrue
        \else
            \xetexorluatexfalse
        \fi
    \fi

\ifxetexorluatex
    \usepackage{fontspec}
    \usepackage{unicode-math}
    \setmainfont[Ligatures=TeX]{CMU Serif}
    \setsansfont[Ligatures=TeX]{CMU Sans Serif}
    \setmathfont[Scale=MatchUppercase]{Asana Math}
    \setmonofont{Monaco for Powerline}[Scale=0.9]

%\newfontfamily{\myfiracode}[Scale=1.5,Contextuals=Alternate]{Fira Code}
%\setmonofont[Scale=0.9,BoldFont={Inconsolata Bold}]{Inconsolata}

    \usepackage{polyglossia}
    \setmainlanguage{russian}
    \setotherlanguage{english}
\else
    \usepackage[utf8]{inputenc}
    \usepackage[T2A]{fontenc}
    \usepackage[russian,english]{babel}
\fi
%\newfontfamily{\cyrillicfonttt}{Monaco for Powerline}
%  [ Contextuals=Alternate
%  , Scale=0.8
%  , BoldFont=MonacoB
%  ]

%\newfontfamily\dejaVuSansMono{DejaVu Sans Mono}
% https://github.com/vjpr/monaco-bold/raw/master/MonacoB/MonacoB.otf
%\newfontfamily\monacoB{MonacoB}

\usepackage{fontawesome}
% \newfontfamily{\FA}{Font Awesome 5 Free} % some glyphs missing
\expandafter\def\csname faicon@facebook\endcsname{{\FA\symbol{"F09A}}}
\def\faQuestionSign{{\FA\symbol{"F059}}}
\def\faQuestion{{\FA\symbol{"F128}}}
\def\faExclamation{{\FA\symbol{"F12A}}}
\def\faUploadAlt{{\FA\symbol{"F093}}}
\def\faLemon{{\FA\symbol{"F094}}}
\def\faPhone{{\FA\symbol{"F095}}}
\def\faCheckEmpty{{\FA\symbol{"F096}}}
\def\faBookmarkEmpty{{\FA\symbol{"F097}}}

\newcommand{\faGood}{\textcolor{ForestGreen}{\faThumbsUp}}
\newcommand{\faBad}{\textcolor{red}{\faThumbsODown}}
\newcommand{\faWrong}{\textcolor{red}{\faTimes}}
\newcommand{\faMaybe}{\textcolor{blue}{\faQuestion}}
\newcommand{\faCheckGreen}{\textcolor{ForestGreen}{\faCheck}}
%%%%%%%%%%%%%%%%%%%%%%%%%%%%%%%%%%%%%%%%%%%%%


\usepackage{relsize} % \mathlarger{..}
\usepackage{hyperref}
%\usepackage{comment}
%\usepackage{easyReview}
%\usepackage{changes}
% What to pick?
%https://mengxiangxi.info/BLOG/research/2021/09/20/Good-practice-revising-latex.html

%%%%%%%%%%%%%%%%%%%%%%%%%%%%%%%%%%%%%%%%%%%%%%%5
\usepackage[useregional]{datetime2}
\usepackage{soul} % for \st that strikes through
\usepackage[normalem]{ulem} % \sout

\usepackage{stmaryrd}
\newcommand{\sem}[1]{\ensuremath{\llbracket #1\rrbracket}}

% We should use the following commands as \ocaml{} to prevent chewing
% extra space
\newcommand{\ocaml}{\textsc{OCaml}}
\newcommand{\OCaml}{\ocaml}
\newcommand{\haskell}{\textsc{Haskell}}
\newcommand{\Haskell}{\haskell}
\newcommand{\Rust}{\textsc{Rust}}
\newcommand{\Coq}{\textsc{Coq}}
\newcommand{\Java}{\textsc{Java}}
\newcommand{\CXX}{\textsc{C++}}

\author{Косарев Дмитрий}
\institute{матмех}

\addtobeamertemplate{title page}{}{
    \begin{center}
        {\tiny Дата сборки: \DTMtoday}
    \end{center}
}

%\let\thefootnote\relax\footnotetext{Put your text here}
\DeclareMathOperator{\arr}{\rightarrow}

%%%%%%%%%%%%%%%%%%%%%%%%%%%%%%%%%%%%%%%%%%%%%%%%%%%%%%%%%%%
\usepackage{tikz}
% For every picture that defines or uses external nodes, you'll have to
% apply the 'remember picture' style. To avoid some typing, we'll apply
% the style to all pictures.
\tikzstyle{every picture}+=[remember picture]

% By default all math in TikZ nodes are set in inline mode. Change this to
% displaystyle so that we don't get small fractions.
\everymath{\displaystyle}
\usetikzlibrary{cd}
\usepackage{tikz-cd}
\usepackage{caption}
\usepackage{subcaption}
\usepackage{amsthm}

%%\DeclareMathOperator{->}{\rightarrow}


\usepackage{amsmath}
\usepackage{amssymb}
%\usepackage{amsfonts}
\usepackage{scalerel}
\DeclareMathOperator*{\myvee}{\scalerel*{\vee}{\sum}}
\DeclareMathOperator*{\mywedge}{\scalerel*{\wedge}{\sum}}

%
%\usepackage{tabulary}
%
%% sudo aptget install ttf-mscorefonts-installer
%%\setmainfont{Times New Roman}
%%\setsansfont[Mapping=tex-text]{DejaVu Sans}
%
%%\setmonofont[Scale=1.0,
%%    BoldFont=lmmonolt10-bold.otf,
%%    ItalicFont=lmmono10-italic.otf,
%%    BoldItalicFont=lmmonoproplt10-boldoblique.otf
%%]{lmmono9-regular.otf}
%
% https://tex.stackexchange.com/questions/380799/warning-when-adding-package-minted
\usepackage[autostyle]{csquotes}

% color options
\definecolor{YellowGreen} {HTML}{B5C28C}
\definecolor{ForestGreen} {HTML}{009B55}
\definecolor{MyPurple}{HTML}{7f007f}
\def\HaskellTypeclassColor\PYG{k+kt}
\def\HaskellCommentColor\PYG{c+c1}

%\ifxetexorluatex
%  \usepackage{fontawesome}
%\else
%\fi


\newif\ifminted
\newif\iflistings

\listingstrue
%\mintedtrue

\iflistings
    \usepackage{listings}

    %\newcommand{\inline}[1]{\lstinline{haskell}{#1}}
    % TODO: https://tex.stackexchange.com/questions/4198/emphasize-word-beginning-with-uppercase-letters-in-code-with-lstlisting-package
    \definecolor{eclipseGreen}{RGB}{63,127,95}

    % https://tex.stackexchange.com/a/4199
    \makeatletter
    \newcommand*\ocamlidstyle{%
            \expandafter\id@style\the\lst@token\relax
    }
    \def\id@style#1#2\relax{%
            \ifcat#1\relax\else
                    \ifnum`#1=\uccode`#1%
                             \ttfamily\bfseries\color{MyPurple}
                    \else
                                                \ttfamily
                    \fi
            \fi
    }
    \makeatother


    \lstdefinelanguage{ocaml}{
        basicstyle=\ttfamily   % Вот тут надо стиль ставить, а не у идентификаторов
%        , identifierstyle=\ocamlidstyle  % конфликтует, если идентификаторы стартуют с подчерка
        %, commentstyle=\HaskellCommentColor\itshape\HaskellCommentColor
        , sensitive=true
        %
        , classoffset=0
        , keywords={ fun, function, and, let, rec, in, match, with, when
            , class, type, of, do, as, val
            , inherit, module, struct, sig
            , if, then, else
            , assert, true, false, begin, end
            , Some, None
        }
        , keywordstyle=\ttfamily\bfseries\color{MyPurple} %\underbar
        , classoffset=1
        , morekeywords={pure,empty,select,branch,oneOf}
        , keywordstyle=\color{MyPurple}
        , classoffset=2
        , morekeywords={Monad,Applicative,Selective,String
            ,Either,Left,Right
            ,Maybe,Some,None
        }
        , keywordstyle=\PYG{k+kt}
        , classoffset=0,
        %keywordstyle=[2]{\color{orange}},
        otherkeywords={::},
        %identifierstyle=\fontfamily{cmtt}\selectfont\ttfamily,
        %basewidth={0.5em,0.5em},
        columns=fixed,
        %fontadjust=true,
        %literate={->}{{$\to$}}3 {===}{{$\equiv$}}1 {=/=}{{$\not\equiv$}}1 {|>}{{$\triangleright$}}3 {\\/}{{$\vee$}}2 {/\\}{{$\wedge$}}2 {>=}{{$\ge$}}1 {<=}{{$\le$}} 1,
        , morecomment=[s]{(*}{*)}
        , commentstyle=\color{eclipseGreen} % style of comments
        %, literate={\$}{{\textcolor{blue}{\$}}}1
        %, literate={<\$>}{{\textcolor{RawSienna}{\ <\$>\ } }}1
        %           {>?>}{{\textcolor{RawSienna}{\ >?>\ } }}1
    }
    \lstset{ language=ocaml }
    \lstnewenvironment{mlisting}[1][]{\lstset{inputencoding=latin1, language=ocaml,#1}%
    }{%
    }

    %\def\mlinline[1]{\lstinline[langauge=ocaml]{#1}} % is not possible
    \ifpdftex
        \usepackage{etoolbox}
        \expandafter\patchcmd\csname \string\lstinline\endcsname{%
            \leavevmode
            \bgroup
        }{%
            \leavevmode
            \ifmmode\hbox\fi
            \bgroup
        }{}{%
            \typeout{Patching of \string\lstinline\space failed!}%
        }
    \fi
\fi

\ifminted
    \usepackage[cache=true]{minted}
    \usemintedstyle{perldoc}
    \def\hsinline{\mintinline{haskell}}
    \def\mlinline{\mintinline[escapeinside=||]{ocaml}}

%\def\hsinline{\mintinline{haskell}}
%\def\inline{\hsinline}
\fi


\usepackage{tikz} % Мощный пакет для создание рисунков, однако может очень сильно замедлять компиляцию
\usetikzlibrary{decorations.pathreplacing,calc,shapes,positioning,tikzmark}
\newcounter{tmkcount}
\usepackage{tikzsymbols}

\tikzset{
    use tikzmark/.style={
        remember picture,
        overlay,
        execute at end picture={
            \stepcounter{tmkcount}
        },
    },
    tikzmark suffix={-\thetmkcount}
}
%\newcommand\centerarc{} % just for safety
\def\tikzHighlight(#1)(#2){
  \draw[fill=gray,opacity=0.1]
    ([shift={(-3pt,2ex)}]pic cs:#1)
    rectangle
    ([shift={(3pt,-0.65ex)}]pic cs:#2);
}


\renewcommand{\epsilon}{\varepsilon}
\renewcommand{\theta}{\vartheta}
\renewcommand{\kappa}{\varkappa}
\renewcommand{\rho}{\varrho}
\renewcommand{\phi}{\varphi}

\usepackage{tabulary}
\usepackage{verbatim}
\usepackage{amsmath}
\usepackage{relsize}

\newcommand{\term}[2]{\textit{#1} (#2)}
\newcommand\alphaequiv{\overset{\alpha}{\equiv}}
\newcommand{\xarr}[1]{\xrightarrow{\ #1\ }}
\renewcommand{\arr}{\rightarrow}
\newcommand{\arrmany}{\xarr{*}}
\newcommand{\cbv}{\xarr{cbv}}
\newcommand{\cbn}{\xarr{cbn}}
\newcommand{\nor}{\xarr{nor}}
\newcommand{\ao}{\xarr{ao}}
\newcommand{\betaarr}{\xarr{\beta}}

\newcommand{\abs}[2]{\ensuremath{(\lambda #1 . #2)}}
\newcommand{\lam}[2]{\ensuremath{\abs{#1}{#2}}}
\newcommand{\app}[2]{\ensuremath{(#1 #2)}}
\newcommand{\subst}[3]{\ensuremath{[#1 \mapsto #2]#3}}
\newcommand{\substt}[3]{\ensuremath{[#2/#1]#3}}

\usepackage{mathpartir}
\usepackage{tabulary}
\usepackage{multirow}
\usepackage{makecell}
\usepackage{appendixnumberbeamer}
%%%%%%%%%%%%%%%%%%
\makeatletter
%\newenvironment{tabminted}{
%  \let\FV@ListVSpace\relax
%  \minted
%}{
%  \endminted
%  \unskip
%  \aftergroup\@tabmintedend
%}
%\newcommand*{\tabminted@finalstrut}[1]{%
%  \ifdim\prevdepth>0pt
%    \ifdim\dp#1>\prevdepth
%      \vskip\dimexpr(\dp#1)-\prevdepth\relax
%    \fi
%  \else
%    \vskip\dimexpr(\dp#1)\relax
%  \fi
%}
%\newcommand*{\@tabmintedend}{%
%  \let\@finalstrut\tabminted@finalstrut
%}
%\renewcommand{\cite}[1]{}
%\makeatother


%%%%%%%%%%%%%%%%%%%%%%%%%%%%%%%%%%%%%%%%%%
\title[]{Лямбда-исчисление}
%\subtitle{Набросок слайдов}
\author{Косарев Дмитрий}

%\institute{матмех СПбГУ}


\AtBeginSection[]
{
  \begin{frame}<beamer>%[allowframebreaks]
    \frametitle{Оглавление}
    \tableofcontents[currentsection,currentsubsection]
  \end{frame}
}
\AtBeginSubsection[]
{
  \begin{frame}<beamer>%[allowframebreaks]
    \frametitle{Оглавление}
    \tableofcontents[currentsection,currentsubsection]
  \end{frame}
}

\newcommand{\verbatimfont}[1]{\def\verbatim@font{#1}}
\setcounter{tocdepth}{2}  % part,chapters,sections
%\newcommand\chap[1]{
%  \addcontentsline{toc}{chapter}{#1}
%}


\begin{document}
\maketitle

% For every picture that defines or uses external nodes, you'll have to
% apply the 'remember picture' style. To avoid some typing, we'll apply
% the style to all pictures.
\tikzstyle{every picture}+=[remember picture]

% By default all math in TikZ nodes are set in inline mode. Change this to
% displaystyle so that we don't get small fractions.
\everymath{\displaystyle}

% Uncomment these lines for an automatically generated outline.
%\begin{frame}{Оглавление}
%  \tableofcontents
%\end{frame}


% !TeX root = lambda2023.tex


\defverbatim[colored]{\cast}{
\begin{minted}{c}
enum Tag { VAR, ABS, APP };
struct ulc {
  Tag tag;
  union body {
    struct Var { char* name; } Var;
    struct Abs { char* name; ulc* body; } Abs;
    struct App { ulc* f; ulc* arg; } App;
  } body;
};
\end{minted}
}

\defverbatim[colored]{\strat}{
\begin{minted}{c}
struct Strategy {
  ulc* (*onVar)(Strategy* self, char* name);
  ulc* (*onApp)(Strategy* self, struct ulc *f, struct ulc *arg);
  ulc* (*onAbs)(Strategy* self, char *name, struct ulc *arg);
};

struct ulc* applyStrategy(Strategy *self, struct ulc *root) {
  switch (root->tag) {
    case VAR: return self->onVar(self, root->body.Var.name);
    case APP: return self->onApp(self, root->body.App.f, root->body.App.arg);
    case ABS: return self->onAbs(self, root->body.Abs.name, root->body.Abs.body);
  }
  assert(false);   return nullptr; // unreachable
}
\end{minted}
}


\defverbatim[colored]{\nostrat}{
  \begin{minted}{c}
struct ulc *evalVar(Strategy *this, char *name) {
  return var(name);
}
struct ulc *dontReduceUnderAbstraction(Strategy *this, char  *name, ulc *body) {
  return abs(name, body);
}
struct ulc *dontReduceApplication(Strategy *this, ulc* f, ulc* arg) {
  return app(f, arg);
}

struct Strategy NoStrategy = {
  .onvar = evalVar,
  .onApp = dontReduceApplication,
  .onAbs = dontReduceUnderAbstraction,
};
  \end{minted}
}

\defverbatim[colored]{\cbvstrat}{
  \begin{minted}{c}
struct ulc *evalApplyByValue(Strategy *self, ulc *f, ulc *a1) {
  auto f2 = applyStrategy(self, f);
  switch (f2->tag) {
    case VAR:    case APP:      return app(f2, a1);
    case ABS: {
      auto a2 = applyStrategy(self, a1);
      auto r = subst(a2, f2->body.Abs.name, f2->body.Abs.body);
      return applyStrategy(self, r);
    }
  }
  assert(false); return nullptr; // unreachable
}
struct Strategy CallByValue = {
  .onvar = evalVar,
  .onApp = evalApplyByValue,
  .onAbs = dontReduceUnderAbstraction };
  \end{minted}
}

\defverbatim[colored]{\inheritance}{
\begin{minted}{c}
struct ulc *evalApplyByValue(Strategy *this, ulc *f, ulc *arg)
struct ulc *evalVar(Strategy *this, char *name);
struct ulc *dontReduceUnderAbstraction(Strategy *this, char *name, ulc *body);
struct ulc *dontReduceApplication(Strategy *this, ulc *f, ulc *arg);

struct Strategy NoStrategy = {
  .onvar = evalVar,
  .onAbs = dontReduceUnderAbstraction,
  .onApp = dontReduceApplication,
};
struct Strategy CallByValue = {
  .onvar = evalVar,
  .onAbs = dontReduceUnderAbstraction
  .onApp = evalApplyByValue,
};
\end{minted}
}


\section*{Введение}

%{
%\setbeamertemplate{headline}{}
%\setbeamertemplate{footline}{}
%\usebackgroundtemplate{
%  \includegraphics[width=\paperwidth]{munch2.jpg}
%}
%\begin{frame}
%\end{frame}
%}

%\section{Введение в Haskell}


%\defverbatim[colored]{\imageA}{
%\begin{tikzpicture}
%    [%%%%%%%%%%%%%%%%%%%%%%%%%%%%%%
%        box/.style={rectangle,draw=black, ultra thick, minimum size=1cm},
%    ]%%%%%%%%%%%%%%%%%%%%%%%%%%%%%%
%
%\foreach \x/\y in {0/9, 1/\faAmazon,2/13,3/19,4/12,5/8,6/7,7/4,8/21,9/2,10/6,11/11}
%        \node[box] at (\x,0){\y};
%
%\end{tikzpicture}
%}


\begin{frame}{Введение}
\begin{center}
\only<1>{
\includegraphics{tikzpics/array1.pdf}
}
\only<2>{
\includegraphics{tikzpics/array2.pdf}
}
\only<3>{
\includegraphics{tikzpics/array1.pdf}
}
\only<4>{
\includegraphics{tikzpics/array1.pdf}
\vspace{2em}

\includegraphics{tikzpics/array2.pdf}
}
\end{center}
\end{frame}

\begin{frame}{Чистые функции}
\begin{definition}{Чистая функция -- это}
  \begin{itemize}
    \item Детерминированная
    \item В процессе работы не совершающая ``побочных эффектов''
  \end{itemize}
\end{definition}
Т.е. запрещены: ввод-вывод, случайные значения, присваивания\\

N.B. Это свойство \emph{функции}, а не языка программирования
\end{frame}

\begin{frame}%{Чистые функции}
\begin{definition}[Неизменяемые структуры данных (immutable data structures)]
  Которые с течением времени не изменяются \faSmileO
\end{definition}

\vspace{1em}

\begin{definition}[Устойчивые структуры данных (persistent data structures)]
Имеют доступ (не уничтожают) предыдущее своё состояние
\end{definition}
Почти то же самое, только акцент смещён\vspace{1em}

\begin{remark}
Так как старые узлы есть, то можно их использовать (share) в новой версии структуры данных
\end{remark}
\begin{definition}[Неустойчивые структуры данных называются \textit{эфемерными (ephemeral)}]
\end{definition}
\end{frame}



\begin{frame}%[allowframebreaks]
  \frametitle<presentation>{Ссылки}
  \begin{thebibliography}{10}
    \bibitem{litvinov}
      {\em Слайды Ю.~Литвинова} \newblock\url{https://github.com/yurii-litvinov/courses/tree/master/structures-and-algorithms/03-lambda-calculus}
  \end{thebibliography}
\end{frame}

%\appendix

%% !TeX root = lambda2023.tex


\section{Проблема останова}

\begin{frame}{Проблема останова (1/2)}
  \begin{block}{Вопрос}
    Можем ли мы написать алгоритм, который будет брать на вход произвольную $\lambda$-абстракцию и аргумент, и говорить посчитается ли для их применения нормальная форма?\\
  \end{block}
  \pause
  Положим наши программы либо зависают, либо выдают значение \texttt{true}.\\

  Положим существует гипотетическая $(Halting\ P\ w)$, которая всегда завершается, и возвращает \texttt{true}, если $(P\ w)$ редуцируется в \texttt{true}, иначе $(Halting\ P\ w)$ возвращает \texttt{false}.\\

  Покажем от противного, что $Halting$ не может существовать.
\end{frame}

\begin{frame}{Проблема останова (2/2)}
  Вопрос: во что редуцируется $E$, в \texttt{true} или в \texttt{false}?
  \[
  E = Halting\big( \lam{m}{not(Halting\ m\ m)},\ \ \lam{m}{not(Halting \ m\ m)} \big)
  \]
  Если E редуцируется в \texttt{true}, то применим функцию $\lam{m}{not(Halting(m,m))}$ к аргументу $\lam{m}{not(Halting(m,m))}$ и получим
  \[
  not ( Halting\big(\lam{m}{not(Halting \ m\ m)},\ \ \lam{m}{not(Halting \ m\ m)}\big)) = \neg E
  \]
  что является отрицание истинного факта выше.\\

  Если $E$ редуцируется в \texttt{false}, то это означает, $Halting$ иногда зависает, что противоречит определению функции $Halting$.
  \qed
\end{frame}


%\section{Две стратегии}
\begin{comment}

  \begin{frame}{Как происходят вычисления (редукция) $\lambda$-исчислении?}
    \begin{definition}[Процесс вычислений регламентирует стратегия]
      Ищем редексы $(\lambda x. P)Q$
      \begin{itemize}
        \item Если редексов нет, то вычисление закончилось
        \item Если редексы есть, стратегия регламентирует какой на данном шаге редекс стоит $\beta$редуцировать
        \item Или же, стратегия может сказать, что все редексы нужно оставить как есть, и выдать ответ.
      \end{itemize}
    \end{definition}
    %Стратегий бывает много разных
    %\begin{enumerate}
    %  \item Строгая стратегия call-by-value .    Для $(\lambda x. P)Q$ вычисляет $Q\cbv Q'$ и потом подставляет $Q'$ вместо $x$ в $P$.
    %  \item Ленивая стратегия call-by-name.   Для $(\lambda x. P)Q$ сразу подставляет $Q$ вместо $x$ в $P$.
    %  \item Обе стратегии оставляют абстракции и переменные как есть
    %\end{enumerate}
  \end{frame}

  \begin{frame}{Ленивая vs. Строгая}
    Пример 1 ($\cbv$ выглядит лучше)\\
    $(\lambda x. f x x)((\lambda x. x)A) \cbv (\lambda x. f x x)A \cbv (f A A) \cbv \dots $\\

    $(\lambda x. f x x)((\lambda x. x)A) \cbn (\lambda x. f ((\lambda x. x)A) ((\lambda x. x)A))A \cbn \dots $

    \vspace{3em}
    Пример 2 ($\cbn$ выглядит лучше)\\
    $(\lambda x. \lambda y. y)((\lambda x. xx)(\lambda x. xx)) \cbv (\lambda x. \lambda y. y)((\lambda x. xx)(\lambda x. xx)) \cbv \dots \text{зависло}$

    $(\lambda x. \lambda y. y)((\lambda x. xx)(\lambda x. xx)) \cbn (\lambda y. y)\quad\text{ответ!}$
  \end{frame}
\end{comment}



\begin{comment}


  \newcommand{\ite}[3]{\ensuremath{(\text{if } #1\text{ then }#2\text{ else }#3})}}
\begin{frame}{Рекурсия в call-by-name}
  $$
  Y\equiv \lam{f}{\lam{x}{f(xx)}\lam{x}{f(xx)}}
  $$
  %Основное свойство
  \[
  YR = \lam{x}{R(xx)} \lam{x}{R(xx)} \nor
  R\big( \lam{x}{R(xx)}\lam{x}{R(xx)} \big) =
  R(YR)
  \]

  \vspace{1em}

  Факториал: $fac \equiv (\lambda self.\lambda n . \ite{n<2}{1}{n \cdot self(n-1)})$
  \begin{align*}
    Y(\lambda self.\lambda n . \ite{n<2}{1}{n \cdot self(n-1)}) 2 &\cbn \\
    (\lambda n . \ite{n<2}{1}{n \cdot Y\ fac(n-1)}) 2 &\cbn \\
    2 \cdot Y\ fac\ (2-1) &\cbn \\
    2 \cdot (Y(\lambda self.\lambda n . \ite{n<2}{1}{n \cdot self(n-1)})\ 1) &\cbn \\
    2 \cdot \ite{1<2}{1}{n \cdot (Y\ fac\ (1-1))} &\cbn \\
    2 \cdot 1 & \cbn 2
  \end{align*}

\end{frame}
\end{comment}



\section{Дополнительные слайды}

\begin{frame}{Нормальные формы}

У нас как минимум четыре возможности
\begin{itemize}
  \item Редуцируем ли под абстракциями? (да/нет)
  \item Редуцируем ли аргументы перед подстановкой? (да/нет)
\end{itemize}

\begin{center}
  \begin{table}[]
    \begin{tabular}{|c|m{6cm}|m{6cm}|}
      \hline\hline
      \multirow{2}{*}{\thead{Редуцируем \\аргументы?}} & \multicolumn{2}{c|}{Редуцируем под абстракциями?} \\ \cline{2-3}
      &   Да(strong)        &     Нет(weak)     \\ \hline\hline
      Да(strict)  & \thead{Normal form \\
        $E ::=\lam{x}{E} \mid xE_1\dots E_n$} &
      \thead{Weak normal form \\
        $E ::=\lam{x}{e} \mid xE_1\dots E_n$}    \\
      Нет(lazy) & \thead{Head normal form \\
        $E ::=\lam{x}{E} \mid xe_1\dots e_n$} &
      \thead{Weak head normal form \\
        $E ::=\lam{x}{e} \mid xe_1\dots e_n$}  \\ \hline\hline
    \end{tabular}
  \end{table}
\end{center}

В таблице $E_j$  -- это выражение в соответствующей нормальной форме, а $e_i$ -- произвольный $\lambda$-терм.\\
\footnotetext{То, что у некоторых $E$ нет индексов -- не опечатка}
\end{frame}

%
\begin{frame}{Порядков редукции бывает много...\cite{setsoft}}
\begin{enumerate}
  \item Call-by-Name
  \item Normal Order
  \item Call-by-Value (OCaml)
  \item Applicative Order
  \item Hybrid Applicative Order
  \item Head Spine Reduction
  \item Hybrid Normal Order
\end{enumerate}
И ещё есть оптимизации связанные с мемоизацией (кешированием) нормальных форм подвыражений.

Так Call-by-Name + кеширование = Call-by-Need (\Haskell{})
\end{frame}



\subsection{Call-By-Name}

\begin{frame}{Call-By-Name $\rightsquigarrow$ Weak Head Normal Form}
Редуцирует \textbf{самый левый внешний} редекс, который \textbf{не под абстракцией}. Например, $\lam{x}{\lam{y}{M}N}$ уже в WHNF, потому что единственный редекс $\lam{y}{M}N$ под абстракцией.
\begin{mathpar}
  \inferrule* [Right=Var] {\\}
  {x \cbn x}
  \and
  \inferrule*  [Right=Abs] {\\}
  {\lam{x}{e} \cbn \lam{x}{e}}
\end{mathpar}
\begin{mathpar}
  \inferrule*  [Right=App-abs]
  {e_1 \cbn \lam{x}{e} \\
    [e_2/x]e \cbn e'
  }
  { (e_1 e_2) \cbn e'}
\end{mathpar}
\begin{mathpar}
  \inferrule*  [Right=App-non-abs]
  {e_1 \cbn e_1' \neq \lam{x}{e}}
  { (e_1 e_2) \cbn (e_1' e_2) }
\end{mathpar}

CBN может посчитать 1 аргумент несколько раз по сравнению с CBV.
\end{frame}

\subsection{Call-By-Value}
\begin{frame}{Call-by-Value $\rightsquigarrow$ Weak Normal Form}
Редуцирует \textbf{самый левый внутренний} редекс, который \textbf{не под абстракцией}. Например, в
$\lam{x}{\lam{y}{U}V} (\lam{z}{M}N)$ самый левый внутренний -- это $\lam{y}{U}V$, но редуцироваться первым будет
$(\lam{z}{M}N)$.

\begin{mathpar}
  \inferrule* [Right=Var]  {\\}
  {x \cbv x}
  \and
  \inferrule*  [Right=Abs] {\\}
  {\lam{x}{e} \cbv \lam{x}{e}}
\end{mathpar}
\begin{mathpar}
  \inferrule* [Right=App-abs]
  { e_1 \cbv \lam{x}{e} \\
    e_2 \cbv e_2' \\
    [e_2'/x]e \cbv e'
  }
  { (e_1 e_2) \cbv e'}
\end{mathpar}
\begin{mathpar}
  \inferrule*  [Right=App-non-abs]
  { e_1 \cbv e_1' \neq \lam{x}{e} \\
    e_2 \cbv e_2'}
  { (e_1 e_2) \cbv (e_1' e_2') }
\end{mathpar}
Стандарт для большинства языков программирования.
\end{frame}

\begin{frame}{Нормальной формы может не быть!}
\begin{definition}
  Нормализация -- процесс поиска соответствующей нормальной формы с помощью применения $\beta$-редукции согласно соответствующей стратегии
\end{definition}\vspace{1cm}

Пример: комбинатор $\Omega = \lam{x}{xx} \lam{x}{xx}$

$$
\lam{x}{xx} \lam{x}{xx} \cbv
[\lam{x}{xx}/x] (xx) \cbv
\lam{x}{xx}\lam{x}{xx} \cbv \dots
$$
\end{frame}

\begin{frame}{CBN vs. CBV}
Call-by-Name чаще завершается
$$
\lam{x}{\lam{y}{y}}\Omega \cbv \text{расходится}
$$
$$
\lam{x}{\lam{y}{y}} \Omega \cbn \lam{y}{y}
$$

\vspace{1cm}
Но Call-by-Name иногда вычисляет аргументы больше одного раза
$$
\lam{x}{(A x)(B x)} (\lam{y}{y}C) \cbn (A (\lam{y}{y}C))\ (B (\lam{y}{y}C))
$$

$$
\lam{x}{(A x)(B x)} (\lam{y}{y}C) \cbv
\lam{x}{(A x)(B x)} C \cbv
(A C) (B C)
$$

\end{frame}

\subsection{Аппликативный порядок}

\begin{frame}{Applicative Order $\rightsquigarrow$ Normal Form}
Редуцирует \textbf{самый левый внутренний} редекс, и \textbf{под абстракцией тоже}.
Например, в $\lam{x}{\lam{y}{U}V} (\lam{z}{M}N)$ самый левый внутренний -- это $\lam{y}{U}V$.
\begin{mathpar}
  \inferrule* [Right=Var] {\\}
  {x \ao x}
  \and
  \inferrule* [Right=Abs] {e \ao e'}
  {\lam{x}{e} \ao \lam{x}{e'}}
\end{mathpar}
\begin{mathpar}
  \inferrule*  [Right=App-abs]
  { e_1 \ao \lam{x}{e} \\
    e_2 \ao e_2' \\
    [e_2'/x]e \ao e'
  }
  { (e_1 e_2) \ao e'}
\end{mathpar}
\begin{mathpar}
  \inferrule*  [Right=App-non-abs]
  { e_1 \ao e' \neq \lam{x}{e} \\
    e_2 \ao e_2'}
  { (e_1 e_2) \ao (e_1' e_2') }
\end{mathpar}
N.B. Аппликативный порядок совершает больше редукций и выдает более простой ответ по сравнению с CBV, но не гарантирует, что редукция завершится.
\end{frame}
%
%\begin{frame}{"Другой" Applicative Order}
%  \begin{center}
%    Иногда (в википедии или книге SICP~\cite{sicp}) аппликативным порядком называют Call-By-Value, где явно упорядочивают порядок вычисления фактических аргументов у $\lambda$-абстракций.\\ \vspace{1em}
%
%    Согласно~\cite{setsoft} это аппликативным порядком не является.\\ \vspace{1em}
%
%    Короче говоря, стратегии с редукцией под абстракциями (applicative order, normal order) в программировании не используются.
%  \end{center}
%\end{frame}

\subsection{"Нормальный" порядок}

\begin{frame}{Normal Order $\rightsquigarrow$ Normal Form}
Сначала редуцирует \textbf{самый левый внешний} редекс. Встретив применение
$(e_1e_2)$ вначале пытается редуцировать $e_1$ как CBN. Если не получилась абстракция -- принимается за аргументы.
\begin{mathpar}
  \inferrule* [Right=Var]
  {\\}
  {x \nor x}
  \and
  \inferrule* [Right=Abs]{e \nor e'}
  {\lam{x}{e} \nor \lam{x}{e'}}
\end{mathpar}
\begin{mathpar}
  \inferrule* [Right=App-abs]
  { e_1 \cbn \lam{x}{e} \\
    [e_2/x]e \nor e'
  }
  { (e_1 e_2) \nor e'}
\end{mathpar}
\begin{mathpar}
  \inferrule* [Right=App-non-abs]
  { e_1 \cbn e_1' \neq \lam{x}{e} \\
    e_1' \nor e_1''\\
    e_2 \nor e_2'}
  { (e_1 e_2) \nor (e_1'' e_2') }
\end{mathpar}
N.B. Нормальный порядок сочетает две стратегии, позволяет получить более простые результаты, чем CBN. Чаще завершается, чем AO.
\end{frame}





\section{Дополнительные слайды}


\begin{frame}{Бывают различные стратегии}
Например,
\begin{itemize}
\item Строгие (например, call-by-value $\cbv$) вычисляют аргумент до его подстановки
\item Ленивые (например, call-by-name $\cbn$)  оставляют вычисление аргумента на потом
\end{itemize}


На практике больше любят стратегии, которые эффективно можно посчитать
\end{frame}

\begin{frame}{Ленивая vs. Строгая}
Пример 1 ($\cbv$ выглядит лучше)\\
$\lam{x}{f x x}(\tr{(\lambda x. x)}\tb{A}) \cbv \tr{(\lambda x. f x x)}\tb{A} \cbv (f A\ A) \cbv \dots $\\

$\tr{\lam{x}{f x x}}\tb{((\lambda x. x)A)} \cbn (\lambda x. f ((\lambda x. x)A) ((\lambda x. x)A))A \cbn \dots $

\vspace{2em}
Пример 2 ($\cbn$ выглядит лучше)\\
$(\lambda x. \lambda y. y)(\tr{(\lambda x. xx)}\tb{(\lambda x. xx)}) \cbv (\lambda x. \lambda y. y)(\tr{(\lambda x. xx)}\tb{(\lambda x. xx)}) \cbv \dots \text{зависло}$

$\tr{(\lambda x. \lambda y. y)}\tb{((\lambda x. xx)(\lambda x. xx)) \cbn (\lambda y. y)}\quad\text{ответ!}$

\vspace{2em}
В обычных языках программирования:
\lstinline[language=c]=(c>0) ? f() : g() =
\end{frame}

\end{document}
