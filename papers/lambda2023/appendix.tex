% !TeX root = lambda2023.tex


\section{Проблема останова}

\begin{frame}{Проблема останова (1/2)}
  \begin{block}{Вопрос}
    Можем ли мы написать алгоритм, который будет брать на вход произвольную $\lambda$-абстракцию и аргумент, и говорить посчитается ли для их применения нормальная форма?\\
  \end{block}
  \pause
  Положим наши программы либо зависают, либо выдают значение \texttt{true}.\\

  Положим существует гипотетическая $(Halting\ P\ w)$, которая всегда завершается, и возвращает \texttt{true}, если $(P\ w)$ редуцируется в \texttt{true}, иначе $(Halting\ P\ w)$ возвращает \texttt{false}.\\

  Покажем от противного, что $Halting$ не может существовать.
\end{frame}

\begin{frame}{Проблема останова (2/2)}
  Вопрос: во что редуцируется $E$, в \texttt{true} или в \texttt{false}?
  \[
  E = Halting\big( \lam{m}{not(Halting\ m\ m)},\ \ \lam{m}{not(Halting \ m\ m)} \big)
  \]
  Если E редуцируется в \texttt{true}, то применим функцию $\lam{m}{not(Halting(m,m))}$ к аргументу $\lam{m}{not(Halting(m,m))}$ и получим
  \[
  not ( Halting\big(\lam{m}{not(Halting \ m\ m)},\ \ \lam{m}{not(Halting \ m\ m)}\big)) = \neg E
  \]
  что является отрицание истинного факта выше.\\

  Если $E$ редуцируется в \texttt{false}, то это означает, $Halting$ иногда зависает, что противоречит определению функции $Halting$.
  \qed
\end{frame}


%\section{Две стратегии}
\begin{comment}

  \begin{frame}{Как происходят вычисления (редукция) $\lambda$-исчислении?}
    \begin{definition}[Процесс вычислений регламентирует стратегия]
      Ищем редексы $(\lambda x. P)Q$
      \begin{itemize}
        \item Если редексов нет, то вычисление закончилось
        \item Если редексы есть, стратегия регламентирует какой на данном шаге редекс стоит $\beta$редуцировать
        \item Или же, стратегия может сказать, что все редексы нужно оставить как есть, и выдать ответ.
      \end{itemize}
    \end{definition}
    %Стратегий бывает много разных
    %\begin{enumerate}
    %  \item Строгая стратегия call-by-value .    Для $(\lambda x. P)Q$ вычисляет $Q\cbv Q'$ и потом подставляет $Q'$ вместо $x$ в $P$.
    %  \item Ленивая стратегия call-by-name.   Для $(\lambda x. P)Q$ сразу подставляет $Q$ вместо $x$ в $P$.
    %  \item Обе стратегии оставляют абстракции и переменные как есть
    %\end{enumerate}
  \end{frame}

  \begin{frame}{Ленивая vs. Строгая}
    Пример 1 ($\cbv$ выглядит лучше)\\
    $(\lambda x. f x x)((\lambda x. x)A) \cbv (\lambda x. f x x)A \cbv (f A A) \cbv \dots $\\

    $(\lambda x. f x x)((\lambda x. x)A) \cbn (\lambda x. f ((\lambda x. x)A) ((\lambda x. x)A))A \cbn \dots $

    \vspace{3em}
    Пример 2 ($\cbn$ выглядит лучше)\\
    $(\lambda x. \lambda y. y)((\lambda x. xx)(\lambda x. xx)) \cbv (\lambda x. \lambda y. y)((\lambda x. xx)(\lambda x. xx)) \cbv \dots \text{зависло}$

    $(\lambda x. \lambda y. y)((\lambda x. xx)(\lambda x. xx)) \cbn (\lambda y. y)\quad\text{ответ!}$
  \end{frame}
\end{comment}



\begin{comment}


  \newcommand{\ite}[3]{\ensuremath{(\text{if } #1\text{ then }#2\text{ else }#3})}}
\begin{frame}{Рекурсия в call-by-name}
  $$
  Y\equiv \lam{f}{\lam{x}{f(xx)}\lam{x}{f(xx)}}
  $$
  %Основное свойство
  \[
  YR = \lam{x}{R(xx)} \lam{x}{R(xx)} \nor
  R\big( \lam{x}{R(xx)}\lam{x}{R(xx)} \big) =
  R(YR)
  \]

  \vspace{1em}

  Факториал: $fac \equiv (\lambda self.\lambda n . \ite{n<2}{1}{n \cdot self(n-1)})$
  \begin{align*}
    Y(\lambda self.\lambda n . \ite{n<2}{1}{n \cdot self(n-1)}) 2 &\cbn \\
    (\lambda n . \ite{n<2}{1}{n \cdot Y\ fac(n-1)}) 2 &\cbn \\
    2 \cdot Y\ fac\ (2-1) &\cbn \\
    2 \cdot (Y(\lambda self.\lambda n . \ite{n<2}{1}{n \cdot self(n-1)})\ 1) &\cbn \\
    2 \cdot \ite{1<2}{1}{n \cdot (Y\ fac\ (1-1))} &\cbn \\
    2 \cdot 1 & \cbn 2
  \end{align*}

\end{frame}
\end{comment}



\section{Дополнительные слайды}

\begin{frame}{Нормальные формы}

У нас как минимум четыре возможности
\begin{itemize}
  \item Редуцируем ли под абстракциями? (да/нет)
  \item Редуцируем ли аргументы перед подстановкой? (да/нет)
\end{itemize}

\begin{center}
  \begin{table}[]
    \begin{tabular}{|c|m{6cm}|m{6cm}|}
      \hline\hline
      \multirow{2}{*}{\thead{Редуцируем \\аргументы?}} & \multicolumn{2}{c|}{Редуцируем под абстракциями?} \\ \cline{2-3}
      &   Да(strong)        &     Нет(weak)     \\ \hline\hline
      Да(strict)  & \thead{Normal form \\
        $E ::=\lam{x}{E} \mid xE_1\dots E_n$} &
      \thead{Weak normal form \\
        $E ::=\lam{x}{e} \mid xE_1\dots E_n$}    \\
      Нет(lazy) & \thead{Head normal form \\
        $E ::=\lam{x}{E} \mid xe_1\dots e_n$} &
      \thead{Weak head normal form \\
        $E ::=\lam{x}{e} \mid xe_1\dots e_n$}  \\ \hline\hline
    \end{tabular}
  \end{table}
\end{center}

В таблице $E_j$  -- это выражение в соответствующей нормальной форме, а $e_i$ -- произвольный $\lambda$-терм.\\
\footnotetext{То, что у некоторых $E$ нет индексов -- не опечатка}
\end{frame}

%
\begin{frame}{Порядков редукции бывает много...\cite{setsoft}}
\begin{enumerate}
  \item Call-by-Name
  \item Normal Order
  \item Call-by-Value (OCaml)
  \item Applicative Order
  \item Hybrid Applicative Order
  \item Head Spine Reduction
  \item Hybrid Normal Order
\end{enumerate}
И ещё есть оптимизации связанные с мемоизацией (кешированием) нормальных форм подвыражений.

Так Call-by-Name + кеширование = Call-by-Need (\Haskell{})
\end{frame}



\subsection{Call-By-Name}

\begin{frame}{Call-By-Name $\rightsquigarrow$ Weak Head Normal Form}
Редуцирует \textbf{самый левый внешний} редекс, который \textbf{не под абстракцией}. Например, $\lam{x}{\lam{y}{M}N}$ уже в WHNF, потому что единственный редекс $\lam{y}{M}N$ под абстракцией.
\begin{mathpar}
  \inferrule* [Right=Var] {\\}
  {x \cbn x}
  \and
  \inferrule*  [Right=Abs] {\\}
  {\lam{x}{e} \cbn \lam{x}{e}}
\end{mathpar}
\begin{mathpar}
  \inferrule*  [Right=App-abs]
  {e_1 \cbn \lam{x}{e} \\
    [e_2/x]e \cbn e'
  }
  { (e_1 e_2) \cbn e'}
\end{mathpar}
\begin{mathpar}
  \inferrule*  [Right=App-non-abs]
  {e_1 \cbn e_1' \neq \lam{x}{e}}
  { (e_1 e_2) \cbn (e_1' e_2) }
\end{mathpar}

CBN может посчитать 1 аргумент несколько раз по сравнению с CBV.
\end{frame}

\subsection{Call-By-Value}
\begin{frame}{Call-by-Value $\rightsquigarrow$ Weak Normal Form}
Редуцирует \textbf{самый левый внутренний} редекс, который \textbf{не под абстракцией}. Например, в
$\lam{x}{\lam{y}{U}V} (\lam{z}{M}N)$ самый левый внутренний -- это $\lam{y}{U}V$, но редуцироваться первым будет
$(\lam{z}{M}N)$.

\begin{mathpar}
  \inferrule* [Right=Var]  {\\}
  {x \cbv x}
  \and
  \inferrule*  [Right=Abs] {\\}
  {\lam{x}{e} \cbv \lam{x}{e}}
\end{mathpar}
\begin{mathpar}
  \inferrule* [Right=App-abs]
  { e_1 \cbv \lam{x}{e} \\
    e_2 \cbv e_2' \\
    [e_2'/x]e \cbv e'
  }
  { (e_1 e_2) \cbv e'}
\end{mathpar}
\begin{mathpar}
  \inferrule*  [Right=App-non-abs]
  { e_1 \cbv e_1' \neq \lam{x}{e} \\
    e_2 \cbv e_2'}
  { (e_1 e_2) \cbv (e_1' e_2') }
\end{mathpar}
Стандарт для большинства языков программирования.
\end{frame}

\begin{frame}{Нормальной формы может не быть!}
\begin{definition}
  Нормализация -- процесс поиска соответствующей нормальной формы с помощью применения $\beta$-редукции согласно соответствующей стратегии
\end{definition}\vspace{1cm}

Пример: комбинатор $\Omega = \lam{x}{xx} \lam{x}{xx}$

$$
\lam{x}{xx} \lam{x}{xx} \cbv
[\lam{x}{xx}/x] (xx) \cbv
\lam{x}{xx}\lam{x}{xx} \cbv \dots
$$
\end{frame}

\begin{frame}{CBN vs. CBV}
Call-by-Name чаще завершается
$$
\lam{x}{\lam{y}{y}}\Omega \cbv \text{расходится}
$$
$$
\lam{x}{\lam{y}{y}} \Omega \cbn \lam{y}{y}
$$

\vspace{1cm}
Но Call-by-Name иногда вычисляет аргументы больше одного раза
$$
\lam{x}{(A x)(B x)} (\lam{y}{y}C) \cbn (A (\lam{y}{y}C))\ (B (\lam{y}{y}C))
$$

$$
\lam{x}{(A x)(B x)} (\lam{y}{y}C) \cbv
\lam{x}{(A x)(B x)} C \cbv
(A C) (B C)
$$

\end{frame}

\subsection{Аппликативный порядок}

\begin{frame}{Applicative Order $\rightsquigarrow$ Normal Form}
Редуцирует \textbf{самый левый внутренний} редекс, и \textbf{под абстракцией тоже}.
Например, в $\lam{x}{\lam{y}{U}V} (\lam{z}{M}N)$ самый левый внутренний -- это $\lam{y}{U}V$.
\begin{mathpar}
  \inferrule* [Right=Var] {\\}
  {x \ao x}
  \and
  \inferrule* [Right=Abs] {e \ao e'}
  {\lam{x}{e} \ao \lam{x}{e'}}
\end{mathpar}
\begin{mathpar}
  \inferrule*  [Right=App-abs]
  { e_1 \ao \lam{x}{e} \\
    e_2 \ao e_2' \\
    [e_2'/x]e \ao e'
  }
  { (e_1 e_2) \ao e'}
\end{mathpar}
\begin{mathpar}
  \inferrule*  [Right=App-non-abs]
  { e_1 \ao e' \neq \lam{x}{e} \\
    e_2 \ao e_2'}
  { (e_1 e_2) \ao (e_1' e_2') }
\end{mathpar}
N.B. Аппликативный порядок совершает больше редукций и выдает более простой ответ по сравнению с CBV, но не гарантирует, что редукция завершится.
\end{frame}
%
%\begin{frame}{"Другой" Applicative Order}
%  \begin{center}
%    Иногда (в википедии или книге SICP~\cite{sicp}) аппликативным порядком называют Call-By-Value, где явно упорядочивают порядок вычисления фактических аргументов у $\lambda$-абстракций.\\ \vspace{1em}
%
%    Согласно~\cite{setsoft} это аппликативным порядком не является.\\ \vspace{1em}
%
%    Короче говоря, стратегии с редукцией под абстракциями (applicative order, normal order) в программировании не используются.
%  \end{center}
%\end{frame}

\subsection{"Нормальный" порядок}

\begin{frame}{Normal Order $\rightsquigarrow$ Normal Form}
Сначала редуцирует \textbf{самый левый внешний} редекс. Встретив применение
$(e_1e_2)$ вначале пытается редуцировать $e_1$ как CBN. Если не получилась абстракция -- принимается за аргументы.
\begin{mathpar}
  \inferrule* [Right=Var]
  {\\}
  {x \nor x}
  \and
  \inferrule* [Right=Abs]{e \nor e'}
  {\lam{x}{e} \nor \lam{x}{e'}}
\end{mathpar}
\begin{mathpar}
  \inferrule* [Right=App-abs]
  { e_1 \cbn \lam{x}{e} \\
    [e_2/x]e \nor e'
  }
  { (e_1 e_2) \nor e'}
\end{mathpar}
\begin{mathpar}
  \inferrule* [Right=App-non-abs]
  { e_1 \cbn e_1' \neq \lam{x}{e} \\
    e_1' \nor e_1''\\
    e_2 \nor e_2'}
  { (e_1 e_2) \nor (e_1'' e_2') }
\end{mathpar}
N.B. Нормальный порядок сочетает две стратегии, позволяет получить более простые результаты, чем CBN. Чаще завершается, чем AO.
\end{frame}



