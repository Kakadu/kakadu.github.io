
\newcommand{\lazy}{\xarr{lazy}}
\newcommand{\strict}{\xarr{strict}}

\begin{frame}{Бывают различные стратегии}
\begin{itemize}
\item Строгие (англ. strict, например, call-by-value) вычисляют аргумент до его подстановки
\item Ленивые (англ. lazy, например, call-by-name)  оставляют вычисление аргумента на потом
\end{itemize}
\vspace{1em}
Классификация по обработке $\lambda$-абстракции
\begin{itemize}
\item Не вычисляют под абстракцией (например, call-by-value $\cbv$)
\item Вычисляют под абстракцией (например, call-by-name $\cbn$)
\end{itemize}
\vspace{2em}

На практике больше любят стратегии, которые эффективно можно посчитать
\end{frame}

\begin{frame}{Ленивая vs. Строгая}
Пример 1 ($\strict$ выглядит лучше)\\
$\lam{x}{f x x}(\tr{(\lambda x. x)}\tb{A}) \strict \tr{(\lambda x. f x x)}\tb{A} \strict (f A\ A) \strict \dots $\\

$\redex{\lam{x}{f x x}}{((\lambda x. x)A)} \lazy f \App{\app{\lam{x}{x}}{A}} {\app{\lam{x}{x}}{A}} \lazy \dots $

\vspace{2em}
Пример 2 ($\lazy$ выглядит лучше)\\
$\lam{x}{\lam{y}{y}}\Redex{\lam{x}{xx}} {\lam{x}{xx}} \strict (\lambda x. \lambda y. y)\Redex{\lam{x}{xx}} {\lam{x}{xx}} \strict \dots \text{зависло}$

$\tr{\lam{x}{\lam{y}{y}}}\tb{\App{\lam{x}{xx}}{\lam{x}{xx}}} \lazy (\lambda y. y)\quad\text{ответ!}$

\vspace{2em}
В обычных языках программирования:
\lstinline[language=c]=(c>0) ? f() : g() =
\end{frame}


\begin{frame}[fragile]{Правила вывода в исчислении}
Пусть дан некоторый язык L, c помощью которого записываются $P_i$ и $C$.\\

\vspace{1em}
Все $(n+1)$-местные правила вывода имеют форму:
\begin{mathpar}
 \inferrule* [leftskip=2em,rightskip=2em,Right={$\qquad$ Название}]
 { P_1\\ ... \\ P_n}
 {C}
\end{mathpar}

\begin{itemize}
  \item $P_i$ --- посылки (premises)
  \item $C_i$ --- заключение (conclusion)
\end{itemize}\vspace{1em}

По смыслу означает <<если и $P_1$, и $P_2$, ..., и $P_n$, то $C$>>

\end{frame}


\begin{frame}{Исчисление}
Состоит из
\begin{itemize}
  \item непустого множества аксиом
  \item множества правил вывода
\end{itemize}
\begin{definition}
Аксиома --- это правило вывода без посылок. Их можно рисовать без черты
\end{definition}
\vspace{2em}
%Формальное определение можно прочитать в книжке Герасимова~\cite{gerasimov}.
\end{frame}


\begin{frame}{Пример исчисления. Дифференциальное исчисление}
Языком L будет язык задания функций (который вообще-то надо формально определять, но не будем)

\begin{mathpar}
  \inferrule* [Right=sin]{\\}
  {sin'(x) = cos(x)}
  \and
  \inferrule* [Right=cos]{\\}
  {cos'(x) = -sin(x)}
  \and
  \inferrule* [Right=var]{\\}
  {x' = 1}
  \and
  \inferrule* [Right=const]{\\}
  {c' = 0, \text{если } c\in N}

\end{mathpar}
\begin{mathpar}
  \inferrule* [Right=mul,leftskip=2em,rightskip=2em]
  {f\ '(x) = u(x) \\ g'(x) = v(x)}
  {((f\cdot g)(x))' = u(x)\cdot g(x) + f(x) \cdot v(x)}
\end{mathpar}
\begin{mathpar}
  \inferrule* [Right=cmps,leftskip=2em,rightskip=2em]
  {f'(x) = u(x) \\ g'(x) = v(x)}
  {(f(g(x)))' = u(g(x)) \cdot v(x)}
\end{mathpar}
\end{frame}


\begin{frame}{Дифференциальное исчисление. Пример вывода}
Вывод обычно рисуется снизу вверх
\begin{mathpar}
\inferrule* []
{ \inferrule* [Right=sin]{ }{\uncover<2- >{sin'(x) =} \uncover<3- >{cos(x)}} \\
  \inferrule* {
    \inferrule* [Right=cos]{ }{\uncover<3- >{cos'(x) =} \uncover<4- >{-sin(x)}} \\
    \inferrule* []
      { \inferrule* [Right=const]{ }{\uncover<4- >{2' = 0}} \\
        \inferrule* [Right=var]{ }{\uncover<4- >{x' = 1}} \\
      }
      {\uncover<3- >{(2\cdot x)' =} \uncover<5- >{0\cdot x + 2\cdot 1} }
  }{\uncover<2- >{cos'(2\cdot x) =} \uncover<6- >{-sin(2\cdot x)\cdot 2} }
}
{(sin(x)\cdot cos(2\cdot x))' = \uncover<7>{cos(x)\cdot cos(2\cdot x) + sin(x)\cdot( -sin(2\cdot x)\cdot 2) }}

\end{mathpar}

На вывод можно смотреть как на \textbf{доказательство} того, что производная действительно посчитана правильно.\\

Результат вывода можно было бы упростить и дальше, но у нас недостаточно правил вывода для этого.
\end{frame}


\begin{frame}{Две стратегии: Call-by-value vs. Full Reduction}
\begin{figure}[ht]
  \begin{subfigure}[t]{.48\textwidth}
     \begin{mathpar}
     \lam{x}{e} \arr \lam{x}{e}
     \and
     \inferrule*  %[Right=App-abs]
     { f \arr \lam{x}{e} \\
       a \arr a_2 \\
       \subst{x}{a_2}{e} \arr r
     }
     { (f a) \arr r}
     \and
     \inferrule* % [Right=App-non-abs]
     { f \arr f_2 \neq \lam{x}{e} \\
       a \arr a_2}
     { (f a) \arr (f_2 a_2) }\\
     v \cbv v\\
    \end{mathpar}
  \end{subfigure}
  \begin{subfigure}[t]{.48\textwidth}
     \begin{mathpar}
     \inferrule* % [Right=App-non-abs]
     { a \arr b }
     { \lam{x}{a} \arr \lam{x}{b}}
      \and
    \inferrule*  %[Right=App-abs]
    { f \arr \lam{x}{e} \\
      a \arr a_2 \\
      \subst{x}{a_2}{e} \arr r
    }
    { (f a) \arr r}
      \and
      \inferrule* % [Right=App-non-abs]
    { f \arr f_2 \neq \lam{x}{e} \\
      a \arr a_2}
    { (f a) \arr (f_2 a_2) }\\
     v \xarr{f\!ull} v
    \end{mathpar}
   \end{subfigure}
\end{figure}


    \begin{figure}[t]
      \begin{subfigure}[t]{0.45\textwidth}
    \begin{itemize}
    \item[\faGood]    Используется в большинстве языков программирования
    \vspace{4em}
    \end{itemize}
      \end{subfigure}
      \begin{subfigure}[t]{0.45\textwidth}
    \begin{itemize}
      \item[\faGood] Cчитает под абстракцией, поэтому короткий ответ
    \end{itemize}
      \end{subfigure}
    \end{figure}
\end{frame}

\begin{frame}{$\lambda$-исчисление. Пример вывода}
Вывод обычно рисуется снизу вверх
\begin{mathpar}
\inferrule* []
{ \inferrule* {
     \inferrule* {
           \inferrule* []{
                \inferrule* []{ }{\uncover<5- >{y \arr y}}
           }{\uncover<4- >{\lam{y}{y} \arr} \uncover<6- >{\lam{y}{y}} } \\
           \inferrule* []{ }{\uncover<4- >{x \arr} \uncover<6- >{x} } \\
           \inferrule* []{ }{\uncover<6- >{\subst{x}{y}{y} \arr x} } \\
        }{\uncover<3- >{ \apppl{\lam{y}{y}}{x} \arr} \uncover<7- >{ x } }
  }{\uncover<2- >{ \lampl{x}{\apppl{\lam{y}{y}}{x}} \arr} \uncover<8- >{ \lam{x}{x} } }\\
  \inferrule* []{ }{\uncover<2- >{a \arr} \uncover<9- >{a}} \\
  \inferrule* []{ }{\uncover<9- >{\subst{a}{x}{x} \arr} \uncover<10- >{a}}
}
{\apppl{\lam{x}{\apppl{\lam{y}{y}}{x}}}{a} \arr \uncover<11- >{a}}
\end{mathpar}


Отличием от дифференциального исчисления будет то, что от результата вывода надо запуститься рекурсивно пока оно не остановится.
Там могло быть также, если бы были правила упрощения (например, $x-x\equiv0$)
%На вывод можно смотреть как на \textbf{доказательство} того, что производная действительно посчитана правильно.\\

%Результат вывода можно было бы упростить и дальше, но у нас недостаточно правил вывода для этого.
\end{frame}
