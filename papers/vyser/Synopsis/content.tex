%% \newcommand{\ifext}[2]{\ifdefined\extflag{#1}\else{#2}\fi}
%% \newcommand{\tick}{\checkmark}%
%% \newcommand{\tickP}{\checkmark}%
%% \newcommand{\tickPP}{\checkmark}%
%% \newcommand{\xmark}{\text{\ding{55}}}
%% \newcommand{\fail}{\xmark}%
%% \newcommand\Cf{\mathit{Prog}}%\textsf{Prog}

\section*{Общая характеристика работы}

\newcommand{\actuality}{\underline{\textbf{\actualityTXT}}}
\newcommand{\progress}{\underline{\textbf{\progressTXT}}}
\newcommand{\aim}{\underline{{\textbf\aimTXT}}}
\newcommand{\tasks}{\underline{\textbf{\tasksTXT}}}
\newcommand{\novelty}{\underline{\textbf{\noveltyTXT}}}
\newcommand{\influence}{\underline{\textbf{\influenceTXT}}}
\newcommand{\methods}{\underline{\textbf{\methodsTXT}}}
\newcommand{\defpositions}{\underline{\textbf{\defpositionsTXT}}}
\newcommand{\reliability}{\underline{\textbf{\reliabilityTXT}}}
\newcommand{\probation}{\underline{\textbf{\probationTXT}}}
\newcommand{\contribution}{\underline{\textbf{\contributionTXT}}}
\newcommand{\publications}{\underline{\textbf{\publicationsTXT}}}

\newcommand{\ccite}[1]{}
\newcommand{\cscite}[1]{}

{\actuality}
Логическое программирование давно привлекает внимание исследователей тем, что оно предоставляет возможность не реализовывать конкретный алгоритм непосредственно, а вместо этого с некоторой степенью декларативности описывать условие решаемой задачи. Роль алгоритма в данном случае выполняет встроенная в систему логического программирования процедура поиска решения.

Recent years have seen a resurgence of interest from theresearch and industry community in the Datalog languageand  its  syntactic  extensions.   Datalog  is  a  recursive  querylanguage that extends relational algebra with recursion, andcan be used to express a wide spectrum of modern data man-agement tasks, such as data integration [7, 13], declarativenetworking [14], graph analysis [20, 21] and program anal-ysis [27].  Datalog offers a simple and declarative interfaceto the developer, while at the same time allowing powerfuloptimizations that can speed up and scale out evaluation\footnote{\url{https://arxiv.org/pdf/1812.03975.pdf}}.

Программы на логическом языке записываются в виде реляций (relations), которые моделируют математические отношения (relations). 
Математические отношения являются подмножествами декартова произведения, в то время как алгоритмические реляции моделируют математические отношения путём генерации элементов подмножества один за другим. Побочным эффектом использования реляций является возможность описания только счетных отношений, а также то, что реляция может зависнуть в процессе вычисления следующего элемента отношения.

Запустить на вычисление логическую программу можно с помощью \emph{запросов}. Запросы осуществляются путём указания конкретной реляции, которая будет вычисляться, и указания аргументов этой реляции с разной степенью конкретности. Разрешается указать аргумент полностью конкретным (ground); не конкретизировать аргумент никак, подставив вместо него свободную переменную; или же совместить эти два способа, конкретизировав известные части аргумента, и подставив свободные переменные в пока неизвестных частях аргумента. В результате своей работы система логического программирования выдает такие подстановки свободных переменных, которые после применения подстановок к указанным аргументам дадут элемент отношения.

Программистские n-местные функции в логическом программировании 
превращаются в (n+1)-местные реляции, путём добавления дополнительного аргумента для результата функции. Эти реляции можно использовать в прямую сторону, конкретизировав аргументы и подставив вместо результата свободную переменную, т.е. как функции. Также является возможным запускать их ``в обратную сторону'', конкретизировав результат и некоторые аргументы, по сути подбирая входные данные, на которых конкретный алгоритм, записанный на логическом языке программирования, выдаст ожидаемый результат. Таким образом, если перед программистом на обычном языке программирования стоит задача обращения некоторого алгоритма, то он вынужден написать новый алгоритм, который непосредственно этим и занимается. Если же пользоваться логическим программированием, то можно попробовать обойтись одной реляцией, правильно подобрав её аргументы.

Обладание свойством декларативности полезно для логических программ. Оно заключается в том, что программист не реализует непосредственно алгоритм как последовательность действий, от него требуется только корректное описание задачи на языке логического программирование. В декларативных описаниях программ результат не зависит от того как именно и в каком порядке описывается предметная область. Это снимает некоторый груз ответственности с программиста.


Процедура поиска, встроенная в систему логического программирования может осуществлять поиск решения руководствуюсь различными стратегиями. Самым распространенным является поиск в глубину, который, если находит ответ, показывает хорошую производительность. Однако он обладает свойством неполноты: алгоритм может бесконечно искать ответ в некоторой ветви дерева поиска решения и в итоге не зайти в другие ветви, где решение находится быстрее.
 Таким образом логическое программирование, использующее поиск в глубину не является декларативным в общем случае. 
 
Антагонистом поиска в глубину является стратегия поиска в ширину. Хотя поиск в ширину является стратегией полного поиска, она не показывает достаточную производительность на практике, и потому почти не используется. Также существуют стратегии сочетающие в себе поиск в глубину и в ширину: инкрементальное погружение (incremental deepening) и поиск с чередованием (interleaving search).

Исторически исследование логического программирования началось с изучения языка Prolog, который использует алгоритм поиска в глубину (deep-first search) и поэтому является не декларативным в общем случае. 
% Этот алгоритм, если находит ответ, показывает хорошую производительность. Однако он обладает свойством неполноты: алгоритм может бесконечно искать ответ в некоторой ветви дерева поиска решения и в итоге не зайти в другие ветви, где решение находится быстрее. Таким образом программы не Prolog не являются декларативными в общем случае.

Недостаток неполноты поиска в литературе разрешается несколькими способами. Одним из вариантов является добавление Prolog дополнительных конструкций для управления поиском: отсечение (cut), копирование термов (copyterm) и других. Но корректная расстановка программистом этих конструкций невозможна без обладания информацией о конкретности аргументов отношения. Более того, при различных запросах к логической программе изменяется степень конкретизации используемых данных, и поэтому правильная расстановка конструкций управления поиском невозможна, так как она существенно зависит от входных данных. Таким образом, дополнительные конструкции не гарантируют свойство декларативности логических программ на языке Prolog.

Другим способом борьбы с неполнотой поиска является подход, применяемый в языке Datalog. Этот язык наследует синтаксис и процедуру поиска из Prolog, но некоторые программы и сочетания синтаксических конструкций считаются некорректными. Ограничения выбраны таким образом, что пространство поиска запросов на Datalog является конечным, и, следовательно, поиск в глубину является полным поиском на таких программах. Однако из-за введенных синтаксических ограничений не все программы языка Prolog можно переписать на Datalog.

Язык Datalog нашел своё применения в различных предметных областях: компиляторы, запросы к базам данных, запросы к интернет-серверам и прочее\footnote{дописать подробнее со ссылками}.

Третьим способом решения проблемы является использование языка программирования miniKanren, который является темой Ph.D. диссертации Уилла Бёрда (Will Byrd). Основных свойств miniKanren можно назвать два.

Во-первых, в нём используется новый алгоритм поиска (interleaving search), являющийся чем-то средним между поисками в глубину и ширину. Как и поиск в ширину он является полным, и, следовательно, если на данный запрос существует ответ, то алгоритм поиска находит этот ответ за конечный, но, возможно, очень длительный, промежуток времени. При этом поиск в miniKanren является направленным в том же смысле, что и поиск в глубину. Программист может записывать программы так, что некоторые случаи (наиболее часто встречающиеся или конечные) будут рассмотрены перед случаями, в которых пространство поиска больше.

Во-вторых, miniKanren не является непосредственно языком программирования, а семейством встраиваемых языков программирования. Его можно реализовывать как библиотеку в языках программирования общего назначения, и так как ядро языка не очень большое, то реализовать miniKanren относительно просто. Флагманской является реализация на различных диалектах Scheme, при этом существуют \footnote{Ссылка на оф. сайт со списком реализаций} реализации miniKanren для других языков: функциональных и императивных, статически и динамически типизированных.

На данный момент сочетанием этих двух свойств обладают только языки семейства miniKanren. В своей Ph.D. диссертации Will Bird вводит термин \emph{реляционное программирование}, приравнивая его к программированию на miniKanren.

За счет использования полного поиска miniKanren обладает свойством декларативности в большей степени, чем Prolog, что позволяет содержательно использовать miniKanren для запуска реляционных запросов в обратную сторону. В 2012 году исследователи W. Byrd, E. Hold и D. Fridman применили miniKanren для написания \emph{реляционного интерпретатора} (интерпретатора некоторого подмножества Scheme на miniKanren) и воспользовались декларативностью и запускаемостью интерпретатора в обратном направлении для синтеза квайнов (quines) для подмножества Scheme.

Реализация miniKanren на строго типизированных языках сопряжена с некоторыми проблемами, так как непосредственное переписывание кода, спроектированного для динамического языка программирования, на строго типизированный языке ведет к ошибкам типизации. Обычно, реализации miniKanren на строго типизированных языках предоставляют для использования встроенный язык программирования с поддержкой некоторого \emph{конечного и заготовленного заранее} набора типов данных. Другими словами программист не может использовать свои типы данных при написании реляционных запросов и вынужден преобразовывать свои данные в реляционный домен.
%\paragraph{} 

Таким образом, для развития реляционного программирования в строго типизированных языках необходимо улучшать реализации miniKanren путём поддержки использования большего количества типов. 

Возможность запуска реляционных запросов в различных направлениях и большая декларативность относительно Prolog являются привлекательными причинами для использования miniKanren в задачах, где достаточно просто описать проблему и довольно сложно написать эффективный алгоритм поиска решения. Такие задачи встречаются, например, при написании компиляторов. Известен метод выделения регистров с помощью решения пазлов, который требует выполнения поиска с возвратами. Также задача компиляции конструкции сопоставления с образцом  (в функциональных языках программирования) в эффективный код требует построение диаграмм решений, а это построение также требует совершения поиска с возвратами. Таким образцом, является интересным изучить возможности реляционного программирования для применения в компиляторах функциональных языков.
\footnote{Возможно, в актуальности стоит ещё сказать про обобщенное программирование в стиле GT и упомянуть F.Pottier в степени разработанности темы, но я пока не понял как.}

{\progress}
Работы по эффективной компиляции конструкции сопоставления с образцом в компиляторах функциональных языков ведутся с 1990х годов. Сейчас в компиляторе OCaml используется подход, предложенный в 2001 году исследователями F.Lefessant и L. Maranget. В 2008 году L.Maranget предложил компилировать конструкцию сопоставления с образцом в эффективные деревья решений. В работе была введена эвристика для осуществления компиляции, которая показывает значительные улучшения на подобранных примерах, но в среднем не дает существенного прироста производительности.

В 2008 году исследователь F.M.Q. Pereira предложил использовать для выделения регистров решение пазлов. Предложенный алгоритм позволяет получать сравнимый по эффективности код, но является существенно более простым, чем аналоги.

Основополагающей работой по реляционному программированию является диссертация исследователя Will Byrd 2009 года. В ней представляется язык реляционного программирования miniKanren как встраиваемый в Scheme предметно-ориентированный язык.
%С 1990-х годов велась работа по
%разработке семантики многопоточности с учетом слабых сценариев поведения.
%Формальные модели для наиболее распространенных процессорных архитектур (x86, Power, ARM)
%были разработаны J. Alglave, S. Ishtiaq, L. Maranget, F. Zappa Nardelli, S. Sarkar, P. Sewell и др.
%исследователями\cscite{Alglave-al:TOPLAS14,Sewell-al:CACM10,Owens-al:TPHOL09}.
%Новые версии моделей продолжают появляться в связи с развитием процессорных архитектур.
%В частности, в 2016 и 2017 годах были представлены модели памяти для архитектур ARMv8.0 и
%ARMv8.3\cscite{Flur-al:POPL16,Pulte-al:POPL18}.
%% В настоящее время также ведутся работы по формализации моделей памяти графических процессоров.
%В 1995 году была стандартизована слабая модель памяти для языка Java\cscite{Manson-al:POPL05};
%в дальнейшем модель существенно менялась вплоть до 2005 года.
%В 2011 году появилась аксиоматическая модель памяти для языков C/C++\cscite{Batty-al:POPL11}.
%% Так модель памяти Java некорректна по отношению к базовым оптимизациям, а модель памяти C/С++ разрешает сценарии поведения
%% программ, в которых появляются т.н. \emph{значения из воздуха} (Out-Of-Thin-Air values),
%% что делает некорректными даже базовые рассуждения о программах.
%
%В 2017 году исследователи J. Kang, C.-K. Hur, O. Lahav, V. Vafeiadis и D. Dreyer
%представили обещающую модель памяти (promising memory model, $\Promise$)\cscite{Kang-al:POPL17}, которая является перспективным
%решением проблемы задания семантики для языка с многопоточностью. Авторы 
%доказали, что модель допускает большинство необходимых оптимизаций, а также показали корректность
%эффективных схем компиляции в архитектуры x86 и Power. Открытым остался вопрос о корректности компиляции
%в архитектуру ARM.

%% Один из основополагающих результатов в области формальной верификации, логика Хоара (Hoare logic), была
%% предложена Hoare в 1969 году. В конце 1990-ых Reynolds, O'Hearn и другие разработали ее расширение,
%% логику разделения (separation logic), которая позволила доказывать свойства однопоточных программ
%% композиционально относительно использованной памяти. Далее O'Hearn и Reynolds предложили логику разделения
%% для многопоточности (concurrent separation logic). Она базируется на денотационной семантике Brookes,
%% которая, в свою очередь, является композициональным представлением модели памяти последовательной
%% консистентности Lamport. С помощью логики разделения для многопоточности десятки многопоточных алгоритмов
%% были верифицированы в модели последовательной консистентности
%% (работы Birkedal, Dreyer, Gardner, Gotsman, Nanevski, Sergey, Vafeiadis и других).

%% Параллельно с развитием логик для последовательной консистентности,

%% \cite{Kang-al:POPL17}

%% Этот раздел должен быть отдельным структурным элементом по
%% ГОСТ, но он, как правило, включается в описание актуальности
%% темы. Нужен он отдельным структурынм элемементом или нет ---
%% смотрите другие диссертации вашего совета, скорее всего не нужен.

{\aim} данной работы является применение реляционного программирования в компиляторах, а также поиск подзадач во время процесса компиляции, которые могут получить выгоду от использования реляционного программирования.

Для~достижения поставленной цели были сформулированы следующие {\tasks}.
\begin{enumerate}
  %% \item Провести обзор существующих операционных моделей слабой памяти.
  \item Разработать и реализовать предметно-ориентированный язык семейства miniKanren, встраиваемый в функциональный язык общего назначения  OCaml. Также позаботиться об удобстве использования OCanren конечным пользователем.
  %% \item Формализовать модель памяти ARMv8 и доказать дополнительные свойства о ней, необходимые для
  %%       доказательства корректности компиляции. 
  %% \item Разработать метод для доказательства корректности компиляции из обещающей модели в модели памяти
  %%       процессорных архитектур.
  %% \item Апробировать применимость метода на примере доказательства корректности компиляции из обещающей модели памяти в
  %%       операционныую модель памяти ARMv8 POP и аксиоматическую модели памяти процессора ARMv8.3.
  \item Исследовать алгоритмы, которые используются в компиляторе OCaml, выделить те части компилятора, где использование реляционного программирования выгодно.
  \item Проверить успешность реляционного программирования для написания частей компилятора функционального языка.
\end{enumerate}

Постановка цели и задач исследования соответствует 
следующим пунктам паспорта специальности 05.13.11: ...
%модели, методы и алгоритмы проектирования и анализа программ и программных систем, их эквивалентных преобразований, верификации и тестирования (пункт 1);
%языки программирования и системы программирования, семантика программ (пункт 2);
%модели и методы создания программ и программных систем для параллельной и распределенной обработки данных,
%языки и инструментальные средства параллельного программирования (пункт 8).

{\methods}
Пока не придумал
%Методология исследования базируется на подходах информатики
%к описанию и анализу формальных семантик языков программирования.
%% Также используется концепция симуляции систем переходов.

%В работе используется представление операционной семантики программы с помощью помеченной системы переходов\cscite{Keller:CACM76},
%а также метода вычислительных контекстов, предложенного M. Felleisen\cscite{Felleisen-Hieb:TCS92}.
%Для доказательств корректности компиляции используется техника прямой симуляции\cscite{Lynch-Vaandrager:IC95,Lynch-Vaandrager:IC96}.
%Программная реализация интерпретатора операционной модели памяти C/C++11 выполнена на языке Racket\cscite{Flatt-PLT:TR10,RacketLang}
%с использованием предметно-ориентированного расширения PLT/Redex\cscite{Felleisen-al:BOOK09,Klein-al:POPL12}.

%% \filbreak
{{\defpositions}
\begin{enumerate}
  %% \item Проведён обзор существующих операционных моделей слабой памяти.
  %%       \todo{(пока не сделано)}
%  \item Предложена операционная модель памяти C/С++11, для этой модели реализован интерпретатор.
  \item Разработка и реализация OCanren в виде библиотеки: предметно-ориентированного, встраиваемого (в OCaml) языка реляционного программирования.
  \item Разработка и реализация библиотеки обобщенного программирования Generic Transformers для языка OCaml, которая упрощает обобщенное программирование в OCaml в общем, а в частности облегчает написание реляционных запросов с помощью OCanren.
  \item Применение реляционного программирования для компилятора функционального языка OCaml (планируется). 
  \begin{itemize}
  \item Применение реляционного алгоритма поиска для компиляции языковой конструкции “сопоставление с образцом” (pattern matching) для языка OCaml, а исследование трудоемкости расширения процедуры компиляции при расширении конструкции сопоставления с образцом.
  \item Добавление расширений компиляции расширений конструкции сопоставления с образцом.
  \item Реализация выделения регистров с помощью решения пазлов (puzzle solving).
  \end{itemize}



  
  
  %% \item Создан компонентный метод описания интерпретаторов для модели C/C++11 в системе PLT/Redex.
  %% \item Разработан реляционный интерпретатор для модели C/C++11, позволяющий исправлять ошибки синхронизации
  %%       в программах.
  %%       \todo{(пока не сделано)}
  %% \item Реализован метод автоматического поиска сертификата в обещающей \; модели памяти.
  %%       \todo{(пока не сделано)}
  %% \item Проведена формализация операционной модели памяти ARMv8 POP, доказаны вспомогательные утверждения про
  %%       модель, полезные для проверки корректности компиляции.
%  \item Доказана корректность компиляции из существенного подмножества обещающей модели в операционную модель памяти ARMv8 POP.
  %% \item Разработан метод доказательства корректности компиляции из обещающей модели памяти в аксиоматические модели памяти,
  %%       который основан на построении операционной семантики по аксиоматической.
%  \item Доказана корректность компиляции из существенного подмножества обещающей модели в аксиоматическую модель памяти ARMv8.3.
  %% с помощью метода, основанного на построении операционной семантики по аксиоматической.
\end{enumerate}
}

{\novelty} результатов, полученных в рамках исследования, заключается в следующем.
\begin{enumerate}
  \item 
%  Альтернативная модель памяти для стандарта C/C++11, предложенная в работе, отличается от обещающей модели
%  памяти\cscite{Kang-al:POPL17} тем, что является запускаемой,
%  т.е. для нее возможно создание интерпретатора (что и было выполнено в рамках данной диссертационной работы).
%  Это отличие является следствием того, что для получения эффекта отложенного чтения предложенная модель использует
%  синтаксический подход (буферизация инструкций), тогда как обещающая модель --- семантический (обещание потоком сделать
%  запись в будущем).
  \item 
%  Доказательство корректности компиляции из обещающей модели памяти в аксиоматическую модель
%        ARMv8.3\cscite{Pulte-al:POPL18}
%        не опирается на специфические свойства целевой модели, такие как представимость модели в виде
%        набора оптимизаций поверх более строгой модели.
%        Это отличает его от аналогичных доказательств для моделей x86 и Power
%        (работы O. Lahav, V.Vafeiadis и других\cscite{Lahav-Vafeiadis:FM16,Kang-al:POPL17}).
  \item 
%  Доказательства корректности компиляции из обещающей модели памяти в
%  модели ARMv8 POP\cscite{Flur-al:POPL16} и ARMv8.3\cscite{Pulte-al:POPL18}, представленные
%  в работе, являются первыми результатами о компиляции для данных моделей.
  %% Кроме того, представленный подход может быть использован и для доказательства
  %% корректности компиляции в модели x86 и Power, а значит обладает большой областью применимости.
\end{enumerate}

{\underline{\textbf{Evaluation}}}
Динамические языки программирования (например, Scheme) помогают заниматься быстрым прототипированием, в то время как строгие статические языки программирования (например, OCaml) требуют дополнительных затрат на изначальное проектирование типов.

\emph{В: Почему для реализации был выбран язык OCaml и можно ли переиспользовать наработки OCanren в функциональных языках со строгой статической типизацией?}
Языки реляционного программирования состоят из двух основных частей: метода проведения поиска (interleaving search) и процедуры унификации.
Языки из семейства Lisp, на которых реализован официальный miniKanren известны наличием универсального представления данных, которое заключается в синтаксисе, основанном на списках. Для универсального представления исследователям Will Byrd и другим удалось реализовать 
\emph{единую} алгоритм унификации, который применим для всех данных, используемых в реляционном домене. В языке OCaml нет универсального представления на уровне синтаксиса, но присутствует универсально представление данных в памяти в среде выполнения языка OCaml. В OCanren также была реализована единый алгоритм полиморфной унификации, который работает для всех типов данных, используемых в OCanren. Если в новом языке программирования, куда необходимо встроить miniKanren, используется универсальное представление, то туда можно легко перенести идею унификации из официального miniKanren или OCanren. Если же нет, то унификацию придется реализовывать ad-hoc способом для каждого типа данных. Вторая часть miniKanren, а именно поиск,  не вызовет трудностей по переносу в язык, где поддерживаются функции высшего порядка.

\emph{В: Что отличает OCanren от других реализацией miniKanren на OCaml и других строго типизированных языках?} Как было сказано выше, OCanren использует универсальный алгоритм полиморфной унификации. Хотя унификация реализована с использованием типонебезопасных возможностей языка OCaml, её интерфейс спроектирован так, чтобы туда можно было передавать только согласованные по типам данные. Без полиморфной унификации перед разработчиками стоят два возможных варианта действий. Либо они могут реализовывать ad-hoc унификацию для каждого используемого типа данных, что требует усилий от пользователя реляционного предметно-ориентированного языка, а также может вызвать серьёзные проблемы с обеспечением типобезопасности. Либо они могут поддержать в предметно-ориентированном языке только наперед заданное конечное количество используемых типов, реализовать унификацию для них и заставлять пользователя реляционного языка конвертировать свои типы данных туда и обратно. Это может оказаться неудобно для пользователя и требовать дополнительных расходов. На момент написания статьи 2016 года все известные реализации miniKanren на строго типизированных языках пользовались вторым подходом.


Иногда, при реализации чего-либо с использованием типов данных приходится платить высокую цену абстракции из-за использования нетривиальных типов.

\emph{В: На сколько производителен OCanren по сравнению с официальной реализацией miniKanren и на сколько большую цену абстракции пришлось заплатить?}
Действительно, самая первая реализация OCanren платила высокую цену абстракции: древовидное представление данных по сравнению с официальной реализацией содержало в два раза большее количество узлов и поэтому унификация работала существенно медленнее. Из-за этого был разработан другой механизм типобезопасного встраивания, который не страдает этой проблемой. Текущая версия OCanren показывает сходную производительность по сравнению с самой быстрой реализацией miniKanren на Chez Scheme -- faster-miniKanren. Конкретные замеры и тесты есть в статье с ML Workshop 2016.

\emph{В: На сколько осложнилось написание реляционных программ с помощью OCanren по сравнению с официальным miniKanren?}
Действительно, подготовка типов данных для использования в реляционном домене требует некоторых усилий. Однако, подготовка типов и функций-конструкторов значений является однотипной для всех используемых типов, поэтому в рамках работы на пунктом 1 были спроектированы и реализованы синтаксические расширения для уменьшения количества шаблонного кода, который требуется писать программисту.

Для использования типов данных в реляционном домене от них требуется наличие функториального действия для доказательства корректности возможности трансляции в логический домен, а также функции конвертации в строковое представление для удобства работы с данными типами. В OCanren обе функции создаются с помощью библиотеки обобщенного программирования Generic Transformers, которая генерирует некоторый код по типам данных. Основной её особенностью является то, что программист имеет удобный способ видоизменять некоторые части сгенерированных функций по своему усмотрению. Наиболее похожим на Generic Transformers решением является библиотека исследователя F.Pottier 2017 года, но её использование для типов, используемых вместе с OCanren невозможно: сгенерированный код не проходит проверку типов. Это является фундаментальной проблемой для работы F.Pottier, поэтому в Generic Transformers были приняты противоположные дизайн-решения, которые позволяют обойти возникающие проблемы.

\emph{В: Как может облегчиться написание реляционных программ с помощью OCanren по сравнению с оригинальным miniKanren?}
В работе 2012 года исследователи W. Byrd, E. Hold и D. Fridman для %корректной работы
 реляционного интерпретатора вводили дополнительные синтаксические конструкции в miniKanren, чтобы гарантировать корректность получаемых результатов. В их реализации синтаксические конструкции интерпретируемого языка смешиваются с данными miniKanren и поэтому возникает необходимость проверять, что некоторые символы языка Scheme не встречаются в подвыражениях. При переписывании символьного интерпретатора на строго статически типизированный OCanren были спроектировны типы данных таким образом, что некорректные данные построить невозможно, т.е. проверку на корректность начала выполнять система типов языка OCaml. Таким образом, с использованием OCanren удалось сократить количество примитивов miniKanren, которые требуются для создания реляционного интерпретатора с семи до пяти, выкинув \texttt{absento} и \texttt{numero}.

\emph{В: Какие улучшения и недостатки может привнести реляционное программирование для задачи компиляции конструкции сопоставления с образцом? (планируется)}
В контексте сопоставления с образцом компилятор выполняет две задач: проверки образцов на полноту и непосредственно компиляцию во внутреннее представление. Задача компиляции в максимально эффективный код является сложной. В работе 2008 год исследователь K.Maranget предложил использовать для этого построение диаграмм решений на основе некоторой эвристики.  Построение же оптимальной диаграммы решений является NP-трудной задачей из-за необходимости произведения поиска с возвратами. В реляционном программировании поиск с возвратами встроен изначально, и поэтому является  естественным применить реляционное программирование для генерации кода сопоставления с образцом.

Задача проверки на полноту обычно в компиляторах решается написанием отдельного алгоритма. Эта задача заключается в проверке того, что для любого сопоставляемого выражения (scrutinee) исполнение продолжится под одной из веток сопоставления с образцом, т.е. не случится исключительной ситуации, когда ни одна ветвь сопоставления с образцом не подходит для данного разбираемого выражения. Используя реляционное программирование, мы сможем переиспользовать реляцию для порождения промежуточного представления, запустив её в обратную сторону: зафиксировав результат и подобрав выражения, на которых ни один из образцов не подойдет.

Использование реляционного программирования может облегчить поддержку и последующее расширение алгоритмов, связанных с компиляцией сопоставления с образцом. Любопытным является попытка поддержать различные расширения сопоставления с образцом и оценить усилия, которые требуются для добавления расширений в классическом и реляционном случае. Например, можно добавить в сопоставление с образцом поддержку целочисленных интервалов. На данный момент в компиляторе OCaml это не поддерживается\footnote{\href{https://github.com/ocaml/ocaml/issues/8504}{\texttt{https://github.com/ocaml/ocaml/issues/8504} багтрекер}} по причине того, что в текущую реализацию компилятор это сложно добавить, не увеличив существенным образом размер генерируемого кода. Однако, при использовании реляционного программирования эта проблема видится тривиальной.

\emph{В: Какие улучшения может привнести реляционное программирование для задачи выделения регистров? (планируется)}
Скорее всего более удобный способ описать алгоритм, мы этим вплотную пока не занимались.

{\underline{\textbf{Что надо сделать, чтобы считать диссер законченным?}}}
\begin{itemize}
\item Статья про сопоставление с образцом
\begin{itemize}
\item Проверка на полноту с запуском в обратную сторону
\item  Поддержка расширений для демонстрации лаконичности реляционного программирования: числовые интервалы, охранные выражения, может быть полиморфные вариантные типы
\end{itemize}
\item Статья про генерацию кода с выделением регистров.
\end{itemize}

{\influence}
%Диссертационное исследование предлагает новый
%операционный способ представления реалистичной семантики многопоточности с помощью меток времени и фронтов, который
%может быть полезен при верификации многопоточных программ с неблокирующей синхронизацией, а также при анализе
%реализации примитивов блокирующей синхронизации.
%
%Предложенный в диссертационном исследовании метод доказательства корректности компиляции из обещающей в
%аксиоматические модели памяти может быть использован для доказательств корректности компиляции
%из обещающей модели в архитектуры других процессоров.
%Последнее актуально в свете того, что
%в комитетах по стандартизации языков C и C++ активно обсуждается вопрос о смене модели памяти, и обещающая модель 
%является одной из возможных альтернатив. 

%% \textcolor{blue}{План публикаций ВАК:}
%% \begin{enumerate}
%%   %% \item \emph{Сертифицируемость шагов обещающей \; семантики в контексте доказательства корректности компиляции
%%   %%   в модель памяти ARMv8 POP.}

%%   %%   В этой статье я собираюсь описать часть доказательства, которая не описана в моей статье на ECOOP'17.
%%   %%   \textcolor{red}{Подготовить черновой вариант за июль.}
%%   %%   \textcolor{green}{Вестник Политеха.}
    
%%   \item \emph{Слабые модели памяти языков программирования. Проблемы и решения.}
%%     Обзорная статья. Можем попробовать успеть опубликовать в первом выпуске НТВ СПбГПУ 2018 года.
    
%%   \item Кирпич нужно подготовить к концу декабря.
%%   %% \item Новый вариант названия ``Слабые модели памяти для языка C/C++''.
%% \end{enumerate}

{\reliability}
%Достоверность и обоснованность результатов исследования
%обеспечивается формальными доказательствами, а также инженерными экспериментами.
%Полученные результаты согласуются с результатами, установленными другими авторами.


%% {\probation}
%Основные результаты работы докладывались~на следующих научных конференциях и семинарах:
%внутренний семинар ННГУ им. Лобачевского (13 декабря 2017, Нижний Новгород, Россия),
%открытая конференция ИСП РАН им. В.П. Иванникова (30 ноября--1 декабря  2017, РАН, Москва, Россия),
%семинар ``Технологии разработки и анализа программ'' (16 ноября 2017, ИСП РАН, Москва, Россия),
%внутренние семинары School of Computing of the University of Kent (август 2017, Кентербери, Великобритания),
%внутренние семинары Department of Computer Science of UCL (август 2017, Лондон, Великобритания),
%внутренние семинары MPI-SWS (май 2017, Кайзерслаутерн, Германия),
%The European Conference on Object-Oriented Programming (ECOOP, 18--23 июня 2017, Барселона, Испания),
%конференция ``Языки программирования и компиляторы'' (PLC, 3--5 апреля 2017, Ростов-на-Дону, Россия),
%Verified Trustworthy Software Systems workshop (VTSS, 4-7 апреля 2016, Лондон, Великобритания),
%POPL 2016 Student Research Competition (21 января 2016, Санкт-Петербург, Флорида, США).

%% {\contribution} Автор принимал активное участие \ldots

%% \publications\ Основные результаты по теме диссертации изложены в ХХ печатных изданиях~\cite{Sokolov,Gaidaenko,Lermontov,Management},
%% Х из которых изданы в журналах, рекомендованных ВАК~\cite{Sokolov,Gaidaenko}, 
%% ХХ --- в тезисах докладов~\cite{Lermontov,Management}.

\ifnumequal{\value{bibliosel}}{0}{% Встроенная реализация с загрузкой файла через движок bibtex8
    {\publications}
%        Основные результаты по теме диссертации изложены в пяти %\arabic{citeauthor}
%        печатных работах, зарегистрированных в РИНЦ.
%        Из них две статьи изданы в журналах из \vakJournals.
%        Одна статья опубликована в издании, входящем в базы цитирования SCOPUS и Web of Science.
%        
%        Личный вклад автора в публикациях, выполненных с соавторами, распределён следующим образом.
%        В статьях \cite{Podkopaev-al:NTV17,Podkopaev-al:ISPRAS17}  автор предложил  схему
%        доказательства корректности компиляции для аксиоматических семантик и выполнил само доказательство для модели ARMv8.3;
%        соавторы участвовали в обсуждении основных идей доказательства.
%        В работах \cite{Podkopaev-al:ECOOP17, Podkopaev-al:PLC17}
%        автор выполнил формализацию семантики ARMv8 POP и доказал корректность компиляции методом симуляции;
%        соавторы участвовали в обсуждении идей доказательства и редактировали тексты статей.
%        В работе \cite{Podkopaev-al:CoRR16} личный вклад автора заключается в предложении идеи меток времени и фронтов как способа
%        операционного задания модели памяти, а также в создании компонентного метода задания семантики и реализации интерпретатора;
%        соавторы предложили синтаксический способ обработки отложенных операций.
        
}{% Реализация пакетом biblatex через движок biber
%Сделана отдельная секция, чтобы не отображались в списке цитированных материалов
    \begin{refsection}[vak,papers,conf]% Подсчет и нумерация авторских работ. Засчитываются только те, которые были прописаны внутри \nocite{}.
        %Чтобы сменить порядок разделов в сгрупированном списке литературы необходимо перетасовать следующие три строчки, а также команды в разделе \newcommand*{\insertbiblioauthorgrouped} в файле biblio/biblatex.tex
        \printbibliography[heading=countauthorvak, env=countauthorvak, keyword=biblioauthorvak, section=1]%
        \printbibliography[heading=countauthorconf, env=countauthorconf, keyword=biblioauthorconf, section=1]%
        \printbibliography[heading=countauthornotvak, env=countauthornotvak, keyword=biblioauthornotvak, section=1]%
        \printbibliography[heading=countauthor, env=countauthor, keyword=biblioauthor, section=1]%
        \nocite{%Порядок перечисления в этом блоке определяет порядок вывода в списке публикаций автора
                Podkopaev-al:NTV17,
                Podkopaev-al:ISPRAS17,
                %% vakbib1,vakbib2,%
                %% confbib1,confbib2,%
                Podkopaev-al:ECOOP17,
                Podkopaev-al:PLC17, Podkopaev-al:CoRR16,
                %% bib1,bib2,%
        }%
        {\publications}
%        Основные результаты по теме диссертации изложены в пяти 
%        печатных работах, зарегистрированных в РИНЦ.
%        Из них две статьи изданы в журналах из \vakJournals.
%        Одна статья опубликована в издании, входящем в базы цитирования Scopus и Web of Science.

%        Личный вклад автора в публикациях, выполненных с соавторами, распределён следующим образом.
%        В статьях \cite{Podkopaev-al:NTV17,Podkopaev-al:ISPRAS17}  автор предложил  схему
%        доказательства корректности компиляции для аксиоматических семантик и выполнил само доказательство для модели ARMv8.3;
%        соавторы участвовали в обсуждении основных идей доказательства.
%        В работах \cite{Podkopaev-al:ECOOP17, Podkopaev-al:PLC17}
%        автор выполнил формализацию семантики ARMv8 POP и доказал корректность компиляции методом симуляции;
%        соавторы участвовали в обсуждении идей доказательства и редактировали текст статей.
%        В работе \cite{Podkopaev-al:CoRR16} личный вклад автора заключается в предложении идеи меток времени и фронтов как способа
%        операционного задания модели памяти, а также в создании компонентного метода задания семантики и реализации интерпретатора;
%        соавторы предложили синтаксический способ обработки отложенных операций.
    \end{refsection}
    \begin{refsection}[vak,papers,conf]%Блок, позволяющий отобрать из всех работ автора наиболее значимые, и только их вывести в автореферате, но считать в блоке выше общее число работ
        \printbibliography[heading=countauthorvak, env=countauthorvak, keyword=biblioauthorvak, section=2]%
        \printbibliography[heading=countauthornotvak, env=countauthornotvak, keyword=biblioauthornotvak, section=2]%
        \printbibliography[heading=countauthorconf, env=countauthorconf, keyword=biblioauthorconf, section=2]%
        \printbibliography[heading=countauthor, env=countauthor, keyword=biblioauthor, section=2]%
        \nocite{Podkopaev-al:NTV17}%vak
        \nocite{Podkopaev-al:ISPRAS17}%vak
        \nocite{Podkopaev-al:ECOOP17}%conf
        \nocite{Podkopaev-al:PLC17,Podkopaev-al:CoRR16}%notvak
    \end{refsection}
}
%% При использовании пакета \verb!biblatex! для автоматического подсчёта
%% количества публикаций автора по теме диссертации, необходимо
%% их здесь перечислить с использованием команды \verb!\nocite!.
 % Характеристика работы по структуре во введении и в автореферате не отличается (ГОСТ Р 7.0.11, пункты 5.3.1 и 9.2.1), потому её загружаем из одного и того же внешнего файла, предварительно задав форму выделения некоторым параметрам

%Диссертационная работа была выполнена при поддержке грантов ...

% TODO
\underline{\textbf{Объем и структура работы.}}
% Диссертация состоит из~введения, четырех глав, заключения и~приложения. Полный объем диссертации \textbf{190}~страниц текста с~\textbf{29}~рисунками и~\textbf{4}~таблицами. Список литературы содержит \textbf{109}~наименований.

%\newpage
\section*{Содержание работы}
\begin{comment}
Во \underline{\textbf{введении}} обосновывается актуальность
исследований, выполненных в рамках данной диссертационной работы,
приводится краткий обзор научной литературы по изучаемой проблеме,
формулируется цель, ставятся задачи работы, излагается научная новизна
и практическая значимость представленного исследования.

\underline{\textbf{Первая глава}} посвящена обзору области исследования.
Рассматриваются требования к реалистичным моделями памяти
языков программирования, предъявляемые через призму наблюдаемых сценариев поведения многопоточных программ,
применяемых компиляторами оптимизаций, а также моделей памяти процессорных архитектур.
Описывается модель памяти C/C++11. Рассматриваются проблемы модели C/C++11, в том числе
проблема ``значений из воздуха''. Приводится описание существующих слабых моделей памяти без ``значений из воздуха'', в частности, обещающей модели.
На основе выполненного обзора делаются следующие выводы.
\begin{itemize}
  \item Модель памяти промышленного языка программирования должна удовлетворять, как минимум, трём критериям.
    Во-первых, должна существовать корректная схема компиляции в модель целевой процессорной
    архитектуры.
    Во-вторых, основные компиляторные оптимизации должны быть корректны в рамках модели.
    В-третьих, у модели должна отсутствовать проблема ``значений из воздуха''.
  \item При разработке новой модели памяти языка программирования нужно доказывать корректность эффективной компиляции
     в модели памяти целевых процессорных архитектур.
  \item Существующие модели памяти промышленных языков программирования не удовлетворяют всем приведённым выше
    критериям.
  \item Требуется разработать операционную модель памяти с синтаксисом модели C/C++11, которая
    не имеет проблемы ``значений из воздуха''.
  %% \item Обещающая модель памяти является перспективной альтернативой существующей модели памяти C/C++.
\end{itemize}

\underline{\textbf{Вторая глава}} посвящена описанию предложенной в диссертации операционной  модели памяти $\OpCpp$ для C/С++. Модель представлена в виде операционной семантики малого шага с помощью редукционных
контекстов. 
Основное отличие слабых моделей памяти от модели последовательной консистентности заключается в том, что
первые не гарантируют для локации в памяти единственность значения, которое может быть прочитано
в каждый конкретный момент времени.
Так, например, следующая программа может завершиться с
результатом $[a = 1, b = 0]$, хотя, казалось бы, $a = 1$ гарантирует, что
в локацию $d$ уже записано новое значение $239$:
\[
\begin{array}{c}
[f] := 0; [d] := 0; \\
\begin{array}{l||l}
  {} [d] := 239; & a := [f]; \\
  {} [f] := 1   & b := [d] \\
\end{array}
\end{array}
\]
Как следствие, оперативная память (далее просто ``память'') в рамках слабых моделей памяти не может быть представлена как функция из локации в значения.

Память в модели $\OpCpp$ представляется как множество \emph{сообщений}. Каждое сообщение содержит целевую локацию, записываемое значение и
\emph{метку времени} --- натуральное число, которое определяет полный порядок на сообщениях, относящихся к одной локации.
Последнее нужно для того, чтобы гарантировать последовательную консистентность для программ, оперирующих только над одной локацией --- эту гарантию
предоставляют большинство слабых моделей памяти, в том числе модель C/C++11.
При выполнении инструкции чтения из некоторой локации поток может недетерминировано выбрать сообщение, относящееся к этой локации, и выполнить из него чтение.
чтение из него.

%% Представление памяти как множества сообщений позволяет симулировать исполнение приведенной выше программы, которое заканчивается результатом
%% $a = 1, b = 0$, следующим образом. После выполнения Сначала левый поток выполняет 

Недетерминированность чтения из локации ограничена гарантией, которую также предоставляет модель C/C++11:
после того, как поток прочитал или записал сообщение в локацию $x$ с меткой времени $t$, он (поток) больше не может читать из сообщений
с меткой времени, которая меньше $t$. Для реализации данного ограничения в модели $\OpCpp$ у каждого потока есть т.н. \emph{базовый фронт}
(current view, current viewfront) --- функция из локаций в метки времени, определяющая осведомленность потока о сообщениях в памяти.

Некоторые программы имеют слабые сценарии поведения, разрешенные моделью C/C++11, которые не могут быть смоделированы только недетерминированной
памятью, а также требуют исполнения инструкций не по порядку. Например, следующая программа может завершиться с
результатом $[a = 1, b = 1]$ в модели C/C++11:
\[
\begin{array}{c}
[x] := 0; [y] := 0; \\
\begin{array}{l||l}
  {} a := [x]; & b := [y]; \\
  {} [y] := 1 & [x] := 1 \\
\end{array}
\end{array}
\]
Для представления таких сценариев поведения в модели $\OpCpp$ у каждого потока есть \emph{буфер отложенных операций}. Так, модель
позволяет потоку в каждый момент отложить текущую операцию вместо её выполнения.
С помощью этого механизма модель $\OpCpp$ может исполнить приведенную выше программу и получить результат $a = 1, b = 1$ следующим образом.
Сначала левый поток откладывает чтение из локации $x$ и выполняет запись в $y$. После этого правый поток читает из вновь добавленного
сообщения и записывает $1$ в $x$. Далее левый поток исполняет отложенное чтение из сообщения, добавленного правым потоком, и получает
результат $[a = 1, b = 1]$.

Для поддержки высвобождающих (release) барьеров и записей, а также приобретающих (acquire) барьеров и чтений
модель $\OpCpp$ использует дополнительные фронты --- высвобождающий и приобретающий для каждого потока,
а также фронт сообщений для каждого сообщения в памяти. Для поддержки чтений с модификатором доступа
consume модель использует динамическую пометку инструкций, зависимых от consume-чтения.

%% Рассмотрим следующую программу, в которой левый поток передает сообщение через локацию $d$ в правый поток:
%% \[
%% \begin{array}{c}
%% [f] := 0; [d] := 0; \\
%% \begin{array}{l||l}
%%   {} [d] := 239; & a := [f]; \\
%%   {} [f] := 1;   & b := [d]; \\
%% \end{array}
%% \end{array}
%% \]
%% Здесь локация $f$ используется как индекс того, что левый поток уже записал нужные данные в локацию $d$.
%% В рамках модели последовательной консистентности гарантируется, что если $a = 1$, т.е. правый поток
%% ``увидел'' запись в $f$, то дальше он прочитает из записи в $d$, сделанной левым потоком, и в результате $b$ будет
%% равняться $239$. Это гарантируется тем, что 

Для предложенной модели был реализован интерпретатор на языке Racket с помощью библиотеки описания редукционных
семантик PLT/Redex. Код проекта доступен по адресу \url{github.com/anlun/OperationalSemanticsC11}.

Апробация предложенной семантики была выполнена на наборе, состоящем более чем из 40 тестов (litmus tests), взятых из  тематической литературы, а также 
на алгоритме RCU (Read-Copy-Update). Поведение модели $\OpCpp$ совпадает с поведением модели C/C++11 на большинстве этих тестов.
Отличие наблюдается на двух следующих категориях тестов. Первая категория --- это программы, которые имеют исполнения со
``значениями из воздуха'' в рамках модели C/C++11. Для таких тестов модель $\OpCpp$ не выдает исполнения со ``значениями из воздуха'',
что является её положительным свойством. Вторая категория --- это программы, в которых существуют антизависимости по управлению,
адресу или значению, ведущие к инструкциям записи. На таких программах $\OpCpp$ не способна получить все возможные в рамках C/C++11
исполнения, поскольку выполнение инструкций не по порядку в $\OpCpp$ реализовано синтаксическим способом. Этот недостаток модели не позволяет ей поддержать все необходимые компиляторные оптимизации.

Одновременно с моделью памяти $\OpCpp$ исследователями J. Kang, C.-K. Hur, O. Lahav, V. Vafeiadis и D. Dreyer была представлена обещающая модель
памяти, которая очень близка $\OpCpp$, но использует другой механизм для выполнения инструкций не по порядку.
Этот механизм позволяет поддержать больше компиляторных оптимизаций, чем модель $\OpCpp$.
Из-за данного преимущества диссертант принял решение продолжить свою исследовательскую работу в рамках обещающей модели.

В \underline{\textbf{третьей главе}} приводится описание обещающая модель памяти и операционной модели ARMv8 POP,
а также представлено доказательство
корректности компиляции из существенного подмножества обещающей модели памяти в модель ARMv8 POP.
%% В конце главы приводятся рассуждения о том, как нужно доработать доказательство, чтобы покрыть всю обещающую модель.

Обещающая модель памяти является операционной моделью для синтаксиса модели C/C++11.
Она использует те же базовые понятия, что и предложенная диссертантом модель $\OpCpp$,
метки времени и фронты, однако вместо механизма откладывания выполнения инструкций она,
в соответствии со своим названием, использует механизм \emph{обещаний}. Так, в каждый момент исполнения поток
обещающей модели памяти может совершить одно из двух действий: либо выполнить следующую инструкцию,
либо пообещать сделать запись в локацию. Последнее может быть выполнено вне зависимости от того, какая
инструкция является следующей.
Если поток выбирает пообещать сделать запись в локацию, то он добавляет соответствующее сообщение в память,
делая это сообщение видимым для других потоков. Далее в ходе исполнения поток должен будет выполнить соответствующую инструкцию записи, таким образом выполняя сделанное ранее обещание.

Для того, чтобы запретить ``значения из воздуха'', после каждого исполненного шага каждый поток должен выполнить
т.н. \emph{сертификацию} --- предъявить, что он может быть локально исполнен таким образом, что
выполнит все оставшиеся обещания. Известно, что задача сертификации является алгоритмически неразрешимой для языков,
 полных по Тьюрингу. Следовательно, для
обещающей модели невозможно разработать интерпретатор, что является её недостатком по сравнению с представленной в
диссертации моделью $\OpCpp$.

Несмотря на этот недостаток, обещающая модель обладает рядом существенных достоинств.
В частности, обещающая модель не имеет проблемы ``значений из воздуха'', что делает возможным для неё разработать
выразительную программную логику. Также для модели была доказана корректность существенного класса компиляторных
оптимизаций и корректность эффективной компиляции в модели памяти процессоров x86 и Power.

Открытой проблемой является доказательство корректности компиляции из обещающей модели памяти в модели памяти архитектуры
ARM, которая, наравне с x86 и Power, является одной из наиболее распространённых процессорных архитектур на данный момент.

Здесь и далее под корректностью компиляции понимается следующее утверждение.\\
\textbf{Определение.} Для языков $L$ и $L'$ с моделями памяти $M$ и $M'$ соответственно схема компиляции $\textsf{compile} : L \rightarrow L'$
называется \emph{корректной}, если выполняется следующее условие:
\[ \forall Prog \in L. \; \sembr{\textsf{compile}(Prog)}_{M'} \subseteq \sembr{Prog}_{M}, \]
где $\sembr{Prog}_{M}$ --- множество результатов сценариев поведения программы $Prog$ в модели памяти $M$.
В доказательствах корректности компиляции в ARMv8 POP и ARMv8.3 результатом сценария поведения считается
финальное состояние памяти.

Рассмотренное подмножество обещающей модели памяти ($\Promise$) состоит из расслабленных (relaxed, rlx) записей и чтений,
а также высвобождающих (release, rel) и приобретающих (acquire, acq) барьеров памяти. При этом подразумевается следующая
схема компиляции:
\[
  \begin{array}{c@{~~}@{~~}l@{~~}|@{~~}l@{~~}|@{~~}l@{~~}|@{~~}l}
    \textbf{Promise:}   & [x]_{\textsf{rlx}} := \; a & a := [x]_{\textsf{rlx}}  &  \acqFence & \relFence \\[2pt]
    \textbf{ARMv8 POP:} & \writeInst{x}{a}    & \readInst{a}{x}  &  \dmbLD & \dmbSY \\
  \end{array}
\]
Первый и второй столбцы подразумевают, что расслабленные операции записи и чтения из языка, на котором определена обещающая модель,
переходят в обычные операции записи и чтения в терминах ARMv8-ассемблера, а приобретающий и высвобождающий барьеры --- в барьер
по чтению и в полный барьер. Такая схема компиляции считается эффективной и применяется в компиляторах GCC и LLVM.
Поскольку схема компиляции в данном случае является биекцией, то далее в этой главе предполагается, что язык задания обещающей и ARMv8 POP
моделей совпадает.

Основной результат главы %про корректность компиляции из обещающей семантики в ARMv8 POP
сформулирован следующим образом. \\
\textbf{Теорема.} Для любой программы $\Cf$ на языке задания модели и её сценария поведения в модели ARMv8 POP существует
такой сценарий поведения $\Cf$ в обещающей модели, что финальное состояние памяти в сценариях поведения совпадает.

В рамках доказательства теоремы по сценарию поведения программы в модели ARMv8 POP строится сценарий поведения в обещающей модели.
Поскольку обе модели заданы операционным способом, то существуют две абстрактные машины, которые представляют данные модели.
Обычно для решения задачи построения сценария поведения одной машины по сценарию другой используют технику симуляции,
которая является специальной формой индукции. В рамках данной техники вводится отношение симуляции,
которое связывает состояния машин, и доказываются две следующие леммы:
отношение симуляции связывает начальные состояния машин (база индукции);
для любого шага симулируемой машины существует ноль или более шагов симулирующей машины, после выполнения которых 
новые состояния машин опять связаны отношением симуляции (индукционный переход).

%% Модель памяти ARMv8 POP --- операционная модель памяти для архитектуры ARMv8.0, предложенная в 2016 г.
%% Данная модель определена в терминах, достаточно близких к физической реализации архитектуры ARM.
%% Абстрактная машина, которая реализует модель ARMv8 POP, состоит из двух компонент: подсистем памяти и управления.

%% Подсистема памяти является иерархической структурой \emph{буферов}, каждый из которых является списком запросов к подсистеме.
%% Запросом, при этом, может быть чтение из локации, запись в локацию или барьер памяти.
%% Идейно подсистема памяти похожа на иерархическую систему кэшей, используемую в современных процессорах.
%% Так, подсистема управления может послать запрос в подсистему хранения. В таком случае запрос сначала попадает в буфер, который локален
%% для соответствующего потока, потом он может быть передан следующему в иерархии буферу, который является общий для некоторого
%% набора потоков, и так далее, пока запрос не попадёт в основную память или, если это запрос на чтение, не будет
%% удовлетворён из сообщения.

%% Подсистема управления выполняет программу каждого потока. Шагом исполнения программы потока является полное или частичное исполнение
%% некоторой инструкции из потока. Например, исполнение инструкции чтения выполняется в три шага --- отправка запроса в подсистему памяти,
%% получение ответа от подсистемы памяти и завершение исполнения инструкции чтения. При этом между упомянутыми шагами исполнения могут
%% исполняться другие инструкции, поскольку подсистема управления может исполнять инструкции не по порядку и спекулятивно.

В доказательстве индукционного перехода сложным является то, что между моделями имеется два существенных различия.
Во-первых, обещающая модель может исполнять не по порядку только инструкции записи, 
в то время как модель ARMv8 POP может исполнять инструкции в несколько шагов, не по порядку и спекулятивно.
Во-вторых, в обещающей модели в тот момент, когда сообщение попадает в память, этому сообщению присваивается некоторая
метка времени, которая служит его порядковым номером в множестве сообщений, относящихся к той же локации. В модели ARMv8 POP
меток времени нет и порядок сообщений одной локации определяется не сразу после того, как
сообщения попадут в её подсистему памяти и станут видимыми для других потоков.

Для того, чтобы обойти первое различие, автор использовал технику ``запаздывающей'' симуляции. В рамках данной техники отношение
симуляции представляется как объединение двух взаимоисключающих отношений, например, $A$ и $B$. Далее индукционный
переход формулируется следующим образом. Если состояние симулируемой машины $x$ связано с состояние симулирующей машины $y$ отношением $A$,
т.е. выполняется $(x,y) \in A$, то для любого состояния $x'$, в которое может перейти симулируемая машина, выполняется $(x', y) \in A \cup B$.
С другой стороны, если $(x, y) \in B$, то существует состояние $y'$, в которое может перейти симулирующая машина, что $(x, y') \in A \cup B$.
Тогда при условии, что не существует такой бесконечной цепочки $\{y'_i\}_{i \in \mathbb{N}}$, что $y'_i$ переходит $y'_{i+1}$ и $(x, y'_i) \in B$ для всех $i$,
из нового варианта индукционного перехода следует изначальное утверждение симуляции.

В доказательстве теоремы отношение $A$ символизирует, что обещающая машина ждёт, пока ARM-машина выполнит действие, которое обещающая
машина может повторить, а отношение $B$ означает, что обещающая машина может симулировать несколько действий, уже выполненных ARM-машиной.

Для того, чтобы снять второе различие между обещающей и ARMv8 POP моделями, в доказательстве определяется ограниченная версия ARM-машины,
которая добавляет метки времени к сообщениям записи в подсистеме памяти, тем самым определяя порядок на сообщениях к одной локации
раньше, чем это делает обычная ARMv8 POP модель. Это изменение также добавляет дополнительные ограничения на сценарии поведения ARM-машины.
Тем не менее, автор приводит доказательство того, что новая модель эквивалентна исходной, что позволяет свести доказательство теоремы 
к доказательству корректности компиляции из обещающей модели в модифицированную ARMv8 POP модель.

В \underline{\textbf{четвертой главе}} обсуждается аксиоматическая модель памяти ARMv8.3.
Приводятся рассуждения о том, почему метод доказательства корректности компиляции из
обещающей модели памяти, использованный её авторами для аксиоматических моделей архитектур x86 и Power,
не подходит для модели ARMv8.3. Далее приводится доказательство корректности компиляции из обещающей модели
памяти в подмножество модели ARMv8.3. Доказательство основано на построении операционной семантики обхода аксиоматических сценариев
поведения программ в модели ARMv8.3.
%% В конце главы приводятся рассуждения о применимости использованного подхода для других аксиоматических моделей.

В рамках аксиоматической (или декларативной) модели памяти сценарий поведения программы представляется в виде графа, в котором вершинами
являются \emph{события} (операции над памятью), а ребрами --- различные отношения на событиях, такие как программный
порядок, отношение ``читает из'' и др. При этом граф считается согласованным с моделью, если выполняются \emph{аксиомы} модели,
которые обычно формулируются как наличие некоторого полного порядка на подмножестве событий или отсутствие  в графе путей определённого
типа.

При доказательстве корректности компиляции из обещающей модели в аксиоматическую техника симуляции напрямую
неприменима, поскольку сценарий поведения в аксиоматической модели не является последовательностью шагов исполнения некоторой абстрактной машины.
Поэтому в доказательстве корректности компиляции в 
модели x86 и Power авторы использовали другой метод. Этот метод состоит из двух частей.
Во-первых, доказывается, что модели x86 и Power могут быть представлены как набор программных оптимизаций поверх
более простых моделей. Эти оптимизации являются доказано корректными в рамках обещающей модели, из чего следует,
что доказательство корректности компиляции может быть сведено к аналогичному доказательству для более простых моделей.
Далее показывается, что эти более простые модели могут быть симулированы моделью, которая является аксиоматическим аналогом
обещающей модели без механизма обещаний.

Автору работы не удалось применить такой подход для модели ARMv8.3, поскольку эта модель не представима как набор тех же
оптимизаций над упрощенной моделью. Поэтому был разработан альтернативный подход,
который заключается в построении операционной семантики обхода аксиоматических сценариев поведения, которая может быть
непосредственно симулирована обещающей моделью.

\emph{Обходом} в диссертационном исследовании называется последовательность переходов между \emph{конфигурациями исполнения} ---
упорядоченными парами подмножеств вершин сценария поведения $\tup{C, \IssuedSet}$.
Подмножество $C$ называется \emph{множеством покрытых событий}, а подмножество $\IssuedSet$ --- \emph{множеством выпущенных событий};
элементы этих подмножеств называются \emph{покрытыми} и \emph{выпущенными} соответственно.

Конфигурация обхода называется \emph{корректной}, если выполняются следующие условия.
\begin{itemize}
  \item Множество покрытых событий префикс-замкнуто по отношению программного порядка.
  \item Множество выпущенных событий содержит только события записи.
  \item Если событие записи покрыто, то оно также является выпущенным.
\end{itemize}
При доказательстве симуляции обещающей моделью обхода покрытые события будут соответствовать инструкциям, выполненным
обещающей машиной, а выпущенные события --- сообщениям в памяти обещающей машины.

Шаги обхода задаются следующим образом:
\begin{mathpar}
\inferrule*{
    a \in \nextset(G, C) \cap \coverable(G, C, \IssuedSet) 
}{
    G \vdash 
    \tup{C, \IssuedSet} \travConfigStep \tup{C \cup \{a\}, \IssuedSet}
} \and
\inferrule*{
    w \in \issuable(G, C, \IssuedSet) \setminus \IssuedSet 
    }{
    G \vdash
    \tup{C, \IssuedSet} \travConfigStep \tup{C, \IssuedSet \cup \{w\}}
}
\end{mathpar}
Здесь первое правило соответствует покрытию события $a$, а второе --- выпуску события $w$; $\nextset(G, C)$ --- обозначает множество событий,
непосредственно следующих в отношении программного порядка за покрытыми; $\coverable(G, C, I)$ и $\issuable(G, C, I)$ --- это
события, покрываемые и выпускаемые в текущей конфигурации, которые определены в соответствии с требованиями обещающей модели
к исполнению инструкции и обещанию сообщения соответственно.

Далее автор работы приводит доказательство следующей теоремы о полноте обхода. \\
\textbf{Теорема.} Для любого корректного сценария поведения $G$ в модели ARMv8.3 существует обход
$\tup{W^{\textrm{init}}, W^{\textrm{init}}} \travConfigStep^{*} \tup{E, W}$,
где $W^{\textrm{init}}$ --- множество инициализирующих записей сценария $G$, $E$ --- все события сценария $G$,
$W$ --- все события записи сценария $G$.

Используя данную теорему для построения операционного исполнения программы в модели ARMv8.3,
автор работы доказывает, что обещающая модель может симулировать сценарии поведения модели ARMv8.3.

%% Модель ARMv8.3 является аксиоматической, что не позволяет напрямую использовать технику симуляции для доказательства
%% корректности компиляции из обещающей модели. 

%% В \underline{\textbf{главе}}
%% Обсуждается структура доказательства корректности компиляции. Доказательство основано на симуляции обещающей моделью
%% памяти операционной семантики обхода исполнения программ в модели ARMv8.3.

В \underline{\textbf{заключении}} приведены основные результаты работы.
%% Согласно ГОСТ Р 7.0.11-2011:
%% 5.3.3 В заключении диссертации излагают итоги выполненного исследования, рекомендации, перспективы дальнейшей разработки темы.
%% 9.2.3 В заключении автореферата диссертации излагают итоги данного исследования, рекомендации и перспективы дальнейшей разработки темы.
\begin{enumerate}
  \item Разработана операционная модель памяти C/C++11.
    Данная модель допускает такие же сценарии поведения, что и модель C/С++11 на большинстве
    тестов, приведенных в литературе, но не обладает сценариями поведения со ``значениями из воздуха''.
    В отличие от обещающей модели, предлагаемая модель является запускаемой, что упрощает
    разработку средств анализа программ для неё. Недостатком модели является то,
    она накладывает синтаксические ограничения на поведения программ.
  \item Доказана корректность компиляции из существенного подмножества обещающей модели в операционную модель
    памяти ARMv8 POP.
  \item Доказана корректность компиляции из существенного подмножества обещающей модели в
    аксиоматическую модель памяти ARMv8.3.
\end{enumerate}

В рамках \textbf{рекомендации по применению результатов работы} в индустрии и научных исследованиях указывается,
что модель памяти промышленного языка программирования должна быть лишена сценариев поведения, имеющих ``значения из воздуха'',
а также либо быть представленной в операционной форме, либо иметь эквивалентный операционный аналог.
Последнее позволяет реализовать интерпретатор модели и выполнять отладку программ в рамках модели.

Также были определены \textbf{перспективы дальнейшей разработки тематики}, основным из которых является
разработка обобщенной аксиоматической модели памяти для процессорных архитектур, которая будет
определена для синтаксиса модели C/C++11 и окажется строгим надмножеством существующих моделей памяти
x86, Power и ARM, а также для которой будет применим предложенный метод доказательства корректности компиляции
из обещающей модели памяти. Это позволит свести дальнейшие доказательства корректности компиляции из
обещающей модели к доказательству корректности компиляции в обобщенную аксиоматическую модель, что
сводится к рассуждениям об ацикличности и вложенности путей на графах.
Кроме того, актуальной является задача разработки эффективной программной логики на базе логики многопоточного разделения
(concurrent separation logic) для операционного аналога модели памяти C/С++11 и обещающей модели памяти.
Такая логика позволит формально доказывать  в рамках моделей сложные свойства программ, такие как соответствие спецификации.


%%  картинку можно добавить так:
%% \begin{figure}[ht] 
%%   \center
%%   \includegraphics [scale=0.27] {latex}
%%   \caption{Подпись к картинке.} 
%%   \label{img:latex}
%% \end{figure}

%% Формулы в строку без номера добавляются так:
%% \[ 
%%   \lambda_{T_s} = K_x\frac{d{x}}{d{T_s}}, \qquad
%%   \lambda_{q_s} = K_x\frac{d{x}}{d{q_s}},
%% \]



%\newpage
%% При использовании пакета \verb!biblatex! список публикаций автора по теме
%% диссертации формируется в разделе <<\publications>>\ файла
%% \verb!../common/characteristic.tex!  при помощи команды \verb!\nocite! 

%% \ifdefmacro{\microtypesetup}{\microtypesetup{protrusion=false}}{} % не рекомендуется применять пакет микротипографики к автоматически генерируемому списку литературы
%% \ifnumequal{\value{bibliosel}}{0}{% Встроенная реализация с загрузкой файла через движок bibtex8
%%   \renewcommand{\bibname}{\large \authorbibtitle}
%%   \nocite{*}
%%   \insertbiblioauthor           % Подключаем Bib-базы
%%   %\insertbiblioother   % !!! bibtex не умеет работать с несколькими библиографиями !!!
%% }{% Реализация пакетом biblatex через движок biber
%%   %% \insertbiblioauthor           % Вывод всех работ автора
%%  \insertbiblioauthorgrouped    % Вывод всех работ автора, сгруппированных по источникам
%% %  \insertbiblioauthorimportant  % Вывод наиболее значимых работ автора (определяется в файле characteristic во второй section)
%%   \insertbiblioother            % Вывод списка литературы, на которую ссылались в тексте автореферата
%% }
%% \ifdefmacro{\microtypesetup}{\microtypesetup{protrusion=true}}{}

\end{comment}

\begin{comment}

\newcounter{firstbib}

\section*{\LARGE Публикации автора по теме диссертации}

Ниже приведён перечень публикаций, где были представлены основные результаты данной  диссертационной работы. \\

\renewcommand{\bibsection}{\noindent \textbf{\refname}}

\renewcommand{\refname}{Статьи из \vakJournals}
\begin{thebibliography}{99}
\bibitem{Podkopaev-al:NTV17} Подкопаев, А. В. О корректности компиляции подмножества обещающей модели памяти в аксиоматическую модель ARMv8.3 / А.В. Подкопаев, О. Лахав, В. Вафеядис // Научно-технические ведомости СПбГПУ. Информатика, Телекоммуникации. Управление. ---~2017. ---~Т.~10, \textnumero~4. ---~C.~51--69.
\bibitem{Podkopaev-al:ISPRAS17} Подкопаев, А. В. Обещающая компиляция в ARMv8.3 / А.В. Подкопаев, О. Лахав, В. Вафеядис // Труды ИСП РАН. ---~2017. ---~Т.~29, \textnumero~5. ---~C.~149--164.
\setcounter{firstbib}{\value{enumiv}}
\end{thebibliography}

\renewcommand{\refname}{Статьи в изданиях, входящих в базы цитирования Web of Science и Scopus}
\begin{thebibliography}{99}
\setcounter{enumiv}{\value{firstbib}}
\bibitem{Podkopaev-al:ECOOP17} Podkopaev, A. Promising compilation to ARMv8 POP / A. Podkopaev, O. Lahav, V. Vafeiadis // 31st European Conference on Object-Oriented Programming (ECOOP 17), Leibniz International Proceedings in Informatics (LIPIcs).  ---~2017. ---~P.~22:1--22:28.
\setcounter{firstbib}{\value{enumiv}}
\end{thebibliography}

\renewcommand{\refname}{Статьи в других изданиях}
\begin{thebibliography}{99}
\setcounter{enumiv}{\value{firstbib}}
\bibitem{Podkopaev-al:PLC17} Подкопаев, А. В. Обещающая компиляция в ARMv8 / А.В. Подкопаев, О. Лахав, В. Вафеядис // Языки программирования и компиляторы. Труды конференции. Ростов-на-Дону, Россия. ---~2017. ---~C.~223--226.
\bibitem{Podkopaev-al:CoRR16} Podkopaev, A. Operational Aspects of {C/C++} Concurrency / A. Podkopaev, I. Sergey, A. Nanevski
  [Электронный ресурс]. --- URL: \url{http://arxiv.org/abs/1606.01400} (дата обращения: 14.11.2017).
\end{thebibliography}
\end{comment}