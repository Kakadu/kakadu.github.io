\chapter*{Введение}                         % Заголовок
\addcontentsline{toc}{chapter}{Введение}    % Добавляем его в оглавление
\textbf{Актуальность работы}


\textbf{Степень разработанности темы}


\textbf{Объект исследования}

Объектом исследования являются методы, алгоритмы и программные средства обработки динамически формируемых программ, а также задача реинжиниринга информационных систем.

\textbf{Цель и задачи диссертационной работы}

\textbf{Целью} данной работы является создание комплексного подхода к статическому синтаксическому анализу динамически формируемых программ.

Достижение поставленной цели обеспечивается решением следующих \textbf{задач}.
\begin{enumerate}
    \item Разработать универсальный алгоритм синтаксического анализа динамически формируемых программ, не зависящий от целевого языка программирования и допускающий реализацию различных видов статического анализа. 
    \item Создать архитектуру инструментария для автоматизации разработки программных средств статического анализа динамически формируемых программ.
    \item Создать метод реинжиниринга динамически формируемых программ.
\end{enumerate}

\textbf{Методология и методы исследования}


\textbf{Положения, выносимые на защиту}
\begin{enumerate}
    \item 1
    \item 2
\end{enumerate}

\textbf{Научная новизна работы}

Научная новизна полученных в ходе исследования результатов заключается в следующем.


\textbf{Теоретическая и практическая значимость работы}


\textbf{Степень достоверности и апробация результатов}

Достоверность и обоснованность результатов исследования опирается на использование формальных методов исследуемой области, проведенные доказательства, рассуждения и эксперименты.

Основные результаты работы были доложены на ряде научных конференций: SECR-2012, SECR-2013, SECR-2014, TMPA-2014, Parsing@SLE-2013, Рабочий семинар ``Наукоемкое программное обеспечение'' при конференции PSI-2014. Доклад на SECR-2014 награждён премией Бертрана Мейера за лучшую исследовательскую работу в области программной инженерии. Дополнительной апробацией является то, что разработка инструментальных средств на основе предложенного алгоритма была поддержана Фондом содействия развитию малых форм предприятий в технической сфере (программа УМНИК\footnote{Список победителей конкурса УМНИК (дата обращения: 29.07.2015): \\ \url{http://www.fasie.ru/obyavleniya/9-obyavleniya-dlya-zayavitelej/1762-opublikovan-spisok-pobeditelej-programmy-umnik-2-go-polugodiya-2014-goda}.}, проекты \textnumero~162ГУ1/2013 и \textnumero~5609ГУ1/2014).

\textbf{Публикации по теме диссертации}



\textbf{Структура работы}


\textbf{Благодарности}

