\def\pgfsysdriver{pgfsys-dvipdfm.def} % required for forests inside
\documentclass[aspectratio=169
  , xcolor={svgnames}
  , hyperref={ colorlinks,citecolor=Blue
             , linkcolor=DarkRed,urlcolor=DarkBlue}
  , russian
  ]{beamer}
\usetheme{CambridgeUS}
\usefonttheme{professionalfonts}

\usepackage{pgfpages}

\usepackage{bibentry}
\usepackage{cite}
\def\newblock{\hskip .11em plus .33em minus .07em}

\usepackage{framed}
\makeatletter
\@ifclassloaded{beamer}{
  % get rid of header navigation bar
  \setbeamertemplate{headline}{}
  % get rid of bottom navigation symbols
  \setbeamertemplate{navigation symbols}{}
  % get rid of footer
  %\setbeamertemplate{footline}{}
}
{}
\makeatother
%%%%%%%%%%%%%%%%%%%%%%%%%%%%%%%%%%%%%%%%%%%%%
\usepackage{fontawesome}
% \newfontfamily{\FA}{Font Awesome 5 Free} % some glyphs missing
\expandafter\def\csname faicon@facebook\endcsname{{\FA\symbol{"F09A}}}
\def\faQuestionSign{{\FA\symbol{"F059}}}
\def\faQuestion{{\FA\symbol{"F128}}}
\def\faExclamation{{\FA\symbol{"F12A}}}
\def\faUploadAlt{{\FA\symbol{"F093}}}
\def\faLemon{{\FA\symbol{"F094}}}
\def\faPhone{{\FA\symbol{"F095}}}
\def\faCheckEmpty{{\FA\symbol{"F096}}}
\def\faBookmarkEmpty{{\FA\symbol{"F097}}}

\newcommand{\faGood}{\textcolor{ForestGreen}{\faThumbsUp}}
\newcommand{\faBad}{\textcolor{red}{\faThumbsODown}}
\newcommand{\faWrong}{\textcolor{red}{\faTimes}}
\newcommand{\faMaybe}{\textcolor{blue}{\faQuestion}}
\newcommand{\faCheckGreen}{\textcolor{ForestGreen}{\faCheck}}
%%%%%%%%%%%%%%%%%%%%%%%%%%%%%%%%%%%%%%%%%%%%%

\usepackage{fontspec}
\usepackage{xunicode}
\usepackage{xltxtra}
\usepackage{xecyr}
\usepackage{hyperref}

\setmainfont[
 Ligatures=TeX,
 Extension=.otf,
 BoldFont=cmunbx,
 ItalicFont=cmunti,
 BoldItalicFont=cmunbi,
% Scale = 1.1
]{cmunrm}
\setsansfont[
 Ligatures=TeX,
 Extension=.otf,
 BoldFont=cmunsx,
 ItalicFont=cmunsi,
%  Scale = 1.2
]{cmunss}
%\setmainfont[Mapping=tex-text]{DejaVu Serif}
%\setsansfont[Mapping=tex-text]{DejaVu Sans}
%\setmonofont{Fira Code}[Contextuals=AlternateOff]
\setmonofont{Fira Code}[Contextuals=Alternate,Scale=0.9]
\newfontfamily{\myfiracode}[Scale=1.5,Contextuals=Alternate]{Fira Code}
%\setmonofont[Scale=0.9,BoldFont={Inconsolata Bold}]{Inconsolata}

\usepackage{polyglossia}
\setmainlanguage{russian}
\setotherlanguage{english}


%\newfontfamily\dejaVuSansMono{DejaVu Sans Mono}
% https://github.com/vjpr/monaco-bold/raw/master/MonacoB/MonacoB.otf
%\newfontfamily\monacoB{MonacoB}
%%%%%%%%%%%%%%%%%%%%%%%%%%%%%%%%%%%%%%%%%%%%%%%5
\usepackage{soul} % for \st that strikes through
\usepackage[normalem]{ulem} % \sout

\usepackage{stmaryrd}
\newcommand{\sem}[1]{\ensuremath{\llbracket #1\rrbracket}}


\usepackage{listings}
%\lstdefinestyle{style1}{
%  language=haskell,
%  numbers=left,
%  stepnumber=1,
%  numbersep=10pt,
%  tabsize=4,
%  showspaces=false,
%  showstringspaces=false
%}
%\lstdefinestyle{hsstyle1}
%{ language=haskell
%%          , basicstyle=\monacoB
%         , deletekeywords={Int,Float,String,List,Void}
%         , breaklines=true
%         , columns=fullflexible
%         , commentstyle=\color{ForestGreen}
%         , escapeinside=§§
%         , escapebegin=\begin{russian}\commentfont
%         , escapeend=\end{russian}
%         , commentstyle=\color{ForestGreen}
%         , escapeinside=§§
%         , escapebegin=\begin{russian}\color{ForestGreen}
%         , escapeend=\end{russian}
%         , mathescape=true
%%          , backgroundcolor = \color{MyBackground}
%}
%
%\newcommand{\inline}[1]{\lstinline{haskell}{#1}}
%\def\hsinline{\mintinline{haskell}}
%\def\inline{\hsinline}
%
%\lstnewenvironment{hslisting} {
%    \lstset { style={hsstyle1} }
%  }
%  {}
%  
%%%%%%%%%%%%%%%%%%%%%%%%%%%%%%%%%%%%%%%%%%%%%%%%%%%%%%%%%%%  
%%\setmainfont[
%% Ligatures=TeX,
%% Extension=.otf,
%% BoldFont=cmunbx,
%% ItalicFont=cmunti,
%% BoldItalicFont=cmunbi,
%%]{cmunrm}
%%% С засечками (для заголовков)
%%\setsansfont[
%% Ligatures=TeX,
%% Extension=.otf,
%% BoldFont=cmunsx,
%% ItalicFont=cmunsi,
%%]{cmunss}
%% \setmonofont[Scale=0.6]{Monaco}
%
%\usefonttheme{professionalfonts}
%\usepackage{times}
\usepackage{tikz}
\usetikzlibrary{cd}
\usepackage{tikz-cd}
\usepackage{caption}
\usepackage{subcaption}

%\renewtheorem{definition}{برهان}[chapter]
%%\DeclareMathOperator{->}{\rightarrow}
%\newcommand\iso{\ensuremath{\cong}}
%\usepackage{verbatim}
%\usepackage{graphicx}
%\usetikzlibrary{arrows,shapes}

%\usepackage{amsmath}
%\usepackage{amsfonts}
\usepackage{scalerel}
\DeclareMathOperator*{\myvee}{\scalerel*{\vee}{\sum}}
\DeclareMathOperator*{\mywedge}{\scalerel*{\wedge}{\sum}}

%
%\usepackage{tabulary}
%
%% sudo aptget install ttf-mscorefonts-installer
%%\setmainfont{Times New Roman}
%%\setsansfont[Mapping=tex-text]{DejaVu Sans}
%
%%\setmonofont[Scale=1.0,
%%    BoldFont=lmmonolt10-bold.otf,
%%    ItalicFont=lmmono10-italic.otf,
%%    BoldItalicFont=lmmonoproplt10-boldoblique.otf
%%]{lmmono9-regular.otf}
%
\usepackage[cache=true]{minted}
\usemintedstyle{perldoc}

\def\hsinline{\mintinline{haskell}}
\def\mlinline{\mintinline{ocaml}}
% color options
\definecolor{YellowGreen} {HTML}{B5C28C}
\definecolor{ForestGreen} {HTML}{009B55}
\definecolor{MyBackground}{HTML}{F0EDAA}


\author{Косарев Дмитрий a.k.a. Kakadu}
\institute{матмех СПбГУ}

\addtobeamertemplate{title page}{}{
  \begin{center}{\tiny Дата сборки: \today}\end{center}
}

\newcommand{\staged}[1]{{\colorbox{light-gray}{#1}}}

%\lstdefinelanguage{ocamllambda}{
%keywords={catch, switch, default, case, with, failwith, exit, true, false, ::},
%sensitive=true,
%commentstyle=\small\itshape\ttfamily,
%keywordstyle=\ttfamily\textbf,
%identifierstyle=\ttfamily,
%basewidth={0.5em,0.5em},
%columns=fixed,
%mathescape=true,
%fontadjust=true,
%literate={->}{{$\to$}}3 {===}{{$\equiv$}}1  {Scru}{{$\bullet$}}2,
%morecomment=[s]{(*}{*)}
%}

\usepackage{subcaption}
\usepackage{etoolbox}


\usepackage{exercise}
\usepackage{tikz}
\usepackage{tikz-qtree}
\usetikzlibrary{trees}
\usepackage[edges]{forest}
\forestset{.style={
%  for tree={l=1em, l sep=1em, s sep=1em}
  forked edges,
    for tree={    grow'=0,    draw,    align=c,    font=\sffamily,
        rounded corners  }
  }}

\newcommand{\lstquot}[1]{``\lstinline{#1}''}
\newcommand{\sembr}[1]{\llbracket{#1}\rrbracket}
%\newcommand\false{$f\!alse$}
%\newcommand\myif{i\!f}
  \definecolor{light-gray}{gray}{0.80}

\def\transarrow{\xrightarrow}
\newcommand{\setarrow}[1]{\def\transarrow{#1}}

\def\padding{\phantom{X}}
\newcommand{\setpadding}[1]{\def\padding{#1}}

\def\subarrow{}
\newcommand{\setsubarrow}[1]{\def\subarrow{#1}}

\newcommand{\trule}[2]{\dfrac{#1}{#2}}
\newcommand{\crule}[3]{\dfrac{#1}{#2},\;{#3}}
\newcommand{\withenv}[2]{{#1}\vdash{#2}}
\newcommand{\trans}[3]{{#1}\transarrow{\padding{\textstyle #2}\padding}\subarrow{#3}}
\newcommand{\ctrans}[4]{{#1}\transarrow{\padding#2\padding}\subarrow{#3},\;{#4}}
\newcommand{\llang}[1]{\mbox{\lstinline[mathescape]|#1|}}
\newcommand{\pair}[2]{\inbr{{#1}\mid{#2}}}
\newcommand{\inbr}[1]{\left<{#1}\right>}
\newcommand{\highlight}[1]{\color{red}{#1}}
%\newcommand{\ruleno}[1]{\eqno[\scriptsize\textsc{#1}]}
\newcommand{\ruleno}[1]{\mbox{[\textsc{#1}]}}
\newcommand{\rulename}[1]{\textsc{#1}}
\newcommand{\inmath}[1]{\mbox{$#1$}}
%\newcommand{\lfp}[1]{fix_{#1}}
%\newcommand{\gfp}[1]{Fix_{#1}}
\newcommand{\vsep}{\vspace{-2mm}}
\newcommand{\supp}[1]{\scriptsize{#1}}
\renewcommand{\sembr}[1]{\llbracket{#1}\rrbracket}
\newcommand{\cd}[1]{\texttt{#1}}
%\newcommand{\free}[1]{\boxed{#1}}
%\newcommand{\binds}{\;\mapsto\;}
%\newcommand{\dbi}[1]{\mbox{\bf{#1}}}
%\newcommand{\sv}[1]{\mbox{\textbf{#1}}}
%\newcommand{\bnd}[2]{{#1}\mkern-9mu\binds\mkern-9mu{#2}}
%\newcommand{\meta}[1]{{\mathcal{#1}}}
%\newcommand{\dom}[1]{\mathtt{dom}\;{#1}}
%\newcommand{\primi}[2]{\mathbf{#1}\;{#2}}
%\renewcommand{\dom}[1]{\mathcal{D}om\,({#1})}
%\newcommand{\ran}[1]{\mathcal{VR}an\,({#1})}
%\newcommand{\fv}[1]{\mathcal{FV}\,({#1})}
%\newcommand{\tr}[1]{\mathcal{T}r_{#1}}
\newcommand{\diseq}{\not\equiv}
%\newcommand{\reprfunset}{\mathcal{R}}
%\newcommand{\reprfun}{\mathfrak{f}}
%\newcommand{\cstore}{\Omega}
%\newcommand{\cstoreinit}{\cstore_\epsilon^{init}}
%\newcommand{\csadd}[3]{add(#1, #2 \diseq #3)}  %{#1 + [#2 \diseq #3]}
%\newcommand{\csupdate}[2]{update(#1, #2)}  %{#1 \cdot #2}
\newcommand{\primi}[1]{\ensuremath{\mathbf{#1}}}
\newcommand{\sem}[1]{\llbracket #1 \rrbracket}
\newcommand{\ir}{\ensuremath{\mathcal{S}}}
\usepackage{tikz}
\newcommand*\circled[1]{\tikz[baseline=(char.base)]{
    \node[shape=circle,draw,inner sep=1pt] (char) {#1};}}

\let\emptyset\varnothing
\let\eps\varepsilon

% for fancy table
%\newcommand{\lheadl}[2]{\multicolumn{1}{|>{\centering\arraybackslash}m{#1}|}{{#2}}}
%\newcommand{\head}[2]{\multicolumn{1}{>{\centering\arraybackslash}m{#1}}{\textbf{\small #2}}}
%\newcommand{\headll}[2]{\multicolumn{1}{>{\centering\arraybackslash}m{#1}||}{\textbf{\small #2}}}
%\newcommand{\lheadll}[2]{\multicolumn{1}{|>{\centering\arraybackslash}m{#1}||}{\textbf{\small #2}}}
%\newcommand{\headl}[2]{\multicolumn{1}{>{\centering\arraybackslash}m{#1}|}{\textbf{\small #2}}}
%\usepackage{longtable}
%\newcommand{\nodata}{}
%\newcommand{\tablenotemark}[1]{#1}

\newcommand{\contributions}{
\begin{enumerate}
\item[I] Спроектировали синтез с помощью комбинации \emph{реляционных интерпретаторов} на \miniKanren{}
\item[II] Заменили $\forall$ на \emph{конечный} набор примеров
\item[III] Сделали оптимизацию методом ветвей и границ с помощью нового примитива \miniKanren{}: \emph{ограничение на структуру (structural constraint)}
%\item[IV] Extension of OCanren called \emph{structural constraint}
\end{enumerate}
}

\title{Реляционный синтез сопоставления с образцом}
\subtitle{Relational Synthesis for Pattern Matching}

\date{9 ноября 2020}
\author{Косарев Дмитрий} 
\institute[]{\normalfont
Опубликовано на IFCP 2020}


\newcommand{\verbatimfont}[1]{\def\verbatim@font{#1}}
\usepackage{verbatimbox}

\forestset{
    .style={
        for tree={
            base=bottom,
            child anchor=north,
            align=center,
            s sep+=1cm,
    straight edge/.style={
        edge path={\noexpand\path[\forestoption{edge},thick,-{Latex}] 
        (!u.parent anchor) -- (.child anchor);}
    },
    if n children={0}
        {tier=word, draw, thick, rectangle}
        {draw, diamond, thick, aspect=2},
    if n=1{%
        edge path={\noexpand\path[\forestoption{edge},thick,-{Latex}] 
        (!u.parent anchor) -| (.child anchor) node[pos=.2, above] {Y};}
        }{
        edge path={\noexpand\path[\forestoption{edge},thick,-{Latex}] 
        (!u.parent anchor) -| (.child anchor) node[pos=.2, above] {N};}
        }
        }
    }
}
  
\AtBeginSection[]
{
  \begin{frame}<beamer>
    \frametitle{Оглавление}
    \tableofcontents[currentsection,currentsubsection]
  \end{frame}
} 
\AtBeginSubsection[]
{
  \begin{frame}<beamer>
    \frametitle{Оглавление}
    \tableofcontents[currentsection,currentsubsection]
  \end{frame}
}

\begin{document}

\begin{frame}[fragile]
  \begin{tabular}{p{5.5cm} p{5.5cm}}
   \begin{center}
%      \includegraphics[height=1.5cm]{pictures/jetbrainsResearch.pdf}
    \end{center}
    &
    \begin{center}
   %   \includegraphics[height=1.5cm]{pictures/SPbGU_Logo.png}
    \end{center}
  \end{tabular}
  \titlepage
\end{frame}


%\maketitle

% For every picture that defines or uses external nodes, you'll have to
% apply the 'remember picture' style. To avoid some typing, we'll apply
% the style to all pictures.
\tikzstyle{every picture}+=[remember picture] 

% By default all math in TikZ nodes are set in inline mode. Change this to
% displaystyle so that we don't get small fractions.
\everymath{\displaystyle}
%\begin{comment}
% Uncomment these lines for an automatically generated outline.
\begin{frame}{}
111
\vspace{1cm}

\result
\end{frame}

\section{Обзор}

\defverbatim[colored]{\matchA}{
\begin{lstlisting}
match x,y,z with
| _,F,T -> 1
| F,T,_ -> 2
| _,_,F -> 3
| _,_,T -> 4
\end{lstlisting}
}

\defverbatim[colored]{\matchB}{
\begin{lstlisting}
if x then
  if y then
    if z then 4 else 3
  else
    if z then 1 else 3
else
  if y then 2
  else
    if z then 1 else 3
\end{lstlisting}
}

\defverbatim[colored]{\matchC}{
\begin{lstlisting}
if y then
  if x then
    if z then 4 else 3
  else 2
else
  if z then 1 else 3
\end{lstlisting}
}


\defverbatim[colored]{\forestB}{
\begin{forest} 
[\texttt{x}%, tikz={\draw[{Latex}-, thick] (.north) --++ (0,1);}
    [\texttt{y}
          [\texttt{z}
                [4] 
                [3] 
            ]   
         [\texttt{z}
               [1] 
               [3] 
           ]    
    ]   
    [\texttt{y}
        [2] 
        [\texttt{z}
            [1] 
            [3] 
        ]   
    ]   
] 
\end{forest}
}

\defverbatim[colored]{\forestC}{
\begin{forest} 
[\texttt{y}%, tikz={\draw[{Latex}-, thick] (.north) --++ (0,1);}
    [\texttt{x}
          [\texttt{z}
                [4] 
                [3] 
            ]   
         [2]    
    ]
    [\texttt{z}
               [1] 
               [3] 
           ]
]
\end{forest}
}


\begin{frame}[fragile]{Пример: использование диаграмм решений}% (1/2)}

\begin{minipage}[t]{0.25\linewidth}
\begin{minipage}{7cm}
\matchA
\end{minipage}
%\caption{Pattern matching}
\end{minipage}
\hspace{0.5cm}
\begin{minipage}[t]{0.32\linewidth}
\only<1>{\begin{minipage}{7cm}
\matchB
\end{minipage}}
\only<2>{\begin{minipage}{7cm}
\matchC
\end{minipage}}

\end{minipage}
\hspace{0.5cm}
\begin{minipage}[t]{0.3\linewidth}
\only<1>{\begin{minipage}{7cm}
\forestB
\end{minipage}}
\only<2>{\begin{minipage}{7cm}
\forestC
\end{minipage}}
\end{minipage}
%\caption{Pattern matching compilation can be non-trivial (example from~\cite{maranget2008}).}\label{fig:match-example}
\end{frame}

\defverbatim[colored]{\btCompAone}{
\begin{lstlisting}[language=ocaml]


    (switch lx with case []: 1
      default:     )
               
      (switch ly with case []: 2
        default:     )
                  
      (switch lx with
        case (::):
          (switch ly with
            case (::) : 3
            default:     )
        default:     )   
                                
\end{lstlisting}
}

\defverbatim[colored]{\btCompA}{
\begin{lstlisting}[language=ocaml]
catch
  (catch
    (switch lx with case []: 1
      default: exit)
    with (catch
      (switch ly with case []: 2
        default: exit)
    with (catch
      (switch lx with
        case (::):
          (switch ly with
            case (::) : 3
            default: exit)
        default: exit))))
with (failwith "Partial match")
\end{lstlisting}
}








%%%%%%%%%%%%%%%%%%%%%%%%%%%%%%%%%%%%%%%%%%%%%%%%%%%%%%%%%%%%%%%
\section{Заслуги работы}

\section{Превращение монадического парсера в апплкиативный}




\defverbatim[colored]{\haskellOptZero}{
\begin{lstlisting}[language=haskell,mathescape=false]
      (f <$> p) <*> (g <$> q)
=> 
      (\x y -> (f x) (g y)) <$> p <*> q
\end{lstlisting}
}

\defverbatim[colored]{\haskellOptOne}{
\begin{lstlisting}[language=haskell,mathescape=false]
      string "ab"
      = 
      char 'a' <:> char 'b' <:> pure []
=>
      satisfy (== 'a') *> satisfy (== 'b') $> "ab"
\end{lstlisting}
}
\defverbatim[colored]{\haskellOptTwo}{
\begin{minted}{haskell}
string "ab"   ===   char 'a' <:> char 'b' <:> pure []
\end{minted}      
}
\defverbatim[colored]{\haskellOptThree}{
\mintinline{haskell}{satisfy (== 'a') *> satisfy (== 'b') $> "ab" }
}



\begin{frame}{}

Это кусок про optimization and analysis
\begin{center}
\mintinline{haskell}{ (f <$> p) <*> (g <$> q) } 

{\Large $\Downarrow$}

\mintinline{haskell}{(\x y -> (f x) (g y)) <$> p <*> q }

\end{center}

\begin{center}
\begin{minipage}{10cm}

\begin{minipage}{10cm}
\haskellOptTwo
\end{minipage}
 
\begin{center}
{\Large $\Downarrow$}
\end{center}

%\mintinline{haskell}{ satisfy (== 'a') *> satisfy (== 'b') $> "ab" }
\begin{minipage}{10cm}
\haskellOptThree
\end{minipage}
\end{minipage}
\end{center}
\end{frame}


\section{Parsec vs Parsley} 

\begin{frame}{Parsec vs Parsley} 
\begin{tabular}{|c|c|c|c|c|c|}
\hline
 &  &  &  &  &  \\
\hline
 &  &  &  &  &  \\
\hline
\end{tabular} 
\end{frame}

  %  \includepdf[page={1}]{pipeline}
    
\begin{frame}[fragile]{}
\includegraphics[page=2]{pipeline}

\begin{minted}{haskell}
newtype Fix (syn :: (* -> *) -> (* -> *)) (a :: *) where 
  In :: syn (Fix syn) a -> Fix syn a

newtype Parser a = Parser (Fix ParserF a)
data ParserF (k :: ∗ -> ∗) (a :: ∗) where
  Pure :: a -> ParserF k a
  Satisfy :: (Char -> Bool) -> ParserF k Char
  Try :: k a -> ParserF k a
  Look :: k a -> ParserF k a
  NegLook :: k () -> ParserF k ()
  (:<*>:) :: k (a -> b) -> k a -> ParserF k b
  (:*>:) :: k a -> k b -> ParserF k b
  (:<*:) :: k a -> k b -> ParserF k a
  (:<|>:) :: k a -> k a -> ParserF k a
  Empty :: ParserF k a
  Branch :: k (Either x y) -> k (x -> a) -> k (y -> a) -> ParserF k a
\end{minted}
\end{frame}



\defverbatim[colored]{\lawApp}{
\begin{minted}{haskell}
pure id <*> p = p  -- (1)
pure f <*> pure x = pure (f x)  -- (2)
u <*> pure x = pure (λf -> f x) <*> u -- (3)
u <*> (v <*> w) = pure (.) <*> u <*> v <*> w -- (4)
\end{minted}
}

\begin{frame}[fragile]
\includegraphics[page=3]{pipeline}

\begin{minted}{haskell}
string :: String -> Parser String
string = traverse char
\end{minted}


\begin{minted}{haskell}
string "ab"
-- urolling
pure (:) <*> char 'a' <*> (pure (:) <*> char 'b' <*> pure [])
-- Applicative fusion optimizations ....
satisfy (== 'a') *> satisfy (== 'b') *> pure "ab"
\end{minted}
%\mintinline{haskell}{string "ab"}
%
%{\Large $\Downarrow$}
%
%\mintinline{haskell}{pure (:) <*> char 'a' <*> (pure (:) <*> char 'b' <*> pure []).}

\begin{center}
\begin{minipage}{50cm}
Законы аппликативов
\lawApp
\end{minipage}
\end{center}
\end{frame}


\defverbatim[colored]{\lawAlt}{
\begin{minted}{haskell}
(p <|> q) <|> r = p <|> (q <|> r) -- (5)
empty <|> p = p <|> empty = p -- (6)
empty <*> p = empty -- (7)
pure x <|> p = pure x -- (8)
\end{minted}
}

\defverbatim[colored]{\lawSelParser}{
\begin{minted}{haskell}
branch (pure (Left x)) p q = p <*> pure x -- (9)
branch (pure (Right y)) p q = q <*> pure y -- (10)
branch b (pure f) (pure g) = pure (either f g)<*>b -- (11)
branch (x ∗> y) p q = x ∗ŋ branch y p q -- (12)
branch b p empty = branch (pure swap <*> b) empty p -- (13)

branch (branch b empty (pure f)) empty k = branch (pure g <*> b) empty k where
g = either (const (Left ())) (either (const (Left ())) Right · f) -- (14)
\end{minted}
}
%%%%%%%%%%%%%%%%%%%%%%%%%%%%%%%%%%%%%%%%%%%%%%%%%%%%%%%%%%%%%%%%%%%%%
\defverbatim[colored]{\haskellLetZero}{
\begin{lstlisting}[language=haskell]
many :: Parser a -> Parser a 
many p = (p <:> many p) <|> (pure p)
\end{lstlisting}
}

\defverbatim[colored]{\haskellLetOne}{
\begin{lstlisting}[language=haskell]
many :: Parser a -> Parser a 
many p = 
  let go = (p <:> many p) <|> (pure p) 
  go 
\end{lstlisting}
}

\begin{frame}{Вставка let}
\includegraphics[page=3]{pipeline}


\haskellLetZero
\haskellLetOne
\end{frame}


\begin{frame}{Различные анализы}
\includegraphics[page=3]{pipeline}
\begin{itemize}
\item Cut 
\begin{itemize}
\item TODO
\end{itemize}
\item Termination
\begin{itemize}
\item Анализ точный для КС-грамматик и наивных аппликатиных парсеров
\item В остальных случаях анализ неточный и ложные срабатывания игнорируются
\end{itemize}
\end{itemize}
\end{frame}




\begin{frame}[fragile]{}
Сказать про CPS, и так как у нас нет монады и КЗ, то обходимся стековой машиной

\begin{minted}{haskell}
data M (k :: [*] -> * -> * -> *) (xs :: [*]) (r :: *) (a :: *) where
  Halt :: M k [a] Void a
  Push :: x -> k (x : xs) r a -> M k xs r a
  Pop  :: k xs r a -> M k (x : xs) r a
  ...
\end{minted}

\begin{itemize}
\item \mintinline{haskell}{k} is the same as in the combinator tree, it represents the shape of
the values contained within each node (often Fix M); 
\item \mintinline{haskell}{xs} is a type-level list representing the types of
the values required on the stack upon entry to the given instruction
\item \mintinline{haskell}{r} represents what type the
machine returns to the caller in the case that this is a recursive call
\item \mintinline{haskell}{a} the final "goal" of
the machine, in other words it directly corresponds to the type of the top-level parser that was
compiled to generate this machine.
\end{itemize}
\end{frame}


\begin{frame}[fragile]{Компиляция (1/n)}
\begin{minted}{haskell}
compile :: Fix ParserF a -> Fix M [ ] Void a
compile = cata compAlg halt
type CodeGen a x = forall xs r . Fix M (x : xs) r a -> Fix M xs r a
compAlg :: ParserF (CodeGen a) x -> Fix M (x : xs) r a -> Fix M xs r a
compAlg (Pure x) = push x
compAlg (p :*>: q) = p . pop . q
compAlg (p :<*: q) = p . q . pop
\end{minted}
\end{frame}

\begin{frame}[fragile]{Компиляция (2/n)}
\begin{minted}{haskell}
data M (k :: [*] -> * -> * -> *) (xs :: [*]) (r :: *) (a :: *) where
  ...
  Lift2 :: (x -> y -> z) -> k (z : xs) r a -> M k (y : x : xs) r a
  Swap :: k (x : y : xs) r a -> M k (y : x : xs) r a
  
app = lift2 id
compAlg (pf :<*>: px) = pf . px . app

\end{minted}
\end{frame}

\begin{frame}[fragile]{Компиляция (1/n). Selectives }
\begin{minted}{haskell}
data M (k :: [*] -> * -> * -> *) (xs :: [*]) (r :: *) (a :: *) where
  ...
  Case :: k (x : xs) r a -> k (y : xs) r a -> M k (Either x y : xs) r a
  
compAlg (Branch b l r) = λk -> b (case (l (swap (app k))) (r (swap (app k))))

\end{minted}
\end{frame}

\begin{frame}[fragile]{Компиляция (1/n). Alternatives}
TODO: рассказать про две реализации alternative выше
TODO: сказать про второй стек
\begin{minted}{haskell}
data M (k :: [*] -> * -> * -> *) (xs :: [*]) (r :: *) (a :: *) where
  ...
  Fail :: M k xs r a
  Catch :: k xs r a -> k (String : xs) r a -> M k xs r a
  Commit :: k xs r a -> M k xs r a
  
handle :: (Fix M xs r a -> Fix M (x : xs) r a) -> Fix M (String : xs) r a
  -> Fix M (x : xs) r a -> Fix M xs r a
handle p h k = catch (p (commit k)) h
compAlg (p :<|>: q) = λk -> handle p (parsecHandle (q k)) k
compAlg Empty = const fail
\end{minted}
\end{frame}

\begin{frame}[fragile]{Компиляция (1/n). Primitive instructions}

\begin{minted}{haskell}
data M k (xs :: [*]) r a where
  ...
  Sat :: (Char -> Bool) -> k (Char : xs) r a -> M k xs r a
  Tell :: k (String : xs) r a -> M k xs r a
  Seek :: k xs r a -> M (String : xs) r a
  Ret :: M k [r] r a
  Call :: MuVar x -> k (x : xs) r a -> M k xs r a

compAlg (Satisfy p) = sat p
compAlg (Try p) = handle p (seek fail)
compAlg (Look p) = tell . p . swap . seek
compAlg (Let μ) = call μ

\end{minted}
\end{frame}

\begin{frame}[fragile]{}
\includegraphics[page=5]{pipeline}
\begin{itemize}
\item Join points ($\varphi$-узлы) 
\begin{minted}{haskell}
compAlg (p :<|>: q) = λk -> handle p (parsecHandle (q k)) k
\end{minted}
Решается с помощью специальной инструкции машины \mintinline{haskell}{Join}, которая вставляется после 
\mintinline{haskell}{Case} и \mintinline{haskell}{Catch}
\item TCO
\begin{itemize}
\item Достигается введением инструкций \mintinline{haskell}{Jump} и \mintinline{haskell}{Call}
\end{itemize}
\item Deep inspection (via histomoprhism)
\end{itemize}

%\begin{itemize}
%\item Cut 
%\begin{itemize}
%\item TODO
%\end{itemize}
%\item Termination
%\begin{itemize}
%\item Анализ точный для КС-грамматик и наивных аппликатиных парсеров
%\item В остальных случаях анализ неточный и ложные срабатывания игнорируются
%\end{itemize}
%\end{itemize}
\end{frame}

\begin{frame}{Consumption analysis revisited}

\end{frame}


\begin{frame}[fragile]{Staging}
\begin{figure}[ht]
\begin{subfigure}[t]{.59\textwidth}
Обычная функция возведения в степень
\begin{minted}{haskell}
power :: Nat -> (Int -> Int)
power 0 = λx -> 1
power n = λx -> x * power (n - 1) x
\end{minted}
Staged функция, где степень известна \emph{статически}, а основание -- динамически
\begin{minted}{haskell}
power_ :: Nat -> Code (Int -> Int)
power_ 0 = [| λx -> 1 |]
power_ n = [| λx -> x * $(power_ (n-1)) |]
\end{minted}
\end{subfigure}
\hspace{.1\textwidth}
\begin{subfigure}[t]{.29\textwidth}
Quoting vs. splicing 
\vspace{1em}

Если \verb=x :: a= то \verb=[|x|] :: Code a= 
\vspace{1em}

и если  \verb=qx :: Code a=, то \verb=$(qx) :: a=.

\end{subfigure}
\end{figure}
\begin{center}
\begin{minipage}{10cm}
\begin{minted}{haskell}
power5 = $(power_ 5)
       = $([| λx -> x * x * x * x * x * 1 |])
       = λx -> x * x * x * x * x * 1
\end{minted}
\end{minipage}
\end{center}
\end{frame}


\begin{frame}[fragile]{Staging}



\begin{minted}[escapeinside=!!]{haskell}
type Eval' xs r a = Code !\staged{(Γ xs r a -> Maybe a)}!
\end{minted}

\end{frame}


\begin{frame}{Заслуги работы}
\Large
%\contributions
\end{frame}



\section{Заключение}

\begin{frame}{Результаты}
Достижения: \contributions
\vspace{1em}

\end{frame}


\begin{frame}{Пути дальнейшего улучшения}
\begin{itemize}
\item 1
\end{itemize}
\end{frame}

\begin{frame}
\begin{center}
{\Huge Спасибо!}
\end{center}
\end{frame}


\begin{frame}%[t, allowframebreaks]
\frametitle{Литература}
\bibliographystyle{amsalpha}
\bibliography{references}
\vspace{1cm}
\end{frame}

\end{document}
