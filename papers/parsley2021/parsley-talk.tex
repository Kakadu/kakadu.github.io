% for images on title page. Should go before \documentclass
%\def\pgfsysdriver{pgfsys-dvipdfm.def}
% required for [rememberpicture]
% FIXME: remove?

\documentclass[aspectratio=169
  , xcolor={svgnames}
  , hyperref={ colorlinks,citecolor=Blue
             , linkcolor=DarkRed,urlcolor=DarkBlue}
  , usenames, dvipsnames
  , russian
  ]{beamer}


\usepackage{pgfpages}
\usepackage{graphicx}   % for \includegraphics{file.pdf}
%\def\theFancyVerbLine{%
%  \rmfamily\tiny\arabic{FancyVerbLine}%
%  {\tikz[remember picture,overlay]\node(minted-\arabic{FancyVerbLine}){};}%
%}

%\usepackage{bibentry}
%\usepackage{cite}
%\def\newblock{\hskip .11em plus .33em minus .07em}

%\usepackage{framed}
\makeatletter
\@ifclassloaded{beamer}{
  % get rid of header navigation bar
  \setbeamertemplate{headline}{}
  % get rid of bottom navigation symbols
  \setbeamertemplate{navigation symbols}{}
  % get rid of footer
  %\setbeamertemplate{footline}{}
}
{}
\makeatother
%%%%%%%%%%%%%%%%%%%%%%%%%%%%%%%%%%%%%%%%%%%%%
\usepackage{fontawesome}
% \newfontfamily{\FA}{Font Awesome 5 Free} % some glyphs missing
\expandafter\def\csname faicon@facebook\endcsname{{\FA\symbol{"F09A}}}
\def\faQuestionSign{{\FA\symbol{"F059}}}
\def\faQuestion{{\FA\symbol{"F128}}}
\def\faExclamation{{\FA\symbol{"F12A}}}
\def\faUploadAlt{{\FA\symbol{"F093}}}
\def\faLemon{{\FA\symbol{"F094}}}
\def\faPhone{{\FA\symbol{"F095}}}
\def\faCheckEmpty{{\FA\symbol{"F096}}}
\def\faBookmarkEmpty{{\FA\symbol{"F097}}}

\newcommand{\faGood}{\textcolor{ForestGreen}{\faThumbsUp}}
\newcommand{\faBad}{\textcolor{red}{\faThumbsODown}}
\newcommand{\faWrong}{\textcolor{red}{\faTimes}}
\newcommand{\faMaybe}{\textcolor{blue}{\faQuestion}}
\newcommand{\faCheckGreen}{\textcolor{ForestGreen}{\faCheck}}
%%%%%%%%%%%%%%%%%%%%%%%%%%%%%%%%%%%%%%%%%%%%%

\usepackage{fontspec}
\usepackage{xunicode}
\usepackage{xltxtra}
\usepackage{xecyr}
\usepackage{hyperref}

\setmainfont[
 Ligatures=TeX,
 Extension=.otf,
 BoldFont=cmunbx,
 ItalicFont=cmunti,
 BoldItalicFont=cmunbi,
% Scale = 1.1
]{cmunrm}
\setsansfont[
 Ligatures=TeX,
 Extension=.otf,
 BoldFont=cmunsx,
 ItalicFont=cmunsi,
%  Scale = 1.2
]{cmunss}
%\setmainfont[Mapping=tex-text]{DejaVu Serif}
%\setsansfont[Mapping=tex-text]{DejaVu Sans}
%\setmonofont{Fira Code}[Contextuals=AlternateOff]
\setmonofont{Fira Code}[Contextuals=Alternate,Scale=0.9]
\newfontfamily{\myfiracode}[Scale=1.5,Contextuals=Alternate]{Fira Code}
%\setmonofont[Scale=0.9,BoldFont={Inconsolata Bold}]{Inconsolata}

\usepackage{polyglossia}
\setmainlanguage{russian}
\setotherlanguage{english}


%\newfontfamily\dejaVuSansMono{DejaVu Sans Mono}
% https://github.com/vjpr/monaco-bold/raw/master/MonacoB/MonacoB.otf
%\newfontfamily\monacoB{MonacoB}
%%%%%%%%%%%%%%%%%%%%%%%%%%%%%%%%%%%%%%%%%%%%%%%5
\usepackage{soul} % for \st that strikes through
\usepackage[normalem]{ulem} % \sout

\usepackage{stmaryrd}
\newcommand{\sem}[1]{\ensuremath{\llbracket #1\rrbracket}}


\usepackage{listings}
%\lstdefinestyle{style1}{
%  language=haskell,
%  numbers=left,
%  stepnumber=1,
%  numbersep=10pt,
%  tabsize=4,
%  showspaces=false,
%  showstringspaces=false
%}
%\lstdefinestyle{hsstyle1}
%{ language=haskell
%%          , basicstyle=\monacoB
%         , deletekeywords={Int,Float,String,List,Void}
%         , breaklines=true
%         , columns=fullflexible
%         , commentstyle=\color{ForestGreen}
%         , escapeinside=§§
%         , escapebegin=\begin{russian}\commentfont
%         , escapeend=\end{russian}
%         , commentstyle=\color{ForestGreen}
%         , escapeinside=§§
%         , escapebegin=\begin{russian}\color{ForestGreen}
%         , escapeend=\end{russian}
%         , mathescape=true
%%          , backgroundcolor = \color{MyBackground}
%}
%
%\newcommand{\inline}[1]{\lstinline{haskell}{#1}}
%\def\hsinline{\mintinline{haskell}}
%\def\inline{\hsinline}
%
%\lstnewenvironment{hslisting} {
%    \lstset { style={hsstyle1} }
%  }
%  {}
%  
%%%%%%%%%%%%%%%%%%%%%%%%%%%%%%%%%%%%%%%%%%%%%%%%%%%%%%%%%%%  
%%\setmainfont[
%% Ligatures=TeX,
%% Extension=.otf,
%% BoldFont=cmunbx,
%% ItalicFont=cmunti,
%% BoldItalicFont=cmunbi,
%%]{cmunrm}
%%% С засечками (для заголовков)
%%\setsansfont[
%% Ligatures=TeX,
%% Extension=.otf,
%% BoldFont=cmunsx,
%% ItalicFont=cmunsi,
%%]{cmunss}
%% \setmonofont[Scale=0.6]{Monaco}
%
%\usefonttheme{professionalfonts}
%\usepackage{times}
\usepackage{tikz}
\usetikzlibrary{cd}
\usepackage{tikz-cd}
\usepackage{caption}
\usepackage{subcaption}

%\renewtheorem{definition}{برهان}[chapter]
%%\DeclareMathOperator{->}{\rightarrow}
%\newcommand\iso{\ensuremath{\cong}}
%\usepackage{verbatim}
%\usepackage{graphicx}
%\usetikzlibrary{arrows,shapes}

%\usepackage{amsmath}
%\usepackage{amsfonts}
\usepackage{scalerel}
\DeclareMathOperator*{\myvee}{\scalerel*{\vee}{\sum}}
\DeclareMathOperator*{\mywedge}{\scalerel*{\wedge}{\sum}}

%
%\usepackage{tabulary}
%
%% sudo aptget install ttf-mscorefonts-installer
%%\setmainfont{Times New Roman}
%%\setsansfont[Mapping=tex-text]{DejaVu Sans}
%
%%\setmonofont[Scale=1.0,
%%    BoldFont=lmmonolt10-bold.otf,
%%    ItalicFont=lmmono10-italic.otf,
%%    BoldItalicFont=lmmonoproplt10-boldoblique.otf
%%]{lmmono9-regular.otf}
%
\usepackage[cache=true]{minted}
\usemintedstyle{perldoc}

\def\hsinline{\mintinline{haskell}}
\def\mlinline{\mintinline{ocaml}}
% color options
\definecolor{YellowGreen} {HTML}{B5C28C}
\definecolor{ForestGreen} {HTML}{009B55}
\definecolor{MyBackground}{HTML}{F0EDAA}


\author{Косарев Дмитрий a.k.a. Kakadu}
\institute{матмех СПбГУ}

\addtobeamertemplate{title page}{}{
  \begin{center}{\tiny Дата сборки: \today}\end{center}
}


%\newcommand{\lstquot}[1]{``\lstinline{#1}''}
%\newcommand{\sembr}[1]{\llbracket{#1}\rrbracket}
%\newcommand\false{$f\!alse$}
%\newcommand\myif{i\!f}



%\newcommand{\pair}[2]{\inbr{{#1}\mid{#2}}}
%\newcommand{\inbr}[1]{\left<{#1}\right>}
%\newcommand{\highlight}[1]{\color{red}{#1}}

%\newcommand{\vsep}{\vspace{-2mm}}
%\newcommand{\supp}[1]{\scriptsize{#1}}
%\renewcommand{\sembr}[1]{\llbracket{#1}\rrbracket}
%\newcommand{\cd}[1]{\texttt{#1}}

%\newcommand{\diseq}{\not\equiv}
%\newcommand{\reprfunset}{\mathcal{R}}
%\newcommand{\reprfun}{\mathfrak{f}}
%\newcommand{\cstore}{\Omega}
%\newcommand{\cstoreinit}{\cstore_\epsilon^{init}}
%\newcommand{\csadd}[3]{add(#1, #2 \diseq #3)}  %{#1 + [#2 \diseq #3]}
%\newcommand{\csupdate}[2]{update(#1, #2)}  %{#1 \cdot #2}
%\newcommand{\primi}[1]{\ensuremath{\mathbf{#1}}}
%\newcommand{\sem}[1]{\llbracket #1 \rrbracket}
%\newcommand{\ir}{\ensuremath{\mathcal{S}}}
%\usepackage{tikz}
%\newcommand*\circled[1]{\tikz[baseline=(char.base)]{
%    \node[shape=circle,draw,inner sep=1pt] (char) {#1};}}


% for fancy table
%\newcommand{\lheadl}[2]{\multicolumn{1}{|>{\centering\arraybackslash}m{#1}|}{{#2}}}

\newcommand{\head}[2]{\multicolumn{1}{>{\centering\arraybackslash}m{#1}}{\textbf{\small #2}}}

%\newcommand{\headll}[2]{\multicolumn{1}{>{\centering\arraybackslash}m{#1}||}{\textbf{\small #2}}}
%\newcommand{\lheadll}[2]{\multicolumn{1}{|>{\centering\arraybackslash}m{#1}||}{\textbf{\small #2}}}
%\newcommand{\headl}[2]{\multicolumn{1}{>{\centering\arraybackslash}m{#1}|}{\textbf{\small #2}}}
%\usepackage{longtable}
%\newcommand{\nodata}{}
%\newcommand{\tablenotemark}[1]{#1}

\title{Staged Selective парсер-комбинаторы}
\subtitle{Staged Selective Parser Combinators}

\date{1 марта 2021}
\author{Косарев Дмитрий} 
\institute[]{\normalfont
По статье <<Staged Selective Parser Combinators>> \\c конференции
IFCP 2020}


\AtBeginSection[]
{
  \begin{frame}<beamer>
    \frametitle{Оглавление}
    \tableofcontents[currentsection,currentsubsection]
  \end{frame}
} 
\AtBeginSubsection[]
{
  \begin{frame}<beamer>
    \frametitle{Оглавление}
    \tableofcontents[ currentsection
                    , currentsubsection
                    ]
  \end{frame}
}
\begin{document}

% Title page 
\begin{frame}
   \tikz [overlay] {
    \node at
        ([yshift=-10cm,xshift=-2cm]current page.east) 
        {\includegraphics[height=2cm]{pictures/SPbGU_Logo.png}};
    \node at
        ([yshift=-10cm,xshift=2cm]current page.west) 
        {\includegraphics[height=1.5cm]{pictures/jetbrainsResearch.pdf}};
   }
   \titlepage
\end{frame}



% For every picture that defines or uses external nodes, you'll have to
% apply the 'remember picture' style. To avoid some typing, we'll apply
% the style to all pictures.
%\tikzstyle{every picture}+=[remember picture] 

% By default all math in TikZ nodes are set in inline mode. Change this to
% displaystyle so that we don't get small fractions.
\everymath{\displaystyle}

\section{Парсер-комбинаторы (кратко)}

%\begin{frame}[fragile]{Парсер-комбинаторы (кратко)}
%\begin{minted}{haskell}
%nonzero :: Parser Char 
%nonzero :: oneOf ['1'..'9']
%digit :: Parser Char 
%digit = char '0' <|> nonzero 
%natural :: Parser Int 
%natural :: read <$> (nonzero <:> many digit)
%\end{minted}
%
%\begin{minted}{haskell}
%ident :: Parser String
%ident = satisfy isAlpha 
%  <:> many (satisfy alphaNum)
%\end{minted}
%\end{frame}


\defverbatim[colored]{\haskellPrimA}{
\begin{minted}{haskell}
item :: Parser Char 
item = satisfy (const True)
char :: Char -> Parser Char 
char c = satisfy (==c)
\end{minted}
}
\defverbatim[colored]{\haskellPrimB}{
\begin{minted}{haskell}
eof :: Parser ()
eof = notFollowedBy item
\end{minted}
}

\begin{frame}[fragile]{Парсер-комбинаторы. Примитивные операции}
\begin{minted}{haskell}
satisfy :: (Char -> Bool) -> Parser Char 
\end{minted}
\uncover<2->{
\haskellPrimA
}
\begin{minted}{haskell}
try :: Parser a -> Parser a 
lookAhead :: Parser a -> Parser a 
notFollowedBy :: Parser a -> Parser ()
\end{minted}
\uncover<2->{
\haskellPrimB
}
\end{frame}

\begin{frame}[fragile]
\begin{minipage}{0.65\linewidth}
\begin{minted}{haskell}
class Functor (f :: * -> *) where
  fmap :: (a -> b) -> f a -> f b

-- application abstracted 
class Functor f => Applicative f where
  -- | Lift a value.
  pure :: a -> f a
  -- | Sequential application.
  (<*>) :: f (a -> b) -> f a -> f b

-- Nondeterminism (or try-catch) abstracted
class Applicative f => Alternative f where
  -- | The identity of '<|>'
  empty :: f a
  -- | An associative binary operation
  (<|>) :: f a -> f a -> f a
\end{minted}
\end{minipage}\hspace{1cm}
\begin{minipage}{0.2\linewidth}
Монад пока нет, это нарочно
\end{minipage}
\end{frame}


\begin{frame}[fragile]
\begin{block}{\emph{Идея парсер-комбинаторов}}
Использовать обычные функции, чтобы строить большие парсеры
\end{block}
\vspace{1em}

%\uncover<2->{
\begin{minted}{haskell}
char :: Char -> Parser Char 
\end{minted}
%}
\newln
\begin{minted}{haskell}
sequence :: Applicative f => [f a] -> f [a]
sequence = foldr (<:>) (pure [])

traverse :: Applicative f => (a -> f b) -> [a] -> f [b]
traverse f = sequence . map f  
\end{minted}
\newln

%\uncover<2->{
\begin{minted}{haskell}
string :: String -> Parser String 
string = traverse char 
\end{minted}
%}
\newln
\begin{minted}{haskell}
string :: String -> Parser String 
oneOf = foldr (<|>) empty . map char 
\end{minted}
%TODO: ещё слайд про предыдущеге из доклада?
\end{frame}

%%%%%%%%%%%%%%%%%%%%%%%%%%%%%%%%%%%%%%%%%%%%%%%%%%%%%%%%%%%%%%%
\section{Заслуги работы (библиотеки Parsley)}


\begin{frame}{Parsec vs. Parsley} 
\def\sad{\textcolor{red}{\Huge\faSadTear}}
\def\yolo{\textcolor{ForestGreen}{\Huge\faGrinStars}}

\Large 
\begin{center}
\renewcommand{\arraystretch}{2}
\begin{tabular}{c|cp{2cm}cccc}
 & Performance & Error messages & Grammar & Debugging & Analysis  \\
\hline
Parsec & \Huge \faMeh & \head{2cm}{\Huge \faMeh} & \yolo & \sad & \sad \\
\pause
Parsley &\Huge  \faGood & \head{2cm}{\Huge 404} & \yolo & \yolo & \Huge  \faGood \\
\end{tabular}
\renewcommand{\arraystretch}{1}
\end{center}
\pause
\newln 
Как этого удалось добиться? \emph{Метапрограммирование!} (Typed Template \Haskell{})
\end{frame}

%\begin{frame}[fragile]{11}
%\includegraphics[page=0]{pipeline}
%
%TODO: слайд с примером
%\end{frame}


\section{Monads  и Selectives}
\input{02selectives.tex}



\section{Превращение монадического парсера в аппликативный}
\input{03conversion}

\section{Законы}


\defverbatim[colored]{\lawApp}{
\begin{minted}{haskell}
    pure id <*> p = p                          -- (1)
pure f <*> pure x = pure (f x)                 -- (2)
     u <*> pure x = pure (λf -> f x) <*> u     -- (3)
  u <*> (v <*> w) = pure (.) <*> u <*> v <*> w -- (4)
\end{minted}
}

\defverbatim[colored]{\lawAlt}{
\begin{minted}{haskell}
(p <|> q) <|> r = p <|> (q <|> r) -- (5)
    empty <|> p = p <|> empty = p -- (6)
    empty <*> p = empty           -- (7)
   pure x <|> p = pure x          -- (8)
\end{minted}
}

\defverbatim[colored]{\lawSelParser}{
\begin{minted}{haskell}
branch (pure (Left  x)) p q = p <*> pure x                     -- (9)
branch (pure (Right y)) p q = q <*> pure y                     -- (10)
branch b (pure f) (pure g)  = pure (either f g) <*> b          -- (11)
        branch (x *> y) p q = x *> branch y p q                -- (12)
           branch b p empty = branch (pure swap <*> b) empty p -- (13)

branch (branch b empty (pure f)) empty k = branch (pure g <*> b) empty k 
  where
    g = either (const (Left ())) (either (const (Left ())) Right. f) -- (14)
\end{minted}
}

\begin{frame}[fragile]{Законы Applicative}
\setminted{fontsize=\large,baselinestretch=1}
\begin{minted}{haskell}
-- application abstracted 
class Functor f => Applicative f where
    -- | Lift a value.
    pure :: a -> f a
    -- | Sequential application.
    (<*>) :: f (a -> b) -> f a -> f b



    pure id <*> p = p                          -- (1)
pure f <*> pure x = pure (f x)                 -- (2)
     u <*> pure x = pure (λf -> f x) <*> u     -- (3)
  u <*> (v <*> w) = pure (.) <*> u <*> v <*> w -- (4)
\end{minted}
\end{frame}



\begin{frame}[fragile]{Законы Selective}
\setminted{fontsize=\large,baselinestretch=1}
\begin{minted}{haskell}
-- case expression abstracted
class Applicative f => Selective f where
  branch :: f (Either a b) -> f (a -> c) -> f (b -> c) -> f c
\end{minted}
%\begin{lstlisting}[language=haskell]
\newln 

\begin{minted}{haskell}
branch (pure (Left  x)) p q = p <*> pure x                     -- (9)
branch (pure (Right y)) p q = q <*> pure y                     -- (10)
branch b (pure f) (pure g)  = pure (either f g) <*> b          -- (11)
        branch (x *> y) p q = x *> branch y p q                -- (12)
           branch b p empty = branch (pure swap <*> b) empty p -- (13)
           
branch (branch b empty (pure f)) empty k = 
      branch (pure g <*> b) empty k  -- (14)
  where
    g = either (const (Left ())) (either (const (Left ())) Right. f) 
\end{minted}
%\end{lstlisting}

\end{frame}


\begin{frame}[fragile]{Законы Alternative}
\setminted{fontsize=\large,baselinestretch=1}
\begin{minted}{haskell}
-- Nondeterminism (or try-catch) abstracted
class Applicative f => Alternative f where
  -- | The identity of '<|>'
  empty :: f a
  -- | An associative binary operation
  (<|>) :: f a -> f a -> f a



(p <|> q) <|> r = p <|> (q <|> r) -- (5)
    empty <|> p = p <|> empty = p -- (6)
    empty <*> p = empty           -- (7)
   pure x <|> p = pure x          -- (8)
\end{minted}
\end{frame}



\begin{frame}[fragile]{Законы парсеров}
\setminted{fontsize=\large,baselinestretch=1}
\begin{minted}{haskell}
try (satisfy f) = satisfy f -- (15)
try (negLook p) = negLook p -- (16)

look empty       = empty    -- (17)
look (pure x)    = pure x   -- (18)
negLook empty    = pure ()  -- (19)
negLook (pure x) = empty    -- (20)

look (look p)       = look p                       -- (21)
look p <|> look q   = look (try p <|> q)           -- (22)
negLook (negLook p) = look p                       -- (23)
look (negLook p)    = negLook (look p) = negLook p -- (24)
-- de Morgan law where *> is conjunction
negLook (try p <|> q)   = negLook p *> negLook q     -- (25)
negLook p <|> negLook q = negLook (look p *> look q) -- (26)
\end{minted}
\end{frame}

\section{Реализация}
\input{05impl}

%
\begin{frame}[fragile]{Замеры производительности}
\includegraphics[scale=.6]{pictures/benchmark1Screen.png}
\end{frame}

\begin{frame}[fragile]
\begin{center}
\includegraphics[scale=.5]{pictures/benchmark2Screen.png}
\end{center}
\end{frame}


\section{Заключение}

\begin{frame}{Заключение}
\Large
Достижения:
\begin{itemize}
\item Оптимизированная библиотека персер-комбинаторов с отличной производительностью
\end{itemize}

Задачи на будущее:
\begin{itemize}
\item Обработка ошибок
\item Обход недостатков из-за отсутствия монад:
\begin{itemize}\large
\item Доступ к предыдущим результатам парсинга (например, через "регистры"~\cite{berlin2020})
\end{itemize}
\end{itemize}

\end{frame}

%\begin{frame}{Пути дальнейшего улучшения}
%\begin{itemize}
%\item 1
%\end{itemize}
%\cite{demos}
%\end{frame}

%\begin{frame}
%\begin{center}
%{\Huge Спасибо!}
%\end{center}
%\end{frame}


%\begin{frame}%[t, allowframebreaks]
%\frametitle{Литература}
%\bibliographystyle{amsalpha}
%\bibliography{references}
%\vspace{1cm}
%\end{frame}

\begin{frame}[allowframebreaks]
\frametitle<presentation>{Ссылки}
\begin{thebibliography}{10}

  \bibitem{icfp2020}
    Staged Selective Parser Combinators
    \newblock {\em Jamie Willis \& Nicolas Wu \& Matthew Pickering}
    \newblock \url{https://doi.org/10.1145/3409002}

  \bibitem{berlin2020}
    Garnishing Parsec With Parsley: A Staged Selective Parser Combinator Library
    \newblock {\em Jamie Willis}
    \newblock \url{https://www.youtube.com/watch?v=tJcyY9L2z84}


  \bibitem{selective}
     Selective Applicative Functors
    \newblock {\em Andrey Mokhov \& Georgy Lukyanov \& Simon Marlow \& Jeremie Dimino}
    \newblock \url{https://doi.org/10.1145/3341694}
 
  \bibitem{cuts}
    Библиотека FastParse для Scala
    \newblock \href{https://webcache.googleusercontent.com/search?q=cache:WSoAEDqEOakJ:https://www.lihaoyi.com/fastparse/}{Documentation}
   
  \bibitem{trylookahead}
    Try vs. lookahead
    \newblock \url{https://stackoverflow.com/questions/20020350}
    
%  \bibitem{gerasimov}
%     Курс математической логики и теории вычислимости
%     \newblock {\em Герасимов А.С.}     
%     \newblock \href{https://www.mccme.ru/free-books/gerasimov-3ed-mccme.pdf}{PDF}
%
%  \bibitem{}
%    A Tutorial Introduction to the Lambda Calculus
%    \newblock {\em Ra\'ul Rojas}     
%    \newblock \href{https://www.inf.fu-berlin.de/lehre/WS03/alpi/lambda.pdf}{PDF}
%
%  \bibitem{sicp}
%    Structure and Interpretation of Computer Programs
%    \newblock {\em Abelson, Harold and Sussman, Gerald Jay and {with~Julie~Sussman}}     
%    \newblock \href{https://web.mit.edu/alexmv/6.037/sicp.pdf}{PDF}
%
%  \bibitem{olegSKI}
%    λ to SKI, Semantically (Declarative Pearl)
%    \newblock {\em Oleg Kiselyov}
%    \newblock \href{http://okmij.org/ftp/tagless-final/ski.pdf}{PDF}
    
\end{thebibliography}
 \end{frame}



\end{document}
