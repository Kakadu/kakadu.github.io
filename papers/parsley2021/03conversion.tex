\subsection{chainr1}

\setminted{fontsize=\Large,baselinestretch=1}
\defverbatim[colored]{\haskellChainrA}{
\begin{minted}{haskell}
chainr1 :: Parser a -> Parser (a -> a -> a) -> Parser a
chainr1 p op = scan
  where
    scan     = do { x <- p; rest x }
    rest x   = do { f <- op
                  ; y <- scan
                  ; return (f x y)
                  }
              <|> return x
\end{minted}
}


\defverbatim[colored]{\haskellChainrB}{
\begin{minted}{haskell}
chainr1 p op = p >>= rest
  where
  
    rest x = (op >>= \f ->
              chainr1 p op >>= \y ->
              pure (f x y))
            <|> pure x
\end{minted}
}

\defverbatim[colored]{\haskellChainrC}{
\begin{minted}[escapeinside=!!]{haskell}
chainr1 :: Parser a -> Parser (a -> a -> a) -> Parser a
chainr1 p op = p <**> rest
  where
    rest :: Parser (a -> a)
    -- N.B. arguments must go away   
    rest !\textcolor{red}{x}! = (op >>= \f ->
              chainr1 p op >>= \y ->
              pure (!\textcolor{red}{f x y}!))
            <|> pure x
          
-- (<**>) :: Applicative f => f a -> f (a -> b) -> f b
\end{minted}
}

\defverbatim[colored]{\haskellChainrD}{
\begin{minted}{haskell}
chainr1 :: Parser a -> Parser (a -> a -> a) -> Parser a
chainr1 p op = p <**> rest
  where

    rest   = (op >>= \f ->
              chainr1 p op >>= \y ->
              pure (flip f y))
            <|> pure id
          
-- Уже компилируется, но надо избавиться от двух >>=
-- и это очень просто
\end{minted}
}

\defverbatim[colored]{\haskellChainrE}{
\begin{minted}{haskell}
chainr1 :: Parser a -> Parser (a -> a -> a) -> Parser a
chainr1 p op = p <**> rest
  where

    rest   = (flip <$> op <*> chainr1 p op)
            <|> pure id
            
-- C chainr1 всё

-- С chainl1 будет посложнее            
\end{minted}
}

%%% Now chainl 


\defverbatim[colored]{\haskellChainlA}{
\begin{minted}{haskell}
chainl1 :: Parser a -> Parser (a -> a -> a) -> Parser a
chainl1 p op = do { x <- p; rest x }
  where
    rest x   = do { f <- op
                  ; y <- p
                  ; rest (f x y)
                  }
                <|> return x
\end{minted}
}

\defverbatim[colored]{\haskellChainlB}{
\begin{minted}{haskell}
chainl1 :: Parser a -> Parser (a -> a -> a) -> Parser a
chainl1 p op = p >>= rest
  where
  
    rest x  = op >>= \f ->
              p  >>= \y ->
              rest (f x y)
              <|> return x
\end{minted}
}

\defverbatim[colored]{\haskellChainlC}{
\begin{minted}[escapeinside=!!]{haskell}
chainl1 :: Parser a -> Parser (a -> a -> a) -> Parser a
chainl1 p op = p !\textcolor{ForestGreen}{<**>}! rest
  where
    -- N.B. Аргумент нужно убирать
    rest !\textcolor{red}{x}!  = op >>= \f ->
              p  >>= \y ->
              rest (!\textcolor{red}{f x y}!)
              <|> return !\textcolor{ForestGreen}{id}!
              
-- рекурсивный вызов мешает
-- Интуиция: реализуем foldl через foldr
\end{minted}
}
\defverbatim[colored]{\haskellChainlD}{
\begin{minted}{haskell}
chainl1 :: Parser a -> Parser (a -> a -> a) -> Parser a
chainl1 p op = p <**> rest
  where
    
    rest   = (op >>= \f ->
              p  >>= \y ->
              (\ g x -> g (f x y)) <$> rest)
              <|> pure id
          
-- Теперь превратим (\ g x -> g (f x y)) в парсер, 
-- чтобы вытолкнуть rest наружу
\end{minted}
}

\defverbatim[colored]{\haskellChainlE}{
\begin{minted}{haskell}
chainl1 :: Parser a -> Parser (a -> a -> a) -> Parser a
chainl1 p op = p <**> rest
  where

    rest   = (op >>= \f ->
              p  >>= \y ->
              pure (\ g x -> g (f x y)) ) <*> rest
            <|> pure id
          
-- (<*>) :: Applicative f => f (a -> b) -> f a -> f b
\end{minted}
}


\defverbatim[colored]{\haskellChainlF}{
\begin{minted}{haskell}
chainl1 :: Parser a -> Parser (a -> a -> a) -> Parser a
chainl1 p op = p <**> rest
  where

    rest   = (op >>= \f ->
              p  >>= \y ->
              -- pure (\ g x -> g (f x y)) 
              -- pure (\ g x -> g (flip f y x)) 
              -- pure (\ g -> g . flip f y)) 
              pure (flip (.) (flip f y))
              ) <*> rest
            <|> pure id
\end{minted}
}

\defverbatim[colored]{\haskellChainlG}{
\begin{minted}{haskell}
chainl1 :: Parser a -> Parser (a -> a -> a) -> Parser a
chainl1 p op = p <**> rest
  where

    rest   = (op >>= \f ->
              p  >>= \y ->
              flip (.) <$> pure (flip f y))
              ) <*> rest
            <|> pure id
          
\end{minted}
}

\defverbatim[colored]{\haskellChainlH}{
\begin{minted}{haskell}
chainl1 :: Parser a -> Parser (a -> a -> a) -> Parser a
chainl1 p op = p <**> rest
  where

    rest   =  flip (.) <$> 
               (op >>= \f ->
                p  >>= \y ->
                pure (flip f y))
                ) <*> rest
            <|> pure id
          

\end{minted}
}

\defverbatim[colored]{\haskellChainlI}{
\begin{minted}{haskell}
chainl1 :: Parser a -> Parser (a -> a -> a) -> Parser a
chainl1 p op = p <**> rest
  where
    rest :: Parser (a -> a)
    rest =  
      flip (.) <$> (flip <$> op <*> chainl1 p op) <*> rest
      <|> pure id
          
-- Конец
\end{minted}
}

\setminted{fontsize=\normalsize,baselinestretch=1}


\begin{frame}[t]
\haskellChainrA
\end{frame}
\begin{frame}[t]
\haskellChainrB
\end{frame}
\begin{frame}[t]
\haskellChainrC
\end{frame}
\begin{frame}[t]
\haskellChainrD
\end{frame}
\begin{frame}[t]
\haskellChainrE
\end{frame}


\subsection{chainl1}

\begin{frame}[t]%{\haskellChainl}
\haskellChainlA
\end{frame}
\begin{frame}[t]
\haskellChainlB
\end{frame}
\begin{frame}[t]
\haskellChainlC
\end{frame}
\begin{frame}[t]
\haskellChainlD
\end{frame}
\begin{frame}[t]
\haskellChainlE
\end{frame}

\begin{frame}[t]
\haskellChainlF
\end{frame}
\begin{frame}[t]
\haskellChainlG
\end{frame}
\begin{frame}[t]
\haskellChainlH
\end{frame}
\begin{frame}[t]
\haskellChainlI
\end{frame}

%% %%%%%%%%%%%%%%%%%%%%%%%%%%%%%%%%%%%%%%%%%%%%%%%%%%%%%%%%%%%