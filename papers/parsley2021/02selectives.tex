\setminted{fontsize=\Large,baselinestretch=1}

\begin{frame}[fragile]{Monad}
\begin{minted}{haskell}
class Applicative m => Monad (m :: * -> *) where
  -- | Inject a value into the monadic type.
  return      :: a -> m a
  return      = pure
  
  -- | Sequentially compose two actions, 
  -- passing any value produced
  -- by the first as an argument to the second.
  (>>=)       :: forall a b. m a -> (a -> m b) -> m b
\end{minted}
\end{frame}


\begin{frame}[fragile]
\begin{minted}{haskell}
satisfy :: (Char -> Bool) -> Parser Char 
satisfy p = item >>= (\c -> 
                        if p c then pure c 
                        else        empty)
  
ident :: Parser String 
ident = (satisfy isAlpha <:> 
         many (satisfy isAlphaNum) ) >>= (\c -> 
              if isKeyword c then empty
              else                pure c)
\end{minted}
\end{frame}

\begin{frame}[fragile]
\begin{minted}{haskell}
(>?>) :: Monad m => m a -> (a -> Bool) -> m a
m >?> f = m >>= \x -> if f x then pure x else empty
\end{minted}
\newln

\begin{minted}{haskell}
satisfy :: (Char -> Bool) -> Parser Char 
satisfy p = item >?> p
  
ident :: Parser String 
ident = (satisfy isAlpha <:> 
         many (satisfy isAlphaNum) ) 
        >?> (not . isKeyword)
\end{minted}
\end{frame}

\setminted{fontsize=\normalsize,baselinestretch=1}


\begin{frame}
\begin{center}
\Large
Монады классные, что же с ними не так и зачем нужны Selective функторы?
\end{center}
\end{frame}

\setminted{fontsize=\Large,baselinestretch=1}
\begin{frame}[fragile]{Selective}
\begin{minted}{haskell}
-- case expression abstracted
class Applicative f => Selective f where
  -- In the original paper 
  select :: f (Either a b) -> f (a -> b) -> f b
  
  -- alternative definition, used by Parsley
  branch :: f (Either a b) -> f (a -> c) -> f (b -> c) 
         -> f c
\end{minted}
\end{frame}



\begin{frame}[fragile]{Реализация \mintinline{haskell}{>?>}}
\Large
С помощью \mintinline{haskell}{Monad}
\begin{lstlisting}[language=haskell
  , basicstyle=\Large\ttfamily
  , escapeinside={!}{!}
  ]
(>?>) :: Monad m => m a -> (a -> Bool) -> m a
m >?> f = m >>= !\tikzmark{a}! \ x -> if f x then pure x else empty !\tikzmark{b}!
\end{lstlisting}
\begin{tikzpicture}[remember picture,overlay]
  \draw[red,rounded corners]
    ([shift={(-3pt,2ex)}]pic cs:a) 
      rectangle 
    ([shift={(3pt,-0.65ex)}]pic cs:b);
\end{tikzpicture}
\pause 

С помощью \mintinline{haskell}{Selective}
\begin{lstlisting}[language=haskell
  , basicstyle=\Large\ttfamily
  , escapechar=!
  , mathescape=false
  ]
(>?>) :: Selective m => m a -> (a -> Bool) -> m a
m >?> f = select ( !\tikzmark{c}!(\ x -> !\tikzmark{d}!
!\tikzmark{e}!  if f x then Right () 
  else        Left ()) !\tikzmark{f}!   <$> fx) empty 
\end{lstlisting}
\begin{tikzpicture}[remember picture,overlay]
\draw[red,rounded corners]
  ([shift={(-3pt,2.5ex)}]pic cs:c) 
    rectangle 
  ([shift={(3pt,-0.5ex)}]pic cs:d);
\draw[red,rounded corners]
  ([shift={(-3pt,2.2ex)}]pic cs:e) 
    rectangle 
  ([shift={(3pt,-0.65ex)}]pic cs:f);
\end{tikzpicture}
% Со второй реализацией статически знаем больше 
\end{frame}

\begin{frame}[fragile]{Чего не умеют \mintinline{haskell}{Selective}?}
\Large 
\begin{itemize}
\item Использовать несколько предыдущих результатов

  \lstinline[language=haskell, basicstyle=\Large\ttfamily]{mf >>= \ f -> mg >>= \ g -> mh >>= \ h -> ... }
\item Монадический join:

 \lstinline[language=haskell, basicstyle=\Large\ttfamily]{join :: m (m a) -> m a}
\end{itemize}
\newln 

\begin{block}{Замечание}
Если \mintinline{haskell}{Selective} не умеют \mintinline{haskell}{join}
или \mintinline{haskell}{Applicative} не умеют КЗ-грамматики, то это не значит, что работающий парсер нельзя написать.
\end{block}
\end{frame}


\setminted{fontsize=\normalsize,baselinestretch=1}