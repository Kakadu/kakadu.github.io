\documentclass[
  russian,
  aspectratio=43,
  xcolor={svgnames},
  hyperref={colorlinks,citecolor=DeepPink4,linkcolor=DarkRed,urlcolor=DarkBlue}]{beamer}
\beamertemplatenavigationsymbolsempty % remove navigation bar
% \setbeamertemplate{headline}  % remove heading

\usetheme{Madrid}
\usecolortheme{beaver}

\makeatletter
\@ifclassloaded{beamer}{
  % get rid of header navigation bar
  \setbeamertemplate{headline}{}
  % get rid of bottom navigation symbols
  \setbeamertemplate{navigation symbols}{}
  % get rid of footer
  %\setbeamertemplate{footline}{}
}
{}
\makeatother
%%%%%%%%%%%%%%%%%%%%%%%%%%%%%%%%%%%%%%%%%%%%%
\usepackage{fontawesome}
% \newfontfamily{\FA}{Font Awesome 5 Free} % some glyphs missing
\expandafter\def\csname faicon@facebook\endcsname{{\FA\symbol{"F09A}}}
\def\faQuestionSign{{\FA\symbol{"F059}}}
\def\faQuestion{{\FA\symbol{"F128}}}
\def\faExclamation{{\FA\symbol{"F12A}}}
\def\faUploadAlt{{\FA\symbol{"F093}}}
\def\faLemon{{\FA\symbol{"F094}}}
\def\faPhone{{\FA\symbol{"F095}}}
\def\faCheckEmpty{{\FA\symbol{"F096}}}
\def\faBookmarkEmpty{{\FA\symbol{"F097}}}

\newcommand{\faGood}{\textcolor{ForestGreen}{\faThumbsUp}}
\newcommand{\faBad}{\textcolor{red}{\faThumbsODown}}
\newcommand{\faWrong}{\textcolor{red}{\faTimes}}
\newcommand{\faMaybe}{\textcolor{blue}{\faQuestion}}
\newcommand{\faCheckGreen}{\textcolor{ForestGreen}{\faCheck}}
%%%%%%%%%%%%%%%%%%%%%%%%%%%%%%%%%%%%%%%%%%%%%

\usepackage{fontspec}
\usepackage{xunicode}
\usepackage{xltxtra}
\usepackage{xecyr}
\usepackage{hyperref}

\setmainfont[
 Ligatures=TeX,
 Extension=.otf,
 BoldFont=cmunbx,
 ItalicFont=cmunti,
 BoldItalicFont=cmunbi,
% Scale = 1.1
]{cmunrm}
\setsansfont[
 Ligatures=TeX,
 Extension=.otf,
 BoldFont=cmunsx,
 ItalicFont=cmunsi,
%  Scale = 1.2
]{cmunss}
%\setmainfont[Mapping=tex-text]{DejaVu Serif}
%\setsansfont[Mapping=tex-text]{DejaVu Sans}
%\setmonofont{Fira Code}[Contextuals=AlternateOff]
\setmonofont{Fira Code}[Contextuals=Alternate,Scale=0.9]
\newfontfamily{\myfiracode}[Scale=1.5,Contextuals=Alternate]{Fira Code}
%\setmonofont[Scale=0.9,BoldFont={Inconsolata Bold}]{Inconsolata}

\usepackage{polyglossia}
\setmainlanguage{russian}
\setotherlanguage{english}


%\newfontfamily\dejaVuSansMono{DejaVu Sans Mono}
% https://github.com/vjpr/monaco-bold/raw/master/MonacoB/MonacoB.otf
%\newfontfamily\monacoB{MonacoB}
%%%%%%%%%%%%%%%%%%%%%%%%%%%%%%%%%%%%%%%%%%%%%%%5
\usepackage{soul} % for \st that strikes through
\usepackage[normalem]{ulem} % \sout

\usepackage{stmaryrd}
\newcommand{\sem}[1]{\ensuremath{\llbracket #1\rrbracket}}


\usepackage{listings}
%\lstdefinestyle{style1}{
%  language=haskell,
%  numbers=left,
%  stepnumber=1,
%  numbersep=10pt,
%  tabsize=4,
%  showspaces=false,
%  showstringspaces=false
%}
%\lstdefinestyle{hsstyle1}
%{ language=haskell
%%          , basicstyle=\monacoB
%         , deletekeywords={Int,Float,String,List,Void}
%         , breaklines=true
%         , columns=fullflexible
%         , commentstyle=\color{ForestGreen}
%         , escapeinside=§§
%         , escapebegin=\begin{russian}\commentfont
%         , escapeend=\end{russian}
%         , commentstyle=\color{ForestGreen}
%         , escapeinside=§§
%         , escapebegin=\begin{russian}\color{ForestGreen}
%         , escapeend=\end{russian}
%         , mathescape=true
%%          , backgroundcolor = \color{MyBackground}
%}
%
%\newcommand{\inline}[1]{\lstinline{haskell}{#1}}
%\def\hsinline{\mintinline{haskell}}
%\def\inline{\hsinline}
%
%\lstnewenvironment{hslisting} {
%    \lstset { style={hsstyle1} }
%  }
%  {}
%  
%%%%%%%%%%%%%%%%%%%%%%%%%%%%%%%%%%%%%%%%%%%%%%%%%%%%%%%%%%%  
%%\setmainfont[
%% Ligatures=TeX,
%% Extension=.otf,
%% BoldFont=cmunbx,
%% ItalicFont=cmunti,
%% BoldItalicFont=cmunbi,
%%]{cmunrm}
%%% С засечками (для заголовков)
%%\setsansfont[
%% Ligatures=TeX,
%% Extension=.otf,
%% BoldFont=cmunsx,
%% ItalicFont=cmunsi,
%%]{cmunss}
%% \setmonofont[Scale=0.6]{Monaco}
%
%\usefonttheme{professionalfonts}
%\usepackage{times}
\usepackage{tikz}
\usetikzlibrary{cd}
\usepackage{tikz-cd}
\usepackage{caption}
\usepackage{subcaption}

%\renewtheorem{definition}{برهان}[chapter]
%%\DeclareMathOperator{->}{\rightarrow}
%\newcommand\iso{\ensuremath{\cong}}
%\usepackage{verbatim}
%\usepackage{graphicx}
%\usetikzlibrary{arrows,shapes}

%\usepackage{amsmath}
%\usepackage{amsfonts}
\usepackage{scalerel}
\DeclareMathOperator*{\myvee}{\scalerel*{\vee}{\sum}}
\DeclareMathOperator*{\mywedge}{\scalerel*{\wedge}{\sum}}

%
%\usepackage{tabulary}
%
%% sudo aptget install ttf-mscorefonts-installer
%%\setmainfont{Times New Roman}
%%\setsansfont[Mapping=tex-text]{DejaVu Sans}
%
%%\setmonofont[Scale=1.0,
%%    BoldFont=lmmonolt10-bold.otf,
%%    ItalicFont=lmmono10-italic.otf,
%%    BoldItalicFont=lmmonoproplt10-boldoblique.otf
%%]{lmmono9-regular.otf}
%
\usepackage[cache=true]{minted}
\usemintedstyle{perldoc}

\def\hsinline{\mintinline{haskell}}
\def\mlinline{\mintinline{ocaml}}
% color options
\definecolor{YellowGreen} {HTML}{B5C28C}
\definecolor{ForestGreen} {HTML}{009B55}
\definecolor{MyBackground}{HTML}{F0EDAA}



\institute{матмех СПбГУ}

\addtobeamertemplate{title page}{}{
  \begin{center}{\tiny Дата сборки: \today}\end{center}
}


\usepackage{subcaption}
\newtheorem{thm}{Theorem}

\newcommand{\naturalto}{%
  \mathrel{\vbox{\offinterlineskip
    \mathsurround=0pt
    \ialign{\hfil##\hfil\cr
      \normalfont\scalebox{1.5}{.}\cr
      \noalign{\kern-.15ex}
      $\longrightarrow$\cr}
  }}%
}


\title{Генерация расширяемого кода по типам данных на языке OCaml}
\subtitle{Обобщенное программирования с комбинаторами и объектами}
\author[Косарев Дмитрий]{Косарев Дмитрий Сергеевич\\
[2mm]{\small 
Научный руководитель: \\доктор физико-математических наук, профессор, \\
Терехов Андрей Николаевич \\ \vspace{1mm} 
Рецензент: \\
доктор технических наук, \\Новиков Федор Александрович}
}

%\institute[]{матмех СПбГУ}
% 
% Научный руководитель:
% доктор физико-математических наук, профессор Терехов Андрей Николаевич
% 
% Рецензент:
% доктор технических наук Новиков Федор Александрович


\date{18 июня 2019} %\today 

% \AtBeginSection[]
% {
%   \begin{frame}<beamer>
%     \frametitle{Содержание}
%     \tableofcontents[currentsection,currentsubsection]
%   \end{frame}
% }

\newcommand{\verbatimfont}[1]{\def\verbatim@font{#1}}
\usepackage{verbatimbox}

\usepackage{amssymb,amsmath}
\usepackage{tikz}
\usepackage{tikz-cd}
\usetikzlibrary{arrows}
\tikzstyle{line}=[draw] % here
\tikzstyle{startstop} = [rectangle, rounded corners, minimum width=3cm,
    minimum height=1cm,text centered, draw=black, fill=red!30]
\usetikzlibrary{babel}


\begin{document}
\maketitle

% For every picture that defines or uses external nodes, you'll have to
% apply the 'remember picture' style. To avoid some typing, we'll apply
% the style to all pictures.
\tikzstyle{every picture}+=[remember picture]

% By default all math in TikZ nodes are set in inline mode. Change this to
% displaystyle so that we don't get small fractions.
\everymath{\displaystyle}

% Uncomment these lines for an automatically generated outline.
% \begin{frame}{Outline}
%   \tableofcontents
% \end{frame}

% \section{Предварительные знания о терии категорий}

\begin{frame}[fragile]{Предыстория}
\begin{itemize}
 \item Встраиваемый DSL для реляционного (логического) программирования OCanren
 \item Статья ``Typed Embedding of a Relational Language in OCaml'' \href{http://dx.doi.org/10.4204/EPTCS.285.1}{DOI ссылка}
 \item Работа раскрыла несовершенство имеющихся подходов к обобщённому программированию
\end{itemize}
\end{frame}

\begin{frame}[fragile]{Обобщённое программирование (1/2)}
\begin{itemize}
 \item Свойственно флагманским языкам функционального программирования: Haskell, OCaml
 \item Скорее не свойственно (но возможно) для статически типизированных ОО языков: Java, C\#, C++.
 \item Вообще не применимо к динамическим языками, там нет этой проблемы.
\end{itemize}
\end{frame}

\begin{frame}[fragile]{Обобщённое программирование (1/2)}
Идея: генерировать (во время компиляции) некоторые функции по описаниям типов данных. 

Например, преобразование в строковое понятное человеку представление; сериализация и десериализация; преобразования похожие свёртки и т.д

\vspace{0.5cm}Пример 

\begin{lstlisting}[style=ocaml1]
type ($\alpha$,$\beta$,$\dots$) t = $\dots$

val show: ($\alpha$->string) -> ($\beta$->string) -> $\dots$ -> 
          ($\alpha$,$\beta$,...) t -> string
\end{lstlisting}
Реализовывать такой однотипный код руками неудобно
\end{frame}

\begin{frame}[fragile]{Избранные работы про обобщённое программирование}

Ralf Lämmel and Simon Peyton Jones (2003): \textit{Scrap Your Boilerplate: A Practical Design Pattern for Generic Programming}, \href{http://dx.doi.org/10.1145/640136.604179}{doi:10.1145/640136.604179} \vspace{1cm}

Dmitry Boulytchev and Sergey Mechtaev (2011): \textit{Efficiently Scrapping Boilerplate Code in OCaml.} \vspace{1cm}

François Pottier (2017): \textit{Visitors Unchained}. \href{http://dx.doi.org/10.1145/3110272}{doi:10.1145/3110272.}
\end{frame}

\begin{frame}[fragile]{Основная задача}

Разработка подхода для обобщенного программирования, который позволит 
порождаться \emph{расширяемые} преобразования\vspace{1cm}

На сегодняшний день такое умеет делать только \textsc{Visitors}
\end{frame}

\begin{frame}[fragile]{Результаты}
Библиотека обобщённого программирования \href{https://github.com/Kakadu/GT/tree/ppx/}{GT} (Generic transformers) для языка OCaml
\begin{itemize}
 \item Позволяет создавать \emph{расширяемые} трансформации с помощью объектов, как и у \textsc{Visitors}
 \item По-другому типизируются объекты
 \item Комбинаторный интерфейс
 \item Поддерживаются большее многообразие типов
 \item Нет потенциальных проблем с безопасностью, которые есть у \textsc{Visitors} и \textsc{Scrap Your Boilerplate}
\end{itemize}

\end{frame}

\begin{frame}[fragile]{Расширяемость с помощью объектов}
\begin{lstlisting}[style=ocaml1]
type $\dots$ t = $C_1$ of $\dots$ | $\dots$ | $C_n$ of $\dots$
class old_transformation = object 
  method $C_1$ =  $\dots$
  $\dots$
end
\end{lstlisting}
Конструкторы кодируются в методы один к одному
\begin{lstlisting}[style=ocaml1]
class new_transformation = object 
  inherit old_transformation
  method $C_i$ =  (* new implementation *)
end
\end{lstlisting}

\end{frame}

\begin{frame}[fragile]{Типизация}
\textsc{Visitors} полагаются на трюк, для сокращения количества типовых параметров до одного

\begin{lstlisting}[style=ocaml1]
class [$\sigma$]new_transformation = object (this: $\sigma$)
  $\dots$
end
\end{lstlisting}

Недостатки: 
\begin{itemize}
 \item Такая типизация \textsc{Visitors} неприменима в файлах интерфейса
 \item Она не позволяет реализовать поддержку полиморфных вариантных типов языка OCaml
\end{itemize}\vspace{1cm}

В \textsc{Generic Transformers} используется стандартный способ использования типовых параметров (они указываются явно). Полиморфные вариантные типы поддержаны
\end{frame}

\begin{frame}[fragile]{Комбинаторный интерфейс}
В \textsc{Visitors} вызов преобразования типа $(\alpha,...)\tau$ выглядит как 
\begin{lstlisting}[style=ocaml1]
(new classname)#visit_$\tau$
\end{lstlisting}\vspace{1cm}

В то время как все остальные библиотеки используют (комбинаторный интерфейс
\begin{lstlisting}[style=ocaml1]
transform$\tau$ transform$\alpha$ ... (x: $\tau$)
\end{lstlisting}%\vspace{1cm}
\end{frame}

\begin{frame}[fragile]{Виды методов и безопасность}
\textsc{SYB} и \textsc{Visitors} позволяют применять преобразования \emph{ко всем вхождениям типа} (например, \texttt{int}) в структуре данных. \vspace{1cm}

Это позволяет преодолевать барьер инкапсуляции.\vspace{1cm}

В \textsc{Generic Transformers} конструкторы алгебраических типов кодируются в методы один к одному. Преобразования типов нельзя переопределить.
\end{frame}

\begin{frame}[fragile]{Результаты}
Библиотека обобщённого программирования \href{https://github.com/Kakadu/GT/tree/ppx/}{GT} (Generic transformers) для языка OCaml
\begin{itemize}
 \item Позволяет создавать \emph{расширяемые} трансформации с помощью объектов
 \item Другая типизация объекты, по сравнению с конкурентом
 \item Комбинаторный интерфейс
 \item Поддерживаются большее многообразие типов
 \item Нет потенциальных проблем с безопасностью
\end{itemize}\pause
\begin{center}
  \Huge Конец
 \end{center}
\end{frame}

\begin{frame}[allowframebreaks]
  \frametitle<presentation>{Ссылки}
  \begin{thebibliography}{10}
  \beamertemplatebookbibitems
  \bibitem{gadt}
    Ralf Lämmel \& Simon Peyton Jones (2003): \emph{Scrap Your Boilerplate: A Practical Design Pattern for Generic Programming. }
    \newblock Доступно по \href{http://dx.doi.org/10.1145/640136.604179}{DOI ссылке}
   \bibitem{}
    François Pottier (2017): \emph{Visitors Unchained.}
    \newblock Доступно по \href{http://doi.acm.org/10.1145/3110272}{DOI ссылке}
  \bibitem{}
    Dmitry Boulytchev \& Sergey Mechtaev (2011): \emph{Efficiently Scrapping Boilerplate Code in OCaml.}
    \newblock Доклад на ML Workshop при ICFP 2011
   \bibitem{}
    Dmitry Kosarev \& Dmitry Boulytchev (2016): \emph{Typed Embedding of a Relational Language in OCaml.}
    \newblock Доступно по \href{http://dx.doi.org/10.4204/EPTCS.285.1}{DOI ссылке}
 \end{thebibliography}
\end{frame}

\end{document}
