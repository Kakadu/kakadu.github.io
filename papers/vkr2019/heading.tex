\usepackage{xcolor}
\definecolor{YellowGreen} {HTML}{B5C28C}
\definecolor{ForestGreen} {HTML}{009B55}
\definecolor{MyBackground}{HTML}{F0EDAA}

\usepackage{xltxtra} % load xunicode
\usepackage{polyglossia}
\setmainlanguage{russian}
\setotherlanguage{english}

% \usepackage[\eu@zf@math]{fontspec}
% \usepackage{amsmath}
% \usepackage{xkeyval}
% \usepackage{mathstyle}
% \usepackage{etoolbox}
% \usepackage{mathspec}
% \usepackage{inconsolata}

\let\cyrillicfonttt\monofamily
\usepackage{listings}

\lstdefinestyle{ocaml1}
  { mathescape = true
  , language=ml 
  , literate={->}{{$\to$}}3 {===}{{$\equiv$}}1 {=/=}{{$\not\equiv$}}1 {|>}{{$\triangleright$}}3 {\\/}{{$\vee$}}2 {/\\}{{$\wedge$}}2 {>=}{{$\ge$}}1 {<=}{{$\le$}} 1
  , morecomment=[s]{(*}{*)}
  , keywords = {@type, function, fun, let, in, match, with, when, class, type,
        nonrec, object, method, of, rec, repeat, until, while, not, do, done, 
        as, val, inherit, and,
        new, module, sig, deriving, datatype, struct, if, then, else, 
        open, private, virtual, include, success, failure,
        lazy, assert, true, false, end   }
  }
\lstdefinestyle{hsstyle1}
         { showstringspaces=false
         , language=haskell
         , keywords={ import,data,type,class,where
                    , instance,case,of,do,family
                    , deriving,if,then,else,forall,exists
                    }
         , deletekeywords={Int,Bool,Float,String,List,Either,Void,id,tail}
         %, basicstyle=\monacoB
%          , showspaces=false
%          , showstringspaces=false
%          , columns=fixed 
%          , breaklines=true
         %, columns=fullflexible
          , escapeinside=§§
%           , escapebegin=\begin{russian}\monacoB\color{DarkGreen}
%           , escapeend=\end{russian}
          , escapebegin=\begin{russian}\color{red}
          , escapeend=\end{russian}
          , commentstyle=\color{DarkGreen}
%          , escapebegin=\begin{russian}\monacoB\color{ForestGreen}
%          , escapeend=\end{russian}
         , mathescape=true
%          , backgroundcolor = \color{MyBackground}
}

\newcommand{\inline}[1]{\lstinline{haskell}{#1}}
\def\hsinline{\lstinline[style={hsstyle1}]}
\def\inline{\hsinline}
\def\myinline{\lstinline[basicstyle=\monacoB]}
\def\hline{\noindent\makebox[\linewidth]{\rule{\paperwidth}{0.4pt}}}

% \renewcommand{\vskip}{\vspace{1cm}}


\lstnewenvironment{hslisting} {
     \lstset { style={hsstyle1} }
  }
  {}
  
\lstset
  { language=haskell
%   , basicstyle  =\ttfamily %\bfseries\monacoFamily
%   , stringstyle =\ttfamily %\bfseries\monacoFamily
%   , keywordstyle=\ttfamily\bfseries %\monacoFamily
  , showstringspaces=false       % no special string spaces
  , keywords={ import,data,type,class,where
             , instance,case,of,do,family
             , deriving,if,then,else,forall
             }
  }


%%%%%%%%%%%%%%%%%%%%%%%%%%%%%%%%%%%%%%%%%%%%%%%%%%%%%%%%%%  
\setmainfont[
 Ligatures=TeX,
 Extension=.otf,
 BoldFont=cmunbx,
 ItalicFont=cmunti,
 BoldItalicFont=cmunbi,
]{cmunrm}
% С засечками (для заголовков)
\setsansfont[
 Ligatures=TeX,
 Extension=.otf,
 BoldFont=cmunsx,
 ItalicFont=cmunsi,
]{cmunss}
% \setmonofont[]{Hack}
% next line fixed error
% (polyglossia) Please define \cyrillicfont with \newfontfamily.
\newfontfamily{\cyrillicfonttt}{MonacoB}
% \newfontfamily{\cyrillicfonttt}{\monacoFamily}

\usepackage{times}
\usepackage{tikz}
\usetikzlibrary{cd}
% \usepackage{tikz-cd}

%\DeclareMathOperator{->}{\rightarrow}
\newcommand\iso{\ensuremath{\cong}}
\usepackage{verbatim}
\usepackage{graphicx}
\usetikzlibrary{arrows,shapes}
\usepackage{verbatimbox} % for verbnobox

\newcommand*{\arr}{\ensuremath{\rightarrow}}
\newcommand*{\natarr}{\ensuremath{\overset{\text{\normalsize .}}\longrightarrow}}

\usepackage{fontawesome}
% \newfontfamily{\FA}{Font Awesome 5 Free} % some glyphs missing
\expandafter\def\csname faicon@facebook\endcsname{{\FA\symbol{"F09A}}}
\def\faQuestionSign{{\FA\symbol{"F059}}}
\def\faQuestion{{\FA\symbol{"F128}}}
\def\faExclamation{{\FA\symbol{"F12A}}}
\def\faUploadAlt{{\FA\symbol{"F093}}}
\def\faLemon{{\FA\symbol{"F094}}}
\def\faPhone{{\FA\symbol{"F095}}}
\def\faCheckEmpty{{\FA\symbol{"F096}}}
\def\faBookmarkEmpty{{\FA\symbol{"F097}}}


\newcommand{\faGood}{\textcolor{ForestGreen}{\faThumbsUp}}
\newcommand{\faBad}{\textcolor{red}{\faThumbsODown}}
\newcommand{\usefulExercise}{\textcolor{red}{\faPlus}}

\usepackage{soul} % for \st that strikes through
\usepackage[normalem]{ulem} % \sout

% \usepackage{minted}
% \newcommand{\inline}[1]{\mintinline{haskell}{#1}}

% \hypersetup{%
%   linkbordercolor=blue
% }
% \usepackage[colorlinks=true, linkcolor=blue]{hyperref} 
