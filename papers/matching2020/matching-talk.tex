\def\pgfsysdriver{pgfsys-dvipdfm.def} % required for forests inside
\documentclass[aspectratio=169
  , xcolor={svgnames}
  , hyperref={ colorlinks,citecolor=Blue
             , linkcolor=DarkRed,urlcolor=DarkBlue}
  , russian
  ]{beamer}
\usetheme{CambridgeUS}
\usefonttheme{professionalfonts}

\usepackage{pgfpages}

\usepackage{bibentry}
\usepackage{cite}
\def\newblock{\hskip .11em plus .33em minus .07em}


\makeatletter
\@ifclassloaded{beamer}{
  % get rid of header navigation bar
  \setbeamertemplate{headline}{}
  % get rid of bottom navigation symbols
  \setbeamertemplate{navigation symbols}{}
  % get rid of footer
  %\setbeamertemplate{footline}{}
}
{}
\makeatother
%%%%%%%%%%%%%%%%%%%%%%%%%%%%%%%%%%%%%%%%%%%%%
\usepackage{fontawesome}
% \newfontfamily{\FA}{Font Awesome 5 Free} % some glyphs missing
\expandafter\def\csname faicon@facebook\endcsname{{\FA\symbol{"F09A}}}
\def\faQuestionSign{{\FA\symbol{"F059}}}
\def\faQuestion{{\FA\symbol{"F128}}}
\def\faExclamation{{\FA\symbol{"F12A}}}
\def\faUploadAlt{{\FA\symbol{"F093}}}
\def\faLemon{{\FA\symbol{"F094}}}
\def\faPhone{{\FA\symbol{"F095}}}
\def\faCheckEmpty{{\FA\symbol{"F096}}}
\def\faBookmarkEmpty{{\FA\symbol{"F097}}}

\newcommand{\faGood}{\textcolor{ForestGreen}{\faThumbsUp}}
\newcommand{\faBad}{\textcolor{red}{\faThumbsODown}}
\newcommand{\faWrong}{\textcolor{red}{\faTimes}}
\newcommand{\faMaybe}{\textcolor{blue}{\faQuestion}}
\newcommand{\faCheckGreen}{\textcolor{ForestGreen}{\faCheck}}
%%%%%%%%%%%%%%%%%%%%%%%%%%%%%%%%%%%%%%%%%%%%%

\usepackage{fontspec}
\usepackage{xunicode}
\usepackage{xltxtra}
\usepackage{xecyr}
\usepackage{hyperref}

\setmainfont[
 Ligatures=TeX,
 Extension=.otf,
 BoldFont=cmunbx,
 ItalicFont=cmunti,
 BoldItalicFont=cmunbi,
% Scale = 1.1
]{cmunrm}
\setsansfont[
 Ligatures=TeX,
 Extension=.otf,
 BoldFont=cmunsx,
 ItalicFont=cmunsi,
%  Scale = 1.2
]{cmunss}
%\setmainfont[Mapping=tex-text]{DejaVu Serif}
%\setsansfont[Mapping=tex-text]{DejaVu Sans}
%\setmonofont{Fira Code}[Contextuals=AlternateOff]
\setmonofont{Fira Code}[Contextuals=Alternate,Scale=0.9]
\newfontfamily{\myfiracode}[Scale=1.5,Contextuals=Alternate]{Fira Code}
%\setmonofont[Scale=0.9,BoldFont={Inconsolata Bold}]{Inconsolata}

\usepackage{polyglossia}
\setmainlanguage{russian}
\setotherlanguage{english}


%\newfontfamily\dejaVuSansMono{DejaVu Sans Mono}
% https://github.com/vjpr/monaco-bold/raw/master/MonacoB/MonacoB.otf
%\newfontfamily\monacoB{MonacoB}
%%%%%%%%%%%%%%%%%%%%%%%%%%%%%%%%%%%%%%%%%%%%%%%5
\usepackage{soul} % for \st that strikes through
\usepackage[normalem]{ulem} % \sout

\usepackage{stmaryrd}
\newcommand{\sem}[1]{\ensuremath{\llbracket #1\rrbracket}}


\usepackage{listings}
%\lstdefinestyle{style1}{
%  language=haskell,
%  numbers=left,
%  stepnumber=1,
%  numbersep=10pt,
%  tabsize=4,
%  showspaces=false,
%  showstringspaces=false
%}
%\lstdefinestyle{hsstyle1}
%{ language=haskell
%%          , basicstyle=\monacoB
%         , deletekeywords={Int,Float,String,List,Void}
%         , breaklines=true
%         , columns=fullflexible
%         , commentstyle=\color{ForestGreen}
%         , escapeinside=§§
%         , escapebegin=\begin{russian}\commentfont
%         , escapeend=\end{russian}
%         , commentstyle=\color{ForestGreen}
%         , escapeinside=§§
%         , escapebegin=\begin{russian}\color{ForestGreen}
%         , escapeend=\end{russian}
%         , mathescape=true
%%          , backgroundcolor = \color{MyBackground}
%}
%
%\newcommand{\inline}[1]{\lstinline{haskell}{#1}}
%\def\hsinline{\mintinline{haskell}}
%\def\inline{\hsinline}
%
%\lstnewenvironment{hslisting} {
%    \lstset { style={hsstyle1} }
%  }
%  {}
%  
%%%%%%%%%%%%%%%%%%%%%%%%%%%%%%%%%%%%%%%%%%%%%%%%%%%%%%%%%%%  
%%\setmainfont[
%% Ligatures=TeX,
%% Extension=.otf,
%% BoldFont=cmunbx,
%% ItalicFont=cmunti,
%% BoldItalicFont=cmunbi,
%%]{cmunrm}
%%% С засечками (для заголовков)
%%\setsansfont[
%% Ligatures=TeX,
%% Extension=.otf,
%% BoldFont=cmunsx,
%% ItalicFont=cmunsi,
%%]{cmunss}
%% \setmonofont[Scale=0.6]{Monaco}
%
%\usefonttheme{professionalfonts}
%\usepackage{times}
\usepackage{tikz}
\usetikzlibrary{cd}
\usepackage{tikz-cd}
\usepackage{caption}
\usepackage{subcaption}

%\renewtheorem{definition}{برهان}[chapter]
%%\DeclareMathOperator{->}{\rightarrow}
%\newcommand\iso{\ensuremath{\cong}}
%\usepackage{verbatim}
%\usepackage{graphicx}
%\usetikzlibrary{arrows,shapes}

%\usepackage{amsmath}
%\usepackage{amsfonts}
\usepackage{scalerel}
\DeclareMathOperator*{\myvee}{\scalerel*{\vee}{\sum}}
\DeclareMathOperator*{\mywedge}{\scalerel*{\wedge}{\sum}}

%
%\usepackage{tabulary}
%
%% sudo aptget install ttf-mscorefonts-installer
%%\setmainfont{Times New Roman}
%%\setsansfont[Mapping=tex-text]{DejaVu Sans}
%
%%\setmonofont[Scale=1.0,
%%    BoldFont=lmmonolt10-bold.otf,
%%    ItalicFont=lmmono10-italic.otf,
%%    BoldItalicFont=lmmonoproplt10-boldoblique.otf
%%]{lmmono9-regular.otf}
%
\usepackage[cache=true]{minted}
\usemintedstyle{perldoc}

\def\hsinline{\mintinline{haskell}}
\def\mlinline{\mintinline{ocaml}}
% color options
\definecolor{YellowGreen} {HTML}{B5C28C}
\definecolor{ForestGreen} {HTML}{009B55}
\definecolor{MyBackground}{HTML}{F0EDAA}


\author{Косарев Дмитрий a.k.a. Kakadu}
\institute{матмех СПбГУ}

\addtobeamertemplate{title page}{}{
  \begin{center}{\tiny Дата сборки: \today}\end{center}
}


\lstdefinelanguage{ocamllambda}{
keywords={catch, switch, default, case, with, failwith, exit, true, false, ::},
sensitive=true,
commentstyle=\small\itshape\ttfamily,
keywordstyle=\ttfamily\textbf,
identifierstyle=\ttfamily,
basewidth={0.5em,0.5em},
columns=fixed,
mathescape=true,
fontadjust=true,
literate={->}{{$\to$}}3 {===}{{$\equiv$}}1  {Scru}{{$\bullet$}}2,
morecomment=[s]{(*}{*)}
}

\usepackage{subcaption}
\usepackage{etoolbox}


\usepackage{exercise}
\usepackage{tikz}
\usepackage{tikz-qtree}
\usetikzlibrary{trees}
\usepackage[edges]{forest}
\forestset{.style={
%  for tree={l=1em, l sep=1em, s sep=1em}
  forked edges,
    for tree={    grow'=0,    draw,    align=c,    font=\sffamily,
        rounded corners  }
  }}

\newcommand{\lstquot}[1]{``\lstinline{#1}''}
\newcommand{\sembr}[1]{\llbracket{#1}\rrbracket}
%\newcommand\false{$f\!alse$}
%\newcommand\myif{i\!f}


\def\transarrow{\xrightarrow}
\newcommand{\setarrow}[1]{\def\transarrow{#1}}

\def\padding{\phantom{X}}
\newcommand{\setpadding}[1]{\def\padding{#1}}

\def\subarrow{}
\newcommand{\setsubarrow}[1]{\def\subarrow{#1}}

\newcommand{\trule}[2]{\dfrac{#1}{#2}}
\newcommand{\crule}[3]{\dfrac{#1}{#2},\;{#3}}
\newcommand{\withenv}[2]{{#1}\vdash{#2}}
\newcommand{\trans}[3]{{#1}\transarrow{\padding{\textstyle #2}\padding}\subarrow{#3}}
\newcommand{\ctrans}[4]{{#1}\transarrow{\padding#2\padding}\subarrow{#3},\;{#4}}
\newcommand{\llang}[1]{\mbox{\lstinline[mathescape]|#1|}}
\newcommand{\pair}[2]{\inbr{{#1}\mid{#2}}}
\newcommand{\inbr}[1]{\left<{#1}\right>}
\newcommand{\highlight}[1]{\color{red}{#1}}
%\newcommand{\ruleno}[1]{\eqno[\scriptsize\textsc{#1}]}
\newcommand{\ruleno}[1]{\mbox{[\textsc{#1}]}}
\newcommand{\rulename}[1]{\textsc{#1}}
\newcommand{\inmath}[1]{\mbox{$#1$}}
%\newcommand{\lfp}[1]{fix_{#1}}
%\newcommand{\gfp}[1]{Fix_{#1}}
\newcommand{\vsep}{\vspace{-2mm}}
\newcommand{\supp}[1]{\scriptsize{#1}}
\renewcommand{\sembr}[1]{\llbracket{#1}\rrbracket}
\newcommand{\cd}[1]{\texttt{#1}}
%\newcommand{\free}[1]{\boxed{#1}}
%\newcommand{\binds}{\;\mapsto\;}
%\newcommand{\dbi}[1]{\mbox{\bf{#1}}}
%\newcommand{\sv}[1]{\mbox{\textbf{#1}}}
%\newcommand{\bnd}[2]{{#1}\mkern-9mu\binds\mkern-9mu{#2}}
%\newcommand{\meta}[1]{{\mathcal{#1}}}
%\newcommand{\dom}[1]{\mathtt{dom}\;{#1}}
%\newcommand{\primi}[2]{\mathbf{#1}\;{#2}}
%\renewcommand{\dom}[1]{\mathcal{D}om\,({#1})}
%\newcommand{\ran}[1]{\mathcal{VR}an\,({#1})}
%\newcommand{\fv}[1]{\mathcal{FV}\,({#1})}
%\newcommand{\tr}[1]{\mathcal{T}r_{#1}}
\newcommand{\diseq}{\not\equiv}
%\newcommand{\reprfunset}{\mathcal{R}}
%\newcommand{\reprfun}{\mathfrak{f}}
%\newcommand{\cstore}{\Omega}
%\newcommand{\cstoreinit}{\cstore_\epsilon^{init}}
%\newcommand{\csadd}[3]{add(#1, #2 \diseq #3)}  %{#1 + [#2 \diseq #3]}
%\newcommand{\csupdate}[2]{update(#1, #2)}  %{#1 \cdot #2}
\newcommand{\primi}[1]{\ensuremath{\mathbf{#1}}}
\newcommand{\sem}[1]{\llbracket #1 \rrbracket}
\newcommand{\ir}{\ensuremath{\mathcal{S}}}
\usepackage{tikz}
\newcommand*\circled[1]{\tikz[baseline=(char.base)]{
    \node[shape=circle,draw,inner sep=1pt] (char) {#1};}}

\let\emptyset\varnothing
\let\eps\varepsilon

% for fancy table
%\newcommand{\lheadl}[2]{\multicolumn{1}{|>{\centering\arraybackslash}m{#1}|}{{#2}}}
%\newcommand{\head}[2]{\multicolumn{1}{>{\centering\arraybackslash}m{#1}}{\textbf{\small #2}}}
%\newcommand{\headll}[2]{\multicolumn{1}{>{\centering\arraybackslash}m{#1}||}{\textbf{\small #2}}}
%\newcommand{\lheadll}[2]{\multicolumn{1}{|>{\centering\arraybackslash}m{#1}||}{\textbf{\small #2}}}
%\newcommand{\headl}[2]{\multicolumn{1}{>{\centering\arraybackslash}m{#1}|}{\textbf{\small #2}}}
%\usepackage{longtable}
%\newcommand{\nodata}{}
%\newcommand{\tablenotemark}[1]{#1}

\newcommand{\contributions}{
\begin{enumerate}
\item[I] Спроектировали синтез с помощью комбинации \emph{реляционных интерпретаторов} на \miniKanren{}
\item[II] Заменили $\forall$ на \emph{конечный} набор примеров
\item[III] Сделали оптимизацию методом ветвей и границ с помощью нового примитива \miniKanren{}: \emph{ограничение на структуру (structural constraint)}
%\item[IV] Extension of OCanren called \emph{structural constraint}
\end{enumerate}
}

\title{Реляционный синтез сопоставления с образцом}
\subtitle{Relational Synthesis for Pattern Matching}

\date{9 ноября 2020}
\author{Косарев Дмитрий} 
\institute[]{\normalfont
Будет опубликовано на \\
Asian Symposium on Programming Languages and Systems (APLAS) 2020}


\newcommand{\verbatimfont}[1]{\def\verbatim@font{#1}}
\usepackage{verbatimbox}

\forestset{
    .style={
        for tree={
            base=bottom,
            child anchor=north,
            align=center,
            s sep+=1cm,
    straight edge/.style={
        edge path={\noexpand\path[\forestoption{edge},thick,-{Latex}] 
        (!u.parent anchor) -- (.child anchor);}
    },
    if n children={0}
        {tier=word, draw, thick, rectangle}
        {draw, diamond, thick, aspect=2},
    if n=1{%
        edge path={\noexpand\path[\forestoption{edge},thick,-{Latex}] 
        (!u.parent anchor) -| (.child anchor) node[pos=.2, above] {Y};}
        }{
        edge path={\noexpand\path[\forestoption{edge},thick,-{Latex}] 
        (!u.parent anchor) -| (.child anchor) node[pos=.2, above] {N};}
        }
        }
    }
}
  
\AtBeginSection[]
{
  \begin{frame}<beamer>
    \frametitle{Оглавление}
    \tableofcontents[currentsection,currentsubsection]
  \end{frame}
} 
\AtBeginSubsection[]
{
  \begin{frame}<beamer>
    \frametitle{Оглавление}
    \tableofcontents[currentsection,currentsubsection]
  \end{frame}
}

\begin{document}
{
\begin{frame}[fragile]
  \begin{tabular}{p{5.5cm} p{5.5cm}}
   \begin{center}
%      \includegraphics[height=1.5cm]{pictures/jetbrainsResearch.pdf}
    \end{center}
    &
    \begin{center}
   %   \includegraphics[height=1.5cm]{pictures/SPbGU_Logo.png}
    \end{center}
  \end{tabular}
  \titlepage
\end{frame}
}

%\maketitle

% For every picture that defines or uses external nodes, you'll have to
% apply the 'remember picture' style. To avoid some typing, we'll apply
% the style to all pictures.
\tikzstyle{every picture}+=[remember picture] 

% By default all math in TikZ nodes are set in inline mode. Change this to
% displaystyle so that we don't get small fractions.
\everymath{\displaystyle}
%\begin{comment}
% Uncomment these lines for an automatically generated outline.
\begin{frame}{Сопоставление с образцом}
Существенная часть функционального программирования
\vspace{1cm}

%\includegraphics{pictures/SPbGU_Logo.png}

Два основных подхода:
\begin{itemize}
\item диаграммы решений
\begin{itemize}
\item минимизируется количество проверок
\end{itemize}
\item автомат с возвратами
\begin{itemize}
\item минимизируется размер кода
\end{itemize}%\pause
\item \textbf{синтез программ} (наш подход)
\end{itemize}
\end{frame}

\section{Обзор}

\defverbatim[colored]{\matchA}{
\begin{lstlisting}
match x,y,z with
| _,F,T -> 1
| F,T,_ -> 2
| _,_,F -> 3
| _,_,T -> 4
\end{lstlisting}
}

\defverbatim[colored]{\matchB}{
\begin{lstlisting}
if x then
  if y then
    if z then 4 else 3
  else
    if z then 1 else 3
else
  if y then 2
  else
    if z then 1 else 3
\end{lstlisting}
}

\defverbatim[colored]{\matchC}{
\begin{lstlisting}
if y then
  if x then
    if z then 4 else 3
  else 2
else
  if z then 1 else 3
\end{lstlisting}
}


\defverbatim[colored]{\forestB}{
\begin{forest} 
[\texttt{x}%, tikz={\draw[{Latex}-, thick] (.north) --++ (0,1);}
    [\texttt{y}
          [\texttt{z}
                [4] 
                [3] 
            ]   
         [\texttt{z}
               [1] 
               [3] 
           ]    
    ]   
    [\texttt{y}
        [2] 
        [\texttt{z}
            [1] 
            [3] 
        ]   
    ]   
] 
\end{forest}
}

\defverbatim[colored]{\forestC}{
\begin{forest} 
[\texttt{y}%, tikz={\draw[{Latex}-, thick] (.north) --++ (0,1);}
    [\texttt{x}
          [\texttt{z}
                [4] 
                [3] 
            ]   
         [2]    
    ]
    [\texttt{z}
               [1] 
               [3] 
           ]
]
\end{forest}
}


\begin{frame}[fragile]{Пример: использование диаграмм решений}% (1/2)}

\begin{minipage}[t]{0.25\linewidth}
\begin{minipage}{7cm}
\matchA
\end{minipage}
%\caption{Pattern matching}
\end{minipage}
\hspace{0.5cm}
\begin{minipage}[t]{0.32\linewidth}
\only<1>{\begin{minipage}{7cm}
\matchB
\end{minipage}}
\only<2>{\begin{minipage}{7cm}
\matchC
\end{minipage}}

\end{minipage}
\hspace{0.5cm}
\begin{minipage}[t]{0.3\linewidth}
\only<1>{\begin{minipage}{7cm}
\forestB
\end{minipage}}
\only<2>{\begin{minipage}{7cm}
\forestC
\end{minipage}}
\end{minipage}
%\caption{Pattern matching compilation can be non-trivial (example from~\cite{maranget2008}).}\label{fig:match-example}
\end{frame}

\defverbatim[colored]{\btCompAone}{
\begin{lstlisting}[language=ocamllambda]


    (switch lx with case []: 1
      default:     )
               
      (switch ly with case []: 2
        default:     )
                  
      (switch lx with
        case (::):
          (switch ly with
            case (::) : 3
            default:     )
        default:     )   
                                
\end{lstlisting}
}

\defverbatim[colored]{\btCompA}{
\begin{lstlisting}[language=ocamllambda]
catch
  (catch
    (switch lx with case []: 1
      default: exit)
    with (catch
      (switch ly with case []: 2
        default: exit)
    with (catch
      (switch lx with
        case (::):
          (switch ly with
            case (::) : 3
            default: exit)
        default: exit))))
with (failwith "Partial match")
\end{lstlisting}
}

\defverbatim[colored]{\btCompBone}{
\begin{lstlisting}[language=ocamllambda]
      
         
    (switch lx with
      case []: 1
      case (::) :
        (switch ly with
          case (::): 3
          default:     ))
      
    (switch ly with
       case []: 2
       default:     )
                               
\end{lstlisting}
}

\defverbatim[colored]{\btCompB}{
\begin{lstlisting}[language=ocamllambda]
catch
  (catch
    (switch lx with
      case []: 1
      case (::) :
        (switch ly with
          case (::): 3
          default: exit))
  with
    (switch ly with
       case []: 2
       default: exit)
with (failwith "Partial match")
\end{lstlisting}
}

\begin{frame}[fragile]{Пример: Компиляция в автомат с возвратами}
\begin{minipage}{0.45\linewidth}\vspace{0pt}
\[
(P\rightarrow L)=    \begin{pmatrix}
      \texttt{[]} & \texttt{\_} & \rightarrow & 1 \\
      \texttt{\_} & \texttt{[]} & \rightarrow & 2  \\
      \texttt{x::xs} & \texttt{y::ys} & \rightarrow & 3
    \end{pmatrix}
\]
\vspace{1em}
\uncover<2->{
\[
(P_1\rightarrow L_1)=    \begin{pmatrix}
      \texttt{[]} & \texttt{\_} & \rightarrow & 1 \\
    \end{pmatrix}
\]
\[
(P_2\rightarrow L_2)=    \begin{pmatrix}
       \texttt{\_} & \texttt{[]} & \rightarrow & 2  \\
    \end{pmatrix}
\]
\[
(P_3\rightarrow L_3)=    \begin{pmatrix}
      \texttt{x::xs} & \texttt{y::ys} & \rightarrow & 3
    \end{pmatrix}
\]}
\end{minipage}\hspace{.3cm}
\begin{minipage}{0.45\linewidth}%\vspace{-1em}
\begin{minipage}{0.45\linewidth}
\only<3>{\btCompAone}
\only<4>{\btCompA}
%\vspace{1em}
\end{minipage}
\end{minipage}
\end{frame}

\begin{frame}[fragile]{Пример: Компиляция в автомат с возвратами + оптимизация}
\begin{minipage}[t]{0.45\linewidth}\vspace{0pt}
\uncover<1>{
\[
(P\rightarrow L)=    \begin{pmatrix}
      \texttt{[]} & \texttt{\_} & \rightarrow & 1 \\
      \texttt{\_} & \texttt{[]} & \rightarrow & 2  \\
      \texttt{x::xs} & \texttt{y::ys} & \rightarrow & 3
    \end{pmatrix}
\]}
%\vspace{1em}
\uncover<2->{
\[
(P\rightarrow L)=    \begin{pmatrix}
      \texttt{[]} & \texttt{\_} & \rightarrow & 1 \\
      \texttt{x::xs} & \texttt{y::ys} & \rightarrow & 3\\
      \texttt{\_} & \texttt{[]} & \rightarrow & 2  
    \end{pmatrix}
\]
}\vspace{1cm}
\uncover<3->{
\[
(P_1\rightarrow L_1)=    \begin{pmatrix}
      \texttt{[]} & \texttt{\_} & \rightarrow & 1 \\
      \texttt{x::xs} & \texttt{y::ys} & \rightarrow & 3
    \end{pmatrix}
\]
\[
(P_2\rightarrow L_2)=    \begin{pmatrix}
       \texttt{\_} & \texttt{[]} & \rightarrow & 2  \\
    \end{pmatrix}
\]
}
\end{minipage}\hspace{.5cm}
\begin{minipage}[t]{0.45\linewidth}\vspace{0pt}
\begin{minipage}{0.44\linewidth}
\only<1-2>{\vspace{18em}}
\only<4>{\btCompBone\vspace{1em}}
\only<5>{\btCompB}
\end{minipage}
\end{minipage}
\end{frame}


\begin{frame}{Ещё оптимизации для автоматов с возвратами}
\begin{itemize}
\item Использование информации о полноте
\item Оптимизация конструкций \lstinline[language=ocamllambda]{exit}, новый синтаксис \lstinline[language=ocamllambda]{exit N}
\begin{itemize}
\item Дешёвая поддержка охранных выражений (pattern guards)
\end{itemize}
\item Ещё кое-что, для избегания повторных проверок
\end{itemize}
\vspace{2em}
В компиляторе \OCaml{}  используется  реализация из~\cite{maranget2001}.
\end{frame}

%%%%%%%%%%%%%%%%%%%%%%%%%%%%%%%%%%%%%%%%%%%%%%%%%%%%%%%%%%%%%%%%%%%%%
\defverbatim[colored]{\haskellDemoHead}{
\begin{lstlisting}[language=haskell]
f :: Bool -> Bool
f True = True 
f x    = f x 

y :: Bool
y = f x 
\end{lstlisting}
}

\defverbatim[colored]{\haskellDemo}{
\begin{lstlisting}[language=haskell]
case x,y of 
  True, False -> 1
  True, True  -> 2
  False,False -> 3
  False,True  -> 4
\end{lstlisting}
}

\begin{frame}[fragile]{Языки с ленивой семантикой}
\begin{minipage}[t]{0.45\linewidth}
\begin{minipage}{0.5\linewidth}
\uncover<2->{\haskellDemoHead}

\haskellDemo
\end{minipage}
\end{minipage}\hspace{0.1cm}
\begin{minipage}[t]{0.45\linewidth}
\uncover<3>{
\begin{minipage}{1\linewidth}
В \textit{ленивых} языках у нас нет свободы в выборе первого столбца $\Rightarrow$ в \textit{строгих} больше простора для оптимизаций
\end{minipage}}
\end{minipage}
\end{frame}

\begin{frame}{Понятие "хорошей" скомпилированной программы}
\begin{minipage}[t]{0.45\linewidth}
В литературе упоминаются~\cite{Scott2000WhenDM} следующие эвристики
\begin{itemize}
\item Слева-направо
\item Справа-налево
\item Small branching factor (малый коэффициент ветвления)
\item Large branching factor 
\item Leaf edges 
\item Arity factor
\item Artificial rule
\item и другие
\end{itemize}
\end{minipage}\hspace{1cm}
\begin{minipage}[t]{0.45\linewidth}
В подавляющем большинстве случаев формально более оптимальные программы показывают сходную производительность
\end{minipage}
\end{frame}

%%%%%%%%%%%%%%%%%%%%%%%%%%%%%%%%%%%%%%%%%%%%%%%%%%%%%%%%%%%%%%%
\section{Заслуги работы}

\begin{frame}{Заслуги работы <<Реляционный синтез сопоставления с образцом>>}
\Large
%"Реляционный синтез сопоставления с образцом"
\contributions
\end{frame}

\subsection{Синтез с помощью реляционных интерпретаторов}

\begin{frame}{Реляционные интерпретаторы}
Семейство языков \miniKanren{} -- реинкарнация логического программирования\\

Программы представляются как вычислимые отношения ("реляции", англ. relation)

\[
  \texttt{Interpret}: \texttt{Program} \times \texttt{Input} \times \texttt{Result}
\]

Их можно запускать\footnote{Если интерпретатор написан достаточно аккуратно} в разные стороны:
\[
  \texttt{Interpret}_{\rightarrow}: \texttt{Program} \times \texttt{Input} \rightarrow \texttt{Result}
\]
\[
  \texttt{Synthesize}_{\leftarrow}:   \texttt{Input} \times \texttt{Result} \rightarrow \texttt{Program}
\]
\end{frame}

\begin{frame}[fragile]{Синтаксис двух языков сопоставления с образцом}
\begin{figure}[ht]
\begin{subfigure}[t]{0.4\linewidth}
\uncover<2->{
\[
match^o: \mathcal{V}\times \mathcal{P}^* \times  \mathbb{N}
\]}
Язык сопоставления с образцом
\[
 \begin{array}{rcll}
    \mathcal{C} & = & \{ C_1^{k_1}, \dots, C_n^{k_n} \}\\
    \mathcal{V} & = & \mathcal{C}\,\mathcal{V}^*\\  
    \mathcal{P} & = & \_ \mid \mathcal{C}\,\mathcal{P}^*
 \end{array}
\]

\[
 \begin{array}{c}
\trans{\inbr{v;\,p_1,\dots,p_k}}{}{i}\\
1\leqslant i\leqslant k+1
 \end{array}
\]
\end{subfigure}
\hspace{0.5cm}
\begin{subfigure}[t]{0.5\linewidth}
\uncover<2->{
\[
eval^o_{\mathcal S}: \mathcal{V}\times \mathcal{S} \times  \mathbb{N}
\]}
Язык скомпилированного представления
\[
\begin{array}{rcl}
  \mathcal M & = & \bullet \\
  &   & \mathcal M\,[\mathbb{N}] \\
  \ir & = & \primi{return}\,\mathbb{N} \\
  &   & \primi{switch}\;\mathcal{M}\;\primi{with}\; [\mathcal{C}\; \primi{\rightarrow}\; \ir]^*\;\primi{otherwise}\;\ir
\end{array}
\]
\end{subfigure}
\end{figure}
\vspace{2em}
Опущены для простоты: типы, охранные выражения, переменные в паттернах и т.д.

\end{frame}

\begin{frame}{Алгоритм синтеза}
%High-level description:
\[
%depth^o\,v\,n \;\; \wedge \;\;
\forall v \quad \forall  (1\leqslant\!i\leqslant \!k\!+\!1) \quad
%\mbox{\lstinline|fresh ($i$)|}\; \{
(match^o\,v\,\,p_1,\dots,p_k\,\,i) \;\; \Leftrightarrow \;
eval^o_{\mathcal S}\,v\,\circled{?}\,i
%\}
\]

\begin{itemize}
%\item $depth^o\,v\,n$ generates examples bound by depth $n$
%\item $\forall v$ -- all possible scrutinees
%\item $\forall i$ -- all possible branches
\item $match^o\,v\,\,p_1,\dots,p_k\,\,i$ -- интерпретатор языка сопоставления с образцом, для каждого сопоставляемого выражения (scrutinee) $v$ выдает номер ветви $i$ 
\item $\circled{?}$ -- программа, которую надо синтезировать
\item $eval^o_{\mathcal S}\,v\,\circled{?}\,i$ -- a интерпретатор скомпилированного представления $\mathcal S$, которая проверяет, что синтезированная программа $\circled{?}$ на примерах $v$ выдает правильные номера ветвей  $i$
\end{itemize}
\vspace{1cm}
\begin{itemize}
\item[\faGood] Интерпретаторы  $match^o$ и $eval^o$ легко реализовать (и для расширений сопоставления с образцом тоже должно быть легко)
\item[\faBad] \miniKanren{} с disequality constraints не умеет работать с кванторами $\forall$
\end{itemize}
\end{frame}

\subsection{Создание конечного набора примеров}

\begin{frame}{Избавление от $\forall$. Создание конечного набора примеров}
Для каждого сопоставления с образцом  мы знаем:
\begin{itemize}
\item тип сопоставляемого выражения
\item все образцы, которые используются
\begin{itemize}
\item можем посчитать максимальную глубину образцов
\end{itemize}
\end{itemize}

\begin{alertblock}{\textbf{Идея}}
Переберём всех жителей типа сопоставляемого выражения до некоторой глубины, и будем использовать этих жителей как примеры
\end{alertblock}
\vspace{1cm}

\begin{itemize}
%\item[\faGood] Is correct (proved in Coq)
\item[\faBad] В худшем случае ---~экспоненциальное количество примеров
\end{itemize}
\end{frame}

\begin{comment}
\begin{frame}{Our Method: Synthesis + Relational Interpreter}
Do \emph{not compile} with specific algorithm but \emph{synthesize} compiled representation on large enough but finite set of examples
\vspace{1cm}

Fast $\Leftrightarrow$ Small: synthesized program should contain less checks.
\vspace{1cm}

We are using relational programming, more precisely OCanren~\cite{OCanrenWeb} from miniKanren~\cite{MiniKanrenWeb} family.
\vspace{1cm}


Our repo on Github: ~\cite{Repo}.
\end{frame}
\end{comment}


\defverbatim[colored]{\exampleNotAFullSet}{
\begin{lstlisting}{ocaml}
match (s : unit list) with 
| _ :: _ :: _ -> 1
| _           -> 2
\end{lstlisting}
}

\begin{frame}{Пример 1: Полный набор примеров из трёх штук}
\centering
\begin{minipage}{.5\textwidth}
\begin{minipage}{.5\textwidth}
\exampleNotAFullSet
\end{minipage}\end{minipage}

\begin{figure}
\begin{subfigure}[t]{0.3\linewidth}
\begin{subfigure}[b][4cm][t]{0.2\linewidth}
\begin{tikzpicture}[sibling distance=3em,
  every node/.style = {shape=rectangle, rounded corners, minimum height=.5cm,
    draw, align=center, font=\ttfamily
    ,top color=white, bottom color=blue!20}]]
  \node {Cons}
    child { node {\phantom{1}\_\phantom{1}} }
    child { node {Cons}
      child { node  {\phantom{1}\_\phantom{1}} }
      child { node {\phantom{1}\_\phantom{1}} } 
    };
\end{tikzpicture}
\end{subfigure}
\hspace{2cm}
\begin{subfigure}[b][4cm][t]{0.1\linewidth}
\begin{tikzpicture}[sibling distance=3em,
  every node/.style = {shape=rectangle, rounded corners, minimum height=.5cm,
    draw, align=center, font=\ttfamily
    ,top color=white, bottom color=blue!20}]]
  \node {\phantom{1}\_\phantom{1}};
\end{tikzpicture}
\end{subfigure}
\caption{Два образца}
\end{subfigure}
\hspace{1.5cm}
\begin{subfigure}[t]{0.5\linewidth}
\begin{subfigure}[b][4cm][t]{0.2\linewidth}
\begin{tikzpicture}[sibling distance=3em,
  every node/.style = {shape=rectangle, rounded corners, minimum height=.5cm,
    draw, align=center, font=\ttfamily
    ,top color=white, bottom color=green!20}]]
  \node {Cons}
    child { node {()} }
    child { node {Cons}
      child { node {()} }
      child { node { Nil } } 
    };
\end{tikzpicture}
\end{subfigure}
\hspace{1cm}
\begin{subfigure}[b][4cm][t]{0.15\linewidth}
\begin{tikzpicture}[sibling distance=3em,
  every node/.style = {shape=rectangle, rounded corners, minimum height=.5cm,
    draw, align=center, font=\ttfamily
    ,top color=white, bottom color=green!20}]]
  \node {Cons}
    child { node {()} }
    child { node {Nil} } ;
\end{tikzpicture}
\end{subfigure}
\hspace{1cm}
\begin{subfigure}[b][4cm][t]{0.15\linewidth}
\begin{tikzpicture}[sibling distance=3em,
  every node/.style = {shape=rectangle, rounded corners, minimum height=.5cm,
    draw, align=center, font=\ttfamily
    , top color=white, bottom color=green!20}]]
  \node {Nil};
\end{tikzpicture}
\end{subfigure}
\caption{Три примера}
\end{subfigure}
\end{figure}
\end{frame}


\begin{frame}[fragile]{Пример 2: Недостаточно полный набор примеров}
\begin{minipage}{0.4\linewidth}
\begin{lstlisting}{ocaml}
match (s : unit list) with 
| []  -> 1
| _   -> 2
\end{lstlisting}
\end{minipage}
\begin{minipage}{0.55\linewidth}
Два образца:\\
\begin{tikzpicture}[sibling distance=3em,
  every node/.style = {shape=rectangle, rounded corners, minimum height=.5cm,
    draw, align=center, font=\ttfamily
    ,top color=white, bottom color=blue!20}]]
  \node { Nil };
\end{tikzpicture}
\begin{tikzpicture}[sibling distance=3em,
  every node/.style = {shape=rectangle, rounded corners, minimum height=.5cm,
    draw, align=center, font=\ttfamily
    ,top color=white, bottom color=blue!20}]]
  \node {\phantom{1}\_\phantom{1}};
\end{tikzpicture}
\vspace{1em}

Набор примеров, ограниченных глубиной 1, состоит только из одного примера\\
\begin{tikzpicture}[sibling distance=2em,
  every node/.style = {shape=rectangle, rounded corners, minimum height=.5cm,
    draw, align=center, font=\ttfamily
    , top color=white, bottom color=green!20}]]
  \node {Nil};
\end{tikzpicture}
\vspace{1em}

Данная программа ведет себя согласовано на наборе примеров \\
\[
 \begin{array}{ll}
    \primi{switch} \dots \primi{with}   & \\
    |\ \text{Nil} \rightarrow 1  & \\  
    |\ \primi{otherwise} \rightarrow  1 & 
 \end{array}
\]

но, очевидно, неправильная
\end{minipage}
\end{frame}


\begin{frame}{Текущий алгоритм для получения примеров}
\begin{figure}
\begin{subfigure}[b]{0.75\linewidth}
Если кратко:
\begin{itemize}
\item Вычислить глубину образцов $h$
\item Синтезируем всех жителей, но
\item на глубине $h+1$ используем заранее подготовленного жителя соответствующего типа
\end{itemize}
\vspace{1cm}

\begin{itemize}
\item[\faBad] В худшем случае ---~экспоненциальное количество примеров
\end{itemize}
\end{subfigure}
\end{figure}
\end{frame}

\subsection{Оптимизации}

\begin{frame}[fragile]{Оптимизация: откидывание эквивалентных программ}
Очевидно, что реляционный интерпретатор языка \ir{} может перебирать различные эквивалентные программы
\begin{center}
  \begin{minipage}[t]{0.2\linewidth}
  \begin{center}
  \begin{lstlisting}[language=ocamllambda,gobble=2]
  switch Scru with 
  true -> ...
  false -> ...
  \end{lstlisting}
  \end{center}
  \end{minipage}\hspace{.5cm}
  \begin{minipage}[t]{0.2\linewidth}
  \huge
  \begin{center}
  \[
  \Leftrightarrow
  \]
  \end{center}
  \end{minipage}\hspace{.5cm}
  \begin{minipage}[t]{0.2\linewidth}
  \begin{lstlisting}[language=ocamllambda,gobble=2]
  switch Scru with 
  false -> ...
  true -> ...
  \end{lstlisting}
  \end{minipage}
\end{center}
С этой проблемой удалось побороться
\begin{itemize}
\item Задав порядок на ветвях \primi{switch}ей, используя информацию о типе
\item Это несколько <<сломало>> интерпретатор, но для синтеза это не существенно
\end{itemize}

\end{frame}

\defverbatim[colored]{\improvementA}{
\begin{lstlisting}{ocaml}
match (s : bool * bool * bool) with 
| (_,_,F) -> 1
| (_,_,T) -> 2
\end{lstlisting}
}

\begin{frame}{Оптимизация: сокращение необходимого набора примеров}
Пример:
\improvementA
\vspace{1em}

Мы можем во время компиляции обнаружить, что
\begin{itemize}
\item Всего $2^3$ жителей у типа \lstinline=bool * bool * bool=
\item Не надо проверять, что сопоставляемое выражение --- это тройка
\item Не надо заглядывать в 1ю и 2ю компоненты, так как важна только третья 
\end{itemize}
\vspace{1em}


Итого, может запускать синтез только на двух примерах $\{(\mathcal{B},\mathcal{B},\text{T}),(\mathcal{B},\mathcal{B},\text{F})\}$
(где $\mathcal{B}$ ---~это любое значение типа \lstinline=bool=)
если мы \textbf{запретим "заглядывание" в ненужные поддеревья} примеров
\end{frame}


\begin{frame}{Оптимизация методом ветвей и границ}
Текущий результат синтеза хранится в  $\circled{?}$ и во время поиска 
%Synthesis answer is stored in variable $\circled{?}$ and 
%Our answer is always in single variable -- $\circled{?}$ and 
мы \emph{только уточняем} результат, добавляя новые конструкции $\primi{switch}$ \\

\begin{alertblock}{\textbf{Идея}}
Если текущее приближение ответа длиннее, чем уже найденный ответ --- прерырываем поиск в этой ветке
\end{alertblock}

Требует модификации примитива \miniKanren{} \lstinline=run=: для каждого найденного ответа
\begin{itemize}
\item считаем размер
\item обновляем минимальный найденный ответ
\end{itemize}

Отсечение ветвей поиска делается помощью нового примитива --- \emph{ограничения на структуру (structural constraint)}

\end{frame}

\begin{frame}{Ограничение на структуру (structural constraint)}
Новый примитив:
\begin{itemize}
\item Принимает промежуточное представление значений и конвертирует их (в текущем состоянии) до логических %a logic value and performs reification in a current state
\item Принимает предикат для логических значений %Takes a predicate which inspects the reified value 
\item Если результат слишком большой ---~прекратить поиск% If there are too many branches, stops the search 
($failure^o$)
\item Иначе продолжать поиск, не меняя состояние 
%Otherwise continues search without changing the state 
($success^o$)
\end{itemize}
\vspace{1em}
%Peculiarities:
Особенности:
\begin{itemize}
\item Используется для вычисления размера текущего решения
\item Может учитывать или не учитывать disequality constraints
\item Можно использовать, чтобы реализовать $absent^o$ и подобные играничения
\item Работает с логическим представлением (reified) значений $\Rightarrow$ медленно
\end{itemize}
\end{frame}



\begin{frame}[fragile]{Критерий минимизации для синтезированных программ}
\begin{figure}
\begin{subfigure}[b]{0.3\linewidth}
\[
  \begin{array}{l}
  \primi{switch}\;\mathcal{M}\;\primi{with}\; \\
  \mathcal{C}_1\; \primi{\rightarrow}\; \ir_1\\
  \dots \\
  \mathcal{C}_n\; \primi{\rightarrow}\; \ir_n\\
  \primi{otherwise}\;\ir\\
  \end{array}\\
\]
\[
  \Updownarrow
\]
\[
  \begin{array}{l}
  \primi{if}\; \mathcal{M} = \mathcal{C}_1\;\primi{then}\; \ir_1\\
  \dots \\
  \primi{else\ if}\; \mathcal{M} = \mathcal{C}_n\; \primi{then}\;  \ir_n\\
  \primi{else}\;\ir\\
  \end{array}\\
\]
\vspace{1cm}

\end{subfigure}
\hspace{1cm}
\begin{subfigure}[b]{0.6\linewidth}
Интуиция: один \primi{switch} с $n$ случаев можно примерно закодировать в  $n$ \primi{if}ов\\

Будем считать, что размер
\begin{itemize}
\item \primi{switch} --- это число веток
\item \primi{return} равен 0
\item программы целиком --- сумма размеров всех входящих в неё \primi{switch}
\end{itemize}
Наш критерий минимизации: уменьшение размера синтезированной программы \\

Но могут быть другие: глубина, коэффициент ветвления, и т.д.
\end{subfigure}
\end{figure}

\end{frame}






%%%%%%%%%%%%%%%%%%%%%%%%%%%%%%%%%%%%%%%%%%%%%%%%
\section{Реализация}

\begin{frame}{Реализация}
Мы пользуемся \OCanren{} --- типобезопасным встраиванием \miniKanren{} в \OCaml{}.
\vspace{2em}

В процессе используется \noCanren{}~\cite{RelConversion} для порождения кода на \OCanren{}~\cite{OCanren,OCanrenWeb}
\vspace{1em}

Основная часть (два реляционных интерпретатора + порождение примеров) сделаны с помощью \noCanren{}.
\vspace{1em}

Репозиторий проекта~\cite{Repo}
\end{frame}

%%%%%%%%%%%%%%%%%%%%%%%%%%%%%%%%%%%%%%%%%%%%%%%%
\section{Проблемы}

\defverbatim[colored]{\synthInputA}{
\begin{lstlisting}
match $\bullet$ with
(_, false, true) -> 1
(false, true, _) -> 2
(_, _, false) -> 3
(_, _, true) -> 4
\end{lstlisting}
}
\defverbatim[colored]{\synthResultA}{
\begin{lstlisting}[language=ocamllambda]
switch $\bullet$[0] with  
| true -> 
    (switch $\bullet$[1] with  
    | true -> 
        (switch $\bullet$[2] with true -> 4 | _ -> 3)
    | _ -> 
        (switch $\bullet$[2] with true -> 1 | _ -> 3))  
| _ -> 
    (switch $\bullet$[1] with  
    | true -> 2   
    | _ -> 
         (switch $\bullet$[2] with true -> 1 | _ -> 3))
\end{lstlisting}
}

\defverbatim[colored]{\synthResultB}{
\begin{lstlisting}[language=ocamllambda]
switch Scru[0] with  
| true -> 
    (switch Scru[2] with  
    | true -> 
       (switch Scru[1] with true -> 4 | _ -> 1)
    | _ -> 3 )  
| _ -> 
    (switch Scru[1] with  
    | true -> 2   
    | _ -> (switch Scru[2] with  true -> 1 | _ -> 3))
\end{lstlisting}
}

\defverbatim[colored]{\synthResultC}{
\begin{lstlisting}[language=ocamllambda]
switch $\bullet$[1] with  
| true -> 
    (switch $\bullet$[0] with  
    | true -> 
        (switch $\bullet$[2] with true -> 4 | _ -> 3)
    | _ -> 2) 
| _ -> 
    (switch $\bullet$[2] with true -> 1 | _ -> 3)
\end{lstlisting}
}

\begin{frame}[fragile]{Пример синтезированной программы: сопоставление выражения типа \lstinline[language=ocaml]{bool*bool*bool}}
\begin{minipage}[c][7cm][t]{0.35\linewidth}\vspace{0em}
\begin{minipage}{0.35\linewidth}
\synthInputA
\end{minipage}
\end{minipage}
\begin{minipage}{0.63\linewidth}\vspace{0em}
\begin{onlyenv}<1>\begin{minipage}{0.63\linewidth}
Ответ размера  6 (за 1.6 секунд)
\synthResultA
\end{minipage}\end{onlyenv}
%
\begin{onlyenv}<2>\begin{minipage}{0.63\linewidth}
Ответ размера 5 (за +0.4 секунд)
\synthResultB
\end{minipage}\end{onlyenv}
\begin{onlyenv}<3>\begin{minipage}{0.63\linewidth}
Ответ размера 4 (за +0.7 секунд)
\synthResultC
\end{minipage}\end{onlyenv}
\end{minipage}
\end{frame}

%%%%%%%%%%%%%%%%%%%%%%%%%%%%%%%%%%%%%%%%%%%%%%%%

\begin{frame}[fragile]{Проблемы с производительностью (1/2)}
\begin{figure}
\begin{subfigure}[t]{0.5\linewidth}
\begin{lstlisting}[basicstyle=\small,language=ocaml]
match a,s,c with
| (_,_,Ldi i::_) -> 1
| (_,_,Push::_)  -> 2
| (Int _,Val (Int _)::_,IOp _::_) -> 3
| (Int _,_,Test (_,_)::c) -> 4
| (Int _,_,Test (_,_)::c) -> 5
| (_,_,Extend::_) -> 6
| (_,_,Search _::_) -> 7
| (_,_,Pushenv::_) -> 8
| (_,Env e::s,Popenv::_) -> 9
| (_,_,Mkclos cc::_) -> 10
| (_,_,Mkclosrec _::_) -> 11
| (Clo (_,_), Val _::_, Apply::_) -> 12
| (_,(Code _::Env _::_),[]) -> 13
| (_,[],[]) -> 14
\end{lstlisting}
\end{subfigure}
\hspace{1cm}
\begin{subfigure}[t]{0.35\linewidth}\vspace{0em}
Интерпретатор PCF (mini-ML) из статьи Г.Плоткина, 1977
\vspace{4em}

Сейчас не работает, потому что слишком много (11102) примеров
\begin{itemize}
\item большая глубина образцов
\item много конструкторов в типах
\end{itemize}
%Doesn't currently work because the types are too large (11102 examples  generated).\\

%But for a reduced example we can synthesize the answer
%\vspace{1cm}
\end{subfigure}
\end{figure}
\end{frame}

\begin{frame}[fragile]{Проблемы с производительностью (2/2)}
\begin{figure}
\begin{subfigure}[b][7cm][b]{0.35\linewidth}\vspace{0em}
\begin{lstlisting}[basicstyle=\small]
type code = 
| Push 
| Ldi of int 
| IOp of int 
| Int of int 
type prog = code list 
type item = 
| Val of code 
| Env of int 
| Code of int
type stack = item list 

match (code,stack,prog) with
| (_, _, (Ldi _)::_)  -> 1
| (_, _, (Push _)::_) -> 2
\end{lstlisting}
\begin{onlyenv}<1>
\vspace{4mm}
\end{onlyenv}
\begin{onlyenv}<2>
\begin{lstlisting}[basicstyle=\small,aboveskip=-0.5em]
| (Int _, _, (IOp _)::_) -> 3
\end{lstlisting}
\end{onlyenv}
\end{subfigure}
\hspace{.5cm}
\begin{subfigure}[b]{0.65\linewidth}
\begin{overlayarea}{8cm}{6cm}
Сокращённый пример 
\begin{itemize}
\item по типам 
\item по веткам
\end{itemize}
Для двух веток  надо 5 примеров\\\vspace{1em}

\begin{onlyenv}<2->
Для трёх веток и тех же типов уже необходимо 20 примеров
\begin{itemize}
\item за 1,5с получим 1й ответ размера 7
\item ещё через полсекунды --- 2й и 3й (последний) ответы размера 6 и 5, соответственно
\item в конце оно тратит 10с, чтобы доказать, что более коротких ответов не существует
\end{itemize}
%\begin{lstlisting}[basicstyle=\small,aboveskip=-0.5em]
%| (Int _, _, (IOp _)::_) -> 3
%\end{lstlisting}
\end{onlyenv}
\end{overlayarea}
\vspace{1cm}
\end{subfigure}
\end{figure}

\end{frame}

\section{Заключение}

\begin{frame}{Результаты}% и задачи на будущее}
Достижения: \contributions
\vspace{1em}

На маленьких примерах подход работает корректно, ... но на больших есть проблемы с производительностью 
\vspace{1em}
\end{frame}


\begin{frame}{Пути дальнейшего улучшения}
\begin{itemize}
\item Построение входного логического значения для  ограничений на структуру можно делать эффективнее (ленивые вычисления)
\item Использование конечнодоменных ограничений вместо disequality constraints 
\item Мемоизация сейчас никак не используется, т.к.  disequality constraints
\item Создание меньшего числа примеров
\item Другое представление языка \ir{} с использованием конструкций \primi{exit}
\begin{itemize}
\item тут может потребоваться номинальная унификация
\end{itemize}
\end{itemize}
\end{frame}

\begin{frame}
\begin{center}
{\Huge Спасибо!}
\end{center}
\end{frame}


\begin{frame}%[t, allowframebreaks]
\frametitle{Литература}
\bibliographystyle{amsalpha}
\bibliography{references}
\vspace{1cm}
\end{frame}

\end{document}
