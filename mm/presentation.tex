\documentclass{beamer}
\usepackage{beamerthemesplit}
\usepackage{wrapfig}
\usetheme{SPbGU}
\usepackage{pdfpages}
\usepackage{amsmath}
\usepackage{cmap} 
\usepackage[T2A]{fontenc} 
\usepackage[utf8]{inputenc}
\usepackage[english,russian]{babel}
\usepackage{indentfirst}
\usepackage{amsmath}
\usepackage{tikz}
\usepackage{multirow}
\usepackage[noend]{algpseudocode}
\usepackage{algorithm}
\usepackage{algorithmicx}
\usepackage{fontawesome}          % sudo apt install texlive-fonts-extra

\usetikzlibrary{shapes,arrows}
%usepackage{fancyvrb}
%\usepackage{minted}
%\usepackage{verbments}

\beamertemplatenavigationsymbolsempty
% \title[]{YaccConstructor}
% \subtitle[YaccConstructor]{Курсовые проекты 2017}
% % То, что в квадратных скобках, отображается в левом нижнем углу. 
% \institute[]{
% Лаборатория языковых инструментов JetBrains \\
% Санкт-Петербургский государственный университет \\
% Математико-механический факультет }
% 
% % То, что в квадратных скобках, отображается в левом нижнем углу.
% \author[YC Team]{}
% 
% \date{28 сентября 2017г.}
% 
% \definecolor{orange}{RGB}{179,36,31}

\begin{document}
% {
% \begin{frame}[fragile]
%   \begin{tabular}{p{2.5cm} p{5.5cm} p{2cm}}
%    \begin{center}
%       \includegraphics[height=1.5cm]{pictures/JBLogo3.pdf}
%     \end{center}
%     &
%     \begin{center}
%       \includegraphics[height=1.5cm]{pictures/SPbGU_Logo.png}
%     \end{center}
%     &
%     \begin{center}
%       \includegraphics[height=1.5cm]{pictures/YC_logo.pdf}
%     \end{center} 
%   \end{tabular}
%   \titlepage
% \end{frame}
% }

\begin{frame}[fragile]
  \transwipe[direction=90]
  \frametitle{á la Active Patterns для OCaml}
  \begin{itemize}
    \item Как вдохновение -- их реализация в F\#
    \item Примеры кода, план реализации и оценки сложности шагов находятся по ссылке
			\href{https://github.com/ocamllabs/compiler-hacking/wiki/Add-a-%22with%22-syntax-for-patterns}
          {\beamergotobutton{in english}}
		\item Описанное выше -- это некоторое упрощение...
		\item ... но сложности предстоят те же
		\item Будущие навыки: синтаксический анализ, внутренности компилятора, OCaml
  \end{itemize}
\end{frame}

\begin{frame}[fragile]
  \transwipe[direction=90]
  \frametitle{Кроссплатформенный вариант Barliman}

  \begin{itemize}
    \item GUI для своеобразного синтеза программ с помощью miniKanren
    \item Сейчас уже \href{https://github.com/webyrd/Barliman}{\beamergotobutton{сделан}} GUI на Cocoa
    \item Планируется Qt/QML 
    \item Язык реализации относительно произвольный
    \item Важная задача 1: запускать код на Scheme из приложения
    \item Важная задача 2: многопоточность
    \item Будущие навыки: Qt/QML и многопоточность
  \end{itemize}
\end{frame}

\begin{frame}
  \transwipe[direction=90]
  \frametitle{Тема: QtCreator}
  \begin{itemize}
    \item В частности фронтендная часть IDE для языка OCaml 
    \item Как независимая мини-задача - для Coq
    \item Светлая цель: интеграция имеющихся средств проектирования GUI
    \item Но обязательно: рефакторинги, автодополнения
    \item Подсветка синтаксиса host-языка и встроенных
    \item (Полу)автоматическое исправление ошибок -- code actions
    \item Будущие навыки: 
        \begin{itemize}
          \item отсутствие страха перед большими проектами, 
          \item языки С++ (OCaml не для всех)
          \item представление о том, как работает IDE
        \end{itemize}
  \end{itemize}
\end{frame}

\begin{frame}
  \transwipe[direction=90]
  \frametitle{Контакты}
  \begin{itemize}
    \item Руководитель: Косарев Дмитрий
    \item Dmitrii.Kosarev@protonmail.ch
    \item \faGithub Kakadu
  \end{itemize}
\end{frame}

\end{document}
